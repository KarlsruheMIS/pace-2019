\documentclass[twoside,leqno,twocolumn]{article}

\newif\ifFull
%\Fullfalse
\Fulltrue

% Comment out the line below if using A4 paper size
\usepackage[letterpaper]{geometry}

\usepackage{ltexpprt}
\usepackage{t1enc}
\usepackage[utf8]{inputenc}
\usepackage{numprint}
\npdecimalsign{.} % we want . not , in numbers
\usepackage{hyperref}
\usepackage{xspace}
\usepackage{doi}
\usepackage{enumerate}
\usepackage{booktabs}

\newcommand{\overbar}[1]{\mkern 1.5mu\overline{\mkern-1.5mu#1\mkern-1.5mu}\mkern 1.5mu}
\def\MdR{\ensuremath{\mathbb{R}}}
\def\MdN{\ensuremath{\mathbb{N}}}
%\DeclareMathOperator{\sgn}{sgn}
\newcommand{\Id}[1]{\texttt{\detokenize{#1}}}
\newcommand{\Is}       {:=}
\newcommand{\setGilt}[2]{\left\{ #1\sodass #2\right\}}
\newcommand{\sodass}{\,:\,}
\newcommand{\set}[1]{\left\{ #1\right\}}
\newcommand{\gilt}{:}
\newcommand{\ie}{i.\,e.,\xspace}
\newcommand{\eg}{e.\,g.,\xspace}
\newcommand{\etal}{et~al.\xspace}
\newcommand{\Wlog}{w.\,l.\,o.\,g.\ }
\newcommand{\wrt}{w.\,r.\,t.\xspace}
\newcommand{\csch}[1]{{\color{red} CS: #1}}

\newcommand{\mytitle}{WeGotYouCovered: The Winning Solver from the PACE 2019 Implementation Challenge, Vertex Cover Track}

\setlength\parfillskip{0pt plus .4\textwidth}
\setlength\emergencystretch{.1\textwidth}
\clubpenalty10000
\widowpenalty10000
\displaywidowpenalty=10000

\usepackage{changes}
\definechangesauthor[name={Darren Strash}, color=blue]{DS}
\definechangesauthor[name={Demian Hespe}, color=blue]{DH}

\newcommand{\AlgName}[1]{\textsf{#1}}



%\subjclass{G.2.2 Graph Theory -- Graph Algorithms, G.4 Mathematical Software -- Algorithm Design and Analysis} 
%\keywords{kernelization, branch-and-reduce, local search}
%\EventEditors{}
%\EventNoEds{0}
%\EventLongTitle{}
%\EventShortTitle{PACE 2019}
%\EventAcronym{PACE}
%\EventYear{2019}
%\EventDate{}
%\EventLocation{}
%\EventLogo{}
%\SeriesVolume{}
%\ArticleNo{}

\begin{document}
\title{\mytitle\thanks{
    The research leading to these results has received funding from the European Research Council under the European Community's Seventh Framework Programme (FP7/2007-2013) /ERC grant agreement No. 340506}}
\author{Demian Hespe\thanks{Karlsruhe Institute of Technology, Karlsruhe, Germany} \and Sebastian Lamm\thanks{Karlsruhe Institute of Technology, Karlsruhe, Germany} \and Christian Schulz\thanks{University of Vienna, Faculty of Computer Science, Austria} \and Darren Strash\thanks{Hamilton College, New York, USA}}

%\affil[1]{Karlsruhe Institute of Technology, Karlsruhe, Germany \\
  %\texttt{hespe@kit.edu}}
%\affil[2]{Karlsruhe Institute of Technology, Karlsruhe, Germany\\
  %\texttt{lamm@kit.edu}}
%\affil[3]{University of Vienna, Faculty of Computer Science, Vienna, Austria\\ \texttt{christian.schulz@univie.ac.at}}
%\affil[4]{Hamilton College, New York, USA,  \texttt{dstrash@hamilton.edu}}

\date{}


%\Copyright{}
\maketitle
\begin{abstract}
We present the winning solver of the PACE 2019 Implementation Challenge, Vertex Cover Track. The minimum vertex cover problem is one of a handful of problems for which \emph{kernelization}---the repeated reducing of the input size via \emph{data reduction rules}---is known to be highly effective in practice. Our algorithm uses a portfolio of techniques, including an aggressive kernelization strategy, local search, branch-and-reduce, and a state-of-the-art branch-and-bound solver. Of particular interest is that several of our techniques were \emph{not} from the literature on the vertex over problem: they were originally published to solve the (complementary) maximum independent set and maximum clique problems. 

%We perform extensive experiments to show the impact of the different solver techniques on the number of instances solved during the challenge.
%
Aside from illustrating our solver's performance in the PACE 2019 Implementation Challenge, our experiments provide several key insights not yet seen before in the literature.
First, kernelization can boost the performance of branch-and-bound clique solvers enough to outperform branch-and-reduce solvers. Second, local search can significantly boost the performance of branch-and-reduce solvers. And finally, somewhat surprisingly, kernelization can sometimes make branch-and-bound algorithms perform \emph{worse} than running branch-and-bound alone.
\end{abstract}

%apply an initial aggressive kernelization strategy, using all known reduction rules for the problem. From there we use local search to produce a high-quality solution on the (hopefully smaller) kernel, which we use as a starting solution for a branch-and-bound solver. Our branch-and-bound solver also applies reduction rules via a branch-and-reduce scheme -- applying rules when possible, and branching otherwise -- though this may be toggled to omit reductions if they are not effective.
%%%%%%%%%%%%%%%%%%%%%%%%%%%%%
\section{Introduction}

A \emph{vertex cover} of a graph $G=(V,E)$ is a set of vertices $S\subseteq V$ of $G$ such that every edge of $G$ has at least one of member of $S$ as an endpoint (i.e., $\forall (u,v) \in E\,\, [u\in S \textrm{ or } v \in S]$).
The minimum vertex cover problem---that of computing a vertex cover of minimum cardinality---is a fundamental NP-hard problem, and has applications spanning many areas. These include computational biology~\cite{cheng2008prediction}, classification~\cite{gottlieb2014efficient}, mesh rendering~\cite{sander2008efficient}, and many more through its complementary problems~\cite{gnntbdmlisaac13,gardiner-docking-2000,harary-clique-1957,zaki-ecommerce-97}.

Complementary to vertex covers are independent sets and cliques. An independent set is a set of vertices $I\subseteq V$, all pairs of which are not adjacent, and an clique is a set of vertices $K\subseteq V$ all pairs of which are adjacent. A maximum independent set (maximum clique) is an independent set (clique) of maximum cardinality. The goal of the maximum independent set problem (maximum clique problem) is to compute a maximum independent set (maximum clique).

Many techniques have been proposed for solving these problems, and papers in the literature usually focus on one of these problems in particular. However, all of these problems are equivalent: a
minimum vertex cover $C$ in $G$ is the complement of a maximum independent set $V\setminus C$ in $G$, which is a maximum clique $V\setminus C$ in $\overline{G}$. Thus, an algorithm that solves one of these problems can be used to~solve~the~others.
%
To win the PACE 2019 Implementation Challenge, we deployed a portfolio of solvers, using techniques from the literature on all three problems. These include data reduction rules and branch-and-reduce for the minimum vertex cover problem~\cite{akiba-tcs-2016}, iterated local search for the maximum independent set problem~\cite{andrade-2012}, and a state-of-the-art branch-and-bound maximum clique solver~\cite{DBLP:journals/cor/LiJM17}.

\paragraph*{Our Results.}
In this paper, we describe our techniques and solver in detail and analyze the results of our experiments on the data sets provided by the challenge.
Not only do our experiments illustrate the power of the techniques spanning the literature, they also provide several new insights not yet seen before.
In particular, kernelization followed by branch-and-bound can outperform branch-and-reduce solvers; seeding branch-and-reduce by an initial solution from local search can significantly boost its performance; and, somewhat surprisingly, kernelization is sometimes counterproductive: branch-and-bound algorithms can perform significantly worse on the kernel than on the original input graph.
%\csch{somehow emphasize that nothing like this has been
%  done before}
%  \comment[id=DH]{But it has for cliques (according to Section 2)}
%  \comment[id=DS]{those reductions have a different purpose, they only work on sparse graphs, removing low-degree vertices (or similar) based on a an initial clique from some inexact method. Clarified.}

\paragraph*{Organization.}
We first briefly describe related work in Section~\ref{sec:related_work}. Then in Section~\ref{sec:techniques} we outline each of the techniques that we use, and in Section~\ref{sec:puttingtogether} finally describe how we combine all of the techniques in our final solver that scored the most points in the PACE 2019 Implementation Challenge. Lastly, in Section~\ref{sec:experiments} we perform an experimental evaluation to show the impact of the components used on the final number of instances solved during the challenge.

\section{Preliminaries}
\label{sec:preliminaries}
We work with an undirected graph $G = (V,E)$ where $V$ is a set of $n$ vertices and $E\subset \{\{u,v\}\mid u,v\in V\}$ is a set of $m$ edges. The open neighborhood of a vertex $v$, denoted $N(v)$, is the set of all vertices $w$ such that $(v,w)\in E$. We further denote the closed neighborhood by $N[v]=N(v)\cup\{v\}$. We similarly define the open and closed neighborhoods of a set of vertices $U$ to be $N(U) = \bigcup_{u\in U}N(u)$ and $N[U] = N(U) \cup U$, respectively. The set of vertices of distance $d$ of a vertex $u$ is denoted by $N^d(u)$, where $N^2(u)$ is called the \emph{two-neighborhood} of $u$. Lastly, for vertices $S\subseteq V$, the induced subgraph $G[S]\subseteq G$ is the graph on the vertices in $S$ with edges in $E$ between vertices in $S$.

\section{Related Work}
\label{sec:related_work}
%We now present important related work. 
Research results in the area can be found through work on the minimum vertex cover problem and its complementary maximum clique and independent set problems, and can often be categorized depending on the angle of attack. For exact exponential (theoretical) algorithms, the maximum independent set problem is canonically studied, for parameterized algorithms, the minimum vertex cover problem is studied, and the maximum clique problem is normally solved exactly in practice (though there are recent exceptions). However, these problems are only \emph{trivially} different --- techniques for solving one problem require only subtle modifications to solve the other two.

\paragraph*{Exponential-time Algorithms.}
The maximum independent set problem is most often considered when designing exact (exponential-time) algorithms, and much research has be devoted to reducing the base of the exponential running time. A primary technique is  to develop rules to modify the graph, removing or contracting subgraphs that can be solved simply, which reduces the graph to a smaller instance. These rules are referred to as \emph{data reduction rules} (often simplified to \emph{reduction rules} or \emph{reductions}).
Reduction rules have been used to reduce the running time of the brute force $O(n^22^n)$ algorithm to the $O(2^{n/3})$ time algorithm of Tarjan and Trojanowski~\cite{tarjan-1977}, and to the current best polynomial space algorithm with running time of $O^*(1.1996^n)$ by Xiao and Nagamochi~\cite{xiao2017exact}. 

The reduction rules used for these algorithms are often staggeringly simple, including \emph{pendant vertex removal}, \emph{vertex folding}~\cite{chen1999} and \emph{twin} reductions~\cite{Xiao201392}, which eliminate nearly all vertices of degree three or less from the graph. 
These algorithms apply reductions during recursion, only branching when the graph can no longer be reduced~\cite{fomin-2010}, and are referred to as \emph{branch-and-reduce} algorithms. Further techniques used to accelerate these algorithms include \emph{branching rules}~\cite{kneis2009fine,fomin2009measure} which eliminate unnecessary branches from the search tree, as well as faster exponential-time algorithms for graphs of small maximum degree~\cite{xiao2017exact}.

\paragraph*{Parameterized Algorithms.}
For parameterized algorithms, we now turn to the minimum vertex cover problem. The most efficient algorithms for computing a minimum vertex cover in both theory and practice repeatedly apply data reduction rules to obtain a (hopefully) much smaller problem instance. If this smaller instance has size bounded by a function of some parameter, it's~called~a~\emph{kernel}, and producing a polynomially-sized kernel gives a fixed-parameter tractable in the chosen parameter. Reductions are surprisingly effective for the minimum vertex cover problem. In particular, letting $k$ be the size of a minimum vertex cover, the well-known crown reduction rule produces a kernel of size $3k$~\cite{chor2005linear} and the LP-relaxation reduction due to Nemhauser and Trotter~\cite{nemhauser-1975}, produces a kernel of size $2k$~\cite{chen1999}. Chen et al.~\cite{chen2010improved} developed the current best parameterized algorithm for minimum vertex cover, giving a branch-and-reduce algorithm with running time $O(1.2738^k +kn)$ and polynomial space.
For more information on the history of vertex cover kernelization, see the recent survey by Fellows et al.~\cite{fellows2018known}.

\paragraph*{Exact Algorithms in Practice.}
The most efficient maximum clique solvers use a branch-and-bound search with advanced vertex reordering strategies and pruning (typically using approximation algorithms for graph coloring, MaxSAT~\cite{li-maxsat-2013} or constraint satisfaction). The long-standing canonical algorithms for finding the maximum clique are the MCS algorithm by Tomita et al.~\cite{tomita-recoloring} and the bit-parallel algorithms of San Segundo et al.~\cite{segundo-recoloring,segundo-bitboard-2011}. However, recently Li et al.~\cite{DBLP:journals/cor/LiJM17} introduced the MoMC algorithm, which uses incremental MaxSAT logic to achieve speed ups of up to \numprint{1000} over MCS. Experiments by Batsyn et al.~\cite{batsyn-mcs-ils-2014} show that MCS can be sped up significantly by giving an initial solution found through local search. However, even with these state-of-the-art algorithms, graphs on thousands of vertices remain intractable. For example, a difficult graph on \numprint{4000} required 39 wall-clock hours in a highly-parallel MapReduce cluster, and is estimated to require over a year of sequential computation~\cite{xiang-2013}. Recent clique solvers for sparse graphs investigate applying simple data reduction rules, using an initial clique given by some inexact method~\cite{verma2015solving,sansegundo2016a,chang2019efficient}. However, these techniques rarely work on dense graphs, such as the complement graphs that we consider here.
A thorough discussion of many results in clique finding can be found in the survey of Wu and Hao~\cite{wu-hao-2015}.

Data reductions have been successfully applied in practice to solve many problems that are intractable with general algorithms. Butenko et al.~\cite{butenko-2002,butenko-correcting-codes-2009} were the first to show that simple reductions could be used to compute exact maximum independent sets on graphs with hundreds vertices for graphs derived from error-correcting codes. Their algorithm works by first applying \emph{isolated clique removal} reductions, then solving the remaining graph with a branch-and-bound algorithm. Later, Butenko and Trukhanov~\cite{butenko-trukhanov} introduced the \emph{critical independent set} reduction, which was able to solve graphs produced by the Sanchis graph generator.
Larson~\cite{larson-2007} later proposed an algorithm to find a \emph{maximum} critical independent set, but in experiments it proved to be slow in practice~\cite{strash2016power}.
Iwata~\etal~\cite{iwata-2014} then showed how to remove a large collection of vertices from a maximum matching all at once; however, it is not known if these reductions are equivalent.

For the minimum vertex cover problem, it has long been known that two such simple reductions, called \emph{pendant vertex removal} and \emph{vertex folding}, are particularly effective in practice. However, two seminal experimental works explored the efficacy of further reductions. Abu-Khzam et al.~\cite{abu-khzam-2007} showed that \emph{crown reductions} are as effective (and sometimes faster) in practice than performing the LP relaxation reduction (which, as they show in the paper, removes crowns) on graphs. We briefly note that critical independent sets, together with their neighborhoods, are in fact crowns, and thus in some ways the work of Butenko and Trukhanov~\cite{butenko-trukhanov} replicates that by Abu-Khzam et al.~\cite{abu-khzam-2007}, though their experiments are run on different graphs.

Later, Akiba and Iwata~\cite{akiba-tcs-2016} showed that an extensive collection of advanced data reduction rules (together with branching rules and lower bounds for pruning search) are also highly effective in practice. Their algorithm finds exact minimum vertex covers on a corpus of large social networks with hundreds of thousands of vertices or more in mere seconds. More details on the reduction rules follow in Section~\ref{sec:techniques}.

We briefly note that we considered other reduction techniques that emphasize fast computation at the cost of a larger (irreducible) graph~\cite{chang2017computing,strash2016power,DBLP:conf/alenex/Hespe0S18}; however, we did not find them as effective as Akiba and Iwata~\cite{akiba-tcs-2016} for exactly solving difficult instances. This is somewhat expected, however, since these techniques are optimized to produce fast high-quality solutions when combined with inexact methods such as local search.

\iffalse %doubt we will ever make another version of the paper, no need to discuss all of the version of local search algorithms, unless reviewers want it.
\paragraph*{Inexact Algorithms.}
\comment[id=DS]{Needed?}
\fi


\section{Techniques}
\label{sec:techniques}
We now describe techniques that we use in~our~solver.
\subsection{Kernelization.}
The most efficient algorithms for computing a minimum vertex cover in both theory and practice use \emph{data reduction rules} to obtain a much smaller problem instance. If this smaller instance has size bounded by a function of some parameter, it's~called~a~\emph{kernel}. 

We use an extensive (though not exhaustive) collection of data reduction rules whose efficacy was studied by Akiba and Iwata~\cite{akiba-tcs-2016}. To compute a kernel, Akiba and Iwata~\cite{akiba-tcs-2016} apply their
reductions~$r_1, \dots, r_j$ by iterating over all reductions and trying to
apply the current reduction $r_i$ to all vertices. If $r_i$ reduces at
least one vertex, they restart with reduction~$r_1$. When reduction~$r_j$ 
is executed, but does not reduce any vertex, all reductions have been applied
exhaustively, and a kernel is found. Following their study we order the reductions
as follows: degree-one vertex (i.e., pendant) removal, unconfined vertex removal~\cite{Xiao201392}, a well-known linear-programming 
relaxation~\cite{iwata-2014,nemhauser-1975} (which, consequently, removes crowns~\cite{abu-khzam-2007}),  vertex folding~\cite{chen1999}, and twin, funnel, and desk reductions~\cite{Xiao201392}.

To be self-contained, we now give a brief description of those reductions, in order of increasing complexity. Each reduction allows us to choose vertices that are either in some minimum vertex cover, or for which we can locally choose a vertex in a minimum vertex cover after solving the remaining graph, by following simple rules. If a minimum vertex cover  is found in the kernel, then each reduction may be undone, producing a minimum vertex cover in the original graph. Refer to Akiba and Iwata~\cite{akiba-tcs-2016} for a more thorough discussion, including implementation details. Our implementation of the reductions is an adaptation of Akiba and Iwata's original code. \\

\noindent\textbf{Pendant vertices:} Any vertex $v$ of degree one, called a \emph{pendant}, then its neighbor is in some minimum vertex cover, therefore $v$ and its neighbor $u$ can be removed from $G$. \\
%We can see this as follows: Either $v$'s neighbor is in a MIS $I$, or not. Suppose it is not in $I$, then $v$ must be in $I$. Suppose it is, then it can be removed from the MIS, and add $v$.

%\noindent\textbf{Dominance:} If there exist two vertices $u$ and $v$, such that $N(u) = N(v) \cup \{v\}$ then in any independent set $\mathcal{I}$ either both $v$ and $u$ will be excluded from $\mathcal{I}$, or exactly one of $u$ or $v$ will be in $\mathcal{I}$. Therefore, we may remove either $v$ or $u$ from the graph.

\noindent\textbf{Vertex folding:} For a vertex $v$ with degree 2 whose neighbors $u$ and $w$ are not adjacent, either $v$ is in some minimum vertex cover, or both $u$ and $w$ are in some minimum vertex cover. Therefore, we can contract $u$, $v$, and $w$ to a single vertex $v'$ and decide which vertices are in the vertex cover after computing a minimum vertex cover on the reduced graph. \\
%If $v'$ is in the computed MIS, then $u$ and $w$ are added to the independent set, otherwise $v$ is added. Thus, a vertex fold contributes an additional vertex to an independent set.

\noindent\textbf{Linear Programming Relaxation:}
First introduced by Nemhauser and Trotter~\cite{nemhauser-1975} for the vertex packing problem, they present a linear programming relaxation with a half-integral solution (i.e., using only values 0, 1/2, and 1) which can be solved using bipartite matching. Their relaxation may be formulated for the minimum vertex cover problem as follows: minimize $\sum_{v\in V}{x_v}$, such at for each edge $(u, v) \in E$, $x_u + x_v \geq 1$ and for each vertex $v \in V$, $x_v \geq 0$. There is a minimum vertex cover containing no vertices with value $1$, and therefore their neighbors are added to the solution and removed together with the vertices from the graph. We use the further improvement from Iwata et al.~\cite{iwata-2014}, which computes a solution whose half-integral part is minimal. \\

\noindent\textbf{Unconfined~\cite{Xiao201392}:} Though there are several definitions of an \emph{unconfined} vertex in the literature, we use the simple one from Akiba and Iwata~\cite{akiba-tcs-2016}. A vertex $v$ is \emph{unconfined} when determined by the following simple algorithm. First, initialize $S = \{v\}$. Then find a $u \in N(S)$ such that $|N(u) \cap S| = 1$ and $|N(u) \setminus N[S]|$ is minimized. If there is no such vertex, then $v$ is confined. If $N(u) \setminus N[S] = \emptyset$, then $v$ is unconfined.  If $N(u)\setminus N[S]$ is a single vertex $w$, then add $w$ to $S$ and repeat the algorithm. Otherwise, $v$ is confined. Unconfined vertices can be removed from the graph, since there always exists a minimum vertex cover that contains unconfined vertices. \\

\noindent\textbf{Twin~\cite{Xiao201392}:} Let $u$ and $v$ be vertices of degree 3 with $N(u) = N(v)$. If $G[N(u)]$ has edges, then add $N(u)$ to the minimum vertex cover and remove $u$, $v$, $N(u)$, $N(v)$ from $G$. Otherwise, some $u$ and $v$ may belong to some minimum vertex cover. We still remove $u$, $v$, $N(u)$ and $N(v)$ from $G$, and add a new gadget vertex $w$ to $G$ with edges to $u$'s two-neighborhood (vertices at a distance 2 from $u$). If $w$ is in the computed minimum vertex cover, then $u$'s (and $v$'s) neighbors are in some minimum vertex cover, otherwise $u$ and $v$ are in a minimum vertex cover.\\

\noindent\textbf{Alternative:} Two sets of vertices $A$ and $B$ are set to be \emph{alternatives} if $|A| = |B| \geq 1$ and there exists an minimum vertex cover $C$ such that $C\cap(A\cup B)$ is either $A$ or $B$. Then we remove $A$ and $B$ and $C = N(A)\cap N(B)$ from $G$ and add edges from each $a \in N(A)\setminus C$ to each $b\in N(B)\setminus C$.
Then we add either $A$ or $B$ to $C$, depending on which neighborhood has vertices in $C$. Two structures are detected as alternatives. First, if $N(v)\setminus \{u\}$ induces a complete graph, then $\{u\}$ and $\{v\}$ are alternatives (a \emph{funnel}). Next, if there is a cordless 4-cycle $a_1b_1a_2b_2$ where each vertex has at least degree 3. Then sets $A=\{a_1, a_2\}$ and $B=\{b_1, b_2\}$ are alternatives (called a \emph{desk}) when $|N(A) \setminus B| \leq 2$, $|N(A) \setminus B| \leq 2$, and $N(A) \cap N(B) = \emptyset$. 

\subsection{Branch-and-Reduce.}
Branch-and-reduce is a paradigm that intermixes data reduction rules and branching. We use the algorithm of Akiba and Iwata, which exhaustively applies their full suite of reduction rules before branching, and includes a number of advanced branching rules as well as lower bounds to prune search. 

\paragraph*{Branching.}
When branching, a vertex of maximum degree is chosen for inclusion into the vertex cover. Mirrors and satellites are detected when branching, in order to eliminate branching on certain vertices. A \emph{mirror} of a vertex $v$ is a vertex $u\in N^2(v)$ such that $N(v)\setminus N(u)$ is a clique or empty. Fomin et al.~\cite{fomin2009measure} show that either the mirrors of $v$ or $N(v)$ is in a minimum vertex cover, and we can therefore branch on all mirrors at once. This branching prevents branching on mirrors individually and decreases the size of the remaining graph (and thus the depth of the search tree). A \emph{satellite} of a vertex $v$ is a vertex $u\in N^2(v)$ such that there exists a vertex $w\in N(v)$ such that $N(w)\setminus N[u] = \{u\}$. If a vertex has no mirrors, then either $v$ is in a minimum vertex cover or the neighbors of $v$'s satellites in a minimum vertex cover. Akiba and Iwata~\cite{akiba-tcs-2016} further introduce \emph{packing} branching, maintaining linear inequalities for each vertex included or excluded from the current vertex cover (called \emph{packing constraints}) throughout recursion; when a constraint is violated, further branching can be eliminated.

\paragraph*{Lower Bounds.} We briefly remark that Akiba and Iwata~\cite{akiba-tcs-2016} implement lower bounds to prune the search space. Their lower bounds are based on clique cover, the LP relaxation, and cycle covers (see their paper for further details). The final lower bound used for pruning is the maximum of these three.

\subsection{Branch-and-Bound.} Experiments by Strash~\cite{strash2016power} show that the full power of branch-and-reduce is only needed \emph{very rarely} in real-world instances; kernelization followed by a standard branch-and-bound solver is sufficient for many real-world instances. Furthermore, branch-and-reduce does not work well on many synthetic benchmark instances, where data reduction rules are ineffective~\cite{akiba-tcs-2016}, and instead add significant overhead to branch-and-bound. We use a state-of-the-art branch-and-bound maximum clique solver (MoMC) by Li et al.~\cite{DBLP:journals/cor/LiJM17}, which uses incremental MaxSAT reasoning to prune search, and a combination of static and dynamic vertex ordering to select the vertex for branching. We run the clique solver on the complement graph, giving a maximum independent set from which we derive a minimum vertex cover. In preliminary experiments, we found that a kernel can sometimes be harder for the solver than the original input; therefore, we run the algorithm on both the kernel and on the original graph.

\subsection{Iterated Local Search.}
Batsyn et al.~\cite{batsyn-mcs-ils-2014} showed that if branch-and-bound search is primed with a high-quality solution from local search, then instances can be solved up to thousands of times faster. 
We use the iterated local search algorithm by Andrade et al.~\cite{andrade-2012} to prime the \emph{branch-and-reduce} solver with a high-quality initial solution. To the best of our knowledge, this has not been tried before. Iterated local search was originally implemented for the maximum independent set problem, and is based on the notion of $(j,k)$-swaps. A $(j,k)$-swap removes $j$ nodes from the current solution and inserts $k$ nodes. The authors present a fast linear-time implementation that, given a maximal independent set, can find a $(1,2)$-swap or prove that none exists. Their algorithm applies $(1,2)$-swaps until reaching a local maximum, then perturbs the solution and repeats. We implemented the algorithm to find a high-quality solution on \emph{the kernel}. Calling local search on the kernel has been shown to produce a high-quality solution much faster than without kernelization~\cite{chang2017computing,dahlum2016accelerating}.

\section{Putting it all Together}
\label{sec:puttingtogether}
Our algorithm first runs a preprocessing phase, followed by 4 phases of solvers.

%Our solver is a combination of different kernelization techniques \cite{DBLP:conf/alenex/Hespe0S18}, local search~\cite{DBLP:conf/wea/AndradeRW08}, as well as branch-and-reduce~\cite{akiba-tcs-2016,DBLP:journals/cor/LiJM17}.

%
%Our algorithm uses a portfolio of solvers, i.e., a branch-and-bound solver for vertex cover~\cite{akiba-tcs-2016} as well as a branch-and-bound solver for the maximum clique problem \cite{DBLP:journals/cor/LiJM17}.

\begin{description}
\item[Phase 1. (Preprocessing)] Our algorithm starts by computing a kernel of the graph using the reductions by Akiba and Iwata~\cite{akiba-tcs-2016}. 
From there we use iterated local search to produce a high-quality solution $S_{\textrm{init}}$ on the (hopefully smaller) kernel. 
\item[Phase 2. (Branch-and-Reduce, short)]
We prime a branch-and-reduce solver with the initial solution $S_{\textrm{init}}$ and run it with a short time limit.
\item[Phase 3. (Branch-and-Bound, short)]
If Phase 2 is unsuccessful, we run the MoMC~\cite{DBLP:journals/cor/LiJM17} clique solver on the complement of the kernel, also using a short time limit\footnote{Note that repeatedly checking the time can slow down a highly optimized branch-and-bound solver considerably; we therefore simulate time checking by using a limit on the number of branches.}. Sometimes kernelization can make the problem harder for MoMC. Therefore, if the first call was unsuccessful we also run MoMC on the complement of the original (unkernelized) input with the same short time limit.

\item[Phase 4. (Branch-and-Reduce, long)]
If we have still not found a solution, we run branch-and-reduce on the kernel using initial solution $S_{\textrm{init}}$ and a longer time limit. We opt for this second phase because, while most graphs amenable to reductions are solved very quickly with branch-and-reduce (less than a second),
experiments by Akiba and Iwata~\cite{akiba-tcs-2016} showed that other slower instances either finish in at most a few minutes, or take significantly longer---more than the time limit allotted for the challenge. This second phase of branch-and-reduce is meant to catch any instances that still benefit from reductions.

\item[Phase 5. (Branch-and-Bound, remaining time)]
If all previous phases were unsuccessful, we run MoMC on the original (unkernelized) input graph until the end of the time given to the program by the challenge. This is meant to capture only the hardest-to-compute instances.
\end{description}

The algorithm time limits (discussed in the next section) and ordering were carefully chosen so that the overall algorithm outputs solutions of the ``easy'' instances \emph{quickly}, while still being able to solve hard instances.
%\vfill\pagebreak
\section{Experimental Results}
\label{sec:experiments}
We now look at the impact of the algorithmic components on the number of instances solved.
Here, we focus on the public instances  of the PACE 2019 Implementation Challenge, Vertex Cover Track A, obtained from \url{https://pacechallenge.org/files/pace2019-vc-exact-public-v2.tar.bz2}. This set contains 100 instances overall. We also summarize the results comparing against the second and third best competing algorithms on the private instances during the challenge (which can be found at \url{https://pacechallenge.org/2019/} and \url{https://www.optil.io/optilion/problem/3155}). Note that further comparisons are not yet possible, as the private instances have not yet been released. 

\subsection{Methodology and Setup.}
All of our experiments were run on a machine with  four sixteen-core Intel Xeon Haswell-EX E7-8867 processors running at $2.5$ GHz, $1$ TB of main memory, and \numprint{32768} KB of L2-Cache.
The machine runs Debian GNU/Linux 9 and Linux kernel version 4.9.0-9.
All algorithms were implemented in C++11 and compiled with gcc~version 6.3.0 with optimization flag \texttt{-O3}. Our source code is publicly available under the MIT license at~\cite{wegotyoucovered2019}.
Each algorithm was run sequentially with a time limit of 30 minutes---the time allotted to solve a single data set in the PACE 2019 Implementation Challenge. Our primary focus is on the total number of instances solved.
\subsection{Evaluation.}
We now explain the main configuration that we use in our experimental setup.
In the following, \AlgName{MoMC} runs the MoMC clique solver by Li et al.~\cite{DBLP:journals/cor/LiJM17} on the complement of the input graph; \AlgName{RMoMC} applies reductions to the input graph exhaustively, and then runs MoMC on the complement of the resulting kernel; \AlgName{LSBnR} applies reductions exhaustively, then runs local search to obtain a high-quality solution on the kernel which is used as a initial bound in the branch-and-reduce algorithm that is run on the kernel; \AlgName{BnR} applies reductions and then runs the branch-and-reduce algorithm on the kernel (no local search is used to improve an initial bound); \AlgName{FullA} is the full algorithm as described in the previous section, using a short time limit of one second\iffalse (and a budget of \numprint{50000} branches for MoMC)\comment[id=DS]{we use 100,000 for the run on the original graph. D'oh.}\fi and a long time limit of thirty seconds.


%\section{Material}
%\begin{itemize}
        %\item MoMC solves 30 / 100 instances 
        %\item Kernel + MoMC solves 68 / 100 instances 
        %\item Kernel + BnR solves 55 / 100 instances 
        %\item Kernel + BnR - Local Search solves 42 / 100 instances 
        %\item Full solver solves 82 / 100 instances 
%\end{itemize}

%\begin{description}
%\item[GitHub:] \url{https://github.com/sebalamm/pace-2019/releases/tag/pace-2019}
%\item[DOI:] \url{https://doi.org/10.5281/zenodo.2816116}
%\end{description}
%\section{TODOs}
%\begin{itemize}
        %\item TODO add more exact stuff, heuristics?
%\item TODO insert final three solvers of pace challenge
%\item create a table with instances solved
%\end{itemize}
\begin{table*}
\centering
\caption{Detailed per instance results. The columns $n$ and $m$ refer to the number of nodes and edges of the input graph, $n'$ and $m'$ refer to the number of nodes and edges of the kernel graph after reductions have been applied exhaustively, and $|VC|$ refers to the size of the minimum vertex cover of the input graph. With `X' we denote if a solver successfully solved an instance, and with `-' we denote if this was not the case.}
\label{tab:detailedresults1}
\begin{tabular}{l@{\hskip 25pt} rrrr|ccccc|rc}
\toprule
inst\# & $n$ &$m$& $n'$& $m'$ & \AlgName{MoMC} & \AlgName{RMoMC} & \AlgName{LSBnR} & \AlgName{BnR} & \AlgName{FullA} & $|VC|$ \\
                \midrule

001 &\numprint{6160}&\numprint{40207}&\numprint{0}&\numprint{0}&-&X&X&X&X&  \numprint{2586}&\\ 
003 &\numprint{60541}&\numprint{74220}&\numprint{0}&\numprint{0}&-&X&X&X&X&  \numprint{12190}&\\ 
005 &\numprint{200}&\numprint{819}&\numprint{192}&\numprint{800}&X&X&X&X&X&  \numprint{129}&\\ 
007 &\numprint{8794}&\numprint{10130}&\numprint{0}&\numprint{0}&-&X&X&X&X&  \numprint{4397}&\\ 
009 &\numprint{38452}&\numprint{174645}&\numprint{0}&\numprint{0}&-&X&X&X&X&  \numprint{21348}&\\ 
011 &\numprint{9877}&\numprint{25973}&\numprint{0}&\numprint{0}&-&X&X&X&X&  \numprint{4981}&\\ 
013 &\numprint{45307}&\numprint{55440}&\numprint{0}&\numprint{0}&-&X&X&X&X&  \numprint{8610}&\\ 
015 &\numprint{53610}&\numprint{65952}&\numprint{0}&\numprint{0}&-&X&X&X&X&  \numprint{10670}&\\ 
017 &\numprint{23541}&\numprint{51747}&\numprint{0}&\numprint{0}&-&X&X&X&X&  \numprint{12082}&\\ 
019 &\numprint{200}&\numprint{884}&\numprint{194}&\numprint{862}&X&X&X&X&X&  \numprint{130}&\\ 
021 &\numprint{24765}&\numprint{30242}&\numprint{0}&\numprint{0}&-&X&X&X&X&  \numprint{5110}&\\ 
023 &\numprint{27717}&\numprint{133665}&\numprint{0}&\numprint{0}&-&X&X&X&X&  \numprint{16013}&\\ 
025 &\numprint{23194}&\numprint{28221}&\numprint{0}&\numprint{0}&-&X&X&X&X&  \numprint{4899}&\\ 
027 &\numprint{65866}&\numprint{81245}&\numprint{0}&\numprint{0}&-&X&X&X&X&  \numprint{13431}&\\ 
029 &\numprint{13431}&\numprint{21999}&\numprint{0}&\numprint{0}&-&X&X&X&X&  \numprint{6622}&\\ 
031 &\numprint{200}&\numprint{813}&\numprint{198}&\numprint{818}&X&X&X&X&X&  \numprint{136}&\\ 
033 &\numprint{4410}&\numprint{6885}&\numprint{138}&\numprint{471}&-&X&X&X&X&  \numprint{2725}&\\ 
035 &\numprint{200}&\numprint{884}&\numprint{189}&\numprint{859}&X&X&X&X&X&  \numprint{133}&\\ 
037 &\numprint{198}&\numprint{824}&\numprint{194}&\numprint{810}&X&X&X&X&X&  \numprint{131}&\\ 
039 &\numprint{6795}&\numprint{10620}&\numprint{219}&\numprint{753}&-&X&X&X&X&  \numprint{4200}&\\ 
041 &\numprint{200}&\numprint{1040}&\numprint{200}&\numprint{1023}&X&X&X&X&X&  \numprint{139}&\\ 
043 &\numprint{200}&\numprint{841}&\numprint{198}&\numprint{844}&X&X&X&X&X&  \numprint{139}&\\ 
045 &\numprint{200}&\numprint{1044}&\numprint{200}&\numprint{1020}&X&X&X&X&X&  \numprint{137}&\\ 
047 &\numprint{200}&\numprint{1120}&\numprint{198}&\numprint{1080}&X&X&X&X&X&  \numprint{140}&\\ 
049 &\numprint{200}&\numprint{957}&\numprint{198}&\numprint{930}&X&X&X&X&X&  \numprint{136}&\\ 
051 &\numprint{200}&\numprint{1135}&\numprint{200}&\numprint{1098}&X&X&X&X&X&  \numprint{140}&\\ 
053 &\numprint{200}&\numprint{1062}&\numprint{200}&\numprint{1026}&X&X&X&X&X&  \numprint{139}&\\ 
055 &\numprint{200}&\numprint{958}&\numprint{194}&\numprint{938}&X&X&X&X&X&  \numprint{134}&\\ 
057 &\numprint{200}&\numprint{1200}&\numprint{197}&\numprint{1139}&X&X&X&X&X&  \numprint{142}&\\ 
059 &\numprint{200}&\numprint{988}&\numprint{193}&\numprint{954}&X&X&X&X&X&  \numprint{137}&\\ 
061 &\numprint{200}&\numprint{952}&\numprint{198}&\numprint{914}&X&X&X&X&X&  \numprint{135}&\\ 
063 &\numprint{200}&\numprint{1040}&\numprint{200}&\numprint{1011}&X&X&X&X&X&  \numprint{138}&\\ 
065 &\numprint{200}&\numprint{1037}&\numprint{200}&\numprint{1011}&X&X&X&X&X&  \numprint{138}&\\ 
067 &\numprint{200}&\numprint{1201}&\numprint{200}&\numprint{1174}&X&X&X&X&X&  \numprint{143}&\\ 
069 &\numprint{200}&\numprint{1120}&\numprint{196}&\numprint{1077}&X&X&X&X&X&  \numprint{140}&\\ 
071 &\numprint{200}&\numprint{984}&\numprint{200}&\numprint{952}&X&X&X&X&X&  \numprint{136}&\\ 
073 &\numprint{200}&\numprint{1107}&\numprint{200}&\numprint{1078}&X&X&X&X&X&  \numprint{139}&\\ 
075 &\numprint{26300}&\numprint{41500}&\numprint{500}&\numprint{3000}&-&-&X&-&X&  \numprint{16300}&\\ 
077 &\numprint{200}&\numprint{988}&\numprint{193}&\numprint{954}&X&X&X&X&X&  \numprint{137}&\\ 
079 &\numprint{26300}&\numprint{41500}&\numprint{500}&\numprint{3000}&-&-&X&-&X&  \numprint{16300}&\\ 
081 &\numprint{199}&\numprint{1124}&\numprint{197}&\numprint{1087}&X&X&X&X&X&  \numprint{141}&\\ 
083 &\numprint{200}&\numprint{1215}&\numprint{198}&\numprint{1182}&X&X&X&X&X&  \numprint{144}&\\ 
085 &\numprint{11470}&\numprint{17408}&\numprint{3539}&\numprint{25955}&-&-&-&-&-&  &\\ 
087 &\numprint{13590}&\numprint{21240}&\numprint{441}&\numprint{1512}&-&X&-&-&X&  \numprint{8400}&\\ 
089 &\numprint{57316}&\numprint{77978}&\numprint{16834}&\numprint{54847}&-&-&-&-&-&  &\\ 
091 &\numprint{200}&\numprint{1196}&\numprint{200}&\numprint{1163}&X&X&X&X&X&  \numprint{145}&\\ 
093 &\numprint{200}&\numprint{1207}&\numprint{200}&\numprint{1162}&X&X&X&X&X&  \numprint{143}&\\ 
095 &\numprint{15783}&\numprint{24663}&\numprint{510}&\numprint{1746}&-&X&-&-&X&  \numprint{9755}&\\ 
097 &\numprint{18096}&\numprint{28281}&\numprint{579}&\numprint{1995}&-&X&-&-&X&  \numprint{11185}&\\ 
099 &\numprint{26300}&\numprint{41500}&\numprint{500}&\numprint{3000}&-&-&X&-&X&  \numprint{16300}&\\ 
\bottomrule
\end{tabular}
\end{table*}

\begin{table*}
\centering

\caption{Detailed per instance results. The columns $n$ and $m$ refer to the number of nodes and edges of the input graph, $n'$ and $m'$ refer to the number of nodes and edges of the kernel graph after reductions have been applied exhaustively, and $|VC|$ refers to the size of the minimum vertex cover of the input graph. With `X' we denote if a solver successfully solved an instance, and with `-' we denote if this was not the case.}
\label{tab:detailedresults2}
\begin{tabular}{l@{\hskip 25pt} rrrr|ccccc|rc}
\toprule
inst\# & $n$ &$m$& $n'$& $m'$ & \AlgName{MoMC} & \AlgName{RMoMC} & \AlgName{LSBnR} & \AlgName{BnR} & \AlgName{FullA} & $|VC|$ \\
                \midrule

101 &\numprint{26300}&\numprint{41500}&\numprint{500}&\numprint{3000}&-&-&X&-&X&  \numprint{16300}&\\ 
103 &\numprint{15783}&\numprint{24663}&\numprint{513}&\numprint{1752}&-&X&-&-&X&  \numprint{9755}&\\ 
105 &\numprint{26300}&\numprint{41500}&\numprint{500}&\numprint{3000}&-&-&X&-&X&  \numprint{16300}&\\ 
107 &\numprint{13590}&\numprint{21240}&\numprint{435}&\numprint{1500}&-&X&-&-&X&  \numprint{8400}&\\ 
109 &\numprint{66992}&\numprint{90970}&\numprint{20336}&\numprint{66350}&-&-&-&-&-&  &\\ 
111 &\numprint{450}&\numprint{17831}&\numprint{450}&\numprint{17831}&X&X&-&-&X&  \numprint{420}&\\ 
113 &\numprint{26300}&\numprint{41500}&\numprint{500}&\numprint{3000}&-&-&X&-&X&  \numprint{16300}&\\ 
115 &\numprint{18096}&\numprint{28281}&\numprint{573}&\numprint{1986}&-&X&-&-&X&  \numprint{11185}&\\ 
117 &\numprint{18096}&\numprint{28281}&\numprint{582}&\numprint{2007}&-&X&-&-&X&  \numprint{11185}&\\ 
119 &\numprint{18096}&\numprint{28281}&\numprint{588}&\numprint{2016}&-&X&-&-&X&  \numprint{11185}&\\ 
121 &\numprint{18096}&\numprint{28281}&\numprint{579}&\numprint{1998}&-&X&-&-&X&  \numprint{11185}&\\ 
123 &\numprint{26300}&\numprint{41500}&\numprint{500}&\numprint{3000}&-&-&X&-&X&  \numprint{16300}&\\ 
125 &\numprint{26300}&\numprint{41500}&\numprint{500}&\numprint{3000}&-&-&X&-&X&  \numprint{16300}&\\ 
127 &\numprint{18096}&\numprint{28281}&\numprint{582}&\numprint{2001}&-&X&-&-&X&  \numprint{11185}&\\ 
129 &\numprint{15783}&\numprint{24663}&\numprint{507}&\numprint{1752}&-&X&-&-&X&  \numprint{9755}&\\ 
131 &\numprint{2980}&\numprint{5360}&\numprint{2179}&\numprint{6951}&X&-&-&-&X&  \numprint{1920}&\\ 
133 &\numprint{15783}&\numprint{24663}&\numprint{507}&\numprint{1746}&-&X&-&-&X&  \numprint{9755}&\\ 
135 &\numprint{26300}&\numprint{41500}&\numprint{500}&\numprint{3000}&-&-&X&-&X&  \numprint{16300}&\\ 
137 &\numprint{26300}&\numprint{41500}&\numprint{500}&\numprint{3000}&-&-&X&-&X&  \numprint{16300}&\\ 
139 &\numprint{18096}&\numprint{28281}&\numprint{579}&\numprint{1995}&-&X&-&-&X&  \numprint{11185}&\\ 
141 &\numprint{18096}&\numprint{28281}&\numprint{576}&\numprint{1995}&-&X&-&-&X&  \numprint{11185}&\\ 
143 &\numprint{18096}&\numprint{28281}&\numprint{582}&\numprint{2001}&-&X&-&-&X&  \numprint{11185}&\\ 
145 &\numprint{18096}&\numprint{28281}&\numprint{576}&\numprint{1989}&-&X&-&-&X&  \numprint{11185}&\\ 
147 &\numprint{18096}&\numprint{28281}&\numprint{567}&\numprint{1974}&-&X&-&-&X&  \numprint{11185}&\\ 
149 &\numprint{26300}&\numprint{41500}&\numprint{500}&\numprint{3000}&-&-&X&-&X&  \numprint{16300}&\\ 
151 &\numprint{15783}&\numprint{24663}&\numprint{501}&\numprint{1728}&-&X&-&-&X&  \numprint{9755}&\\ 
153 &\numprint{29076}&\numprint{45570}&\numprint{2124}&\numprint{16266}&-&-&-&-&-&  &\\ 
155 &\numprint{26300}&\numprint{41500}&\numprint{500}&\numprint{3000}&-&-&X&-&X&  \numprint{16300}&\\ 
157 &\numprint{2980}&\numprint{5360}&\numprint{2169}&\numprint{6898}&X&-&-&-&X&  \numprint{1920}&\\ 
159 &\numprint{18096}&\numprint{28281}&\numprint{582}&\numprint{2004}&-&X&-&-&X&  \numprint{11185}&\\ 
161 &\numprint{138141}&\numprint{227241}&\numprint{41926}&\numprint{202869}&-&-&-&-&-&  &\\ 
163 &\numprint{18096}&\numprint{28281}&\numprint{582}&\numprint{2004}&-&X&-&-&X&  \numprint{11185}&\\ 
165 &\numprint{18096}&\numprint{28281}&\numprint{576}&\numprint{1995}&-&X&-&-&X&  \numprint{11185}&\\ 
167 &\numprint{15783}&\numprint{24663}&\numprint{510}&\numprint{1746}&-&X&-&-&X&  \numprint{9755}&\\ 
169 &\numprint{4768}&\numprint{8576}&\numprint{3458}&\numprint{11014}&-&-&-&-&-&  &\\ 
171 &\numprint{18096}&\numprint{28281}&\numprint{576}&\numprint{1989}&-&X&-&-&X&  \numprint{11185}&\\ 
173 &\numprint{56860}&\numprint{77264}&\numprint{17090}&\numprint{55568}&-&-&-&-&-&  &\\ 
175 &\numprint{3523}&\numprint{6446}&\numprint{2723}&\numprint{8570}&-&-&-&-&-&  &\\ 
177 &\numprint{5066}&\numprint{9112}&\numprint{3704}&\numprint{11797}&-&-&-&-&-&  &\\ 
179 &\numprint{15783}&\numprint{24663}&\numprint{504}&\numprint{1740}&-&X&-&-&X&  \numprint{9755}&\\ 
181 &\numprint{18096}&\numprint{28281}&\numprint{573}&\numprint{1989}&-&X&X&-&X&  \numprint{11185}&\\ 
183 &\numprint{72420}&\numprint{118362}&\numprint{30340}&\numprint{133872}&-&-&-&-&-&  &\\ 
185 &\numprint{3523}&\numprint{6446}&\numprint{2723}&\numprint{8568}&-&-&-&-&-&  &\\ 
187 &\numprint{4227}&\numprint{7734}&\numprint{3264}&\numprint{10286}&-&-&-&-&-&  &\\ 
189 &\numprint{7400}&\numprint{13600}&\numprint{5802}&\numprint{18212}&-&-&-&-&-&  &\\ 
191 &\numprint{4579}&\numprint{8378}&\numprint{3539}&\numprint{11137}&-&-&-&-&-&  &\\ 
193 &\numprint{7030}&\numprint{12920}&\numprint{5510}&\numprint{17294}&-&-&-&-&-&  &\\ 
195 &\numprint{1150}&\numprint{81068}&\numprint{1150}&\numprint{81068}&-&-&-&-&-&  &\\ 
197 &\numprint{1534}&\numprint{127011}&\numprint{1534}&\numprint{127011}&-&-&-&-&-&  &\\ 
199 &\numprint{1534}&\numprint{126163}&\numprint{1534}&\numprint{126163}&-&-&-&-&-&  &\\ 

\bottomrule
\end{tabular}
\end{table*}
Tables~\ref{tab:detailedresults1} and \ref{tab:detailedresults2} give an overview of the instances that each of the solver solved, including the kernel size, and the minimum vertex cover size for those instances solved by any of the four algorithms.
Overall, \AlgName{MoMC} can solve 30 out of the 100 instances. 
Applying reductions first enables \AlgName{RMoMC} to solve 68 instances. However, curiously, there are two instances (instances 131 and 157) that \AlgName{MoMC} solves, but that \AlgName{RMoMC} can not solve. 
In these cases, kernelization reduced the number of nodes, but \emph{increased} the number of edges. This is due to the \emph{alternative} reduction, which in some cases can create more edges than initially present. This is why we choose to also run MoMC on the unkernelized input graph in \AlgName{FullA} (in order to solve those instances as well).

\AlgName{LSBnR} solves 55 of the 100 instances. Priming the branch-and-reduce algorithm with an initial solution computed by local search has a significant impact: \AlgName{LSBnR} solves 13 more instances than \AlgName{BnR}, which solve 42 instances. In particular, using local search to find an initial bound helps to solve large instances in which the initial kernelization step does not reduce the graph fully. Surprisingly, \AlgName{RMoMC} solves 26 instances that \AlgName{BnR} does not (and even \AlgName{LSBnR} is only able to solve one of these instances). To the best of our knowledge, this is the first time that kernelization followed by branch-and-bound is shown to significantly outperform branch-and-reduce. Our full algorithm \AlgName{FullA} solves 82 of the 100 instances and, as expected, dominates each of the other configurations. This can be further seen from the plot in Figure~\ref{fig:solution_time}, which shows how many instances each algorithm solves over time. Note that \AlgName{LSBnR} and \AlgName{RMoMC} solve more instances in narrow time gaps, due to \AlgName{FullA}'s set up cost and  running multiple algorithms. However, \AlgName{FullA} quickly makes up for this and overtakes all algorithms at approximately eight seconds.

\begin{figure}
    \centering
    %% Creator: Matplotlib, PGF backend
%%
%% To include the figure in your LaTeX document, write
%%   \input{<filename>.pgf}
%%
%% Make sure the required packages are loaded in your preamble
%%   \usepackage{pgf}
%%
%% Figures using additional raster images can only be included by \input if
%% they are in the same directory as the main LaTeX file. For loading figures
%% from other directories you can use the `import` package
%%   \usepackage{import}
%% and then include the figures with
%%   \import{<path to file>}{<filename>.pgf}
%%
%% Matplotlib used the following preamble
%%   \renewcommand{\sfdefault}{phv}
%%   \renewcommand{\rmdefault}{ptm}
%%   \renewcommand{\ttdefault}{pcr}
%%   \normalfont\selectfont
%%
\begingroup%
\makeatletter%
\begin{pgfpicture}%
\pgfpathrectangle{\pgfpointorigin}{\pgfqpoint{3.049339in}{2.769518in}}%
\pgfusepath{use as bounding box}%
\begin{pgfscope}%
\pgfsetbuttcap%
\pgfsetroundjoin%
\definecolor{currentfill}{rgb}{1.000000,1.000000,1.000000}%
\pgfsetfillcolor{currentfill}%
\pgfsetlinewidth{0.000000pt}%
\definecolor{currentstroke}{rgb}{1.000000,1.000000,1.000000}%
\pgfsetstrokecolor{currentstroke}%
\pgfsetdash{}{0pt}%
\pgfpathmoveto{\pgfqpoint{0.000000in}{0.000000in}}%
\pgfpathlineto{\pgfqpoint{3.049339in}{0.000000in}}%
\pgfpathlineto{\pgfqpoint{3.049339in}{2.769518in}}%
\pgfpathlineto{\pgfqpoint{0.000000in}{2.769518in}}%
\pgfpathclose%
\pgfusepath{fill}%
\end{pgfscope}%
\begin{pgfscope}%
\pgfsetbuttcap%
\pgfsetroundjoin%
\definecolor{currentfill}{rgb}{1.000000,1.000000,1.000000}%
\pgfsetfillcolor{currentfill}%
\pgfsetlinewidth{0.000000pt}%
\definecolor{currentstroke}{rgb}{0.000000,0.000000,0.000000}%
\pgfsetstrokecolor{currentstroke}%
\pgfsetstrokeopacity{0.000000}%
\pgfsetdash{}{0pt}%
\pgfpathmoveto{\pgfqpoint{0.506010in}{1.121191in}}%
\pgfpathlineto{\pgfqpoint{2.831010in}{1.121191in}}%
\pgfpathlineto{\pgfqpoint{2.831010in}{2.521191in}}%
\pgfpathlineto{\pgfqpoint{0.506010in}{2.521191in}}%
\pgfpathclose%
\pgfusepath{fill}%
\end{pgfscope}%
\begin{pgfscope}%
\pgfpathrectangle{\pgfqpoint{0.506010in}{1.121191in}}{\pgfqpoint{2.325000in}{1.400000in}} %
\pgfusepath{clip}%
\pgfsetrectcap%
\pgfsetroundjoin%
\pgfsetlinewidth{2.007500pt}%
\definecolor{currentstroke}{rgb}{0.000000,0.500000,0.000000}%
\pgfsetstrokecolor{currentstroke}%
\pgfsetdash{}{0pt}%
\pgfpathmoveto{\pgfqpoint{1.133284in}{1.135191in}}%
\pgfpathlineto{\pgfqpoint{1.151540in}{1.149191in}}%
\pgfpathlineto{\pgfqpoint{1.191290in}{1.163191in}}%
\pgfpathlineto{\pgfqpoint{1.199514in}{1.177191in}}%
\pgfpathlineto{\pgfqpoint{1.215198in}{1.191191in}}%
\pgfpathlineto{\pgfqpoint{1.215198in}{1.205191in}}%
\pgfpathlineto{\pgfqpoint{1.216714in}{1.219191in}}%
\pgfpathlineto{\pgfqpoint{1.222689in}{1.233191in}}%
\pgfpathlineto{\pgfqpoint{1.238426in}{1.247191in}}%
\pgfpathlineto{\pgfqpoint{1.245268in}{1.261191in}}%
\pgfpathlineto{\pgfqpoint{1.251929in}{1.275191in}}%
\pgfpathlineto{\pgfqpoint{1.259697in}{1.289191in}}%
\pgfpathlineto{\pgfqpoint{1.260968in}{1.303191in}}%
\pgfpathlineto{\pgfqpoint{1.285143in}{1.317191in}}%
\pgfpathlineto{\pgfqpoint{1.287438in}{1.331191in}}%
\pgfpathlineto{\pgfqpoint{1.287438in}{1.345191in}}%
\pgfpathlineto{\pgfqpoint{1.296414in}{1.359191in}}%
\pgfpathlineto{\pgfqpoint{1.301867in}{1.373191in}}%
\pgfpathlineto{\pgfqpoint{1.312430in}{1.387191in}}%
\pgfpathlineto{\pgfqpoint{1.320574in}{1.401191in}}%
\pgfpathlineto{\pgfqpoint{1.324549in}{1.415191in}}%
\pgfpathlineto{\pgfqpoint{1.327490in}{1.429191in}}%
\pgfpathlineto{\pgfqpoint{1.339853in}{1.443191in}}%
\pgfpathlineto{\pgfqpoint{1.378501in}{1.457191in}}%
\pgfpathlineto{\pgfqpoint{1.397663in}{1.471191in}}%
\pgfpathlineto{\pgfqpoint{1.409204in}{1.485191in}}%
\pgfpathlineto{\pgfqpoint{1.426244in}{1.499191in}}%
\pgfpathlineto{\pgfqpoint{1.451916in}{1.513191in}}%
\pgfpathlineto{\pgfqpoint{2.567316in}{1.527191in}}%
\pgfpathlineto{\pgfqpoint{2.579904in}{1.541191in}}%
\pgfusepath{stroke}%
\end{pgfscope}%
\begin{pgfscope}%
\pgfpathrectangle{\pgfqpoint{0.506010in}{1.121191in}}{\pgfqpoint{2.325000in}{1.400000in}} %
\pgfusepath{clip}%
\pgfsetrectcap%
\pgfsetroundjoin%
\pgfsetlinewidth{2.007500pt}%
\definecolor{currentstroke}{rgb}{1.000000,0.000000,0.000000}%
\pgfsetstrokecolor{currentstroke}%
\pgfsetdash{}{0pt}%
\pgfpathmoveto{\pgfqpoint{0.849566in}{1.135191in}}%
\pgfpathlineto{\pgfqpoint{0.849566in}{1.149191in}}%
\pgfpathlineto{\pgfqpoint{0.855957in}{1.163191in}}%
\pgfpathlineto{\pgfqpoint{0.855957in}{1.177191in}}%
\pgfpathlineto{\pgfqpoint{0.855957in}{1.191191in}}%
\pgfpathlineto{\pgfqpoint{0.862190in}{1.205191in}}%
\pgfpathlineto{\pgfqpoint{0.862190in}{1.219191in}}%
\pgfpathlineto{\pgfqpoint{0.868274in}{1.233191in}}%
\pgfpathlineto{\pgfqpoint{0.868274in}{1.247191in}}%
\pgfpathlineto{\pgfqpoint{0.868274in}{1.261191in}}%
\pgfpathlineto{\pgfqpoint{0.874213in}{1.275191in}}%
\pgfpathlineto{\pgfqpoint{0.874213in}{1.289191in}}%
\pgfpathlineto{\pgfqpoint{0.880017in}{1.303191in}}%
\pgfpathlineto{\pgfqpoint{0.891238in}{1.317191in}}%
\pgfpathlineto{\pgfqpoint{0.907186in}{1.331191in}}%
\pgfpathlineto{\pgfqpoint{0.907186in}{1.345191in}}%
\pgfpathlineto{\pgfqpoint{0.974602in}{1.359191in}}%
\pgfpathlineto{\pgfqpoint{0.978516in}{1.373191in}}%
\pgfpathlineto{\pgfqpoint{0.978516in}{1.387191in}}%
\pgfpathlineto{\pgfqpoint{0.997223in}{1.401191in}}%
\pgfpathlineto{\pgfqpoint{1.027756in}{1.415191in}}%
\pgfpathlineto{\pgfqpoint{1.034067in}{1.429191in}}%
\pgfpathlineto{\pgfqpoint{1.037164in}{1.443191in}}%
\pgfpathlineto{\pgfqpoint{1.043247in}{1.457191in}}%
\pgfpathlineto{\pgfqpoint{1.049187in}{1.471191in}}%
\pgfpathlineto{\pgfqpoint{1.057843in}{1.485191in}}%
\pgfpathlineto{\pgfqpoint{1.071640in}{1.499191in}}%
\pgfpathlineto{\pgfqpoint{1.071640in}{1.513191in}}%
\pgfpathlineto{\pgfqpoint{1.082160in}{1.527191in}}%
\pgfpathlineto{\pgfqpoint{1.082160in}{1.541191in}}%
\pgfpathlineto{\pgfqpoint{1.087260in}{1.555191in}}%
\pgfpathlineto{\pgfqpoint{1.092259in}{1.569191in}}%
\pgfpathlineto{\pgfqpoint{1.094721in}{1.583191in}}%
\pgfpathlineto{\pgfqpoint{1.113604in}{1.597191in}}%
\pgfpathlineto{\pgfqpoint{1.149576in}{1.611191in}}%
\pgfpathlineto{\pgfqpoint{1.163016in}{1.625191in}}%
\pgfpathlineto{\pgfqpoint{1.181055in}{1.639191in}}%
\pgfpathlineto{\pgfqpoint{1.181055in}{1.653191in}}%
\pgfpathlineto{\pgfqpoint{1.189613in}{1.667191in}}%
\pgfpathlineto{\pgfqpoint{1.249285in}{1.681191in}}%
\pgfpathlineto{\pgfqpoint{1.259697in}{1.695191in}}%
\pgfpathlineto{\pgfqpoint{1.264745in}{1.709191in}}%
\pgfpathlineto{\pgfqpoint{1.301867in}{1.723191in}}%
\pgfpathlineto{\pgfqpoint{1.305082in}{1.737191in}}%
\pgfpathlineto{\pgfqpoint{1.306145in}{1.751191in}}%
\pgfpathlineto{\pgfqpoint{1.327490in}{1.765191in}}%
\pgfpathlineto{\pgfqpoint{1.351638in}{1.779191in}}%
\pgfpathlineto{\pgfqpoint{1.351638in}{1.793191in}}%
\pgfpathlineto{\pgfqpoint{1.362050in}{1.807191in}}%
\pgfpathlineto{\pgfqpoint{1.368759in}{1.821191in}}%
\pgfpathlineto{\pgfqpoint{1.371230in}{1.835191in}}%
\pgfpathlineto{\pgfqpoint{1.376903in}{1.849191in}}%
\pgfpathlineto{\pgfqpoint{1.396923in}{1.863191in}}%
\pgfpathlineto{\pgfqpoint{1.396923in}{1.877191in}}%
\pgfpathlineto{\pgfqpoint{1.399867in}{1.891191in}}%
\pgfpathlineto{\pgfqpoint{1.404220in}{1.905191in}}%
\pgfpathlineto{\pgfqpoint{1.407790in}{1.919191in}}%
\pgfpathlineto{\pgfqpoint{1.409909in}{1.933191in}}%
\pgfpathlineto{\pgfqpoint{1.414093in}{1.947191in}}%
\pgfpathlineto{\pgfqpoint{1.417527in}{1.961191in}}%
\pgfpathlineto{\pgfqpoint{1.422927in}{1.975191in}}%
\pgfpathlineto{\pgfqpoint{1.427559in}{1.989191in}}%
\pgfpathlineto{\pgfqpoint{1.434670in}{2.003191in}}%
\pgfpathlineto{\pgfqpoint{1.435307in}{2.017191in}}%
\pgfpathlineto{\pgfqpoint{1.436575in}{2.031191in}}%
\pgfpathlineto{\pgfqpoint{1.437838in}{2.045191in}}%
\pgfpathlineto{\pgfqpoint{1.447714in}{2.059191in}}%
\pgfpathlineto{\pgfqpoint{1.475759in}{2.073191in}}%
\pgfusepath{stroke}%
\end{pgfscope}%
\begin{pgfscope}%
\pgfpathrectangle{\pgfqpoint{0.506010in}{1.121191in}}{\pgfqpoint{2.325000in}{1.400000in}} %
\pgfusepath{clip}%
\pgfsetrectcap%
\pgfsetroundjoin%
\pgfsetlinewidth{2.007500pt}%
\definecolor{currentstroke}{rgb}{0.000000,0.000000,1.000000}%
\pgfsetstrokecolor{currentstroke}%
\pgfsetdash{}{0pt}%
\pgfpathmoveto{\pgfqpoint{0.654387in}{1.135191in}}%
\pgfpathlineto{\pgfqpoint{0.693300in}{1.149191in}}%
\pgfpathlineto{\pgfqpoint{0.693300in}{1.163191in}}%
\pgfpathlineto{\pgfqpoint{0.693300in}{1.177191in}}%
\pgfpathlineto{\pgfqpoint{0.705043in}{1.191191in}}%
\pgfpathlineto{\pgfqpoint{0.716264in}{1.205191in}}%
\pgfpathlineto{\pgfqpoint{0.727008in}{1.219191in}}%
\pgfpathlineto{\pgfqpoint{0.727008in}{1.233191in}}%
\pgfpathlineto{\pgfqpoint{0.727008in}{1.247191in}}%
\pgfpathlineto{\pgfqpoint{0.737312in}{1.261191in}}%
\pgfpathlineto{\pgfqpoint{0.737312in}{1.275191in}}%
\pgfpathlineto{\pgfqpoint{0.747213in}{1.289191in}}%
\pgfpathlineto{\pgfqpoint{0.765920in}{1.303191in}}%
\pgfpathlineto{\pgfqpoint{0.791614in}{1.317191in}}%
\pgfpathlineto{\pgfqpoint{0.799628in}{1.331191in}}%
\pgfpathlineto{\pgfqpoint{0.799628in}{1.345191in}}%
\pgfpathlineto{\pgfqpoint{0.807396in}{1.359191in}}%
\pgfpathlineto{\pgfqpoint{0.807396in}{1.373191in}}%
\pgfpathlineto{\pgfqpoint{0.814932in}{1.387191in}}%
\pgfpathlineto{\pgfqpoint{0.814932in}{1.401191in}}%
\pgfpathlineto{\pgfqpoint{0.814932in}{1.415191in}}%
\pgfpathlineto{\pgfqpoint{0.814932in}{1.429191in}}%
\pgfpathlineto{\pgfqpoint{0.822249in}{1.443191in}}%
\pgfpathlineto{\pgfqpoint{0.829361in}{1.457191in}}%
\pgfpathlineto{\pgfqpoint{0.829361in}{1.471191in}}%
\pgfpathlineto{\pgfqpoint{0.862190in}{1.485191in}}%
\pgfpathlineto{\pgfqpoint{1.084723in}{1.499191in}}%
\pgfpathlineto{\pgfqpoint{1.307204in}{1.513191in}}%
\pgfpathlineto{\pgfqpoint{1.415472in}{1.527191in}}%
\pgfpathlineto{\pgfqpoint{1.598234in}{1.541191in}}%
\pgfpathlineto{\pgfqpoint{1.627485in}{1.555191in}}%
\pgfpathlineto{\pgfqpoint{1.704448in}{1.569191in}}%
\pgfpathlineto{\pgfqpoint{1.760908in}{1.583191in}}%
\pgfpathlineto{\pgfqpoint{1.781249in}{1.597191in}}%
\pgfpathlineto{\pgfqpoint{1.858312in}{1.611191in}}%
\pgfpathlineto{\pgfqpoint{1.868802in}{1.625191in}}%
\pgfpathlineto{\pgfqpoint{1.934566in}{1.639191in}}%
\pgfpathlineto{\pgfqpoint{2.008152in}{1.653191in}}%
\pgfpathlineto{\pgfqpoint{2.012390in}{1.667191in}}%
\pgfpathlineto{\pgfqpoint{2.056250in}{1.681191in}}%
\pgfpathlineto{\pgfqpoint{2.059596in}{1.695191in}}%
\pgfpathlineto{\pgfqpoint{2.070868in}{1.709191in}}%
\pgfpathlineto{\pgfqpoint{2.125396in}{1.723191in}}%
\pgfpathlineto{\pgfqpoint{2.132163in}{1.737191in}}%
\pgfpathlineto{\pgfqpoint{2.132846in}{1.751191in}}%
\pgfpathlineto{\pgfqpoint{2.139223in}{1.765191in}}%
\pgfpathlineto{\pgfqpoint{2.159867in}{1.779191in}}%
\pgfpathlineto{\pgfqpoint{2.161484in}{1.793191in}}%
\pgfpathlineto{\pgfqpoint{2.232767in}{1.807191in}}%
\pgfpathlineto{\pgfqpoint{2.271282in}{1.821191in}}%
\pgfpathlineto{\pgfqpoint{2.314118in}{1.835191in}}%
\pgfpathlineto{\pgfqpoint{2.343764in}{1.849191in}}%
\pgfpathlineto{\pgfqpoint{2.403617in}{1.863191in}}%
\pgfpathlineto{\pgfqpoint{2.408307in}{1.877191in}}%
\pgfpathlineto{\pgfqpoint{2.460475in}{1.891191in}}%
\pgfusepath{stroke}%
\end{pgfscope}%
\begin{pgfscope}%
\pgfpathrectangle{\pgfqpoint{0.506010in}{1.121191in}}{\pgfqpoint{2.325000in}{1.400000in}} %
\pgfusepath{clip}%
\pgfsetrectcap%
\pgfsetroundjoin%
\pgfsetlinewidth{2.007500pt}%
\definecolor{currentstroke}{rgb}{0.750000,0.750000,0.000000}%
\pgfsetstrokecolor{currentstroke}%
\pgfsetdash{}{0pt}%
\pgfpathmoveto{\pgfqpoint{0.836277in}{1.135191in}}%
\pgfpathlineto{\pgfqpoint{0.843009in}{1.149191in}}%
\pgfpathlineto{\pgfqpoint{0.843009in}{1.163191in}}%
\pgfpathlineto{\pgfqpoint{0.843009in}{1.177191in}}%
\pgfpathlineto{\pgfqpoint{0.843009in}{1.191191in}}%
\pgfpathlineto{\pgfqpoint{0.849566in}{1.205191in}}%
\pgfpathlineto{\pgfqpoint{0.855957in}{1.219191in}}%
\pgfpathlineto{\pgfqpoint{0.855957in}{1.233191in}}%
\pgfpathlineto{\pgfqpoint{0.855957in}{1.247191in}}%
\pgfpathlineto{\pgfqpoint{0.862190in}{1.261191in}}%
\pgfpathlineto{\pgfqpoint{0.862190in}{1.275191in}}%
\pgfpathlineto{\pgfqpoint{0.868274in}{1.289191in}}%
\pgfpathlineto{\pgfqpoint{0.868274in}{1.303191in}}%
\pgfpathlineto{\pgfqpoint{1.079571in}{1.317191in}}%
\pgfpathlineto{\pgfqpoint{1.307204in}{1.331191in}}%
\pgfpathlineto{\pgfqpoint{1.426244in}{1.345191in}}%
\pgfpathlineto{\pgfqpoint{1.620255in}{1.359191in}}%
\pgfpathlineto{\pgfqpoint{1.650733in}{1.373191in}}%
\pgfpathlineto{\pgfqpoint{1.721996in}{1.387191in}}%
\pgfpathlineto{\pgfqpoint{1.787785in}{1.401191in}}%
\pgfpathlineto{\pgfqpoint{1.788257in}{1.415191in}}%
\pgfpathlineto{\pgfqpoint{1.875561in}{1.429191in}}%
\pgfpathlineto{\pgfqpoint{1.895741in}{1.443191in}}%
\pgfpathlineto{\pgfqpoint{1.941594in}{1.457191in}}%
\pgfpathlineto{\pgfqpoint{2.029689in}{1.471191in}}%
\pgfpathlineto{\pgfqpoint{2.035537in}{1.485191in}}%
\pgfpathlineto{\pgfqpoint{2.073977in}{1.499191in}}%
\pgfpathlineto{\pgfqpoint{2.075643in}{1.513191in}}%
\pgfpathlineto{\pgfqpoint{2.114850in}{1.527191in}}%
\pgfpathlineto{\pgfqpoint{2.137339in}{1.541191in}}%
\pgfpathlineto{\pgfqpoint{2.149418in}{1.555191in}}%
\pgfpathlineto{\pgfqpoint{2.155945in}{1.569191in}}%
\pgfpathlineto{\pgfqpoint{2.179917in}{1.583191in}}%
\pgfpathlineto{\pgfqpoint{2.184012in}{1.597191in}}%
\pgfpathlineto{\pgfqpoint{2.199233in}{1.611191in}}%
\pgfpathlineto{\pgfqpoint{2.251922in}{1.625191in}}%
\pgfpathlineto{\pgfqpoint{2.283124in}{1.639191in}}%
\pgfpathlineto{\pgfqpoint{2.323395in}{1.653191in}}%
\pgfpathlineto{\pgfqpoint{2.360178in}{1.667191in}}%
\pgfpathlineto{\pgfqpoint{2.409222in}{1.681191in}}%
\pgfpathlineto{\pgfqpoint{2.419066in}{1.695191in}}%
\pgfpathlineto{\pgfqpoint{2.468995in}{1.709191in}}%
\pgfusepath{stroke}%
\end{pgfscope}%
\begin{pgfscope}%
\pgfpathrectangle{\pgfqpoint{0.506010in}{1.121191in}}{\pgfqpoint{2.325000in}{1.400000in}} %
\pgfusepath{clip}%
\pgfsetrectcap%
\pgfsetroundjoin%
\pgfsetlinewidth{2.007500pt}%
\definecolor{currentstroke}{rgb}{0.000000,0.750000,0.750000}%
\pgfsetstrokecolor{currentstroke}%
\pgfsetdash{}{0pt}%
\pgfpathmoveto{\pgfqpoint{0.654387in}{1.135191in}}%
\pgfpathlineto{\pgfqpoint{0.680984in}{1.149191in}}%
\pgfpathlineto{\pgfqpoint{0.680984in}{1.163191in}}%
\pgfpathlineto{\pgfqpoint{0.693300in}{1.177191in}}%
\pgfpathlineto{\pgfqpoint{0.693300in}{1.191191in}}%
\pgfpathlineto{\pgfqpoint{0.693300in}{1.205191in}}%
\pgfpathlineto{\pgfqpoint{0.705043in}{1.219191in}}%
\pgfpathlineto{\pgfqpoint{0.737312in}{1.233191in}}%
\pgfpathlineto{\pgfqpoint{0.747213in}{1.247191in}}%
\pgfpathlineto{\pgfqpoint{0.747213in}{1.261191in}}%
\pgfpathlineto{\pgfqpoint{0.747213in}{1.275191in}}%
\pgfpathlineto{\pgfqpoint{0.747213in}{1.289191in}}%
\pgfpathlineto{\pgfqpoint{0.756740in}{1.303191in}}%
\pgfpathlineto{\pgfqpoint{0.829361in}{1.317191in}}%
\pgfpathlineto{\pgfqpoint{0.836277in}{1.331191in}}%
\pgfpathlineto{\pgfqpoint{0.843009in}{1.345191in}}%
\pgfpathlineto{\pgfqpoint{0.843009in}{1.359191in}}%
\pgfpathlineto{\pgfqpoint{0.843009in}{1.373191in}}%
\pgfpathlineto{\pgfqpoint{0.849566in}{1.387191in}}%
\pgfpathlineto{\pgfqpoint{0.849566in}{1.401191in}}%
\pgfpathlineto{\pgfqpoint{0.855957in}{1.415191in}}%
\pgfpathlineto{\pgfqpoint{0.855957in}{1.429191in}}%
\pgfpathlineto{\pgfqpoint{0.862190in}{1.443191in}}%
\pgfpathlineto{\pgfqpoint{0.862190in}{1.457191in}}%
\pgfpathlineto{\pgfqpoint{0.868274in}{1.471191in}}%
\pgfpathlineto{\pgfqpoint{0.874213in}{1.485191in}}%
\pgfpathlineto{\pgfqpoint{1.094721in}{1.499191in}}%
\pgfpathlineto{\pgfqpoint{1.189613in}{1.513191in}}%
\pgfpathlineto{\pgfqpoint{1.194612in}{1.527191in}}%
\pgfpathlineto{\pgfqpoint{1.201127in}{1.541191in}}%
\pgfpathlineto{\pgfqpoint{1.210594in}{1.555191in}}%
\pgfpathlineto{\pgfqpoint{1.216714in}{1.569191in}}%
\pgfpathlineto{\pgfqpoint{1.216714in}{1.583191in}}%
\pgfpathlineto{\pgfqpoint{1.231394in}{1.597191in}}%
\pgfpathlineto{\pgfqpoint{1.241185in}{1.611191in}}%
\pgfpathlineto{\pgfqpoint{1.247953in}{1.625191in}}%
\pgfpathlineto{\pgfqpoint{1.249285in}{1.639191in}}%
\pgfpathlineto{\pgfqpoint{1.251929in}{1.653191in}}%
\pgfpathlineto{\pgfqpoint{1.255842in}{1.667191in}}%
\pgfpathlineto{\pgfqpoint{1.255842in}{1.681191in}}%
\pgfpathlineto{\pgfqpoint{1.258418in}{1.695191in}}%
\pgfpathlineto{\pgfqpoint{1.260968in}{1.709191in}}%
\pgfpathlineto{\pgfqpoint{1.267232in}{1.723191in}}%
\pgfpathlineto{\pgfqpoint{1.270918in}{1.737191in}}%
\pgfpathlineto{\pgfqpoint{1.272134in}{1.751191in}}%
\pgfpathlineto{\pgfqpoint{1.291966in}{1.765191in}}%
\pgfpathlineto{\pgfqpoint{1.293085in}{1.779191in}}%
\pgfpathlineto{\pgfqpoint{1.296414in}{1.793191in}}%
\pgfpathlineto{\pgfqpoint{1.298609in}{1.807191in}}%
\pgfpathlineto{\pgfqpoint{1.300785in}{1.821191in}}%
\pgfpathlineto{\pgfqpoint{1.317551in}{1.835191in}}%
\pgfpathlineto{\pgfqpoint{1.322569in}{1.849191in}}%
\pgfpathlineto{\pgfqpoint{1.346267in}{1.863191in}}%
\pgfpathlineto{\pgfqpoint{1.368759in}{1.877191in}}%
\pgfpathlineto{\pgfqpoint{1.401326in}{1.891191in}}%
\pgfpathlineto{\pgfqpoint{1.402777in}{1.905191in}}%
\pgfpathlineto{\pgfqpoint{1.416844in}{1.919191in}}%
\pgfpathlineto{\pgfqpoint{1.416844in}{1.933191in}}%
\pgfpathlineto{\pgfqpoint{1.431463in}{1.947191in}}%
\pgfpathlineto{\pgfqpoint{1.433392in}{1.961191in}}%
\pgfpathlineto{\pgfqpoint{1.451320in}{1.975191in}}%
\pgfpathlineto{\pgfqpoint{1.457218in}{1.989191in}}%
\pgfpathlineto{\pgfqpoint{1.460692in}{2.003191in}}%
\pgfpathlineto{\pgfqpoint{1.464119in}{2.017191in}}%
\pgfpathlineto{\pgfqpoint{1.469729in}{2.031191in}}%
\pgfpathlineto{\pgfqpoint{1.471388in}{2.045191in}}%
\pgfpathlineto{\pgfqpoint{1.477379in}{2.059191in}}%
\pgfpathlineto{\pgfqpoint{1.477917in}{2.073191in}}%
\pgfpathlineto{\pgfqpoint{1.478989in}{2.087191in}}%
\pgfpathlineto{\pgfqpoint{1.482177in}{2.101191in}}%
\pgfpathlineto{\pgfqpoint{1.483231in}{2.115191in}}%
\pgfpathlineto{\pgfqpoint{1.483757in}{2.129191in}}%
\pgfpathlineto{\pgfqpoint{1.489465in}{2.143191in}}%
\pgfpathlineto{\pgfqpoint{1.489977in}{2.157191in}}%
\pgfpathlineto{\pgfqpoint{1.493536in}{2.171191in}}%
\pgfpathlineto{\pgfqpoint{1.493536in}{2.185191in}}%
\pgfpathlineto{\pgfqpoint{1.497046in}{2.199191in}}%
\pgfpathlineto{\pgfqpoint{1.519880in}{2.213191in}}%
\pgfpathlineto{\pgfqpoint{1.522144in}{2.227191in}}%
\pgfpathlineto{\pgfqpoint{2.011288in}{2.241191in}}%
\pgfpathlineto{\pgfqpoint{2.595507in}{2.255191in}}%
\pgfpathlineto{\pgfqpoint{2.602575in}{2.269191in}}%
\pgfusepath{stroke}%
\end{pgfscope}%
\begin{pgfscope}%
\pgfpathrectangle{\pgfqpoint{0.506010in}{1.121191in}}{\pgfqpoint{2.325000in}{1.400000in}} %
\pgfusepath{clip}%
\pgfsetbuttcap%
\pgfsetroundjoin%
\pgfsetlinewidth{0.501875pt}%
\definecolor{currentstroke}{rgb}{0.000000,0.000000,0.000000}%
\pgfsetstrokecolor{currentstroke}%
\pgfsetdash{{1.000000pt}{3.000000pt}}{0.000000pt}%
\pgfpathmoveto{\pgfqpoint{0.506010in}{1.121191in}}%
\pgfpathlineto{\pgfqpoint{0.506010in}{2.521191in}}%
\pgfusepath{stroke}%
\end{pgfscope}%
\begin{pgfscope}%
\pgfsetbuttcap%
\pgfsetroundjoin%
\definecolor{currentfill}{rgb}{0.000000,0.000000,0.000000}%
\pgfsetfillcolor{currentfill}%
\pgfsetlinewidth{0.501875pt}%
\definecolor{currentstroke}{rgb}{0.000000,0.000000,0.000000}%
\pgfsetstrokecolor{currentstroke}%
\pgfsetdash{}{0pt}%
\pgfsys@defobject{currentmarker}{\pgfqpoint{0.000000in}{0.000000in}}{\pgfqpoint{0.000000in}{0.055556in}}{%
\pgfpathmoveto{\pgfqpoint{0.000000in}{0.000000in}}%
\pgfpathlineto{\pgfqpoint{0.000000in}{0.055556in}}%
\pgfusepath{stroke,fill}%
}%
\begin{pgfscope}%
\pgfsys@transformshift{0.506010in}{1.121191in}%
\pgfsys@useobject{currentmarker}{}%
\end{pgfscope}%
\end{pgfscope}%
\begin{pgfscope}%
\pgfsetbuttcap%
\pgfsetroundjoin%
\definecolor{currentfill}{rgb}{0.000000,0.000000,0.000000}%
\pgfsetfillcolor{currentfill}%
\pgfsetlinewidth{0.501875pt}%
\definecolor{currentstroke}{rgb}{0.000000,0.000000,0.000000}%
\pgfsetstrokecolor{currentstroke}%
\pgfsetdash{}{0pt}%
\pgfsys@defobject{currentmarker}{\pgfqpoint{0.000000in}{-0.055556in}}{\pgfqpoint{0.000000in}{0.000000in}}{%
\pgfpathmoveto{\pgfqpoint{0.000000in}{0.000000in}}%
\pgfpathlineto{\pgfqpoint{0.000000in}{-0.055556in}}%
\pgfusepath{stroke,fill}%
}%
\begin{pgfscope}%
\pgfsys@transformshift{0.506010in}{2.521191in}%
\pgfsys@useobject{currentmarker}{}%
\end{pgfscope}%
\end{pgfscope}%
\begin{pgfscope}%
\pgftext[x=0.506010in,y=1.065635in,,top]{{\rmfamily\fontsize{8.328000}{9.993600}\selectfont 0.1}}%
\end{pgfscope}%
\begin{pgfscope}%
\pgfpathrectangle{\pgfqpoint{0.506010in}{1.121191in}}{\pgfqpoint{2.325000in}{1.400000in}} %
\pgfusepath{clip}%
\pgfsetbuttcap%
\pgfsetroundjoin%
\pgfsetlinewidth{0.501875pt}%
\definecolor{currentstroke}{rgb}{0.000000,0.000000,0.000000}%
\pgfsetstrokecolor{currentstroke}%
\pgfsetdash{{1.000000pt}{3.000000pt}}{0.000000pt}%
\pgfpathmoveto{\pgfqpoint{1.087260in}{1.121191in}}%
\pgfpathlineto{\pgfqpoint{1.087260in}{2.521191in}}%
\pgfusepath{stroke}%
\end{pgfscope}%
\begin{pgfscope}%
\pgfsetbuttcap%
\pgfsetroundjoin%
\definecolor{currentfill}{rgb}{0.000000,0.000000,0.000000}%
\pgfsetfillcolor{currentfill}%
\pgfsetlinewidth{0.501875pt}%
\definecolor{currentstroke}{rgb}{0.000000,0.000000,0.000000}%
\pgfsetstrokecolor{currentstroke}%
\pgfsetdash{}{0pt}%
\pgfsys@defobject{currentmarker}{\pgfqpoint{0.000000in}{0.000000in}}{\pgfqpoint{0.000000in}{0.055556in}}{%
\pgfpathmoveto{\pgfqpoint{0.000000in}{0.000000in}}%
\pgfpathlineto{\pgfqpoint{0.000000in}{0.055556in}}%
\pgfusepath{stroke,fill}%
}%
\begin{pgfscope}%
\pgfsys@transformshift{1.087260in}{1.121191in}%
\pgfsys@useobject{currentmarker}{}%
\end{pgfscope}%
\end{pgfscope}%
\begin{pgfscope}%
\pgfsetbuttcap%
\pgfsetroundjoin%
\definecolor{currentfill}{rgb}{0.000000,0.000000,0.000000}%
\pgfsetfillcolor{currentfill}%
\pgfsetlinewidth{0.501875pt}%
\definecolor{currentstroke}{rgb}{0.000000,0.000000,0.000000}%
\pgfsetstrokecolor{currentstroke}%
\pgfsetdash{}{0pt}%
\pgfsys@defobject{currentmarker}{\pgfqpoint{0.000000in}{-0.055556in}}{\pgfqpoint{0.000000in}{0.000000in}}{%
\pgfpathmoveto{\pgfqpoint{0.000000in}{0.000000in}}%
\pgfpathlineto{\pgfqpoint{0.000000in}{-0.055556in}}%
\pgfusepath{stroke,fill}%
}%
\begin{pgfscope}%
\pgfsys@transformshift{1.087260in}{2.521191in}%
\pgfsys@useobject{currentmarker}{}%
\end{pgfscope}%
\end{pgfscope}%
\begin{pgfscope}%
\pgftext[x=1.087260in,y=1.065635in,,top]{{\rmfamily\fontsize{8.328000}{9.993600}\selectfont 1}}%
\end{pgfscope}%
\begin{pgfscope}%
\pgfpathrectangle{\pgfqpoint{0.506010in}{1.121191in}}{\pgfqpoint{2.325000in}{1.400000in}} %
\pgfusepath{clip}%
\pgfsetbuttcap%
\pgfsetroundjoin%
\pgfsetlinewidth{0.501875pt}%
\definecolor{currentstroke}{rgb}{0.000000,0.000000,0.000000}%
\pgfsetstrokecolor{currentstroke}%
\pgfsetdash{{1.000000pt}{3.000000pt}}{0.000000pt}%
\pgfpathmoveto{\pgfqpoint{1.668510in}{1.121191in}}%
\pgfpathlineto{\pgfqpoint{1.668510in}{2.521191in}}%
\pgfusepath{stroke}%
\end{pgfscope}%
\begin{pgfscope}%
\pgfsetbuttcap%
\pgfsetroundjoin%
\definecolor{currentfill}{rgb}{0.000000,0.000000,0.000000}%
\pgfsetfillcolor{currentfill}%
\pgfsetlinewidth{0.501875pt}%
\definecolor{currentstroke}{rgb}{0.000000,0.000000,0.000000}%
\pgfsetstrokecolor{currentstroke}%
\pgfsetdash{}{0pt}%
\pgfsys@defobject{currentmarker}{\pgfqpoint{0.000000in}{0.000000in}}{\pgfqpoint{0.000000in}{0.055556in}}{%
\pgfpathmoveto{\pgfqpoint{0.000000in}{0.000000in}}%
\pgfpathlineto{\pgfqpoint{0.000000in}{0.055556in}}%
\pgfusepath{stroke,fill}%
}%
\begin{pgfscope}%
\pgfsys@transformshift{1.668510in}{1.121191in}%
\pgfsys@useobject{currentmarker}{}%
\end{pgfscope}%
\end{pgfscope}%
\begin{pgfscope}%
\pgfsetbuttcap%
\pgfsetroundjoin%
\definecolor{currentfill}{rgb}{0.000000,0.000000,0.000000}%
\pgfsetfillcolor{currentfill}%
\pgfsetlinewidth{0.501875pt}%
\definecolor{currentstroke}{rgb}{0.000000,0.000000,0.000000}%
\pgfsetstrokecolor{currentstroke}%
\pgfsetdash{}{0pt}%
\pgfsys@defobject{currentmarker}{\pgfqpoint{0.000000in}{-0.055556in}}{\pgfqpoint{0.000000in}{0.000000in}}{%
\pgfpathmoveto{\pgfqpoint{0.000000in}{0.000000in}}%
\pgfpathlineto{\pgfqpoint{0.000000in}{-0.055556in}}%
\pgfusepath{stroke,fill}%
}%
\begin{pgfscope}%
\pgfsys@transformshift{1.668510in}{2.521191in}%
\pgfsys@useobject{currentmarker}{}%
\end{pgfscope}%
\end{pgfscope}%
\begin{pgfscope}%
\pgftext[x=1.668510in,y=1.065635in,,top]{{\rmfamily\fontsize{8.328000}{9.993600}\selectfont 10}}%
\end{pgfscope}%
\begin{pgfscope}%
\pgfpathrectangle{\pgfqpoint{0.506010in}{1.121191in}}{\pgfqpoint{2.325000in}{1.400000in}} %
\pgfusepath{clip}%
\pgfsetbuttcap%
\pgfsetroundjoin%
\pgfsetlinewidth{0.501875pt}%
\definecolor{currentstroke}{rgb}{0.000000,0.000000,0.000000}%
\pgfsetstrokecolor{currentstroke}%
\pgfsetdash{{1.000000pt}{3.000000pt}}{0.000000pt}%
\pgfpathmoveto{\pgfqpoint{2.249760in}{1.121191in}}%
\pgfpathlineto{\pgfqpoint{2.249760in}{2.521191in}}%
\pgfusepath{stroke}%
\end{pgfscope}%
\begin{pgfscope}%
\pgfsetbuttcap%
\pgfsetroundjoin%
\definecolor{currentfill}{rgb}{0.000000,0.000000,0.000000}%
\pgfsetfillcolor{currentfill}%
\pgfsetlinewidth{0.501875pt}%
\definecolor{currentstroke}{rgb}{0.000000,0.000000,0.000000}%
\pgfsetstrokecolor{currentstroke}%
\pgfsetdash{}{0pt}%
\pgfsys@defobject{currentmarker}{\pgfqpoint{0.000000in}{0.000000in}}{\pgfqpoint{0.000000in}{0.055556in}}{%
\pgfpathmoveto{\pgfqpoint{0.000000in}{0.000000in}}%
\pgfpathlineto{\pgfqpoint{0.000000in}{0.055556in}}%
\pgfusepath{stroke,fill}%
}%
\begin{pgfscope}%
\pgfsys@transformshift{2.249760in}{1.121191in}%
\pgfsys@useobject{currentmarker}{}%
\end{pgfscope}%
\end{pgfscope}%
\begin{pgfscope}%
\pgfsetbuttcap%
\pgfsetroundjoin%
\definecolor{currentfill}{rgb}{0.000000,0.000000,0.000000}%
\pgfsetfillcolor{currentfill}%
\pgfsetlinewidth{0.501875pt}%
\definecolor{currentstroke}{rgb}{0.000000,0.000000,0.000000}%
\pgfsetstrokecolor{currentstroke}%
\pgfsetdash{}{0pt}%
\pgfsys@defobject{currentmarker}{\pgfqpoint{0.000000in}{-0.055556in}}{\pgfqpoint{0.000000in}{0.000000in}}{%
\pgfpathmoveto{\pgfqpoint{0.000000in}{0.000000in}}%
\pgfpathlineto{\pgfqpoint{0.000000in}{-0.055556in}}%
\pgfusepath{stroke,fill}%
}%
\begin{pgfscope}%
\pgfsys@transformshift{2.249760in}{2.521191in}%
\pgfsys@useobject{currentmarker}{}%
\end{pgfscope}%
\end{pgfscope}%
\begin{pgfscope}%
\pgftext[x=2.249760in,y=1.065635in,,top]{{\rmfamily\fontsize{8.328000}{9.993600}\selectfont 100}}%
\end{pgfscope}%
\begin{pgfscope}%
\pgfpathrectangle{\pgfqpoint{0.506010in}{1.121191in}}{\pgfqpoint{2.325000in}{1.400000in}} %
\pgfusepath{clip}%
\pgfsetbuttcap%
\pgfsetroundjoin%
\pgfsetlinewidth{0.501875pt}%
\definecolor{currentstroke}{rgb}{0.000000,0.000000,0.000000}%
\pgfsetstrokecolor{currentstroke}%
\pgfsetdash{{1.000000pt}{3.000000pt}}{0.000000pt}%
\pgfpathmoveto{\pgfqpoint{2.831010in}{1.121191in}}%
\pgfpathlineto{\pgfqpoint{2.831010in}{2.521191in}}%
\pgfusepath{stroke}%
\end{pgfscope}%
\begin{pgfscope}%
\pgfsetbuttcap%
\pgfsetroundjoin%
\definecolor{currentfill}{rgb}{0.000000,0.000000,0.000000}%
\pgfsetfillcolor{currentfill}%
\pgfsetlinewidth{0.501875pt}%
\definecolor{currentstroke}{rgb}{0.000000,0.000000,0.000000}%
\pgfsetstrokecolor{currentstroke}%
\pgfsetdash{}{0pt}%
\pgfsys@defobject{currentmarker}{\pgfqpoint{0.000000in}{0.000000in}}{\pgfqpoint{0.000000in}{0.055556in}}{%
\pgfpathmoveto{\pgfqpoint{0.000000in}{0.000000in}}%
\pgfpathlineto{\pgfqpoint{0.000000in}{0.055556in}}%
\pgfusepath{stroke,fill}%
}%
\begin{pgfscope}%
\pgfsys@transformshift{2.831010in}{1.121191in}%
\pgfsys@useobject{currentmarker}{}%
\end{pgfscope}%
\end{pgfscope}%
\begin{pgfscope}%
\pgfsetbuttcap%
\pgfsetroundjoin%
\definecolor{currentfill}{rgb}{0.000000,0.000000,0.000000}%
\pgfsetfillcolor{currentfill}%
\pgfsetlinewidth{0.501875pt}%
\definecolor{currentstroke}{rgb}{0.000000,0.000000,0.000000}%
\pgfsetstrokecolor{currentstroke}%
\pgfsetdash{}{0pt}%
\pgfsys@defobject{currentmarker}{\pgfqpoint{0.000000in}{-0.055556in}}{\pgfqpoint{0.000000in}{0.000000in}}{%
\pgfpathmoveto{\pgfqpoint{0.000000in}{0.000000in}}%
\pgfpathlineto{\pgfqpoint{0.000000in}{-0.055556in}}%
\pgfusepath{stroke,fill}%
}%
\begin{pgfscope}%
\pgfsys@transformshift{2.831010in}{2.521191in}%
\pgfsys@useobject{currentmarker}{}%
\end{pgfscope}%
\end{pgfscope}%
\begin{pgfscope}%
\pgftext[x=2.831010in,y=1.065635in,,top]{{\rmfamily\fontsize{8.328000}{9.993600}\selectfont 1000}}%
\end{pgfscope}%
\begin{pgfscope}%
\pgfsetbuttcap%
\pgfsetroundjoin%
\definecolor{currentfill}{rgb}{0.000000,0.000000,0.000000}%
\pgfsetfillcolor{currentfill}%
\pgfsetlinewidth{0.501875pt}%
\definecolor{currentstroke}{rgb}{0.000000,0.000000,0.000000}%
\pgfsetstrokecolor{currentstroke}%
\pgfsetdash{}{0pt}%
\pgfsys@defobject{currentmarker}{\pgfqpoint{0.000000in}{0.000000in}}{\pgfqpoint{0.000000in}{0.027778in}}{%
\pgfpathmoveto{\pgfqpoint{0.000000in}{0.000000in}}%
\pgfpathlineto{\pgfqpoint{0.000000in}{0.027778in}}%
\pgfusepath{stroke,fill}%
}%
\begin{pgfscope}%
\pgfsys@transformshift{0.680984in}{1.121191in}%
\pgfsys@useobject{currentmarker}{}%
\end{pgfscope}%
\end{pgfscope}%
\begin{pgfscope}%
\pgfsetbuttcap%
\pgfsetroundjoin%
\definecolor{currentfill}{rgb}{0.000000,0.000000,0.000000}%
\pgfsetfillcolor{currentfill}%
\pgfsetlinewidth{0.501875pt}%
\definecolor{currentstroke}{rgb}{0.000000,0.000000,0.000000}%
\pgfsetstrokecolor{currentstroke}%
\pgfsetdash{}{0pt}%
\pgfsys@defobject{currentmarker}{\pgfqpoint{0.000000in}{-0.027778in}}{\pgfqpoint{0.000000in}{0.000000in}}{%
\pgfpathmoveto{\pgfqpoint{0.000000in}{0.000000in}}%
\pgfpathlineto{\pgfqpoint{0.000000in}{-0.027778in}}%
\pgfusepath{stroke,fill}%
}%
\begin{pgfscope}%
\pgfsys@transformshift{0.680984in}{2.521191in}%
\pgfsys@useobject{currentmarker}{}%
\end{pgfscope}%
\end{pgfscope}%
\begin{pgfscope}%
\pgfsetbuttcap%
\pgfsetroundjoin%
\definecolor{currentfill}{rgb}{0.000000,0.000000,0.000000}%
\pgfsetfillcolor{currentfill}%
\pgfsetlinewidth{0.501875pt}%
\definecolor{currentstroke}{rgb}{0.000000,0.000000,0.000000}%
\pgfsetstrokecolor{currentstroke}%
\pgfsetdash{}{0pt}%
\pgfsys@defobject{currentmarker}{\pgfqpoint{0.000000in}{0.000000in}}{\pgfqpoint{0.000000in}{0.027778in}}{%
\pgfpathmoveto{\pgfqpoint{0.000000in}{0.000000in}}%
\pgfpathlineto{\pgfqpoint{0.000000in}{0.027778in}}%
\pgfusepath{stroke,fill}%
}%
\begin{pgfscope}%
\pgfsys@transformshift{0.783337in}{1.121191in}%
\pgfsys@useobject{currentmarker}{}%
\end{pgfscope}%
\end{pgfscope}%
\begin{pgfscope}%
\pgfsetbuttcap%
\pgfsetroundjoin%
\definecolor{currentfill}{rgb}{0.000000,0.000000,0.000000}%
\pgfsetfillcolor{currentfill}%
\pgfsetlinewidth{0.501875pt}%
\definecolor{currentstroke}{rgb}{0.000000,0.000000,0.000000}%
\pgfsetstrokecolor{currentstroke}%
\pgfsetdash{}{0pt}%
\pgfsys@defobject{currentmarker}{\pgfqpoint{0.000000in}{-0.027778in}}{\pgfqpoint{0.000000in}{0.000000in}}{%
\pgfpathmoveto{\pgfqpoint{0.000000in}{0.000000in}}%
\pgfpathlineto{\pgfqpoint{0.000000in}{-0.027778in}}%
\pgfusepath{stroke,fill}%
}%
\begin{pgfscope}%
\pgfsys@transformshift{0.783337in}{2.521191in}%
\pgfsys@useobject{currentmarker}{}%
\end{pgfscope}%
\end{pgfscope}%
\begin{pgfscope}%
\pgfsetbuttcap%
\pgfsetroundjoin%
\definecolor{currentfill}{rgb}{0.000000,0.000000,0.000000}%
\pgfsetfillcolor{currentfill}%
\pgfsetlinewidth{0.501875pt}%
\definecolor{currentstroke}{rgb}{0.000000,0.000000,0.000000}%
\pgfsetstrokecolor{currentstroke}%
\pgfsetdash{}{0pt}%
\pgfsys@defobject{currentmarker}{\pgfqpoint{0.000000in}{0.000000in}}{\pgfqpoint{0.000000in}{0.027778in}}{%
\pgfpathmoveto{\pgfqpoint{0.000000in}{0.000000in}}%
\pgfpathlineto{\pgfqpoint{0.000000in}{0.027778in}}%
\pgfusepath{stroke,fill}%
}%
\begin{pgfscope}%
\pgfsys@transformshift{0.855957in}{1.121191in}%
\pgfsys@useobject{currentmarker}{}%
\end{pgfscope}%
\end{pgfscope}%
\begin{pgfscope}%
\pgfsetbuttcap%
\pgfsetroundjoin%
\definecolor{currentfill}{rgb}{0.000000,0.000000,0.000000}%
\pgfsetfillcolor{currentfill}%
\pgfsetlinewidth{0.501875pt}%
\definecolor{currentstroke}{rgb}{0.000000,0.000000,0.000000}%
\pgfsetstrokecolor{currentstroke}%
\pgfsetdash{}{0pt}%
\pgfsys@defobject{currentmarker}{\pgfqpoint{0.000000in}{-0.027778in}}{\pgfqpoint{0.000000in}{0.000000in}}{%
\pgfpathmoveto{\pgfqpoint{0.000000in}{0.000000in}}%
\pgfpathlineto{\pgfqpoint{0.000000in}{-0.027778in}}%
\pgfusepath{stroke,fill}%
}%
\begin{pgfscope}%
\pgfsys@transformshift{0.855957in}{2.521191in}%
\pgfsys@useobject{currentmarker}{}%
\end{pgfscope}%
\end{pgfscope}%
\begin{pgfscope}%
\pgfsetbuttcap%
\pgfsetroundjoin%
\definecolor{currentfill}{rgb}{0.000000,0.000000,0.000000}%
\pgfsetfillcolor{currentfill}%
\pgfsetlinewidth{0.501875pt}%
\definecolor{currentstroke}{rgb}{0.000000,0.000000,0.000000}%
\pgfsetstrokecolor{currentstroke}%
\pgfsetdash{}{0pt}%
\pgfsys@defobject{currentmarker}{\pgfqpoint{0.000000in}{0.000000in}}{\pgfqpoint{0.000000in}{0.027778in}}{%
\pgfpathmoveto{\pgfqpoint{0.000000in}{0.000000in}}%
\pgfpathlineto{\pgfqpoint{0.000000in}{0.027778in}}%
\pgfusepath{stroke,fill}%
}%
\begin{pgfscope}%
\pgfsys@transformshift{0.912286in}{1.121191in}%
\pgfsys@useobject{currentmarker}{}%
\end{pgfscope}%
\end{pgfscope}%
\begin{pgfscope}%
\pgfsetbuttcap%
\pgfsetroundjoin%
\definecolor{currentfill}{rgb}{0.000000,0.000000,0.000000}%
\pgfsetfillcolor{currentfill}%
\pgfsetlinewidth{0.501875pt}%
\definecolor{currentstroke}{rgb}{0.000000,0.000000,0.000000}%
\pgfsetstrokecolor{currentstroke}%
\pgfsetdash{}{0pt}%
\pgfsys@defobject{currentmarker}{\pgfqpoint{0.000000in}{-0.027778in}}{\pgfqpoint{0.000000in}{0.000000in}}{%
\pgfpathmoveto{\pgfqpoint{0.000000in}{0.000000in}}%
\pgfpathlineto{\pgfqpoint{0.000000in}{-0.027778in}}%
\pgfusepath{stroke,fill}%
}%
\begin{pgfscope}%
\pgfsys@transformshift{0.912286in}{2.521191in}%
\pgfsys@useobject{currentmarker}{}%
\end{pgfscope}%
\end{pgfscope}%
\begin{pgfscope}%
\pgfsetbuttcap%
\pgfsetroundjoin%
\definecolor{currentfill}{rgb}{0.000000,0.000000,0.000000}%
\pgfsetfillcolor{currentfill}%
\pgfsetlinewidth{0.501875pt}%
\definecolor{currentstroke}{rgb}{0.000000,0.000000,0.000000}%
\pgfsetstrokecolor{currentstroke}%
\pgfsetdash{}{0pt}%
\pgfsys@defobject{currentmarker}{\pgfqpoint{0.000000in}{0.000000in}}{\pgfqpoint{0.000000in}{0.027778in}}{%
\pgfpathmoveto{\pgfqpoint{0.000000in}{0.000000in}}%
\pgfpathlineto{\pgfqpoint{0.000000in}{0.027778in}}%
\pgfusepath{stroke,fill}%
}%
\begin{pgfscope}%
\pgfsys@transformshift{0.958310in}{1.121191in}%
\pgfsys@useobject{currentmarker}{}%
\end{pgfscope}%
\end{pgfscope}%
\begin{pgfscope}%
\pgfsetbuttcap%
\pgfsetroundjoin%
\definecolor{currentfill}{rgb}{0.000000,0.000000,0.000000}%
\pgfsetfillcolor{currentfill}%
\pgfsetlinewidth{0.501875pt}%
\definecolor{currentstroke}{rgb}{0.000000,0.000000,0.000000}%
\pgfsetstrokecolor{currentstroke}%
\pgfsetdash{}{0pt}%
\pgfsys@defobject{currentmarker}{\pgfqpoint{0.000000in}{-0.027778in}}{\pgfqpoint{0.000000in}{0.000000in}}{%
\pgfpathmoveto{\pgfqpoint{0.000000in}{0.000000in}}%
\pgfpathlineto{\pgfqpoint{0.000000in}{-0.027778in}}%
\pgfusepath{stroke,fill}%
}%
\begin{pgfscope}%
\pgfsys@transformshift{0.958310in}{2.521191in}%
\pgfsys@useobject{currentmarker}{}%
\end{pgfscope}%
\end{pgfscope}%
\begin{pgfscope}%
\pgfsetbuttcap%
\pgfsetroundjoin%
\definecolor{currentfill}{rgb}{0.000000,0.000000,0.000000}%
\pgfsetfillcolor{currentfill}%
\pgfsetlinewidth{0.501875pt}%
\definecolor{currentstroke}{rgb}{0.000000,0.000000,0.000000}%
\pgfsetstrokecolor{currentstroke}%
\pgfsetdash{}{0pt}%
\pgfsys@defobject{currentmarker}{\pgfqpoint{0.000000in}{0.000000in}}{\pgfqpoint{0.000000in}{0.027778in}}{%
\pgfpathmoveto{\pgfqpoint{0.000000in}{0.000000in}}%
\pgfpathlineto{\pgfqpoint{0.000000in}{0.027778in}}%
\pgfusepath{stroke,fill}%
}%
\begin{pgfscope}%
\pgfsys@transformshift{0.997223in}{1.121191in}%
\pgfsys@useobject{currentmarker}{}%
\end{pgfscope}%
\end{pgfscope}%
\begin{pgfscope}%
\pgfsetbuttcap%
\pgfsetroundjoin%
\definecolor{currentfill}{rgb}{0.000000,0.000000,0.000000}%
\pgfsetfillcolor{currentfill}%
\pgfsetlinewidth{0.501875pt}%
\definecolor{currentstroke}{rgb}{0.000000,0.000000,0.000000}%
\pgfsetstrokecolor{currentstroke}%
\pgfsetdash{}{0pt}%
\pgfsys@defobject{currentmarker}{\pgfqpoint{0.000000in}{-0.027778in}}{\pgfqpoint{0.000000in}{0.000000in}}{%
\pgfpathmoveto{\pgfqpoint{0.000000in}{0.000000in}}%
\pgfpathlineto{\pgfqpoint{0.000000in}{-0.027778in}}%
\pgfusepath{stroke,fill}%
}%
\begin{pgfscope}%
\pgfsys@transformshift{0.997223in}{2.521191in}%
\pgfsys@useobject{currentmarker}{}%
\end{pgfscope}%
\end{pgfscope}%
\begin{pgfscope}%
\pgfsetbuttcap%
\pgfsetroundjoin%
\definecolor{currentfill}{rgb}{0.000000,0.000000,0.000000}%
\pgfsetfillcolor{currentfill}%
\pgfsetlinewidth{0.501875pt}%
\definecolor{currentstroke}{rgb}{0.000000,0.000000,0.000000}%
\pgfsetstrokecolor{currentstroke}%
\pgfsetdash{}{0pt}%
\pgfsys@defobject{currentmarker}{\pgfqpoint{0.000000in}{0.000000in}}{\pgfqpoint{0.000000in}{0.027778in}}{%
\pgfpathmoveto{\pgfqpoint{0.000000in}{0.000000in}}%
\pgfpathlineto{\pgfqpoint{0.000000in}{0.027778in}}%
\pgfusepath{stroke,fill}%
}%
\begin{pgfscope}%
\pgfsys@transformshift{1.030931in}{1.121191in}%
\pgfsys@useobject{currentmarker}{}%
\end{pgfscope}%
\end{pgfscope}%
\begin{pgfscope}%
\pgfsetbuttcap%
\pgfsetroundjoin%
\definecolor{currentfill}{rgb}{0.000000,0.000000,0.000000}%
\pgfsetfillcolor{currentfill}%
\pgfsetlinewidth{0.501875pt}%
\definecolor{currentstroke}{rgb}{0.000000,0.000000,0.000000}%
\pgfsetstrokecolor{currentstroke}%
\pgfsetdash{}{0pt}%
\pgfsys@defobject{currentmarker}{\pgfqpoint{0.000000in}{-0.027778in}}{\pgfqpoint{0.000000in}{0.000000in}}{%
\pgfpathmoveto{\pgfqpoint{0.000000in}{0.000000in}}%
\pgfpathlineto{\pgfqpoint{0.000000in}{-0.027778in}}%
\pgfusepath{stroke,fill}%
}%
\begin{pgfscope}%
\pgfsys@transformshift{1.030931in}{2.521191in}%
\pgfsys@useobject{currentmarker}{}%
\end{pgfscope}%
\end{pgfscope}%
\begin{pgfscope}%
\pgfsetbuttcap%
\pgfsetroundjoin%
\definecolor{currentfill}{rgb}{0.000000,0.000000,0.000000}%
\pgfsetfillcolor{currentfill}%
\pgfsetlinewidth{0.501875pt}%
\definecolor{currentstroke}{rgb}{0.000000,0.000000,0.000000}%
\pgfsetstrokecolor{currentstroke}%
\pgfsetdash{}{0pt}%
\pgfsys@defobject{currentmarker}{\pgfqpoint{0.000000in}{0.000000in}}{\pgfqpoint{0.000000in}{0.027778in}}{%
\pgfpathmoveto{\pgfqpoint{0.000000in}{0.000000in}}%
\pgfpathlineto{\pgfqpoint{0.000000in}{0.027778in}}%
\pgfusepath{stroke,fill}%
}%
\begin{pgfscope}%
\pgfsys@transformshift{1.060663in}{1.121191in}%
\pgfsys@useobject{currentmarker}{}%
\end{pgfscope}%
\end{pgfscope}%
\begin{pgfscope}%
\pgfsetbuttcap%
\pgfsetroundjoin%
\definecolor{currentfill}{rgb}{0.000000,0.000000,0.000000}%
\pgfsetfillcolor{currentfill}%
\pgfsetlinewidth{0.501875pt}%
\definecolor{currentstroke}{rgb}{0.000000,0.000000,0.000000}%
\pgfsetstrokecolor{currentstroke}%
\pgfsetdash{}{0pt}%
\pgfsys@defobject{currentmarker}{\pgfqpoint{0.000000in}{-0.027778in}}{\pgfqpoint{0.000000in}{0.000000in}}{%
\pgfpathmoveto{\pgfqpoint{0.000000in}{0.000000in}}%
\pgfpathlineto{\pgfqpoint{0.000000in}{-0.027778in}}%
\pgfusepath{stroke,fill}%
}%
\begin{pgfscope}%
\pgfsys@transformshift{1.060663in}{2.521191in}%
\pgfsys@useobject{currentmarker}{}%
\end{pgfscope}%
\end{pgfscope}%
\begin{pgfscope}%
\pgfsetbuttcap%
\pgfsetroundjoin%
\definecolor{currentfill}{rgb}{0.000000,0.000000,0.000000}%
\pgfsetfillcolor{currentfill}%
\pgfsetlinewidth{0.501875pt}%
\definecolor{currentstroke}{rgb}{0.000000,0.000000,0.000000}%
\pgfsetstrokecolor{currentstroke}%
\pgfsetdash{}{0pt}%
\pgfsys@defobject{currentmarker}{\pgfqpoint{0.000000in}{0.000000in}}{\pgfqpoint{0.000000in}{0.027778in}}{%
\pgfpathmoveto{\pgfqpoint{0.000000in}{0.000000in}}%
\pgfpathlineto{\pgfqpoint{0.000000in}{0.027778in}}%
\pgfusepath{stroke,fill}%
}%
\begin{pgfscope}%
\pgfsys@transformshift{1.262234in}{1.121191in}%
\pgfsys@useobject{currentmarker}{}%
\end{pgfscope}%
\end{pgfscope}%
\begin{pgfscope}%
\pgfsetbuttcap%
\pgfsetroundjoin%
\definecolor{currentfill}{rgb}{0.000000,0.000000,0.000000}%
\pgfsetfillcolor{currentfill}%
\pgfsetlinewidth{0.501875pt}%
\definecolor{currentstroke}{rgb}{0.000000,0.000000,0.000000}%
\pgfsetstrokecolor{currentstroke}%
\pgfsetdash{}{0pt}%
\pgfsys@defobject{currentmarker}{\pgfqpoint{0.000000in}{-0.027778in}}{\pgfqpoint{0.000000in}{0.000000in}}{%
\pgfpathmoveto{\pgfqpoint{0.000000in}{0.000000in}}%
\pgfpathlineto{\pgfqpoint{0.000000in}{-0.027778in}}%
\pgfusepath{stroke,fill}%
}%
\begin{pgfscope}%
\pgfsys@transformshift{1.262234in}{2.521191in}%
\pgfsys@useobject{currentmarker}{}%
\end{pgfscope}%
\end{pgfscope}%
\begin{pgfscope}%
\pgfsetbuttcap%
\pgfsetroundjoin%
\definecolor{currentfill}{rgb}{0.000000,0.000000,0.000000}%
\pgfsetfillcolor{currentfill}%
\pgfsetlinewidth{0.501875pt}%
\definecolor{currentstroke}{rgb}{0.000000,0.000000,0.000000}%
\pgfsetstrokecolor{currentstroke}%
\pgfsetdash{}{0pt}%
\pgfsys@defobject{currentmarker}{\pgfqpoint{0.000000in}{0.000000in}}{\pgfqpoint{0.000000in}{0.027778in}}{%
\pgfpathmoveto{\pgfqpoint{0.000000in}{0.000000in}}%
\pgfpathlineto{\pgfqpoint{0.000000in}{0.027778in}}%
\pgfusepath{stroke,fill}%
}%
\begin{pgfscope}%
\pgfsys@transformshift{1.364587in}{1.121191in}%
\pgfsys@useobject{currentmarker}{}%
\end{pgfscope}%
\end{pgfscope}%
\begin{pgfscope}%
\pgfsetbuttcap%
\pgfsetroundjoin%
\definecolor{currentfill}{rgb}{0.000000,0.000000,0.000000}%
\pgfsetfillcolor{currentfill}%
\pgfsetlinewidth{0.501875pt}%
\definecolor{currentstroke}{rgb}{0.000000,0.000000,0.000000}%
\pgfsetstrokecolor{currentstroke}%
\pgfsetdash{}{0pt}%
\pgfsys@defobject{currentmarker}{\pgfqpoint{0.000000in}{-0.027778in}}{\pgfqpoint{0.000000in}{0.000000in}}{%
\pgfpathmoveto{\pgfqpoint{0.000000in}{0.000000in}}%
\pgfpathlineto{\pgfqpoint{0.000000in}{-0.027778in}}%
\pgfusepath{stroke,fill}%
}%
\begin{pgfscope}%
\pgfsys@transformshift{1.364587in}{2.521191in}%
\pgfsys@useobject{currentmarker}{}%
\end{pgfscope}%
\end{pgfscope}%
\begin{pgfscope}%
\pgfsetbuttcap%
\pgfsetroundjoin%
\definecolor{currentfill}{rgb}{0.000000,0.000000,0.000000}%
\pgfsetfillcolor{currentfill}%
\pgfsetlinewidth{0.501875pt}%
\definecolor{currentstroke}{rgb}{0.000000,0.000000,0.000000}%
\pgfsetstrokecolor{currentstroke}%
\pgfsetdash{}{0pt}%
\pgfsys@defobject{currentmarker}{\pgfqpoint{0.000000in}{0.000000in}}{\pgfqpoint{0.000000in}{0.027778in}}{%
\pgfpathmoveto{\pgfqpoint{0.000000in}{0.000000in}}%
\pgfpathlineto{\pgfqpoint{0.000000in}{0.027778in}}%
\pgfusepath{stroke,fill}%
}%
\begin{pgfscope}%
\pgfsys@transformshift{1.437207in}{1.121191in}%
\pgfsys@useobject{currentmarker}{}%
\end{pgfscope}%
\end{pgfscope}%
\begin{pgfscope}%
\pgfsetbuttcap%
\pgfsetroundjoin%
\definecolor{currentfill}{rgb}{0.000000,0.000000,0.000000}%
\pgfsetfillcolor{currentfill}%
\pgfsetlinewidth{0.501875pt}%
\definecolor{currentstroke}{rgb}{0.000000,0.000000,0.000000}%
\pgfsetstrokecolor{currentstroke}%
\pgfsetdash{}{0pt}%
\pgfsys@defobject{currentmarker}{\pgfqpoint{0.000000in}{-0.027778in}}{\pgfqpoint{0.000000in}{0.000000in}}{%
\pgfpathmoveto{\pgfqpoint{0.000000in}{0.000000in}}%
\pgfpathlineto{\pgfqpoint{0.000000in}{-0.027778in}}%
\pgfusepath{stroke,fill}%
}%
\begin{pgfscope}%
\pgfsys@transformshift{1.437207in}{2.521191in}%
\pgfsys@useobject{currentmarker}{}%
\end{pgfscope}%
\end{pgfscope}%
\begin{pgfscope}%
\pgfsetbuttcap%
\pgfsetroundjoin%
\definecolor{currentfill}{rgb}{0.000000,0.000000,0.000000}%
\pgfsetfillcolor{currentfill}%
\pgfsetlinewidth{0.501875pt}%
\definecolor{currentstroke}{rgb}{0.000000,0.000000,0.000000}%
\pgfsetstrokecolor{currentstroke}%
\pgfsetdash{}{0pt}%
\pgfsys@defobject{currentmarker}{\pgfqpoint{0.000000in}{0.000000in}}{\pgfqpoint{0.000000in}{0.027778in}}{%
\pgfpathmoveto{\pgfqpoint{0.000000in}{0.000000in}}%
\pgfpathlineto{\pgfqpoint{0.000000in}{0.027778in}}%
\pgfusepath{stroke,fill}%
}%
\begin{pgfscope}%
\pgfsys@transformshift{1.493536in}{1.121191in}%
\pgfsys@useobject{currentmarker}{}%
\end{pgfscope}%
\end{pgfscope}%
\begin{pgfscope}%
\pgfsetbuttcap%
\pgfsetroundjoin%
\definecolor{currentfill}{rgb}{0.000000,0.000000,0.000000}%
\pgfsetfillcolor{currentfill}%
\pgfsetlinewidth{0.501875pt}%
\definecolor{currentstroke}{rgb}{0.000000,0.000000,0.000000}%
\pgfsetstrokecolor{currentstroke}%
\pgfsetdash{}{0pt}%
\pgfsys@defobject{currentmarker}{\pgfqpoint{0.000000in}{-0.027778in}}{\pgfqpoint{0.000000in}{0.000000in}}{%
\pgfpathmoveto{\pgfqpoint{0.000000in}{0.000000in}}%
\pgfpathlineto{\pgfqpoint{0.000000in}{-0.027778in}}%
\pgfusepath{stroke,fill}%
}%
\begin{pgfscope}%
\pgfsys@transformshift{1.493536in}{2.521191in}%
\pgfsys@useobject{currentmarker}{}%
\end{pgfscope}%
\end{pgfscope}%
\begin{pgfscope}%
\pgfsetbuttcap%
\pgfsetroundjoin%
\definecolor{currentfill}{rgb}{0.000000,0.000000,0.000000}%
\pgfsetfillcolor{currentfill}%
\pgfsetlinewidth{0.501875pt}%
\definecolor{currentstroke}{rgb}{0.000000,0.000000,0.000000}%
\pgfsetstrokecolor{currentstroke}%
\pgfsetdash{}{0pt}%
\pgfsys@defobject{currentmarker}{\pgfqpoint{0.000000in}{0.000000in}}{\pgfqpoint{0.000000in}{0.027778in}}{%
\pgfpathmoveto{\pgfqpoint{0.000000in}{0.000000in}}%
\pgfpathlineto{\pgfqpoint{0.000000in}{0.027778in}}%
\pgfusepath{stroke,fill}%
}%
\begin{pgfscope}%
\pgfsys@transformshift{1.539560in}{1.121191in}%
\pgfsys@useobject{currentmarker}{}%
\end{pgfscope}%
\end{pgfscope}%
\begin{pgfscope}%
\pgfsetbuttcap%
\pgfsetroundjoin%
\definecolor{currentfill}{rgb}{0.000000,0.000000,0.000000}%
\pgfsetfillcolor{currentfill}%
\pgfsetlinewidth{0.501875pt}%
\definecolor{currentstroke}{rgb}{0.000000,0.000000,0.000000}%
\pgfsetstrokecolor{currentstroke}%
\pgfsetdash{}{0pt}%
\pgfsys@defobject{currentmarker}{\pgfqpoint{0.000000in}{-0.027778in}}{\pgfqpoint{0.000000in}{0.000000in}}{%
\pgfpathmoveto{\pgfqpoint{0.000000in}{0.000000in}}%
\pgfpathlineto{\pgfqpoint{0.000000in}{-0.027778in}}%
\pgfusepath{stroke,fill}%
}%
\begin{pgfscope}%
\pgfsys@transformshift{1.539560in}{2.521191in}%
\pgfsys@useobject{currentmarker}{}%
\end{pgfscope}%
\end{pgfscope}%
\begin{pgfscope}%
\pgfsetbuttcap%
\pgfsetroundjoin%
\definecolor{currentfill}{rgb}{0.000000,0.000000,0.000000}%
\pgfsetfillcolor{currentfill}%
\pgfsetlinewidth{0.501875pt}%
\definecolor{currentstroke}{rgb}{0.000000,0.000000,0.000000}%
\pgfsetstrokecolor{currentstroke}%
\pgfsetdash{}{0pt}%
\pgfsys@defobject{currentmarker}{\pgfqpoint{0.000000in}{0.000000in}}{\pgfqpoint{0.000000in}{0.027778in}}{%
\pgfpathmoveto{\pgfqpoint{0.000000in}{0.000000in}}%
\pgfpathlineto{\pgfqpoint{0.000000in}{0.027778in}}%
\pgfusepath{stroke,fill}%
}%
\begin{pgfscope}%
\pgfsys@transformshift{1.578473in}{1.121191in}%
\pgfsys@useobject{currentmarker}{}%
\end{pgfscope}%
\end{pgfscope}%
\begin{pgfscope}%
\pgfsetbuttcap%
\pgfsetroundjoin%
\definecolor{currentfill}{rgb}{0.000000,0.000000,0.000000}%
\pgfsetfillcolor{currentfill}%
\pgfsetlinewidth{0.501875pt}%
\definecolor{currentstroke}{rgb}{0.000000,0.000000,0.000000}%
\pgfsetstrokecolor{currentstroke}%
\pgfsetdash{}{0pt}%
\pgfsys@defobject{currentmarker}{\pgfqpoint{0.000000in}{-0.027778in}}{\pgfqpoint{0.000000in}{0.000000in}}{%
\pgfpathmoveto{\pgfqpoint{0.000000in}{0.000000in}}%
\pgfpathlineto{\pgfqpoint{0.000000in}{-0.027778in}}%
\pgfusepath{stroke,fill}%
}%
\begin{pgfscope}%
\pgfsys@transformshift{1.578473in}{2.521191in}%
\pgfsys@useobject{currentmarker}{}%
\end{pgfscope}%
\end{pgfscope}%
\begin{pgfscope}%
\pgfsetbuttcap%
\pgfsetroundjoin%
\definecolor{currentfill}{rgb}{0.000000,0.000000,0.000000}%
\pgfsetfillcolor{currentfill}%
\pgfsetlinewidth{0.501875pt}%
\definecolor{currentstroke}{rgb}{0.000000,0.000000,0.000000}%
\pgfsetstrokecolor{currentstroke}%
\pgfsetdash{}{0pt}%
\pgfsys@defobject{currentmarker}{\pgfqpoint{0.000000in}{0.000000in}}{\pgfqpoint{0.000000in}{0.027778in}}{%
\pgfpathmoveto{\pgfqpoint{0.000000in}{0.000000in}}%
\pgfpathlineto{\pgfqpoint{0.000000in}{0.027778in}}%
\pgfusepath{stroke,fill}%
}%
\begin{pgfscope}%
\pgfsys@transformshift{1.612181in}{1.121191in}%
\pgfsys@useobject{currentmarker}{}%
\end{pgfscope}%
\end{pgfscope}%
\begin{pgfscope}%
\pgfsetbuttcap%
\pgfsetroundjoin%
\definecolor{currentfill}{rgb}{0.000000,0.000000,0.000000}%
\pgfsetfillcolor{currentfill}%
\pgfsetlinewidth{0.501875pt}%
\definecolor{currentstroke}{rgb}{0.000000,0.000000,0.000000}%
\pgfsetstrokecolor{currentstroke}%
\pgfsetdash{}{0pt}%
\pgfsys@defobject{currentmarker}{\pgfqpoint{0.000000in}{-0.027778in}}{\pgfqpoint{0.000000in}{0.000000in}}{%
\pgfpathmoveto{\pgfqpoint{0.000000in}{0.000000in}}%
\pgfpathlineto{\pgfqpoint{0.000000in}{-0.027778in}}%
\pgfusepath{stroke,fill}%
}%
\begin{pgfscope}%
\pgfsys@transformshift{1.612181in}{2.521191in}%
\pgfsys@useobject{currentmarker}{}%
\end{pgfscope}%
\end{pgfscope}%
\begin{pgfscope}%
\pgfsetbuttcap%
\pgfsetroundjoin%
\definecolor{currentfill}{rgb}{0.000000,0.000000,0.000000}%
\pgfsetfillcolor{currentfill}%
\pgfsetlinewidth{0.501875pt}%
\definecolor{currentstroke}{rgb}{0.000000,0.000000,0.000000}%
\pgfsetstrokecolor{currentstroke}%
\pgfsetdash{}{0pt}%
\pgfsys@defobject{currentmarker}{\pgfqpoint{0.000000in}{0.000000in}}{\pgfqpoint{0.000000in}{0.027778in}}{%
\pgfpathmoveto{\pgfqpoint{0.000000in}{0.000000in}}%
\pgfpathlineto{\pgfqpoint{0.000000in}{0.027778in}}%
\pgfusepath{stroke,fill}%
}%
\begin{pgfscope}%
\pgfsys@transformshift{1.641913in}{1.121191in}%
\pgfsys@useobject{currentmarker}{}%
\end{pgfscope}%
\end{pgfscope}%
\begin{pgfscope}%
\pgfsetbuttcap%
\pgfsetroundjoin%
\definecolor{currentfill}{rgb}{0.000000,0.000000,0.000000}%
\pgfsetfillcolor{currentfill}%
\pgfsetlinewidth{0.501875pt}%
\definecolor{currentstroke}{rgb}{0.000000,0.000000,0.000000}%
\pgfsetstrokecolor{currentstroke}%
\pgfsetdash{}{0pt}%
\pgfsys@defobject{currentmarker}{\pgfqpoint{0.000000in}{-0.027778in}}{\pgfqpoint{0.000000in}{0.000000in}}{%
\pgfpathmoveto{\pgfqpoint{0.000000in}{0.000000in}}%
\pgfpathlineto{\pgfqpoint{0.000000in}{-0.027778in}}%
\pgfusepath{stroke,fill}%
}%
\begin{pgfscope}%
\pgfsys@transformshift{1.641913in}{2.521191in}%
\pgfsys@useobject{currentmarker}{}%
\end{pgfscope}%
\end{pgfscope}%
\begin{pgfscope}%
\pgfsetbuttcap%
\pgfsetroundjoin%
\definecolor{currentfill}{rgb}{0.000000,0.000000,0.000000}%
\pgfsetfillcolor{currentfill}%
\pgfsetlinewidth{0.501875pt}%
\definecolor{currentstroke}{rgb}{0.000000,0.000000,0.000000}%
\pgfsetstrokecolor{currentstroke}%
\pgfsetdash{}{0pt}%
\pgfsys@defobject{currentmarker}{\pgfqpoint{0.000000in}{0.000000in}}{\pgfqpoint{0.000000in}{0.027778in}}{%
\pgfpathmoveto{\pgfqpoint{0.000000in}{0.000000in}}%
\pgfpathlineto{\pgfqpoint{0.000000in}{0.027778in}}%
\pgfusepath{stroke,fill}%
}%
\begin{pgfscope}%
\pgfsys@transformshift{1.843484in}{1.121191in}%
\pgfsys@useobject{currentmarker}{}%
\end{pgfscope}%
\end{pgfscope}%
\begin{pgfscope}%
\pgfsetbuttcap%
\pgfsetroundjoin%
\definecolor{currentfill}{rgb}{0.000000,0.000000,0.000000}%
\pgfsetfillcolor{currentfill}%
\pgfsetlinewidth{0.501875pt}%
\definecolor{currentstroke}{rgb}{0.000000,0.000000,0.000000}%
\pgfsetstrokecolor{currentstroke}%
\pgfsetdash{}{0pt}%
\pgfsys@defobject{currentmarker}{\pgfqpoint{0.000000in}{-0.027778in}}{\pgfqpoint{0.000000in}{0.000000in}}{%
\pgfpathmoveto{\pgfqpoint{0.000000in}{0.000000in}}%
\pgfpathlineto{\pgfqpoint{0.000000in}{-0.027778in}}%
\pgfusepath{stroke,fill}%
}%
\begin{pgfscope}%
\pgfsys@transformshift{1.843484in}{2.521191in}%
\pgfsys@useobject{currentmarker}{}%
\end{pgfscope}%
\end{pgfscope}%
\begin{pgfscope}%
\pgfsetbuttcap%
\pgfsetroundjoin%
\definecolor{currentfill}{rgb}{0.000000,0.000000,0.000000}%
\pgfsetfillcolor{currentfill}%
\pgfsetlinewidth{0.501875pt}%
\definecolor{currentstroke}{rgb}{0.000000,0.000000,0.000000}%
\pgfsetstrokecolor{currentstroke}%
\pgfsetdash{}{0pt}%
\pgfsys@defobject{currentmarker}{\pgfqpoint{0.000000in}{0.000000in}}{\pgfqpoint{0.000000in}{0.027778in}}{%
\pgfpathmoveto{\pgfqpoint{0.000000in}{0.000000in}}%
\pgfpathlineto{\pgfqpoint{0.000000in}{0.027778in}}%
\pgfusepath{stroke,fill}%
}%
\begin{pgfscope}%
\pgfsys@transformshift{1.945837in}{1.121191in}%
\pgfsys@useobject{currentmarker}{}%
\end{pgfscope}%
\end{pgfscope}%
\begin{pgfscope}%
\pgfsetbuttcap%
\pgfsetroundjoin%
\definecolor{currentfill}{rgb}{0.000000,0.000000,0.000000}%
\pgfsetfillcolor{currentfill}%
\pgfsetlinewidth{0.501875pt}%
\definecolor{currentstroke}{rgb}{0.000000,0.000000,0.000000}%
\pgfsetstrokecolor{currentstroke}%
\pgfsetdash{}{0pt}%
\pgfsys@defobject{currentmarker}{\pgfqpoint{0.000000in}{-0.027778in}}{\pgfqpoint{0.000000in}{0.000000in}}{%
\pgfpathmoveto{\pgfqpoint{0.000000in}{0.000000in}}%
\pgfpathlineto{\pgfqpoint{0.000000in}{-0.027778in}}%
\pgfusepath{stroke,fill}%
}%
\begin{pgfscope}%
\pgfsys@transformshift{1.945837in}{2.521191in}%
\pgfsys@useobject{currentmarker}{}%
\end{pgfscope}%
\end{pgfscope}%
\begin{pgfscope}%
\pgfsetbuttcap%
\pgfsetroundjoin%
\definecolor{currentfill}{rgb}{0.000000,0.000000,0.000000}%
\pgfsetfillcolor{currentfill}%
\pgfsetlinewidth{0.501875pt}%
\definecolor{currentstroke}{rgb}{0.000000,0.000000,0.000000}%
\pgfsetstrokecolor{currentstroke}%
\pgfsetdash{}{0pt}%
\pgfsys@defobject{currentmarker}{\pgfqpoint{0.000000in}{0.000000in}}{\pgfqpoint{0.000000in}{0.027778in}}{%
\pgfpathmoveto{\pgfqpoint{0.000000in}{0.000000in}}%
\pgfpathlineto{\pgfqpoint{0.000000in}{0.027778in}}%
\pgfusepath{stroke,fill}%
}%
\begin{pgfscope}%
\pgfsys@transformshift{2.018457in}{1.121191in}%
\pgfsys@useobject{currentmarker}{}%
\end{pgfscope}%
\end{pgfscope}%
\begin{pgfscope}%
\pgfsetbuttcap%
\pgfsetroundjoin%
\definecolor{currentfill}{rgb}{0.000000,0.000000,0.000000}%
\pgfsetfillcolor{currentfill}%
\pgfsetlinewidth{0.501875pt}%
\definecolor{currentstroke}{rgb}{0.000000,0.000000,0.000000}%
\pgfsetstrokecolor{currentstroke}%
\pgfsetdash{}{0pt}%
\pgfsys@defobject{currentmarker}{\pgfqpoint{0.000000in}{-0.027778in}}{\pgfqpoint{0.000000in}{0.000000in}}{%
\pgfpathmoveto{\pgfqpoint{0.000000in}{0.000000in}}%
\pgfpathlineto{\pgfqpoint{0.000000in}{-0.027778in}}%
\pgfusepath{stroke,fill}%
}%
\begin{pgfscope}%
\pgfsys@transformshift{2.018457in}{2.521191in}%
\pgfsys@useobject{currentmarker}{}%
\end{pgfscope}%
\end{pgfscope}%
\begin{pgfscope}%
\pgfsetbuttcap%
\pgfsetroundjoin%
\definecolor{currentfill}{rgb}{0.000000,0.000000,0.000000}%
\pgfsetfillcolor{currentfill}%
\pgfsetlinewidth{0.501875pt}%
\definecolor{currentstroke}{rgb}{0.000000,0.000000,0.000000}%
\pgfsetstrokecolor{currentstroke}%
\pgfsetdash{}{0pt}%
\pgfsys@defobject{currentmarker}{\pgfqpoint{0.000000in}{0.000000in}}{\pgfqpoint{0.000000in}{0.027778in}}{%
\pgfpathmoveto{\pgfqpoint{0.000000in}{0.000000in}}%
\pgfpathlineto{\pgfqpoint{0.000000in}{0.027778in}}%
\pgfusepath{stroke,fill}%
}%
\begin{pgfscope}%
\pgfsys@transformshift{2.074786in}{1.121191in}%
\pgfsys@useobject{currentmarker}{}%
\end{pgfscope}%
\end{pgfscope}%
\begin{pgfscope}%
\pgfsetbuttcap%
\pgfsetroundjoin%
\definecolor{currentfill}{rgb}{0.000000,0.000000,0.000000}%
\pgfsetfillcolor{currentfill}%
\pgfsetlinewidth{0.501875pt}%
\definecolor{currentstroke}{rgb}{0.000000,0.000000,0.000000}%
\pgfsetstrokecolor{currentstroke}%
\pgfsetdash{}{0pt}%
\pgfsys@defobject{currentmarker}{\pgfqpoint{0.000000in}{-0.027778in}}{\pgfqpoint{0.000000in}{0.000000in}}{%
\pgfpathmoveto{\pgfqpoint{0.000000in}{0.000000in}}%
\pgfpathlineto{\pgfqpoint{0.000000in}{-0.027778in}}%
\pgfusepath{stroke,fill}%
}%
\begin{pgfscope}%
\pgfsys@transformshift{2.074786in}{2.521191in}%
\pgfsys@useobject{currentmarker}{}%
\end{pgfscope}%
\end{pgfscope}%
\begin{pgfscope}%
\pgfsetbuttcap%
\pgfsetroundjoin%
\definecolor{currentfill}{rgb}{0.000000,0.000000,0.000000}%
\pgfsetfillcolor{currentfill}%
\pgfsetlinewidth{0.501875pt}%
\definecolor{currentstroke}{rgb}{0.000000,0.000000,0.000000}%
\pgfsetstrokecolor{currentstroke}%
\pgfsetdash{}{0pt}%
\pgfsys@defobject{currentmarker}{\pgfqpoint{0.000000in}{0.000000in}}{\pgfqpoint{0.000000in}{0.027778in}}{%
\pgfpathmoveto{\pgfqpoint{0.000000in}{0.000000in}}%
\pgfpathlineto{\pgfqpoint{0.000000in}{0.027778in}}%
\pgfusepath{stroke,fill}%
}%
\begin{pgfscope}%
\pgfsys@transformshift{2.120810in}{1.121191in}%
\pgfsys@useobject{currentmarker}{}%
\end{pgfscope}%
\end{pgfscope}%
\begin{pgfscope}%
\pgfsetbuttcap%
\pgfsetroundjoin%
\definecolor{currentfill}{rgb}{0.000000,0.000000,0.000000}%
\pgfsetfillcolor{currentfill}%
\pgfsetlinewidth{0.501875pt}%
\definecolor{currentstroke}{rgb}{0.000000,0.000000,0.000000}%
\pgfsetstrokecolor{currentstroke}%
\pgfsetdash{}{0pt}%
\pgfsys@defobject{currentmarker}{\pgfqpoint{0.000000in}{-0.027778in}}{\pgfqpoint{0.000000in}{0.000000in}}{%
\pgfpathmoveto{\pgfqpoint{0.000000in}{0.000000in}}%
\pgfpathlineto{\pgfqpoint{0.000000in}{-0.027778in}}%
\pgfusepath{stroke,fill}%
}%
\begin{pgfscope}%
\pgfsys@transformshift{2.120810in}{2.521191in}%
\pgfsys@useobject{currentmarker}{}%
\end{pgfscope}%
\end{pgfscope}%
\begin{pgfscope}%
\pgfsetbuttcap%
\pgfsetroundjoin%
\definecolor{currentfill}{rgb}{0.000000,0.000000,0.000000}%
\pgfsetfillcolor{currentfill}%
\pgfsetlinewidth{0.501875pt}%
\definecolor{currentstroke}{rgb}{0.000000,0.000000,0.000000}%
\pgfsetstrokecolor{currentstroke}%
\pgfsetdash{}{0pt}%
\pgfsys@defobject{currentmarker}{\pgfqpoint{0.000000in}{0.000000in}}{\pgfqpoint{0.000000in}{0.027778in}}{%
\pgfpathmoveto{\pgfqpoint{0.000000in}{0.000000in}}%
\pgfpathlineto{\pgfqpoint{0.000000in}{0.027778in}}%
\pgfusepath{stroke,fill}%
}%
\begin{pgfscope}%
\pgfsys@transformshift{2.159723in}{1.121191in}%
\pgfsys@useobject{currentmarker}{}%
\end{pgfscope}%
\end{pgfscope}%
\begin{pgfscope}%
\pgfsetbuttcap%
\pgfsetroundjoin%
\definecolor{currentfill}{rgb}{0.000000,0.000000,0.000000}%
\pgfsetfillcolor{currentfill}%
\pgfsetlinewidth{0.501875pt}%
\definecolor{currentstroke}{rgb}{0.000000,0.000000,0.000000}%
\pgfsetstrokecolor{currentstroke}%
\pgfsetdash{}{0pt}%
\pgfsys@defobject{currentmarker}{\pgfqpoint{0.000000in}{-0.027778in}}{\pgfqpoint{0.000000in}{0.000000in}}{%
\pgfpathmoveto{\pgfqpoint{0.000000in}{0.000000in}}%
\pgfpathlineto{\pgfqpoint{0.000000in}{-0.027778in}}%
\pgfusepath{stroke,fill}%
}%
\begin{pgfscope}%
\pgfsys@transformshift{2.159723in}{2.521191in}%
\pgfsys@useobject{currentmarker}{}%
\end{pgfscope}%
\end{pgfscope}%
\begin{pgfscope}%
\pgfsetbuttcap%
\pgfsetroundjoin%
\definecolor{currentfill}{rgb}{0.000000,0.000000,0.000000}%
\pgfsetfillcolor{currentfill}%
\pgfsetlinewidth{0.501875pt}%
\definecolor{currentstroke}{rgb}{0.000000,0.000000,0.000000}%
\pgfsetstrokecolor{currentstroke}%
\pgfsetdash{}{0pt}%
\pgfsys@defobject{currentmarker}{\pgfqpoint{0.000000in}{0.000000in}}{\pgfqpoint{0.000000in}{0.027778in}}{%
\pgfpathmoveto{\pgfqpoint{0.000000in}{0.000000in}}%
\pgfpathlineto{\pgfqpoint{0.000000in}{0.027778in}}%
\pgfusepath{stroke,fill}%
}%
\begin{pgfscope}%
\pgfsys@transformshift{2.193431in}{1.121191in}%
\pgfsys@useobject{currentmarker}{}%
\end{pgfscope}%
\end{pgfscope}%
\begin{pgfscope}%
\pgfsetbuttcap%
\pgfsetroundjoin%
\definecolor{currentfill}{rgb}{0.000000,0.000000,0.000000}%
\pgfsetfillcolor{currentfill}%
\pgfsetlinewidth{0.501875pt}%
\definecolor{currentstroke}{rgb}{0.000000,0.000000,0.000000}%
\pgfsetstrokecolor{currentstroke}%
\pgfsetdash{}{0pt}%
\pgfsys@defobject{currentmarker}{\pgfqpoint{0.000000in}{-0.027778in}}{\pgfqpoint{0.000000in}{0.000000in}}{%
\pgfpathmoveto{\pgfqpoint{0.000000in}{0.000000in}}%
\pgfpathlineto{\pgfqpoint{0.000000in}{-0.027778in}}%
\pgfusepath{stroke,fill}%
}%
\begin{pgfscope}%
\pgfsys@transformshift{2.193431in}{2.521191in}%
\pgfsys@useobject{currentmarker}{}%
\end{pgfscope}%
\end{pgfscope}%
\begin{pgfscope}%
\pgfsetbuttcap%
\pgfsetroundjoin%
\definecolor{currentfill}{rgb}{0.000000,0.000000,0.000000}%
\pgfsetfillcolor{currentfill}%
\pgfsetlinewidth{0.501875pt}%
\definecolor{currentstroke}{rgb}{0.000000,0.000000,0.000000}%
\pgfsetstrokecolor{currentstroke}%
\pgfsetdash{}{0pt}%
\pgfsys@defobject{currentmarker}{\pgfqpoint{0.000000in}{0.000000in}}{\pgfqpoint{0.000000in}{0.027778in}}{%
\pgfpathmoveto{\pgfqpoint{0.000000in}{0.000000in}}%
\pgfpathlineto{\pgfqpoint{0.000000in}{0.027778in}}%
\pgfusepath{stroke,fill}%
}%
\begin{pgfscope}%
\pgfsys@transformshift{2.223163in}{1.121191in}%
\pgfsys@useobject{currentmarker}{}%
\end{pgfscope}%
\end{pgfscope}%
\begin{pgfscope}%
\pgfsetbuttcap%
\pgfsetroundjoin%
\definecolor{currentfill}{rgb}{0.000000,0.000000,0.000000}%
\pgfsetfillcolor{currentfill}%
\pgfsetlinewidth{0.501875pt}%
\definecolor{currentstroke}{rgb}{0.000000,0.000000,0.000000}%
\pgfsetstrokecolor{currentstroke}%
\pgfsetdash{}{0pt}%
\pgfsys@defobject{currentmarker}{\pgfqpoint{0.000000in}{-0.027778in}}{\pgfqpoint{0.000000in}{0.000000in}}{%
\pgfpathmoveto{\pgfqpoint{0.000000in}{0.000000in}}%
\pgfpathlineto{\pgfqpoint{0.000000in}{-0.027778in}}%
\pgfusepath{stroke,fill}%
}%
\begin{pgfscope}%
\pgfsys@transformshift{2.223163in}{2.521191in}%
\pgfsys@useobject{currentmarker}{}%
\end{pgfscope}%
\end{pgfscope}%
\begin{pgfscope}%
\pgfsetbuttcap%
\pgfsetroundjoin%
\definecolor{currentfill}{rgb}{0.000000,0.000000,0.000000}%
\pgfsetfillcolor{currentfill}%
\pgfsetlinewidth{0.501875pt}%
\definecolor{currentstroke}{rgb}{0.000000,0.000000,0.000000}%
\pgfsetstrokecolor{currentstroke}%
\pgfsetdash{}{0pt}%
\pgfsys@defobject{currentmarker}{\pgfqpoint{0.000000in}{0.000000in}}{\pgfqpoint{0.000000in}{0.027778in}}{%
\pgfpathmoveto{\pgfqpoint{0.000000in}{0.000000in}}%
\pgfpathlineto{\pgfqpoint{0.000000in}{0.027778in}}%
\pgfusepath{stroke,fill}%
}%
\begin{pgfscope}%
\pgfsys@transformshift{2.424734in}{1.121191in}%
\pgfsys@useobject{currentmarker}{}%
\end{pgfscope}%
\end{pgfscope}%
\begin{pgfscope}%
\pgfsetbuttcap%
\pgfsetroundjoin%
\definecolor{currentfill}{rgb}{0.000000,0.000000,0.000000}%
\pgfsetfillcolor{currentfill}%
\pgfsetlinewidth{0.501875pt}%
\definecolor{currentstroke}{rgb}{0.000000,0.000000,0.000000}%
\pgfsetstrokecolor{currentstroke}%
\pgfsetdash{}{0pt}%
\pgfsys@defobject{currentmarker}{\pgfqpoint{0.000000in}{-0.027778in}}{\pgfqpoint{0.000000in}{0.000000in}}{%
\pgfpathmoveto{\pgfqpoint{0.000000in}{0.000000in}}%
\pgfpathlineto{\pgfqpoint{0.000000in}{-0.027778in}}%
\pgfusepath{stroke,fill}%
}%
\begin{pgfscope}%
\pgfsys@transformshift{2.424734in}{2.521191in}%
\pgfsys@useobject{currentmarker}{}%
\end{pgfscope}%
\end{pgfscope}%
\begin{pgfscope}%
\pgfsetbuttcap%
\pgfsetroundjoin%
\definecolor{currentfill}{rgb}{0.000000,0.000000,0.000000}%
\pgfsetfillcolor{currentfill}%
\pgfsetlinewidth{0.501875pt}%
\definecolor{currentstroke}{rgb}{0.000000,0.000000,0.000000}%
\pgfsetstrokecolor{currentstroke}%
\pgfsetdash{}{0pt}%
\pgfsys@defobject{currentmarker}{\pgfqpoint{0.000000in}{0.000000in}}{\pgfqpoint{0.000000in}{0.027778in}}{%
\pgfpathmoveto{\pgfqpoint{0.000000in}{0.000000in}}%
\pgfpathlineto{\pgfqpoint{0.000000in}{0.027778in}}%
\pgfusepath{stroke,fill}%
}%
\begin{pgfscope}%
\pgfsys@transformshift{2.527087in}{1.121191in}%
\pgfsys@useobject{currentmarker}{}%
\end{pgfscope}%
\end{pgfscope}%
\begin{pgfscope}%
\pgfsetbuttcap%
\pgfsetroundjoin%
\definecolor{currentfill}{rgb}{0.000000,0.000000,0.000000}%
\pgfsetfillcolor{currentfill}%
\pgfsetlinewidth{0.501875pt}%
\definecolor{currentstroke}{rgb}{0.000000,0.000000,0.000000}%
\pgfsetstrokecolor{currentstroke}%
\pgfsetdash{}{0pt}%
\pgfsys@defobject{currentmarker}{\pgfqpoint{0.000000in}{-0.027778in}}{\pgfqpoint{0.000000in}{0.000000in}}{%
\pgfpathmoveto{\pgfqpoint{0.000000in}{0.000000in}}%
\pgfpathlineto{\pgfqpoint{0.000000in}{-0.027778in}}%
\pgfusepath{stroke,fill}%
}%
\begin{pgfscope}%
\pgfsys@transformshift{2.527087in}{2.521191in}%
\pgfsys@useobject{currentmarker}{}%
\end{pgfscope}%
\end{pgfscope}%
\begin{pgfscope}%
\pgfsetbuttcap%
\pgfsetroundjoin%
\definecolor{currentfill}{rgb}{0.000000,0.000000,0.000000}%
\pgfsetfillcolor{currentfill}%
\pgfsetlinewidth{0.501875pt}%
\definecolor{currentstroke}{rgb}{0.000000,0.000000,0.000000}%
\pgfsetstrokecolor{currentstroke}%
\pgfsetdash{}{0pt}%
\pgfsys@defobject{currentmarker}{\pgfqpoint{0.000000in}{0.000000in}}{\pgfqpoint{0.000000in}{0.027778in}}{%
\pgfpathmoveto{\pgfqpoint{0.000000in}{0.000000in}}%
\pgfpathlineto{\pgfqpoint{0.000000in}{0.027778in}}%
\pgfusepath{stroke,fill}%
}%
\begin{pgfscope}%
\pgfsys@transformshift{2.599707in}{1.121191in}%
\pgfsys@useobject{currentmarker}{}%
\end{pgfscope}%
\end{pgfscope}%
\begin{pgfscope}%
\pgfsetbuttcap%
\pgfsetroundjoin%
\definecolor{currentfill}{rgb}{0.000000,0.000000,0.000000}%
\pgfsetfillcolor{currentfill}%
\pgfsetlinewidth{0.501875pt}%
\definecolor{currentstroke}{rgb}{0.000000,0.000000,0.000000}%
\pgfsetstrokecolor{currentstroke}%
\pgfsetdash{}{0pt}%
\pgfsys@defobject{currentmarker}{\pgfqpoint{0.000000in}{-0.027778in}}{\pgfqpoint{0.000000in}{0.000000in}}{%
\pgfpathmoveto{\pgfqpoint{0.000000in}{0.000000in}}%
\pgfpathlineto{\pgfqpoint{0.000000in}{-0.027778in}}%
\pgfusepath{stroke,fill}%
}%
\begin{pgfscope}%
\pgfsys@transformshift{2.599707in}{2.521191in}%
\pgfsys@useobject{currentmarker}{}%
\end{pgfscope}%
\end{pgfscope}%
\begin{pgfscope}%
\pgfsetbuttcap%
\pgfsetroundjoin%
\definecolor{currentfill}{rgb}{0.000000,0.000000,0.000000}%
\pgfsetfillcolor{currentfill}%
\pgfsetlinewidth{0.501875pt}%
\definecolor{currentstroke}{rgb}{0.000000,0.000000,0.000000}%
\pgfsetstrokecolor{currentstroke}%
\pgfsetdash{}{0pt}%
\pgfsys@defobject{currentmarker}{\pgfqpoint{0.000000in}{0.000000in}}{\pgfqpoint{0.000000in}{0.027778in}}{%
\pgfpathmoveto{\pgfqpoint{0.000000in}{0.000000in}}%
\pgfpathlineto{\pgfqpoint{0.000000in}{0.027778in}}%
\pgfusepath{stroke,fill}%
}%
\begin{pgfscope}%
\pgfsys@transformshift{2.656036in}{1.121191in}%
\pgfsys@useobject{currentmarker}{}%
\end{pgfscope}%
\end{pgfscope}%
\begin{pgfscope}%
\pgfsetbuttcap%
\pgfsetroundjoin%
\definecolor{currentfill}{rgb}{0.000000,0.000000,0.000000}%
\pgfsetfillcolor{currentfill}%
\pgfsetlinewidth{0.501875pt}%
\definecolor{currentstroke}{rgb}{0.000000,0.000000,0.000000}%
\pgfsetstrokecolor{currentstroke}%
\pgfsetdash{}{0pt}%
\pgfsys@defobject{currentmarker}{\pgfqpoint{0.000000in}{-0.027778in}}{\pgfqpoint{0.000000in}{0.000000in}}{%
\pgfpathmoveto{\pgfqpoint{0.000000in}{0.000000in}}%
\pgfpathlineto{\pgfqpoint{0.000000in}{-0.027778in}}%
\pgfusepath{stroke,fill}%
}%
\begin{pgfscope}%
\pgfsys@transformshift{2.656036in}{2.521191in}%
\pgfsys@useobject{currentmarker}{}%
\end{pgfscope}%
\end{pgfscope}%
\begin{pgfscope}%
\pgfsetbuttcap%
\pgfsetroundjoin%
\definecolor{currentfill}{rgb}{0.000000,0.000000,0.000000}%
\pgfsetfillcolor{currentfill}%
\pgfsetlinewidth{0.501875pt}%
\definecolor{currentstroke}{rgb}{0.000000,0.000000,0.000000}%
\pgfsetstrokecolor{currentstroke}%
\pgfsetdash{}{0pt}%
\pgfsys@defobject{currentmarker}{\pgfqpoint{0.000000in}{0.000000in}}{\pgfqpoint{0.000000in}{0.027778in}}{%
\pgfpathmoveto{\pgfqpoint{0.000000in}{0.000000in}}%
\pgfpathlineto{\pgfqpoint{0.000000in}{0.027778in}}%
\pgfusepath{stroke,fill}%
}%
\begin{pgfscope}%
\pgfsys@transformshift{2.702060in}{1.121191in}%
\pgfsys@useobject{currentmarker}{}%
\end{pgfscope}%
\end{pgfscope}%
\begin{pgfscope}%
\pgfsetbuttcap%
\pgfsetroundjoin%
\definecolor{currentfill}{rgb}{0.000000,0.000000,0.000000}%
\pgfsetfillcolor{currentfill}%
\pgfsetlinewidth{0.501875pt}%
\definecolor{currentstroke}{rgb}{0.000000,0.000000,0.000000}%
\pgfsetstrokecolor{currentstroke}%
\pgfsetdash{}{0pt}%
\pgfsys@defobject{currentmarker}{\pgfqpoint{0.000000in}{-0.027778in}}{\pgfqpoint{0.000000in}{0.000000in}}{%
\pgfpathmoveto{\pgfqpoint{0.000000in}{0.000000in}}%
\pgfpathlineto{\pgfqpoint{0.000000in}{-0.027778in}}%
\pgfusepath{stroke,fill}%
}%
\begin{pgfscope}%
\pgfsys@transformshift{2.702060in}{2.521191in}%
\pgfsys@useobject{currentmarker}{}%
\end{pgfscope}%
\end{pgfscope}%
\begin{pgfscope}%
\pgfsetbuttcap%
\pgfsetroundjoin%
\definecolor{currentfill}{rgb}{0.000000,0.000000,0.000000}%
\pgfsetfillcolor{currentfill}%
\pgfsetlinewidth{0.501875pt}%
\definecolor{currentstroke}{rgb}{0.000000,0.000000,0.000000}%
\pgfsetstrokecolor{currentstroke}%
\pgfsetdash{}{0pt}%
\pgfsys@defobject{currentmarker}{\pgfqpoint{0.000000in}{0.000000in}}{\pgfqpoint{0.000000in}{0.027778in}}{%
\pgfpathmoveto{\pgfqpoint{0.000000in}{0.000000in}}%
\pgfpathlineto{\pgfqpoint{0.000000in}{0.027778in}}%
\pgfusepath{stroke,fill}%
}%
\begin{pgfscope}%
\pgfsys@transformshift{2.740973in}{1.121191in}%
\pgfsys@useobject{currentmarker}{}%
\end{pgfscope}%
\end{pgfscope}%
\begin{pgfscope}%
\pgfsetbuttcap%
\pgfsetroundjoin%
\definecolor{currentfill}{rgb}{0.000000,0.000000,0.000000}%
\pgfsetfillcolor{currentfill}%
\pgfsetlinewidth{0.501875pt}%
\definecolor{currentstroke}{rgb}{0.000000,0.000000,0.000000}%
\pgfsetstrokecolor{currentstroke}%
\pgfsetdash{}{0pt}%
\pgfsys@defobject{currentmarker}{\pgfqpoint{0.000000in}{-0.027778in}}{\pgfqpoint{0.000000in}{0.000000in}}{%
\pgfpathmoveto{\pgfqpoint{0.000000in}{0.000000in}}%
\pgfpathlineto{\pgfqpoint{0.000000in}{-0.027778in}}%
\pgfusepath{stroke,fill}%
}%
\begin{pgfscope}%
\pgfsys@transformshift{2.740973in}{2.521191in}%
\pgfsys@useobject{currentmarker}{}%
\end{pgfscope}%
\end{pgfscope}%
\begin{pgfscope}%
\pgfsetbuttcap%
\pgfsetroundjoin%
\definecolor{currentfill}{rgb}{0.000000,0.000000,0.000000}%
\pgfsetfillcolor{currentfill}%
\pgfsetlinewidth{0.501875pt}%
\definecolor{currentstroke}{rgb}{0.000000,0.000000,0.000000}%
\pgfsetstrokecolor{currentstroke}%
\pgfsetdash{}{0pt}%
\pgfsys@defobject{currentmarker}{\pgfqpoint{0.000000in}{0.000000in}}{\pgfqpoint{0.000000in}{0.027778in}}{%
\pgfpathmoveto{\pgfqpoint{0.000000in}{0.000000in}}%
\pgfpathlineto{\pgfqpoint{0.000000in}{0.027778in}}%
\pgfusepath{stroke,fill}%
}%
\begin{pgfscope}%
\pgfsys@transformshift{2.774681in}{1.121191in}%
\pgfsys@useobject{currentmarker}{}%
\end{pgfscope}%
\end{pgfscope}%
\begin{pgfscope}%
\pgfsetbuttcap%
\pgfsetroundjoin%
\definecolor{currentfill}{rgb}{0.000000,0.000000,0.000000}%
\pgfsetfillcolor{currentfill}%
\pgfsetlinewidth{0.501875pt}%
\definecolor{currentstroke}{rgb}{0.000000,0.000000,0.000000}%
\pgfsetstrokecolor{currentstroke}%
\pgfsetdash{}{0pt}%
\pgfsys@defobject{currentmarker}{\pgfqpoint{0.000000in}{-0.027778in}}{\pgfqpoint{0.000000in}{0.000000in}}{%
\pgfpathmoveto{\pgfqpoint{0.000000in}{0.000000in}}%
\pgfpathlineto{\pgfqpoint{0.000000in}{-0.027778in}}%
\pgfusepath{stroke,fill}%
}%
\begin{pgfscope}%
\pgfsys@transformshift{2.774681in}{2.521191in}%
\pgfsys@useobject{currentmarker}{}%
\end{pgfscope}%
\end{pgfscope}%
\begin{pgfscope}%
\pgfsetbuttcap%
\pgfsetroundjoin%
\definecolor{currentfill}{rgb}{0.000000,0.000000,0.000000}%
\pgfsetfillcolor{currentfill}%
\pgfsetlinewidth{0.501875pt}%
\definecolor{currentstroke}{rgb}{0.000000,0.000000,0.000000}%
\pgfsetstrokecolor{currentstroke}%
\pgfsetdash{}{0pt}%
\pgfsys@defobject{currentmarker}{\pgfqpoint{0.000000in}{0.000000in}}{\pgfqpoint{0.000000in}{0.027778in}}{%
\pgfpathmoveto{\pgfqpoint{0.000000in}{0.000000in}}%
\pgfpathlineto{\pgfqpoint{0.000000in}{0.027778in}}%
\pgfusepath{stroke,fill}%
}%
\begin{pgfscope}%
\pgfsys@transformshift{2.804413in}{1.121191in}%
\pgfsys@useobject{currentmarker}{}%
\end{pgfscope}%
\end{pgfscope}%
\begin{pgfscope}%
\pgfsetbuttcap%
\pgfsetroundjoin%
\definecolor{currentfill}{rgb}{0.000000,0.000000,0.000000}%
\pgfsetfillcolor{currentfill}%
\pgfsetlinewidth{0.501875pt}%
\definecolor{currentstroke}{rgb}{0.000000,0.000000,0.000000}%
\pgfsetstrokecolor{currentstroke}%
\pgfsetdash{}{0pt}%
\pgfsys@defobject{currentmarker}{\pgfqpoint{0.000000in}{-0.027778in}}{\pgfqpoint{0.000000in}{0.000000in}}{%
\pgfpathmoveto{\pgfqpoint{0.000000in}{0.000000in}}%
\pgfpathlineto{\pgfqpoint{0.000000in}{-0.027778in}}%
\pgfusepath{stroke,fill}%
}%
\begin{pgfscope}%
\pgfsys@transformshift{2.804413in}{2.521191in}%
\pgfsys@useobject{currentmarker}{}%
\end{pgfscope}%
\end{pgfscope}%
\begin{pgfscope}%
\pgftext[x=1.668510in,y=0.892267in,,top]{{\rmfamily\fontsize{8.328000}{9.993600}\selectfont Time \(\displaystyle t\) (s)}}%
\end{pgfscope}%
\begin{pgfscope}%
\pgfpathrectangle{\pgfqpoint{0.506010in}{1.121191in}}{\pgfqpoint{2.325000in}{1.400000in}} %
\pgfusepath{clip}%
\pgfsetbuttcap%
\pgfsetroundjoin%
\pgfsetlinewidth{0.501875pt}%
\definecolor{currentstroke}{rgb}{0.000000,0.000000,0.000000}%
\pgfsetstrokecolor{currentstroke}%
\pgfsetdash{{1.000000pt}{3.000000pt}}{0.000000pt}%
\pgfpathmoveto{\pgfqpoint{0.506010in}{1.121191in}}%
\pgfpathlineto{\pgfqpoint{2.831010in}{1.121191in}}%
\pgfusepath{stroke}%
\end{pgfscope}%
\begin{pgfscope}%
\pgfsetbuttcap%
\pgfsetroundjoin%
\definecolor{currentfill}{rgb}{0.000000,0.000000,0.000000}%
\pgfsetfillcolor{currentfill}%
\pgfsetlinewidth{0.501875pt}%
\definecolor{currentstroke}{rgb}{0.000000,0.000000,0.000000}%
\pgfsetstrokecolor{currentstroke}%
\pgfsetdash{}{0pt}%
\pgfsys@defobject{currentmarker}{\pgfqpoint{0.000000in}{0.000000in}}{\pgfqpoint{0.055556in}{0.000000in}}{%
\pgfpathmoveto{\pgfqpoint{0.000000in}{0.000000in}}%
\pgfpathlineto{\pgfqpoint{0.055556in}{0.000000in}}%
\pgfusepath{stroke,fill}%
}%
\begin{pgfscope}%
\pgfsys@transformshift{0.506010in}{1.121191in}%
\pgfsys@useobject{currentmarker}{}%
\end{pgfscope}%
\end{pgfscope}%
\begin{pgfscope}%
\pgfsetbuttcap%
\pgfsetroundjoin%
\definecolor{currentfill}{rgb}{0.000000,0.000000,0.000000}%
\pgfsetfillcolor{currentfill}%
\pgfsetlinewidth{0.501875pt}%
\definecolor{currentstroke}{rgb}{0.000000,0.000000,0.000000}%
\pgfsetstrokecolor{currentstroke}%
\pgfsetdash{}{0pt}%
\pgfsys@defobject{currentmarker}{\pgfqpoint{-0.055556in}{0.000000in}}{\pgfqpoint{0.000000in}{0.000000in}}{%
\pgfpathmoveto{\pgfqpoint{0.000000in}{0.000000in}}%
\pgfpathlineto{\pgfqpoint{-0.055556in}{0.000000in}}%
\pgfusepath{stroke,fill}%
}%
\begin{pgfscope}%
\pgfsys@transformshift{2.831010in}{1.121191in}%
\pgfsys@useobject{currentmarker}{}%
\end{pgfscope}%
\end{pgfscope}%
\begin{pgfscope}%
\pgftext[x=0.450454in,y=1.121191in,right,]{{\rmfamily\fontsize{8.328000}{9.993600}\selectfont \(\displaystyle 0\)}}%
\end{pgfscope}%
\begin{pgfscope}%
\pgfpathrectangle{\pgfqpoint{0.506010in}{1.121191in}}{\pgfqpoint{2.325000in}{1.400000in}} %
\pgfusepath{clip}%
\pgfsetbuttcap%
\pgfsetroundjoin%
\pgfsetlinewidth{0.501875pt}%
\definecolor{currentstroke}{rgb}{0.000000,0.000000,0.000000}%
\pgfsetstrokecolor{currentstroke}%
\pgfsetdash{{1.000000pt}{3.000000pt}}{0.000000pt}%
\pgfpathmoveto{\pgfqpoint{0.506010in}{1.401191in}}%
\pgfpathlineto{\pgfqpoint{2.831010in}{1.401191in}}%
\pgfusepath{stroke}%
\end{pgfscope}%
\begin{pgfscope}%
\pgfsetbuttcap%
\pgfsetroundjoin%
\definecolor{currentfill}{rgb}{0.000000,0.000000,0.000000}%
\pgfsetfillcolor{currentfill}%
\pgfsetlinewidth{0.501875pt}%
\definecolor{currentstroke}{rgb}{0.000000,0.000000,0.000000}%
\pgfsetstrokecolor{currentstroke}%
\pgfsetdash{}{0pt}%
\pgfsys@defobject{currentmarker}{\pgfqpoint{0.000000in}{0.000000in}}{\pgfqpoint{0.055556in}{0.000000in}}{%
\pgfpathmoveto{\pgfqpoint{0.000000in}{0.000000in}}%
\pgfpathlineto{\pgfqpoint{0.055556in}{0.000000in}}%
\pgfusepath{stroke,fill}%
}%
\begin{pgfscope}%
\pgfsys@transformshift{0.506010in}{1.401191in}%
\pgfsys@useobject{currentmarker}{}%
\end{pgfscope}%
\end{pgfscope}%
\begin{pgfscope}%
\pgfsetbuttcap%
\pgfsetroundjoin%
\definecolor{currentfill}{rgb}{0.000000,0.000000,0.000000}%
\pgfsetfillcolor{currentfill}%
\pgfsetlinewidth{0.501875pt}%
\definecolor{currentstroke}{rgb}{0.000000,0.000000,0.000000}%
\pgfsetstrokecolor{currentstroke}%
\pgfsetdash{}{0pt}%
\pgfsys@defobject{currentmarker}{\pgfqpoint{-0.055556in}{0.000000in}}{\pgfqpoint{0.000000in}{0.000000in}}{%
\pgfpathmoveto{\pgfqpoint{0.000000in}{0.000000in}}%
\pgfpathlineto{\pgfqpoint{-0.055556in}{0.000000in}}%
\pgfusepath{stroke,fill}%
}%
\begin{pgfscope}%
\pgfsys@transformshift{2.831010in}{1.401191in}%
\pgfsys@useobject{currentmarker}{}%
\end{pgfscope}%
\end{pgfscope}%
\begin{pgfscope}%
\pgftext[x=0.450454in,y=1.401191in,right,]{{\rmfamily\fontsize{8.328000}{9.993600}\selectfont \(\displaystyle 20\)}}%
\end{pgfscope}%
\begin{pgfscope}%
\pgfpathrectangle{\pgfqpoint{0.506010in}{1.121191in}}{\pgfqpoint{2.325000in}{1.400000in}} %
\pgfusepath{clip}%
\pgfsetbuttcap%
\pgfsetroundjoin%
\pgfsetlinewidth{0.501875pt}%
\definecolor{currentstroke}{rgb}{0.000000,0.000000,0.000000}%
\pgfsetstrokecolor{currentstroke}%
\pgfsetdash{{1.000000pt}{3.000000pt}}{0.000000pt}%
\pgfpathmoveto{\pgfqpoint{0.506010in}{1.681191in}}%
\pgfpathlineto{\pgfqpoint{2.831010in}{1.681191in}}%
\pgfusepath{stroke}%
\end{pgfscope}%
\begin{pgfscope}%
\pgfsetbuttcap%
\pgfsetroundjoin%
\definecolor{currentfill}{rgb}{0.000000,0.000000,0.000000}%
\pgfsetfillcolor{currentfill}%
\pgfsetlinewidth{0.501875pt}%
\definecolor{currentstroke}{rgb}{0.000000,0.000000,0.000000}%
\pgfsetstrokecolor{currentstroke}%
\pgfsetdash{}{0pt}%
\pgfsys@defobject{currentmarker}{\pgfqpoint{0.000000in}{0.000000in}}{\pgfqpoint{0.055556in}{0.000000in}}{%
\pgfpathmoveto{\pgfqpoint{0.000000in}{0.000000in}}%
\pgfpathlineto{\pgfqpoint{0.055556in}{0.000000in}}%
\pgfusepath{stroke,fill}%
}%
\begin{pgfscope}%
\pgfsys@transformshift{0.506010in}{1.681191in}%
\pgfsys@useobject{currentmarker}{}%
\end{pgfscope}%
\end{pgfscope}%
\begin{pgfscope}%
\pgfsetbuttcap%
\pgfsetroundjoin%
\definecolor{currentfill}{rgb}{0.000000,0.000000,0.000000}%
\pgfsetfillcolor{currentfill}%
\pgfsetlinewidth{0.501875pt}%
\definecolor{currentstroke}{rgb}{0.000000,0.000000,0.000000}%
\pgfsetstrokecolor{currentstroke}%
\pgfsetdash{}{0pt}%
\pgfsys@defobject{currentmarker}{\pgfqpoint{-0.055556in}{0.000000in}}{\pgfqpoint{0.000000in}{0.000000in}}{%
\pgfpathmoveto{\pgfqpoint{0.000000in}{0.000000in}}%
\pgfpathlineto{\pgfqpoint{-0.055556in}{0.000000in}}%
\pgfusepath{stroke,fill}%
}%
\begin{pgfscope}%
\pgfsys@transformshift{2.831010in}{1.681191in}%
\pgfsys@useobject{currentmarker}{}%
\end{pgfscope}%
\end{pgfscope}%
\begin{pgfscope}%
\pgftext[x=0.450454in,y=1.681191in,right,]{{\rmfamily\fontsize{8.328000}{9.993600}\selectfont \(\displaystyle 40\)}}%
\end{pgfscope}%
\begin{pgfscope}%
\pgfpathrectangle{\pgfqpoint{0.506010in}{1.121191in}}{\pgfqpoint{2.325000in}{1.400000in}} %
\pgfusepath{clip}%
\pgfsetbuttcap%
\pgfsetroundjoin%
\pgfsetlinewidth{0.501875pt}%
\definecolor{currentstroke}{rgb}{0.000000,0.000000,0.000000}%
\pgfsetstrokecolor{currentstroke}%
\pgfsetdash{{1.000000pt}{3.000000pt}}{0.000000pt}%
\pgfpathmoveto{\pgfqpoint{0.506010in}{1.961191in}}%
\pgfpathlineto{\pgfqpoint{2.831010in}{1.961191in}}%
\pgfusepath{stroke}%
\end{pgfscope}%
\begin{pgfscope}%
\pgfsetbuttcap%
\pgfsetroundjoin%
\definecolor{currentfill}{rgb}{0.000000,0.000000,0.000000}%
\pgfsetfillcolor{currentfill}%
\pgfsetlinewidth{0.501875pt}%
\definecolor{currentstroke}{rgb}{0.000000,0.000000,0.000000}%
\pgfsetstrokecolor{currentstroke}%
\pgfsetdash{}{0pt}%
\pgfsys@defobject{currentmarker}{\pgfqpoint{0.000000in}{0.000000in}}{\pgfqpoint{0.055556in}{0.000000in}}{%
\pgfpathmoveto{\pgfqpoint{0.000000in}{0.000000in}}%
\pgfpathlineto{\pgfqpoint{0.055556in}{0.000000in}}%
\pgfusepath{stroke,fill}%
}%
\begin{pgfscope}%
\pgfsys@transformshift{0.506010in}{1.961191in}%
\pgfsys@useobject{currentmarker}{}%
\end{pgfscope}%
\end{pgfscope}%
\begin{pgfscope}%
\pgfsetbuttcap%
\pgfsetroundjoin%
\definecolor{currentfill}{rgb}{0.000000,0.000000,0.000000}%
\pgfsetfillcolor{currentfill}%
\pgfsetlinewidth{0.501875pt}%
\definecolor{currentstroke}{rgb}{0.000000,0.000000,0.000000}%
\pgfsetstrokecolor{currentstroke}%
\pgfsetdash{}{0pt}%
\pgfsys@defobject{currentmarker}{\pgfqpoint{-0.055556in}{0.000000in}}{\pgfqpoint{0.000000in}{0.000000in}}{%
\pgfpathmoveto{\pgfqpoint{0.000000in}{0.000000in}}%
\pgfpathlineto{\pgfqpoint{-0.055556in}{0.000000in}}%
\pgfusepath{stroke,fill}%
}%
\begin{pgfscope}%
\pgfsys@transformshift{2.831010in}{1.961191in}%
\pgfsys@useobject{currentmarker}{}%
\end{pgfscope}%
\end{pgfscope}%
\begin{pgfscope}%
\pgftext[x=0.450454in,y=1.961191in,right,]{{\rmfamily\fontsize{8.328000}{9.993600}\selectfont \(\displaystyle 60\)}}%
\end{pgfscope}%
\begin{pgfscope}%
\pgfpathrectangle{\pgfqpoint{0.506010in}{1.121191in}}{\pgfqpoint{2.325000in}{1.400000in}} %
\pgfusepath{clip}%
\pgfsetbuttcap%
\pgfsetroundjoin%
\pgfsetlinewidth{0.501875pt}%
\definecolor{currentstroke}{rgb}{0.000000,0.000000,0.000000}%
\pgfsetstrokecolor{currentstroke}%
\pgfsetdash{{1.000000pt}{3.000000pt}}{0.000000pt}%
\pgfpathmoveto{\pgfqpoint{0.506010in}{2.241191in}}%
\pgfpathlineto{\pgfqpoint{2.831010in}{2.241191in}}%
\pgfusepath{stroke}%
\end{pgfscope}%
\begin{pgfscope}%
\pgfsetbuttcap%
\pgfsetroundjoin%
\definecolor{currentfill}{rgb}{0.000000,0.000000,0.000000}%
\pgfsetfillcolor{currentfill}%
\pgfsetlinewidth{0.501875pt}%
\definecolor{currentstroke}{rgb}{0.000000,0.000000,0.000000}%
\pgfsetstrokecolor{currentstroke}%
\pgfsetdash{}{0pt}%
\pgfsys@defobject{currentmarker}{\pgfqpoint{0.000000in}{0.000000in}}{\pgfqpoint{0.055556in}{0.000000in}}{%
\pgfpathmoveto{\pgfqpoint{0.000000in}{0.000000in}}%
\pgfpathlineto{\pgfqpoint{0.055556in}{0.000000in}}%
\pgfusepath{stroke,fill}%
}%
\begin{pgfscope}%
\pgfsys@transformshift{0.506010in}{2.241191in}%
\pgfsys@useobject{currentmarker}{}%
\end{pgfscope}%
\end{pgfscope}%
\begin{pgfscope}%
\pgfsetbuttcap%
\pgfsetroundjoin%
\definecolor{currentfill}{rgb}{0.000000,0.000000,0.000000}%
\pgfsetfillcolor{currentfill}%
\pgfsetlinewidth{0.501875pt}%
\definecolor{currentstroke}{rgb}{0.000000,0.000000,0.000000}%
\pgfsetstrokecolor{currentstroke}%
\pgfsetdash{}{0pt}%
\pgfsys@defobject{currentmarker}{\pgfqpoint{-0.055556in}{0.000000in}}{\pgfqpoint{0.000000in}{0.000000in}}{%
\pgfpathmoveto{\pgfqpoint{0.000000in}{0.000000in}}%
\pgfpathlineto{\pgfqpoint{-0.055556in}{0.000000in}}%
\pgfusepath{stroke,fill}%
}%
\begin{pgfscope}%
\pgfsys@transformshift{2.831010in}{2.241191in}%
\pgfsys@useobject{currentmarker}{}%
\end{pgfscope}%
\end{pgfscope}%
\begin{pgfscope}%
\pgftext[x=0.450454in,y=2.241191in,right,]{{\rmfamily\fontsize{8.328000}{9.993600}\selectfont \(\displaystyle 80\)}}%
\end{pgfscope}%
\begin{pgfscope}%
\pgfpathrectangle{\pgfqpoint{0.506010in}{1.121191in}}{\pgfqpoint{2.325000in}{1.400000in}} %
\pgfusepath{clip}%
\pgfsetbuttcap%
\pgfsetroundjoin%
\pgfsetlinewidth{0.501875pt}%
\definecolor{currentstroke}{rgb}{0.000000,0.000000,0.000000}%
\pgfsetstrokecolor{currentstroke}%
\pgfsetdash{{1.000000pt}{3.000000pt}}{0.000000pt}%
\pgfpathmoveto{\pgfqpoint{0.506010in}{2.521191in}}%
\pgfpathlineto{\pgfqpoint{2.831010in}{2.521191in}}%
\pgfusepath{stroke}%
\end{pgfscope}%
\begin{pgfscope}%
\pgfsetbuttcap%
\pgfsetroundjoin%
\definecolor{currentfill}{rgb}{0.000000,0.000000,0.000000}%
\pgfsetfillcolor{currentfill}%
\pgfsetlinewidth{0.501875pt}%
\definecolor{currentstroke}{rgb}{0.000000,0.000000,0.000000}%
\pgfsetstrokecolor{currentstroke}%
\pgfsetdash{}{0pt}%
\pgfsys@defobject{currentmarker}{\pgfqpoint{0.000000in}{0.000000in}}{\pgfqpoint{0.055556in}{0.000000in}}{%
\pgfpathmoveto{\pgfqpoint{0.000000in}{0.000000in}}%
\pgfpathlineto{\pgfqpoint{0.055556in}{0.000000in}}%
\pgfusepath{stroke,fill}%
}%
\begin{pgfscope}%
\pgfsys@transformshift{0.506010in}{2.521191in}%
\pgfsys@useobject{currentmarker}{}%
\end{pgfscope}%
\end{pgfscope}%
\begin{pgfscope}%
\pgfsetbuttcap%
\pgfsetroundjoin%
\definecolor{currentfill}{rgb}{0.000000,0.000000,0.000000}%
\pgfsetfillcolor{currentfill}%
\pgfsetlinewidth{0.501875pt}%
\definecolor{currentstroke}{rgb}{0.000000,0.000000,0.000000}%
\pgfsetstrokecolor{currentstroke}%
\pgfsetdash{}{0pt}%
\pgfsys@defobject{currentmarker}{\pgfqpoint{-0.055556in}{0.000000in}}{\pgfqpoint{0.000000in}{0.000000in}}{%
\pgfpathmoveto{\pgfqpoint{0.000000in}{0.000000in}}%
\pgfpathlineto{\pgfqpoint{-0.055556in}{0.000000in}}%
\pgfusepath{stroke,fill}%
}%
\begin{pgfscope}%
\pgfsys@transformshift{2.831010in}{2.521191in}%
\pgfsys@useobject{currentmarker}{}%
\end{pgfscope}%
\end{pgfscope}%
\begin{pgfscope}%
\pgftext[x=0.450454in,y=2.521191in,right,]{{\rmfamily\fontsize{8.328000}{9.993600}\selectfont \(\displaystyle 100\)}}%
\end{pgfscope}%
\begin{pgfscope}%
\pgftext[x=0.203924in,y=1.821191in,,bottom,rotate=90.000000]{{\rmfamily\fontsize{8.328000}{9.993600}\selectfont Instances solved}}%
\end{pgfscope}%
\begin{pgfscope}%
\pgfsetbuttcap%
\pgfsetroundjoin%
\pgfsetlinewidth{1.003750pt}%
\definecolor{currentstroke}{rgb}{0.000000,0.000000,0.000000}%
\pgfsetstrokecolor{currentstroke}%
\pgfsetdash{}{0pt}%
\pgfpathmoveto{\pgfqpoint{0.506010in}{2.521191in}}%
\pgfpathlineto{\pgfqpoint{2.831010in}{2.521191in}}%
\pgfusepath{stroke}%
\end{pgfscope}%
\begin{pgfscope}%
\pgfsetbuttcap%
\pgfsetroundjoin%
\pgfsetlinewidth{1.003750pt}%
\definecolor{currentstroke}{rgb}{0.000000,0.000000,0.000000}%
\pgfsetstrokecolor{currentstroke}%
\pgfsetdash{}{0pt}%
\pgfpathmoveto{\pgfqpoint{2.831010in}{1.121191in}}%
\pgfpathlineto{\pgfqpoint{2.831010in}{2.521191in}}%
\pgfusepath{stroke}%
\end{pgfscope}%
\begin{pgfscope}%
\pgfsetbuttcap%
\pgfsetroundjoin%
\pgfsetlinewidth{1.003750pt}%
\definecolor{currentstroke}{rgb}{0.000000,0.000000,0.000000}%
\pgfsetstrokecolor{currentstroke}%
\pgfsetdash{}{0pt}%
\pgfpathmoveto{\pgfqpoint{0.506010in}{1.121191in}}%
\pgfpathlineto{\pgfqpoint{2.831010in}{1.121191in}}%
\pgfusepath{stroke}%
\end{pgfscope}%
\begin{pgfscope}%
\pgfsetbuttcap%
\pgfsetroundjoin%
\pgfsetlinewidth{1.003750pt}%
\definecolor{currentstroke}{rgb}{0.000000,0.000000,0.000000}%
\pgfsetstrokecolor{currentstroke}%
\pgfsetdash{}{0pt}%
\pgfpathmoveto{\pgfqpoint{0.506010in}{1.121191in}}%
\pgfpathlineto{\pgfqpoint{0.506010in}{2.521191in}}%
\pgfusepath{stroke}%
\end{pgfscope}%
\begin{pgfscope}%
\pgftext[x=1.668510in,y=2.590635in,,base]{{\rmfamily\fontsize{8.328000}{9.993600}\selectfont Instances solved over time}}%
\end{pgfscope}%
\begin{pgfscope}%
\pgfsetbuttcap%
\pgfsetroundjoin%
\definecolor{currentfill}{rgb}{1.000000,1.000000,1.000000}%
\pgfsetfillcolor{currentfill}%
\pgfsetlinewidth{1.003750pt}%
\definecolor{currentstroke}{rgb}{0.000000,0.000000,0.000000}%
\pgfsetstrokecolor{currentstroke}%
\pgfsetdash{}{0pt}%
\pgfpathmoveto{\pgfqpoint{0.387681in}{0.100000in}}%
\pgfpathlineto{\pgfqpoint{2.949339in}{0.100000in}}%
\pgfpathlineto{\pgfqpoint{2.949339in}{0.462381in}}%
\pgfpathlineto{\pgfqpoint{0.387681in}{0.462381in}}%
\pgfpathlineto{\pgfqpoint{0.387681in}{0.100000in}}%
\pgfpathclose%
\pgfusepath{stroke,fill}%
\end{pgfscope}%
\begin{pgfscope}%
\pgfsetrectcap%
\pgfsetroundjoin%
\pgfsetlinewidth{2.007500pt}%
\definecolor{currentstroke}{rgb}{0.000000,0.500000,0.000000}%
\pgfsetstrokecolor{currentstroke}%
\pgfsetdash{}{0pt}%
\pgfpathmoveto{\pgfqpoint{0.468647in}{0.375631in}}%
\pgfpathlineto{\pgfqpoint{0.630581in}{0.375631in}}%
\pgfusepath{stroke}%
\end{pgfscope}%
\begin{pgfscope}%
\pgftext[x=0.757814in,y=0.335148in,left,base]{{\rmfamily\fontsize{8.328000}{9.993600}\selectfont MoMC}}%
\end{pgfscope}%
\begin{pgfscope}%
\pgfsetrectcap%
\pgfsetroundjoin%
\pgfsetlinewidth{2.007500pt}%
\definecolor{currentstroke}{rgb}{1.000000,0.000000,0.000000}%
\pgfsetstrokecolor{currentstroke}%
\pgfsetdash{}{0pt}%
\pgfpathmoveto{\pgfqpoint{0.468647in}{0.211791in}}%
\pgfpathlineto{\pgfqpoint{0.630581in}{0.211791in}}%
\pgfusepath{stroke}%
\end{pgfscope}%
\begin{pgfscope}%
\pgftext[x=0.757814in,y=0.171307in,left,base]{{\rmfamily\fontsize{8.328000}{9.993600}\selectfont RMoMC}}%
\end{pgfscope}%
\begin{pgfscope}%
\pgfsetrectcap%
\pgfsetroundjoin%
\pgfsetlinewidth{2.007500pt}%
\definecolor{currentstroke}{rgb}{0.000000,0.000000,1.000000}%
\pgfsetstrokecolor{currentstroke}%
\pgfsetdash{}{0pt}%
\pgfpathmoveto{\pgfqpoint{1.441631in}{0.375631in}}%
\pgfpathlineto{\pgfqpoint{1.603564in}{0.375631in}}%
\pgfusepath{stroke}%
\end{pgfscope}%
\begin{pgfscope}%
\pgftext[x=1.730797in,y=0.335148in,left,base]{{\rmfamily\fontsize{8.328000}{9.993600}\selectfont LSBnR}}%
\end{pgfscope}%
\begin{pgfscope}%
\pgfsetrectcap%
\pgfsetroundjoin%
\pgfsetlinewidth{2.007500pt}%
\definecolor{currentstroke}{rgb}{0.750000,0.750000,0.000000}%
\pgfsetstrokecolor{currentstroke}%
\pgfsetdash{}{0pt}%
\pgfpathmoveto{\pgfqpoint{1.441631in}{0.211791in}}%
\pgfpathlineto{\pgfqpoint{1.603564in}{0.211791in}}%
\pgfusepath{stroke}%
\end{pgfscope}%
\begin{pgfscope}%
\pgftext[x=1.730797in,y=0.171307in,left,base]{{\rmfamily\fontsize{8.328000}{9.993600}\selectfont BnR}}%
\end{pgfscope}%
\begin{pgfscope}%
\pgfsetrectcap%
\pgfsetroundjoin%
\pgfsetlinewidth{2.007500pt}%
\definecolor{currentstroke}{rgb}{0.000000,0.750000,0.750000}%
\pgfsetstrokecolor{currentstroke}%
\pgfsetdash{}{0pt}%
\pgfpathmoveto{\pgfqpoint{2.343943in}{0.375631in}}%
\pgfpathlineto{\pgfqpoint{2.505877in}{0.375631in}}%
\pgfusepath{stroke}%
\end{pgfscope}%
\begin{pgfscope}%
\pgftext[x=2.633110in,y=0.335148in,left,base]{{\rmfamily\fontsize{8.328000}{9.993600}\selectfont FullA}}%
\end{pgfscope}%
\end{pgfpicture}%
\makeatother%
\endgroup%

  \caption{Number of instances solved over time by each algorithm. At each time step $t$, we count each instance solved by the algorithm in at most $t$ seconds.} 
  \label{fig:solution_time}
\end{figure}

In addition to the 100 public instances, the PACE Implementation Challenge tests all submissions on 100 private instances, which are not yet available for additional experiments.
%\footnote{However, the results are published at \url{https://pacechallenge.org/2019/} and \url{https://www.optil.io/optilion/problem/3155}.} 
On the private instances, our full algorithm solved 87 of the 100 instances, which is 10 more instances than the second-place submission (\textsf{peaty}~\cite{james_trimble_2019_3082356}, solving 77), and 11 more than the third-place submission (\textsf{bogdan}~\cite{zbogdan_2019_3228802}), solving 76). Our solver dominates these other solvers: with the exception of one graph, our algorithm solves all instances that \textsf{peaty} and \textsf{bogdan} can solve combined. 

We briefly describe these two solvers. The \textsf{peaty} solver uses reductions to compute a problem kernel of the input followed by an unpublished maximum weight clique solver on the complement of each of the connected components of the kernel to assemble a solution. The clique solver is similar to MaxCLQ by Li and Quan~\cite{DBLP:conf/aaai/LiQ10}, but is more general. Local search is used to obtain an initial solution. On the other hand, \textsf{bogdan} implemented a small suite of simple reductions (including vertex folding, isolated clique removal, and degree-one removal) together with a recent maximum clique solver by Szab\'o and Zavalnij~\cite{szabo2018different}. 

Lastly, we note that our choice of using MoMC as our chosen branch-and-bound solver is significant on the private instances. Eight instances solved exclusively by our solver are solved in Phase 5, where MoMC is run until the end of the challenge time limit.

\iffalse %not helpful
\begin{figure}
    \centering
    %% Creator: Matplotlib, PGF backend
%%
%% To include the figure in your LaTeX document, write
%%   \input{<filename>.pgf}
%%
%% Make sure the required packages are loaded in your preamble
%%   \usepackage{pgf}
%%
%% Figures using additional raster images can only be included by \input if
%% they are in the same directory as the main LaTeX file. For loading figures
%% from other directories you can use the `import` package
%%   \usepackage{import}
%% and then include the figures with
%%   \import{<path to file>}{<filename>.pgf}
%%
%% Matplotlib used the following preamble
%%   \renewcommand{\sfdefault}{phv}
%%   \renewcommand{\rmdefault}{ptm}
%%   \renewcommand{\ttdefault}{pcr}
%%   \normalfont\selectfont
%%
\begingroup%
\makeatletter%
\begin{pgfpicture}%
\pgfpathrectangle{\pgfpointorigin}{\pgfqpoint{3.049339in}{2.769518in}}%
\pgfusepath{use as bounding box}%
\begin{pgfscope}%
\pgfsetbuttcap%
\pgfsetroundjoin%
\definecolor{currentfill}{rgb}{1.000000,1.000000,1.000000}%
\pgfsetfillcolor{currentfill}%
\pgfsetlinewidth{0.000000pt}%
\definecolor{currentstroke}{rgb}{1.000000,1.000000,1.000000}%
\pgfsetstrokecolor{currentstroke}%
\pgfsetdash{}{0pt}%
\pgfpathmoveto{\pgfqpoint{0.000000in}{0.000000in}}%
\pgfpathlineto{\pgfqpoint{3.049339in}{0.000000in}}%
\pgfpathlineto{\pgfqpoint{3.049339in}{2.769518in}}%
\pgfpathlineto{\pgfqpoint{0.000000in}{2.769518in}}%
\pgfpathclose%
\pgfusepath{fill}%
\end{pgfscope}%
\begin{pgfscope}%
\pgfsetbuttcap%
\pgfsetroundjoin%
\definecolor{currentfill}{rgb}{1.000000,1.000000,1.000000}%
\pgfsetfillcolor{currentfill}%
\pgfsetlinewidth{0.000000pt}%
\definecolor{currentstroke}{rgb}{0.000000,0.000000,0.000000}%
\pgfsetstrokecolor{currentstroke}%
\pgfsetstrokeopacity{0.000000}%
\pgfsetdash{}{0pt}%
\pgfpathmoveto{\pgfqpoint{0.506010in}{1.121191in}}%
\pgfpathlineto{\pgfqpoint{2.831010in}{1.121191in}}%
\pgfpathlineto{\pgfqpoint{2.831010in}{2.521191in}}%
\pgfpathlineto{\pgfqpoint{0.506010in}{2.521191in}}%
\pgfpathclose%
\pgfusepath{fill}%
\end{pgfscope}%
\begin{pgfscope}%
\pgfpathrectangle{\pgfqpoint{0.506010in}{1.121191in}}{\pgfqpoint{2.325000in}{1.400000in}} %
\pgfusepath{clip}%
\pgfsetbuttcap%
\pgfsetroundjoin%
\definecolor{currentfill}{rgb}{0.000000,0.500000,0.000000}%
\pgfsetfillcolor{currentfill}%
\pgfsetlinewidth{1.003750pt}%
\definecolor{currentstroke}{rgb}{0.000000,0.500000,0.000000}%
\pgfsetstrokecolor{currentstroke}%
\pgfsetdash{}{0pt}%
\pgfpathmoveto{\pgfqpoint{1.151065in}{1.102590in}}%
\pgfpathcurveto{\pgfqpoint{1.156889in}{1.102590in}}{\pgfqpoint{1.162476in}{1.104904in}}{\pgfqpoint{1.166594in}{1.109022in}}%
\pgfpathcurveto{\pgfqpoint{1.170712in}{1.113141in}}{\pgfqpoint{1.173026in}{1.118727in}}{\pgfqpoint{1.173026in}{1.124551in}}%
\pgfpathcurveto{\pgfqpoint{1.173026in}{1.130375in}}{\pgfqpoint{1.170712in}{1.135961in}}{\pgfqpoint{1.166594in}{1.140079in}}%
\pgfpathcurveto{\pgfqpoint{1.162476in}{1.144197in}}{\pgfqpoint{1.156889in}{1.146511in}}{\pgfqpoint{1.151065in}{1.146511in}}%
\pgfpathcurveto{\pgfqpoint{1.145241in}{1.146511in}}{\pgfqpoint{1.139655in}{1.144197in}}{\pgfqpoint{1.135537in}{1.140079in}}%
\pgfpathcurveto{\pgfqpoint{1.131419in}{1.135961in}}{\pgfqpoint{1.129105in}{1.130375in}}{\pgfqpoint{1.129105in}{1.124551in}}%
\pgfpathcurveto{\pgfqpoint{1.129105in}{1.118727in}}{\pgfqpoint{1.131419in}{1.113141in}}{\pgfqpoint{1.135537in}{1.109022in}}%
\pgfpathcurveto{\pgfqpoint{1.139655in}{1.104904in}}{\pgfqpoint{1.145241in}{1.102590in}}{\pgfqpoint{1.151065in}{1.102590in}}%
\pgfpathclose%
\pgfusepath{stroke,fill}%
\end{pgfscope}%
\begin{pgfscope}%
\pgfpathrectangle{\pgfqpoint{0.506010in}{1.121191in}}{\pgfqpoint{2.325000in}{1.400000in}} %
\pgfusepath{clip}%
\pgfsetbuttcap%
\pgfsetroundjoin%
\definecolor{currentfill}{rgb}{0.000000,0.500000,0.000000}%
\pgfsetfillcolor{currentfill}%
\pgfsetlinewidth{1.003750pt}%
\definecolor{currentstroke}{rgb}{0.000000,0.500000,0.000000}%
\pgfsetstrokecolor{currentstroke}%
\pgfsetdash{}{0pt}%
\pgfpathmoveto{\pgfqpoint{0.772689in}{1.102842in}}%
\pgfpathcurveto{\pgfqpoint{0.778513in}{1.102842in}}{\pgfqpoint{0.784099in}{1.105156in}}{\pgfqpoint{0.788217in}{1.109274in}}%
\pgfpathcurveto{\pgfqpoint{0.792336in}{1.113393in}}{\pgfqpoint{0.794650in}{1.118979in}}{\pgfqpoint{0.794650in}{1.124803in}}%
\pgfpathcurveto{\pgfqpoint{0.794650in}{1.130627in}}{\pgfqpoint{0.792336in}{1.136213in}}{\pgfqpoint{0.788217in}{1.140331in}}%
\pgfpathcurveto{\pgfqpoint{0.784099in}{1.144449in}}{\pgfqpoint{0.778513in}{1.146763in}}{\pgfqpoint{0.772689in}{1.146763in}}%
\pgfpathcurveto{\pgfqpoint{0.766865in}{1.146763in}}{\pgfqpoint{0.761279in}{1.144449in}}{\pgfqpoint{0.757161in}{1.140331in}}%
\pgfpathcurveto{\pgfqpoint{0.753043in}{1.136213in}}{\pgfqpoint{0.750729in}{1.130627in}}{\pgfqpoint{0.750729in}{1.124803in}}%
\pgfpathcurveto{\pgfqpoint{0.750729in}{1.118979in}}{\pgfqpoint{0.753043in}{1.113393in}}{\pgfqpoint{0.757161in}{1.109274in}}%
\pgfpathcurveto{\pgfqpoint{0.761279in}{1.105156in}}{\pgfqpoint{0.766865in}{1.102842in}}{\pgfqpoint{0.772689in}{1.102842in}}%
\pgfpathclose%
\pgfusepath{stroke,fill}%
\end{pgfscope}%
\begin{pgfscope}%
\pgfpathrectangle{\pgfqpoint{0.506010in}{1.121191in}}{\pgfqpoint{2.325000in}{1.400000in}} %
\pgfusepath{clip}%
\pgfsetbuttcap%
\pgfsetroundjoin%
\definecolor{currentfill}{rgb}{0.000000,0.500000,0.000000}%
\pgfsetfillcolor{currentfill}%
\pgfsetlinewidth{1.003750pt}%
\definecolor{currentstroke}{rgb}{0.000000,0.500000,0.000000}%
\pgfsetstrokecolor{currentstroke}%
\pgfsetdash{}{0pt}%
\pgfpathmoveto{\pgfqpoint{1.222833in}{1.103458in}}%
\pgfpathcurveto{\pgfqpoint{1.228657in}{1.103458in}}{\pgfqpoint{1.234243in}{1.105772in}}{\pgfqpoint{1.238361in}{1.109890in}}%
\pgfpathcurveto{\pgfqpoint{1.242479in}{1.114009in}}{\pgfqpoint{1.244793in}{1.119595in}}{\pgfqpoint{1.244793in}{1.125419in}}%
\pgfpathcurveto{\pgfqpoint{1.244793in}{1.131243in}}{\pgfqpoint{1.242479in}{1.136829in}}{\pgfqpoint{1.238361in}{1.140947in}}%
\pgfpathcurveto{\pgfqpoint{1.234243in}{1.145065in}}{\pgfqpoint{1.228657in}{1.147379in}}{\pgfqpoint{1.222833in}{1.147379in}}%
\pgfpathcurveto{\pgfqpoint{1.217009in}{1.147379in}}{\pgfqpoint{1.211423in}{1.145065in}}{\pgfqpoint{1.207305in}{1.140947in}}%
\pgfpathcurveto{\pgfqpoint{1.203187in}{1.136829in}}{\pgfqpoint{1.200873in}{1.131243in}}{\pgfqpoint{1.200873in}{1.125419in}}%
\pgfpathcurveto{\pgfqpoint{1.200873in}{1.119595in}}{\pgfqpoint{1.203187in}{1.114009in}}{\pgfqpoint{1.207305in}{1.109890in}}%
\pgfpathcurveto{\pgfqpoint{1.211423in}{1.105772in}}{\pgfqpoint{1.217009in}{1.103458in}}{\pgfqpoint{1.222833in}{1.103458in}}%
\pgfpathclose%
\pgfusepath{stroke,fill}%
\end{pgfscope}%
\begin{pgfscope}%
\pgfpathrectangle{\pgfqpoint{0.506010in}{1.121191in}}{\pgfqpoint{2.325000in}{1.400000in}} %
\pgfusepath{clip}%
\pgfsetbuttcap%
\pgfsetroundjoin%
\definecolor{currentfill}{rgb}{0.000000,0.500000,0.000000}%
\pgfsetfillcolor{currentfill}%
\pgfsetlinewidth{1.003750pt}%
\definecolor{currentstroke}{rgb}{0.000000,0.500000,0.000000}%
\pgfsetstrokecolor{currentstroke}%
\pgfsetdash{}{0pt}%
\pgfpathmoveto{\pgfqpoint{0.980706in}{1.103598in}}%
\pgfpathcurveto{\pgfqpoint{0.986530in}{1.103598in}}{\pgfqpoint{0.992116in}{1.105912in}}{\pgfqpoint{0.996234in}{1.110030in}}%
\pgfpathcurveto{\pgfqpoint{1.000352in}{1.114149in}}{\pgfqpoint{1.002666in}{1.119735in}}{\pgfqpoint{1.002666in}{1.125559in}}%
\pgfpathcurveto{\pgfqpoint{1.002666in}{1.131383in}}{\pgfqpoint{1.000352in}{1.136969in}}{\pgfqpoint{0.996234in}{1.141087in}}%
\pgfpathcurveto{\pgfqpoint{0.992116in}{1.145205in}}{\pgfqpoint{0.986530in}{1.147519in}}{\pgfqpoint{0.980706in}{1.147519in}}%
\pgfpathcurveto{\pgfqpoint{0.974882in}{1.147519in}}{\pgfqpoint{0.969296in}{1.145205in}}{\pgfqpoint{0.965177in}{1.141087in}}%
\pgfpathcurveto{\pgfqpoint{0.961059in}{1.136969in}}{\pgfqpoint{0.958745in}{1.131383in}}{\pgfqpoint{0.958745in}{1.125559in}}%
\pgfpathcurveto{\pgfqpoint{0.958745in}{1.119735in}}{\pgfqpoint{0.961059in}{1.114149in}}{\pgfqpoint{0.965177in}{1.110030in}}%
\pgfpathcurveto{\pgfqpoint{0.969296in}{1.105912in}}{\pgfqpoint{0.974882in}{1.103598in}}{\pgfqpoint{0.980706in}{1.103598in}}%
\pgfpathclose%
\pgfusepath{stroke,fill}%
\end{pgfscope}%
\begin{pgfscope}%
\pgfpathrectangle{\pgfqpoint{0.506010in}{1.121191in}}{\pgfqpoint{2.325000in}{1.400000in}} %
\pgfusepath{clip}%
\pgfsetbuttcap%
\pgfsetroundjoin%
\definecolor{currentfill}{rgb}{0.000000,0.500000,0.000000}%
\pgfsetfillcolor{currentfill}%
\pgfsetlinewidth{1.003750pt}%
\definecolor{currentstroke}{rgb}{0.000000,0.500000,0.000000}%
\pgfsetstrokecolor{currentstroke}%
\pgfsetdash{}{0pt}%
\pgfpathmoveto{\pgfqpoint{0.968203in}{1.103878in}}%
\pgfpathcurveto{\pgfqpoint{0.974027in}{1.103878in}}{\pgfqpoint{0.979613in}{1.106192in}}{\pgfqpoint{0.983731in}{1.110310in}}%
\pgfpathcurveto{\pgfqpoint{0.987849in}{1.114429in}}{\pgfqpoint{0.990163in}{1.120015in}}{\pgfqpoint{0.990163in}{1.125839in}}%
\pgfpathcurveto{\pgfqpoint{0.990163in}{1.131663in}}{\pgfqpoint{0.987849in}{1.137249in}}{\pgfqpoint{0.983731in}{1.141367in}}%
\pgfpathcurveto{\pgfqpoint{0.979613in}{1.145485in}}{\pgfqpoint{0.974027in}{1.147799in}}{\pgfqpoint{0.968203in}{1.147799in}}%
\pgfpathcurveto{\pgfqpoint{0.962379in}{1.147799in}}{\pgfqpoint{0.956793in}{1.145485in}}{\pgfqpoint{0.952675in}{1.141367in}}%
\pgfpathcurveto{\pgfqpoint{0.948557in}{1.137249in}}{\pgfqpoint{0.946243in}{1.131663in}}{\pgfqpoint{0.946243in}{1.125839in}}%
\pgfpathcurveto{\pgfqpoint{0.946243in}{1.120015in}}{\pgfqpoint{0.948557in}{1.114429in}}{\pgfqpoint{0.952675in}{1.110310in}}%
\pgfpathcurveto{\pgfqpoint{0.956793in}{1.106192in}}{\pgfqpoint{0.962379in}{1.103878in}}{\pgfqpoint{0.968203in}{1.103878in}}%
\pgfpathclose%
\pgfusepath{stroke,fill}%
\end{pgfscope}%
\begin{pgfscope}%
\pgfpathrectangle{\pgfqpoint{0.506010in}{1.121191in}}{\pgfqpoint{2.325000in}{1.400000in}} %
\pgfusepath{clip}%
\pgfsetbuttcap%
\pgfsetroundjoin%
\definecolor{currentfill}{rgb}{0.000000,0.500000,0.000000}%
\pgfsetfillcolor{currentfill}%
\pgfsetlinewidth{1.003750pt}%
\definecolor{currentstroke}{rgb}{0.000000,0.500000,0.000000}%
\pgfsetstrokecolor{currentstroke}%
\pgfsetdash{}{0pt}%
\pgfpathmoveto{\pgfqpoint{1.097148in}{1.103878in}}%
\pgfpathcurveto{\pgfqpoint{1.102972in}{1.103878in}}{\pgfqpoint{1.108558in}{1.106192in}}{\pgfqpoint{1.112677in}{1.110310in}}%
\pgfpathcurveto{\pgfqpoint{1.116795in}{1.114429in}}{\pgfqpoint{1.119109in}{1.120015in}}{\pgfqpoint{1.119109in}{1.125839in}}%
\pgfpathcurveto{\pgfqpoint{1.119109in}{1.131663in}}{\pgfqpoint{1.116795in}{1.137249in}}{\pgfqpoint{1.112677in}{1.141367in}}%
\pgfpathcurveto{\pgfqpoint{1.108558in}{1.145485in}}{\pgfqpoint{1.102972in}{1.147799in}}{\pgfqpoint{1.097148in}{1.147799in}}%
\pgfpathcurveto{\pgfqpoint{1.091324in}{1.147799in}}{\pgfqpoint{1.085738in}{1.145485in}}{\pgfqpoint{1.081620in}{1.141367in}}%
\pgfpathcurveto{\pgfqpoint{1.077502in}{1.137249in}}{\pgfqpoint{1.075188in}{1.131663in}}{\pgfqpoint{1.075188in}{1.125839in}}%
\pgfpathcurveto{\pgfqpoint{1.075188in}{1.120015in}}{\pgfqpoint{1.077502in}{1.114429in}}{\pgfqpoint{1.081620in}{1.110310in}}%
\pgfpathcurveto{\pgfqpoint{1.085738in}{1.106192in}}{\pgfqpoint{1.091324in}{1.103878in}}{\pgfqpoint{1.097148in}{1.103878in}}%
\pgfpathclose%
\pgfusepath{stroke,fill}%
\end{pgfscope}%
\begin{pgfscope}%
\pgfpathrectangle{\pgfqpoint{0.506010in}{1.121191in}}{\pgfqpoint{2.325000in}{1.400000in}} %
\pgfusepath{clip}%
\pgfsetbuttcap%
\pgfsetroundjoin%
\definecolor{currentfill}{rgb}{0.000000,0.500000,0.000000}%
\pgfsetfillcolor{currentfill}%
\pgfsetlinewidth{1.003750pt}%
\definecolor{currentstroke}{rgb}{0.000000,0.500000,0.000000}%
\pgfsetstrokecolor{currentstroke}%
\pgfsetdash{}{0pt}%
\pgfpathmoveto{\pgfqpoint{0.813731in}{1.103906in}}%
\pgfpathcurveto{\pgfqpoint{0.819555in}{1.103906in}}{\pgfqpoint{0.825141in}{1.106220in}}{\pgfqpoint{0.829259in}{1.110338in}}%
\pgfpathcurveto{\pgfqpoint{0.833378in}{1.114457in}}{\pgfqpoint{0.835691in}{1.120043in}}{\pgfqpoint{0.835691in}{1.125867in}}%
\pgfpathcurveto{\pgfqpoint{0.835691in}{1.131691in}}{\pgfqpoint{0.833378in}{1.137277in}}{\pgfqpoint{0.829259in}{1.141395in}}%
\pgfpathcurveto{\pgfqpoint{0.825141in}{1.145513in}}{\pgfqpoint{0.819555in}{1.147827in}}{\pgfqpoint{0.813731in}{1.147827in}}%
\pgfpathcurveto{\pgfqpoint{0.807907in}{1.147827in}}{\pgfqpoint{0.802321in}{1.145513in}}{\pgfqpoint{0.798203in}{1.141395in}}%
\pgfpathcurveto{\pgfqpoint{0.794085in}{1.137277in}}{\pgfqpoint{0.791771in}{1.131691in}}{\pgfqpoint{0.791771in}{1.125867in}}%
\pgfpathcurveto{\pgfqpoint{0.791771in}{1.120043in}}{\pgfqpoint{0.794085in}{1.114457in}}{\pgfqpoint{0.798203in}{1.110338in}}%
\pgfpathcurveto{\pgfqpoint{0.802321in}{1.106220in}}{\pgfqpoint{0.807907in}{1.103906in}}{\pgfqpoint{0.813731in}{1.103906in}}%
\pgfpathclose%
\pgfusepath{stroke,fill}%
\end{pgfscope}%
\begin{pgfscope}%
\pgfpathrectangle{\pgfqpoint{0.506010in}{1.121191in}}{\pgfqpoint{2.325000in}{1.400000in}} %
\pgfusepath{clip}%
\pgfsetbuttcap%
\pgfsetroundjoin%
\definecolor{currentfill}{rgb}{0.000000,0.500000,0.000000}%
\pgfsetfillcolor{currentfill}%
\pgfsetlinewidth{1.003750pt}%
\definecolor{currentstroke}{rgb}{0.000000,0.500000,0.000000}%
\pgfsetstrokecolor{currentstroke}%
\pgfsetdash{}{0pt}%
\pgfpathmoveto{\pgfqpoint{1.308740in}{1.104018in}}%
\pgfpathcurveto{\pgfqpoint{1.314564in}{1.104018in}}{\pgfqpoint{1.320150in}{1.106332in}}{\pgfqpoint{1.324269in}{1.110450in}}%
\pgfpathcurveto{\pgfqpoint{1.328387in}{1.114569in}}{\pgfqpoint{1.330701in}{1.120155in}}{\pgfqpoint{1.330701in}{1.125979in}}%
\pgfpathcurveto{\pgfqpoint{1.330701in}{1.131803in}}{\pgfqpoint{1.328387in}{1.137389in}}{\pgfqpoint{1.324269in}{1.141507in}}%
\pgfpathcurveto{\pgfqpoint{1.320150in}{1.145625in}}{\pgfqpoint{1.314564in}{1.147939in}}{\pgfqpoint{1.308740in}{1.147939in}}%
\pgfpathcurveto{\pgfqpoint{1.302916in}{1.147939in}}{\pgfqpoint{1.297330in}{1.145625in}}{\pgfqpoint{1.293212in}{1.141507in}}%
\pgfpathcurveto{\pgfqpoint{1.289094in}{1.137389in}}{\pgfqpoint{1.286780in}{1.131803in}}{\pgfqpoint{1.286780in}{1.125979in}}%
\pgfpathcurveto{\pgfqpoint{1.286780in}{1.120155in}}{\pgfqpoint{1.289094in}{1.114569in}}{\pgfqpoint{1.293212in}{1.110450in}}%
\pgfpathcurveto{\pgfqpoint{1.297330in}{1.106332in}}{\pgfqpoint{1.302916in}{1.104018in}}{\pgfqpoint{1.308740in}{1.104018in}}%
\pgfpathclose%
\pgfusepath{stroke,fill}%
\end{pgfscope}%
\begin{pgfscope}%
\pgfpathrectangle{\pgfqpoint{0.506010in}{1.121191in}}{\pgfqpoint{2.325000in}{1.400000in}} %
\pgfusepath{clip}%
\pgfsetbuttcap%
\pgfsetroundjoin%
\definecolor{currentfill}{rgb}{0.000000,0.500000,0.000000}%
\pgfsetfillcolor{currentfill}%
\pgfsetlinewidth{1.003750pt}%
\definecolor{currentstroke}{rgb}{0.000000,0.500000,0.000000}%
\pgfsetstrokecolor{currentstroke}%
\pgfsetdash{}{0pt}%
\pgfpathmoveto{\pgfqpoint{0.846633in}{1.104326in}}%
\pgfpathcurveto{\pgfqpoint{0.852457in}{1.104326in}}{\pgfqpoint{0.858043in}{1.106640in}}{\pgfqpoint{0.862161in}{1.110758in}}%
\pgfpathcurveto{\pgfqpoint{0.866279in}{1.114877in}}{\pgfqpoint{0.868593in}{1.120463in}}{\pgfqpoint{0.868593in}{1.126287in}}%
\pgfpathcurveto{\pgfqpoint{0.868593in}{1.132111in}}{\pgfqpoint{0.866279in}{1.137697in}}{\pgfqpoint{0.862161in}{1.141815in}}%
\pgfpathcurveto{\pgfqpoint{0.858043in}{1.145933in}}{\pgfqpoint{0.852457in}{1.148247in}}{\pgfqpoint{0.846633in}{1.148247in}}%
\pgfpathcurveto{\pgfqpoint{0.840809in}{1.148247in}}{\pgfqpoint{0.835223in}{1.145933in}}{\pgfqpoint{0.831105in}{1.141815in}}%
\pgfpathcurveto{\pgfqpoint{0.826986in}{1.137697in}}{\pgfqpoint{0.824673in}{1.132111in}}{\pgfqpoint{0.824673in}{1.126287in}}%
\pgfpathcurveto{\pgfqpoint{0.824673in}{1.120463in}}{\pgfqpoint{0.826986in}{1.114877in}}{\pgfqpoint{0.831105in}{1.110758in}}%
\pgfpathcurveto{\pgfqpoint{0.835223in}{1.106640in}}{\pgfqpoint{0.840809in}{1.104326in}}{\pgfqpoint{0.846633in}{1.104326in}}%
\pgfpathclose%
\pgfusepath{stroke,fill}%
\end{pgfscope}%
\begin{pgfscope}%
\pgfpathrectangle{\pgfqpoint{0.506010in}{1.121191in}}{\pgfqpoint{2.325000in}{1.400000in}} %
\pgfusepath{clip}%
\pgfsetbuttcap%
\pgfsetroundjoin%
\definecolor{currentfill}{rgb}{0.000000,0.500000,0.000000}%
\pgfsetfillcolor{currentfill}%
\pgfsetlinewidth{1.003750pt}%
\definecolor{currentstroke}{rgb}{0.000000,0.500000,0.000000}%
\pgfsetstrokecolor{currentstroke}%
\pgfsetdash{}{0pt}%
\pgfpathmoveto{\pgfqpoint{1.144617in}{1.104466in}}%
\pgfpathcurveto{\pgfqpoint{1.150441in}{1.104466in}}{\pgfqpoint{1.156027in}{1.106780in}}{\pgfqpoint{1.160145in}{1.110898in}}%
\pgfpathcurveto{\pgfqpoint{1.164263in}{1.115017in}}{\pgfqpoint{1.166577in}{1.120603in}}{\pgfqpoint{1.166577in}{1.126427in}}%
\pgfpathcurveto{\pgfqpoint{1.166577in}{1.132251in}}{\pgfqpoint{1.164263in}{1.137837in}}{\pgfqpoint{1.160145in}{1.141955in}}%
\pgfpathcurveto{\pgfqpoint{1.156027in}{1.146073in}}{\pgfqpoint{1.150441in}{1.148387in}}{\pgfqpoint{1.144617in}{1.148387in}}%
\pgfpathcurveto{\pgfqpoint{1.138793in}{1.148387in}}{\pgfqpoint{1.133207in}{1.146073in}}{\pgfqpoint{1.129089in}{1.141955in}}%
\pgfpathcurveto{\pgfqpoint{1.124970in}{1.137837in}}{\pgfqpoint{1.122657in}{1.132251in}}{\pgfqpoint{1.122657in}{1.126427in}}%
\pgfpathcurveto{\pgfqpoint{1.122657in}{1.120603in}}{\pgfqpoint{1.124970in}{1.115017in}}{\pgfqpoint{1.129089in}{1.110898in}}%
\pgfpathcurveto{\pgfqpoint{1.133207in}{1.106780in}}{\pgfqpoint{1.138793in}{1.104466in}}{\pgfqpoint{1.144617in}{1.104466in}}%
\pgfpathclose%
\pgfusepath{stroke,fill}%
\end{pgfscope}%
\begin{pgfscope}%
\pgfpathrectangle{\pgfqpoint{0.506010in}{1.121191in}}{\pgfqpoint{2.325000in}{1.400000in}} %
\pgfusepath{clip}%
\pgfsetbuttcap%
\pgfsetroundjoin%
\definecolor{currentfill}{rgb}{0.000000,0.500000,0.000000}%
\pgfsetfillcolor{currentfill}%
\pgfsetlinewidth{1.003750pt}%
\definecolor{currentstroke}{rgb}{0.000000,0.500000,0.000000}%
\pgfsetstrokecolor{currentstroke}%
\pgfsetdash{}{0pt}%
\pgfpathmoveto{\pgfqpoint{1.165504in}{1.104606in}}%
\pgfpathcurveto{\pgfqpoint{1.171328in}{1.104606in}}{\pgfqpoint{1.176914in}{1.106920in}}{\pgfqpoint{1.181032in}{1.111038in}}%
\pgfpathcurveto{\pgfqpoint{1.185150in}{1.115157in}}{\pgfqpoint{1.187464in}{1.120743in}}{\pgfqpoint{1.187464in}{1.126567in}}%
\pgfpathcurveto{\pgfqpoint{1.187464in}{1.132391in}}{\pgfqpoint{1.185150in}{1.137977in}}{\pgfqpoint{1.181032in}{1.142095in}}%
\pgfpathcurveto{\pgfqpoint{1.176914in}{1.146213in}}{\pgfqpoint{1.171328in}{1.148527in}}{\pgfqpoint{1.165504in}{1.148527in}}%
\pgfpathcurveto{\pgfqpoint{1.159680in}{1.148527in}}{\pgfqpoint{1.154094in}{1.146213in}}{\pgfqpoint{1.149976in}{1.142095in}}%
\pgfpathcurveto{\pgfqpoint{1.145857in}{1.137977in}}{\pgfqpoint{1.143544in}{1.132391in}}{\pgfqpoint{1.143544in}{1.126567in}}%
\pgfpathcurveto{\pgfqpoint{1.143544in}{1.120743in}}{\pgfqpoint{1.145857in}{1.115157in}}{\pgfqpoint{1.149976in}{1.111038in}}%
\pgfpathcurveto{\pgfqpoint{1.154094in}{1.106920in}}{\pgfqpoint{1.159680in}{1.104606in}}{\pgfqpoint{1.165504in}{1.104606in}}%
\pgfpathclose%
\pgfusepath{stroke,fill}%
\end{pgfscope}%
\begin{pgfscope}%
\pgfpathrectangle{\pgfqpoint{0.506010in}{1.121191in}}{\pgfqpoint{2.325000in}{1.400000in}} %
\pgfusepath{clip}%
\pgfsetbuttcap%
\pgfsetroundjoin%
\definecolor{currentfill}{rgb}{0.000000,0.500000,0.000000}%
\pgfsetfillcolor{currentfill}%
\pgfsetlinewidth{1.003750pt}%
\definecolor{currentstroke}{rgb}{0.000000,0.500000,0.000000}%
\pgfsetstrokecolor{currentstroke}%
\pgfsetdash{}{0pt}%
\pgfpathmoveto{\pgfqpoint{0.788475in}{1.104774in}}%
\pgfpathcurveto{\pgfqpoint{0.794299in}{1.104774in}}{\pgfqpoint{0.799885in}{1.107088in}}{\pgfqpoint{0.804003in}{1.111206in}}%
\pgfpathcurveto{\pgfqpoint{0.808121in}{1.115325in}}{\pgfqpoint{0.810435in}{1.120911in}}{\pgfqpoint{0.810435in}{1.126735in}}%
\pgfpathcurveto{\pgfqpoint{0.810435in}{1.132559in}}{\pgfqpoint{0.808121in}{1.138145in}}{\pgfqpoint{0.804003in}{1.142263in}}%
\pgfpathcurveto{\pgfqpoint{0.799885in}{1.146381in}}{\pgfqpoint{0.794299in}{1.148695in}}{\pgfqpoint{0.788475in}{1.148695in}}%
\pgfpathcurveto{\pgfqpoint{0.782651in}{1.148695in}}{\pgfqpoint{0.777065in}{1.146381in}}{\pgfqpoint{0.772947in}{1.142263in}}%
\pgfpathcurveto{\pgfqpoint{0.768829in}{1.138145in}}{\pgfqpoint{0.766515in}{1.132559in}}{\pgfqpoint{0.766515in}{1.126735in}}%
\pgfpathcurveto{\pgfqpoint{0.766515in}{1.120911in}}{\pgfqpoint{0.768829in}{1.115325in}}{\pgfqpoint{0.772947in}{1.111206in}}%
\pgfpathcurveto{\pgfqpoint{0.777065in}{1.107088in}}{\pgfqpoint{0.782651in}{1.104774in}}{\pgfqpoint{0.788475in}{1.104774in}}%
\pgfpathclose%
\pgfusepath{stroke,fill}%
\end{pgfscope}%
\begin{pgfscope}%
\pgfpathrectangle{\pgfqpoint{0.506010in}{1.121191in}}{\pgfqpoint{2.325000in}{1.400000in}} %
\pgfusepath{clip}%
\pgfsetbuttcap%
\pgfsetroundjoin%
\definecolor{currentfill}{rgb}{0.000000,0.500000,0.000000}%
\pgfsetfillcolor{currentfill}%
\pgfsetlinewidth{1.003750pt}%
\definecolor{currentstroke}{rgb}{0.000000,0.500000,0.000000}%
\pgfsetstrokecolor{currentstroke}%
\pgfsetdash{}{0pt}%
\pgfpathmoveto{\pgfqpoint{1.587679in}{1.104802in}}%
\pgfpathcurveto{\pgfqpoint{1.593503in}{1.104802in}}{\pgfqpoint{1.599089in}{1.107116in}}{\pgfqpoint{1.603207in}{1.111234in}}%
\pgfpathcurveto{\pgfqpoint{1.607325in}{1.115353in}}{\pgfqpoint{1.609639in}{1.120939in}}{\pgfqpoint{1.609639in}{1.126763in}}%
\pgfpathcurveto{\pgfqpoint{1.609639in}{1.132587in}}{\pgfqpoint{1.607325in}{1.138173in}}{\pgfqpoint{1.603207in}{1.142291in}}%
\pgfpathcurveto{\pgfqpoint{1.599089in}{1.146409in}}{\pgfqpoint{1.593503in}{1.148723in}}{\pgfqpoint{1.587679in}{1.148723in}}%
\pgfpathcurveto{\pgfqpoint{1.581855in}{1.148723in}}{\pgfqpoint{1.576269in}{1.146409in}}{\pgfqpoint{1.572151in}{1.142291in}}%
\pgfpathcurveto{\pgfqpoint{1.568033in}{1.138173in}}{\pgfqpoint{1.565719in}{1.132587in}}{\pgfqpoint{1.565719in}{1.126763in}}%
\pgfpathcurveto{\pgfqpoint{1.565719in}{1.120939in}}{\pgfqpoint{1.568033in}{1.115353in}}{\pgfqpoint{1.572151in}{1.111234in}}%
\pgfpathcurveto{\pgfqpoint{1.576269in}{1.107116in}}{\pgfqpoint{1.581855in}{1.104802in}}{\pgfqpoint{1.587679in}{1.104802in}}%
\pgfpathclose%
\pgfusepath{stroke,fill}%
\end{pgfscope}%
\begin{pgfscope}%
\pgfpathrectangle{\pgfqpoint{0.506010in}{1.121191in}}{\pgfqpoint{2.325000in}{1.400000in}} %
\pgfusepath{clip}%
\pgfsetbuttcap%
\pgfsetroundjoin%
\definecolor{currentfill}{rgb}{0.000000,0.500000,0.000000}%
\pgfsetfillcolor{currentfill}%
\pgfsetlinewidth{1.003750pt}%
\definecolor{currentstroke}{rgb}{0.000000,0.500000,0.000000}%
\pgfsetstrokecolor{currentstroke}%
\pgfsetdash{}{0pt}%
\pgfpathmoveto{\pgfqpoint{1.366286in}{1.105362in}}%
\pgfpathcurveto{\pgfqpoint{1.372110in}{1.105362in}}{\pgfqpoint{1.377696in}{1.107676in}}{\pgfqpoint{1.381815in}{1.111794in}}%
\pgfpathcurveto{\pgfqpoint{1.385933in}{1.115913in}}{\pgfqpoint{1.388247in}{1.121499in}}{\pgfqpoint{1.388247in}{1.127323in}}%
\pgfpathcurveto{\pgfqpoint{1.388247in}{1.133147in}}{\pgfqpoint{1.385933in}{1.138733in}}{\pgfqpoint{1.381815in}{1.142851in}}%
\pgfpathcurveto{\pgfqpoint{1.377696in}{1.146969in}}{\pgfqpoint{1.372110in}{1.149283in}}{\pgfqpoint{1.366286in}{1.149283in}}%
\pgfpathcurveto{\pgfqpoint{1.360462in}{1.149283in}}{\pgfqpoint{1.354876in}{1.146969in}}{\pgfqpoint{1.350758in}{1.142851in}}%
\pgfpathcurveto{\pgfqpoint{1.346640in}{1.138733in}}{\pgfqpoint{1.344326in}{1.133147in}}{\pgfqpoint{1.344326in}{1.127323in}}%
\pgfpathcurveto{\pgfqpoint{1.344326in}{1.121499in}}{\pgfqpoint{1.346640in}{1.115913in}}{\pgfqpoint{1.350758in}{1.111794in}}%
\pgfpathcurveto{\pgfqpoint{1.354876in}{1.107676in}}{\pgfqpoint{1.360462in}{1.105362in}}{\pgfqpoint{1.366286in}{1.105362in}}%
\pgfpathclose%
\pgfusepath{stroke,fill}%
\end{pgfscope}%
\begin{pgfscope}%
\pgfpathrectangle{\pgfqpoint{0.506010in}{1.121191in}}{\pgfqpoint{2.325000in}{1.400000in}} %
\pgfusepath{clip}%
\pgfsetbuttcap%
\pgfsetroundjoin%
\definecolor{currentfill}{rgb}{0.000000,0.500000,0.000000}%
\pgfsetfillcolor{currentfill}%
\pgfsetlinewidth{1.003750pt}%
\definecolor{currentstroke}{rgb}{0.000000,0.500000,0.000000}%
\pgfsetstrokecolor{currentstroke}%
\pgfsetdash{}{0pt}%
\pgfpathmoveto{\pgfqpoint{1.279201in}{1.105418in}}%
\pgfpathcurveto{\pgfqpoint{1.285024in}{1.105418in}}{\pgfqpoint{1.290611in}{1.107732in}}{\pgfqpoint{1.294729in}{1.111850in}}%
\pgfpathcurveto{\pgfqpoint{1.298847in}{1.115969in}}{\pgfqpoint{1.301161in}{1.121555in}}{\pgfqpoint{1.301161in}{1.127379in}}%
\pgfpathcurveto{\pgfqpoint{1.301161in}{1.133203in}}{\pgfqpoint{1.298847in}{1.138789in}}{\pgfqpoint{1.294729in}{1.142907in}}%
\pgfpathcurveto{\pgfqpoint{1.290611in}{1.147025in}}{\pgfqpoint{1.285024in}{1.149339in}}{\pgfqpoint{1.279201in}{1.149339in}}%
\pgfpathcurveto{\pgfqpoint{1.273377in}{1.149339in}}{\pgfqpoint{1.267790in}{1.147025in}}{\pgfqpoint{1.263672in}{1.142907in}}%
\pgfpathcurveto{\pgfqpoint{1.259554in}{1.138789in}}{\pgfqpoint{1.257240in}{1.133203in}}{\pgfqpoint{1.257240in}{1.127379in}}%
\pgfpathcurveto{\pgfqpoint{1.257240in}{1.121555in}}{\pgfqpoint{1.259554in}{1.115969in}}{\pgfqpoint{1.263672in}{1.111850in}}%
\pgfpathcurveto{\pgfqpoint{1.267790in}{1.107732in}}{\pgfqpoint{1.273377in}{1.105418in}}{\pgfqpoint{1.279201in}{1.105418in}}%
\pgfpathclose%
\pgfusepath{stroke,fill}%
\end{pgfscope}%
\begin{pgfscope}%
\pgfpathrectangle{\pgfqpoint{0.506010in}{1.121191in}}{\pgfqpoint{2.325000in}{1.400000in}} %
\pgfusepath{clip}%
\pgfsetbuttcap%
\pgfsetroundjoin%
\definecolor{currentfill}{rgb}{0.000000,0.500000,0.000000}%
\pgfsetfillcolor{currentfill}%
\pgfsetlinewidth{1.003750pt}%
\definecolor{currentstroke}{rgb}{0.000000,0.500000,0.000000}%
\pgfsetstrokecolor{currentstroke}%
\pgfsetdash{}{0pt}%
\pgfpathmoveto{\pgfqpoint{2.264030in}{1.105418in}}%
\pgfpathcurveto{\pgfqpoint{2.269854in}{1.105418in}}{\pgfqpoint{2.275440in}{1.107732in}}{\pgfqpoint{2.279558in}{1.111850in}}%
\pgfpathcurveto{\pgfqpoint{2.283676in}{1.115969in}}{\pgfqpoint{2.285990in}{1.121555in}}{\pgfqpoint{2.285990in}{1.127379in}}%
\pgfpathcurveto{\pgfqpoint{2.285990in}{1.133203in}}{\pgfqpoint{2.283676in}{1.138789in}}{\pgfqpoint{2.279558in}{1.142907in}}%
\pgfpathcurveto{\pgfqpoint{2.275440in}{1.147025in}}{\pgfqpoint{2.269854in}{1.149339in}}{\pgfqpoint{2.264030in}{1.149339in}}%
\pgfpathcurveto{\pgfqpoint{2.258206in}{1.149339in}}{\pgfqpoint{2.252620in}{1.147025in}}{\pgfqpoint{2.248502in}{1.142907in}}%
\pgfpathcurveto{\pgfqpoint{2.244383in}{1.138789in}}{\pgfqpoint{2.242070in}{1.133203in}}{\pgfqpoint{2.242070in}{1.127379in}}%
\pgfpathcurveto{\pgfqpoint{2.242070in}{1.121555in}}{\pgfqpoint{2.244383in}{1.115969in}}{\pgfqpoint{2.248502in}{1.111850in}}%
\pgfpathcurveto{\pgfqpoint{2.252620in}{1.107732in}}{\pgfqpoint{2.258206in}{1.105418in}}{\pgfqpoint{2.264030in}{1.105418in}}%
\pgfpathclose%
\pgfusepath{stroke,fill}%
\end{pgfscope}%
\begin{pgfscope}%
\pgfpathrectangle{\pgfqpoint{0.506010in}{1.121191in}}{\pgfqpoint{2.325000in}{1.400000in}} %
\pgfusepath{clip}%
\pgfsetbuttcap%
\pgfsetroundjoin%
\definecolor{currentfill}{rgb}{0.000000,0.500000,0.000000}%
\pgfsetfillcolor{currentfill}%
\pgfsetlinewidth{1.003750pt}%
\definecolor{currentstroke}{rgb}{0.000000,0.500000,0.000000}%
\pgfsetstrokecolor{currentstroke}%
\pgfsetdash{}{0pt}%
\pgfpathmoveto{\pgfqpoint{2.191001in}{1.105642in}}%
\pgfpathcurveto{\pgfqpoint{2.196825in}{1.105642in}}{\pgfqpoint{2.202412in}{1.107956in}}{\pgfqpoint{2.206530in}{1.112074in}}%
\pgfpathcurveto{\pgfqpoint{2.210648in}{1.116193in}}{\pgfqpoint{2.212962in}{1.121779in}}{\pgfqpoint{2.212962in}{1.127603in}}%
\pgfpathcurveto{\pgfqpoint{2.212962in}{1.133427in}}{\pgfqpoint{2.210648in}{1.139013in}}{\pgfqpoint{2.206530in}{1.143131in}}%
\pgfpathcurveto{\pgfqpoint{2.202412in}{1.147249in}}{\pgfqpoint{2.196825in}{1.149563in}}{\pgfqpoint{2.191001in}{1.149563in}}%
\pgfpathcurveto{\pgfqpoint{2.185178in}{1.149563in}}{\pgfqpoint{2.179591in}{1.147249in}}{\pgfqpoint{2.175473in}{1.143131in}}%
\pgfpathcurveto{\pgfqpoint{2.171355in}{1.139013in}}{\pgfqpoint{2.169041in}{1.133427in}}{\pgfqpoint{2.169041in}{1.127603in}}%
\pgfpathcurveto{\pgfqpoint{2.169041in}{1.121779in}}{\pgfqpoint{2.171355in}{1.116193in}}{\pgfqpoint{2.175473in}{1.112074in}}%
\pgfpathcurveto{\pgfqpoint{2.179591in}{1.107956in}}{\pgfqpoint{2.185178in}{1.105642in}}{\pgfqpoint{2.191001in}{1.105642in}}%
\pgfpathclose%
\pgfusepath{stroke,fill}%
\end{pgfscope}%
\begin{pgfscope}%
\pgfpathrectangle{\pgfqpoint{0.506010in}{1.121191in}}{\pgfqpoint{2.325000in}{1.400000in}} %
\pgfusepath{clip}%
\pgfsetbuttcap%
\pgfsetroundjoin%
\definecolor{currentfill}{rgb}{0.000000,0.500000,0.000000}%
\pgfsetfillcolor{currentfill}%
\pgfsetlinewidth{1.003750pt}%
\definecolor{currentstroke}{rgb}{0.000000,0.500000,0.000000}%
\pgfsetstrokecolor{currentstroke}%
\pgfsetdash{}{0pt}%
\pgfpathmoveto{\pgfqpoint{2.263023in}{1.105782in}}%
\pgfpathcurveto{\pgfqpoint{2.268847in}{1.105782in}}{\pgfqpoint{2.274434in}{1.108096in}}{\pgfqpoint{2.278552in}{1.112214in}}%
\pgfpathcurveto{\pgfqpoint{2.282670in}{1.116333in}}{\pgfqpoint{2.284984in}{1.121919in}}{\pgfqpoint{2.284984in}{1.127743in}}%
\pgfpathcurveto{\pgfqpoint{2.284984in}{1.133567in}}{\pgfqpoint{2.282670in}{1.139153in}}{\pgfqpoint{2.278552in}{1.143271in}}%
\pgfpathcurveto{\pgfqpoint{2.274434in}{1.147389in}}{\pgfqpoint{2.268847in}{1.149703in}}{\pgfqpoint{2.263023in}{1.149703in}}%
\pgfpathcurveto{\pgfqpoint{2.257200in}{1.149703in}}{\pgfqpoint{2.251613in}{1.147389in}}{\pgfqpoint{2.247495in}{1.143271in}}%
\pgfpathcurveto{\pgfqpoint{2.243377in}{1.139153in}}{\pgfqpoint{2.241063in}{1.133567in}}{\pgfqpoint{2.241063in}{1.127743in}}%
\pgfpathcurveto{\pgfqpoint{2.241063in}{1.121919in}}{\pgfqpoint{2.243377in}{1.116333in}}{\pgfqpoint{2.247495in}{1.112214in}}%
\pgfpathcurveto{\pgfqpoint{2.251613in}{1.108096in}}{\pgfqpoint{2.257200in}{1.105782in}}{\pgfqpoint{2.263023in}{1.105782in}}%
\pgfpathclose%
\pgfusepath{stroke,fill}%
\end{pgfscope}%
\begin{pgfscope}%
\pgfpathrectangle{\pgfqpoint{0.506010in}{1.121191in}}{\pgfqpoint{2.325000in}{1.400000in}} %
\pgfusepath{clip}%
\pgfsetbuttcap%
\pgfsetroundjoin%
\definecolor{currentfill}{rgb}{0.000000,0.500000,0.000000}%
\pgfsetfillcolor{currentfill}%
\pgfsetlinewidth{1.003750pt}%
\definecolor{currentstroke}{rgb}{0.000000,0.500000,0.000000}%
\pgfsetstrokecolor{currentstroke}%
\pgfsetdash{}{0pt}%
\pgfpathmoveto{\pgfqpoint{2.251667in}{1.106062in}}%
\pgfpathcurveto{\pgfqpoint{2.257491in}{1.106062in}}{\pgfqpoint{2.263077in}{1.108376in}}{\pgfqpoint{2.267195in}{1.112494in}}%
\pgfpathcurveto{\pgfqpoint{2.271313in}{1.116613in}}{\pgfqpoint{2.273627in}{1.122199in}}{\pgfqpoint{2.273627in}{1.128023in}}%
\pgfpathcurveto{\pgfqpoint{2.273627in}{1.133847in}}{\pgfqpoint{2.271313in}{1.139433in}}{\pgfqpoint{2.267195in}{1.143551in}}%
\pgfpathcurveto{\pgfqpoint{2.263077in}{1.147669in}}{\pgfqpoint{2.257491in}{1.149983in}}{\pgfqpoint{2.251667in}{1.149983in}}%
\pgfpathcurveto{\pgfqpoint{2.245843in}{1.149983in}}{\pgfqpoint{2.240257in}{1.147669in}}{\pgfqpoint{2.236139in}{1.143551in}}%
\pgfpathcurveto{\pgfqpoint{2.232020in}{1.139433in}}{\pgfqpoint{2.229707in}{1.133847in}}{\pgfqpoint{2.229707in}{1.128023in}}%
\pgfpathcurveto{\pgfqpoint{2.229707in}{1.122199in}}{\pgfqpoint{2.232020in}{1.116613in}}{\pgfqpoint{2.236139in}{1.112494in}}%
\pgfpathcurveto{\pgfqpoint{2.240257in}{1.108376in}}{\pgfqpoint{2.245843in}{1.106062in}}{\pgfqpoint{2.251667in}{1.106062in}}%
\pgfpathclose%
\pgfusepath{stroke,fill}%
\end{pgfscope}%
\begin{pgfscope}%
\pgfpathrectangle{\pgfqpoint{0.506010in}{1.121191in}}{\pgfqpoint{2.325000in}{1.400000in}} %
\pgfusepath{clip}%
\pgfsetbuttcap%
\pgfsetroundjoin%
\definecolor{currentfill}{rgb}{0.000000,0.500000,0.000000}%
\pgfsetfillcolor{currentfill}%
\pgfsetlinewidth{1.003750pt}%
\definecolor{currentstroke}{rgb}{0.000000,0.500000,0.000000}%
\pgfsetstrokecolor{currentstroke}%
\pgfsetdash{}{0pt}%
\pgfpathmoveto{\pgfqpoint{2.223598in}{1.106286in}}%
\pgfpathcurveto{\pgfqpoint{2.229422in}{1.106286in}}{\pgfqpoint{2.235008in}{1.108600in}}{\pgfqpoint{2.239126in}{1.112718in}}%
\pgfpathcurveto{\pgfqpoint{2.243244in}{1.116837in}}{\pgfqpoint{2.245558in}{1.122423in}}{\pgfqpoint{2.245558in}{1.128247in}}%
\pgfpathcurveto{\pgfqpoint{2.245558in}{1.134071in}}{\pgfqpoint{2.243244in}{1.139657in}}{\pgfqpoint{2.239126in}{1.143775in}}%
\pgfpathcurveto{\pgfqpoint{2.235008in}{1.147893in}}{\pgfqpoint{2.229422in}{1.150207in}}{\pgfqpoint{2.223598in}{1.150207in}}%
\pgfpathcurveto{\pgfqpoint{2.217774in}{1.150207in}}{\pgfqpoint{2.212188in}{1.147893in}}{\pgfqpoint{2.208070in}{1.143775in}}%
\pgfpathcurveto{\pgfqpoint{2.203952in}{1.139657in}}{\pgfqpoint{2.201638in}{1.134071in}}{\pgfqpoint{2.201638in}{1.128247in}}%
\pgfpathcurveto{\pgfqpoint{2.201638in}{1.122423in}}{\pgfqpoint{2.203952in}{1.116837in}}{\pgfqpoint{2.208070in}{1.112718in}}%
\pgfpathcurveto{\pgfqpoint{2.212188in}{1.108600in}}{\pgfqpoint{2.217774in}{1.106286in}}{\pgfqpoint{2.223598in}{1.106286in}}%
\pgfpathclose%
\pgfusepath{stroke,fill}%
\end{pgfscope}%
\begin{pgfscope}%
\pgfpathrectangle{\pgfqpoint{0.506010in}{1.121191in}}{\pgfqpoint{2.325000in}{1.400000in}} %
\pgfusepath{clip}%
\pgfsetbuttcap%
\pgfsetroundjoin%
\definecolor{currentfill}{rgb}{0.000000,0.500000,0.000000}%
\pgfsetfillcolor{currentfill}%
\pgfsetlinewidth{1.003750pt}%
\definecolor{currentstroke}{rgb}{0.000000,0.500000,0.000000}%
\pgfsetstrokecolor{currentstroke}%
\pgfsetdash{}{0pt}%
\pgfpathmoveto{\pgfqpoint{2.223598in}{1.106398in}}%
\pgfpathcurveto{\pgfqpoint{2.229422in}{1.106398in}}{\pgfqpoint{2.235008in}{1.108712in}}{\pgfqpoint{2.239126in}{1.112830in}}%
\pgfpathcurveto{\pgfqpoint{2.243244in}{1.116949in}}{\pgfqpoint{2.245558in}{1.122535in}}{\pgfqpoint{2.245558in}{1.128359in}}%
\pgfpathcurveto{\pgfqpoint{2.245558in}{1.134183in}}{\pgfqpoint{2.243244in}{1.139769in}}{\pgfqpoint{2.239126in}{1.143887in}}%
\pgfpathcurveto{\pgfqpoint{2.235008in}{1.148005in}}{\pgfqpoint{2.229422in}{1.150319in}}{\pgfqpoint{2.223598in}{1.150319in}}%
\pgfpathcurveto{\pgfqpoint{2.217774in}{1.150319in}}{\pgfqpoint{2.212188in}{1.148005in}}{\pgfqpoint{2.208070in}{1.143887in}}%
\pgfpathcurveto{\pgfqpoint{2.203952in}{1.139769in}}{\pgfqpoint{2.201638in}{1.134183in}}{\pgfqpoint{2.201638in}{1.128359in}}%
\pgfpathcurveto{\pgfqpoint{2.201638in}{1.122535in}}{\pgfqpoint{2.203952in}{1.116949in}}{\pgfqpoint{2.208070in}{1.112830in}}%
\pgfpathcurveto{\pgfqpoint{2.212188in}{1.108712in}}{\pgfqpoint{2.217774in}{1.106398in}}{\pgfqpoint{2.223598in}{1.106398in}}%
\pgfpathclose%
\pgfusepath{stroke,fill}%
\end{pgfscope}%
\begin{pgfscope}%
\pgfpathrectangle{\pgfqpoint{0.506010in}{1.121191in}}{\pgfqpoint{2.325000in}{1.400000in}} %
\pgfusepath{clip}%
\pgfsetbuttcap%
\pgfsetroundjoin%
\definecolor{currentfill}{rgb}{0.000000,0.500000,0.000000}%
\pgfsetfillcolor{currentfill}%
\pgfsetlinewidth{1.003750pt}%
\definecolor{currentstroke}{rgb}{0.000000,0.500000,0.000000}%
\pgfsetstrokecolor{currentstroke}%
\pgfsetdash{}{0pt}%
\pgfpathmoveto{\pgfqpoint{1.007822in}{1.106482in}}%
\pgfpathcurveto{\pgfqpoint{1.013646in}{1.106482in}}{\pgfqpoint{1.019232in}{1.108796in}}{\pgfqpoint{1.023351in}{1.112914in}}%
\pgfpathcurveto{\pgfqpoint{1.027469in}{1.117033in}}{\pgfqpoint{1.029783in}{1.122619in}}{\pgfqpoint{1.029783in}{1.128443in}}%
\pgfpathcurveto{\pgfqpoint{1.029783in}{1.134267in}}{\pgfqpoint{1.027469in}{1.139853in}}{\pgfqpoint{1.023351in}{1.143971in}}%
\pgfpathcurveto{\pgfqpoint{1.019232in}{1.148089in}}{\pgfqpoint{1.013646in}{1.150403in}}{\pgfqpoint{1.007822in}{1.150403in}}%
\pgfpathcurveto{\pgfqpoint{1.001998in}{1.150403in}}{\pgfqpoint{0.996412in}{1.148089in}}{\pgfqpoint{0.992294in}{1.143971in}}%
\pgfpathcurveto{\pgfqpoint{0.988176in}{1.139853in}}{\pgfqpoint{0.985862in}{1.134267in}}{\pgfqpoint{0.985862in}{1.128443in}}%
\pgfpathcurveto{\pgfqpoint{0.985862in}{1.122619in}}{\pgfqpoint{0.988176in}{1.117033in}}{\pgfqpoint{0.992294in}{1.112914in}}%
\pgfpathcurveto{\pgfqpoint{0.996412in}{1.108796in}}{\pgfqpoint{1.001998in}{1.106482in}}{\pgfqpoint{1.007822in}{1.106482in}}%
\pgfpathclose%
\pgfusepath{stroke,fill}%
\end{pgfscope}%
\begin{pgfscope}%
\pgfpathrectangle{\pgfqpoint{0.506010in}{1.121191in}}{\pgfqpoint{2.325000in}{1.400000in}} %
\pgfusepath{clip}%
\pgfsetbuttcap%
\pgfsetroundjoin%
\definecolor{currentfill}{rgb}{0.000000,0.500000,0.000000}%
\pgfsetfillcolor{currentfill}%
\pgfsetlinewidth{1.003750pt}%
\definecolor{currentstroke}{rgb}{0.000000,0.500000,0.000000}%
\pgfsetstrokecolor{currentstroke}%
\pgfsetdash{}{0pt}%
\pgfpathmoveto{\pgfqpoint{1.007822in}{1.106846in}}%
\pgfpathcurveto{\pgfqpoint{1.013646in}{1.106846in}}{\pgfqpoint{1.019232in}{1.109160in}}{\pgfqpoint{1.023351in}{1.113278in}}%
\pgfpathcurveto{\pgfqpoint{1.027469in}{1.117397in}}{\pgfqpoint{1.029783in}{1.122983in}}{\pgfqpoint{1.029783in}{1.128807in}}%
\pgfpathcurveto{\pgfqpoint{1.029783in}{1.134631in}}{\pgfqpoint{1.027469in}{1.140217in}}{\pgfqpoint{1.023351in}{1.144335in}}%
\pgfpathcurveto{\pgfqpoint{1.019232in}{1.148453in}}{\pgfqpoint{1.013646in}{1.150767in}}{\pgfqpoint{1.007822in}{1.150767in}}%
\pgfpathcurveto{\pgfqpoint{1.001998in}{1.150767in}}{\pgfqpoint{0.996412in}{1.148453in}}{\pgfqpoint{0.992294in}{1.144335in}}%
\pgfpathcurveto{\pgfqpoint{0.988176in}{1.140217in}}{\pgfqpoint{0.985862in}{1.134631in}}{\pgfqpoint{0.985862in}{1.128807in}}%
\pgfpathcurveto{\pgfqpoint{0.985862in}{1.122983in}}{\pgfqpoint{0.988176in}{1.117397in}}{\pgfqpoint{0.992294in}{1.113278in}}%
\pgfpathcurveto{\pgfqpoint{0.996412in}{1.109160in}}{\pgfqpoint{1.001998in}{1.106846in}}{\pgfqpoint{1.007822in}{1.106846in}}%
\pgfpathclose%
\pgfusepath{stroke,fill}%
\end{pgfscope}%
\begin{pgfscope}%
\pgfpathrectangle{\pgfqpoint{0.506010in}{1.121191in}}{\pgfqpoint{2.325000in}{1.400000in}} %
\pgfusepath{clip}%
\pgfsetbuttcap%
\pgfsetroundjoin%
\definecolor{currentfill}{rgb}{0.000000,0.500000,0.000000}%
\pgfsetfillcolor{currentfill}%
\pgfsetlinewidth{1.003750pt}%
\definecolor{currentstroke}{rgb}{0.000000,0.500000,0.000000}%
\pgfsetstrokecolor{currentstroke}%
\pgfsetdash{}{0pt}%
\pgfpathmoveto{\pgfqpoint{2.381458in}{1.108106in}}%
\pgfpathcurveto{\pgfqpoint{2.387281in}{1.108106in}}{\pgfqpoint{2.392868in}{1.110420in}}{\pgfqpoint{2.396986in}{1.114538in}}%
\pgfpathcurveto{\pgfqpoint{2.401104in}{1.118657in}}{\pgfqpoint{2.403418in}{1.124243in}}{\pgfqpoint{2.403418in}{1.130067in}}%
\pgfpathcurveto{\pgfqpoint{2.403418in}{1.135891in}}{\pgfqpoint{2.401104in}{1.141477in}}{\pgfqpoint{2.396986in}{1.145595in}}%
\pgfpathcurveto{\pgfqpoint{2.392868in}{1.149713in}}{\pgfqpoint{2.387281in}{1.152027in}}{\pgfqpoint{2.381458in}{1.152027in}}%
\pgfpathcurveto{\pgfqpoint{2.375634in}{1.152027in}}{\pgfqpoint{2.370047in}{1.149713in}}{\pgfqpoint{2.365929in}{1.145595in}}%
\pgfpathcurveto{\pgfqpoint{2.361811in}{1.141477in}}{\pgfqpoint{2.359497in}{1.135891in}}{\pgfqpoint{2.359497in}{1.130067in}}%
\pgfpathcurveto{\pgfqpoint{2.359497in}{1.124243in}}{\pgfqpoint{2.361811in}{1.118657in}}{\pgfqpoint{2.365929in}{1.114538in}}%
\pgfpathcurveto{\pgfqpoint{2.370047in}{1.110420in}}{\pgfqpoint{2.375634in}{1.108106in}}{\pgfqpoint{2.381458in}{1.108106in}}%
\pgfpathclose%
\pgfusepath{stroke,fill}%
\end{pgfscope}%
\begin{pgfscope}%
\pgfpathrectangle{\pgfqpoint{0.506010in}{1.121191in}}{\pgfqpoint{2.325000in}{1.400000in}} %
\pgfusepath{clip}%
\pgfsetbuttcap%
\pgfsetroundjoin%
\definecolor{currentfill}{rgb}{0.000000,0.500000,0.000000}%
\pgfsetfillcolor{currentfill}%
\pgfsetlinewidth{1.003750pt}%
\definecolor{currentstroke}{rgb}{0.000000,0.500000,0.000000}%
\pgfsetstrokecolor{currentstroke}%
\pgfsetdash{}{0pt}%
\pgfpathmoveto{\pgfqpoint{2.380105in}{1.108806in}}%
\pgfpathcurveto{\pgfqpoint{2.385929in}{1.108806in}}{\pgfqpoint{2.391515in}{1.111120in}}{\pgfqpoint{2.395633in}{1.115238in}}%
\pgfpathcurveto{\pgfqpoint{2.399751in}{1.119357in}}{\pgfqpoint{2.402065in}{1.124943in}}{\pgfqpoint{2.402065in}{1.130767in}}%
\pgfpathcurveto{\pgfqpoint{2.402065in}{1.136591in}}{\pgfqpoint{2.399751in}{1.142177in}}{\pgfqpoint{2.395633in}{1.146295in}}%
\pgfpathcurveto{\pgfqpoint{2.391515in}{1.150413in}}{\pgfqpoint{2.385929in}{1.152727in}}{\pgfqpoint{2.380105in}{1.152727in}}%
\pgfpathcurveto{\pgfqpoint{2.374281in}{1.152727in}}{\pgfqpoint{2.368695in}{1.150413in}}{\pgfqpoint{2.364576in}{1.146295in}}%
\pgfpathcurveto{\pgfqpoint{2.360458in}{1.142177in}}{\pgfqpoint{2.358144in}{1.136591in}}{\pgfqpoint{2.358144in}{1.130767in}}%
\pgfpathcurveto{\pgfqpoint{2.358144in}{1.124943in}}{\pgfqpoint{2.360458in}{1.119357in}}{\pgfqpoint{2.364576in}{1.115238in}}%
\pgfpathcurveto{\pgfqpoint{2.368695in}{1.111120in}}{\pgfqpoint{2.374281in}{1.108806in}}{\pgfqpoint{2.380105in}{1.108806in}}%
\pgfpathclose%
\pgfusepath{stroke,fill}%
\end{pgfscope}%
\begin{pgfscope}%
\pgfpathrectangle{\pgfqpoint{0.506010in}{1.121191in}}{\pgfqpoint{2.325000in}{1.400000in}} %
\pgfusepath{clip}%
\pgfsetbuttcap%
\pgfsetroundjoin%
\definecolor{currentfill}{rgb}{0.000000,0.500000,0.000000}%
\pgfsetfillcolor{currentfill}%
\pgfsetlinewidth{1.003750pt}%
\definecolor{currentstroke}{rgb}{0.000000,0.500000,0.000000}%
\pgfsetstrokecolor{currentstroke}%
\pgfsetdash{}{0pt}%
\pgfpathmoveto{\pgfqpoint{2.407155in}{1.109254in}}%
\pgfpathcurveto{\pgfqpoint{2.412979in}{1.109254in}}{\pgfqpoint{2.418565in}{1.111568in}}{\pgfqpoint{2.422683in}{1.115686in}}%
\pgfpathcurveto{\pgfqpoint{2.426801in}{1.119805in}}{\pgfqpoint{2.429115in}{1.125391in}}{\pgfqpoint{2.429115in}{1.131215in}}%
\pgfpathcurveto{\pgfqpoint{2.429115in}{1.137039in}}{\pgfqpoint{2.426801in}{1.142625in}}{\pgfqpoint{2.422683in}{1.146743in}}%
\pgfpathcurveto{\pgfqpoint{2.418565in}{1.150861in}}{\pgfqpoint{2.412979in}{1.153175in}}{\pgfqpoint{2.407155in}{1.153175in}}%
\pgfpathcurveto{\pgfqpoint{2.401331in}{1.153175in}}{\pgfqpoint{2.395745in}{1.150861in}}{\pgfqpoint{2.391627in}{1.146743in}}%
\pgfpathcurveto{\pgfqpoint{2.387509in}{1.142625in}}{\pgfqpoint{2.385195in}{1.137039in}}{\pgfqpoint{2.385195in}{1.131215in}}%
\pgfpathcurveto{\pgfqpoint{2.385195in}{1.125391in}}{\pgfqpoint{2.387509in}{1.119805in}}{\pgfqpoint{2.391627in}{1.115686in}}%
\pgfpathcurveto{\pgfqpoint{2.395745in}{1.111568in}}{\pgfqpoint{2.401331in}{1.109254in}}{\pgfqpoint{2.407155in}{1.109254in}}%
\pgfpathclose%
\pgfusepath{stroke,fill}%
\end{pgfscope}%
\begin{pgfscope}%
\pgfpathrectangle{\pgfqpoint{0.506010in}{1.121191in}}{\pgfqpoint{2.325000in}{1.400000in}} %
\pgfusepath{clip}%
\pgfsetbuttcap%
\pgfsetroundjoin%
\definecolor{currentfill}{rgb}{0.000000,0.500000,0.000000}%
\pgfsetfillcolor{currentfill}%
\pgfsetlinewidth{1.003750pt}%
\definecolor{currentstroke}{rgb}{0.000000,0.500000,0.000000}%
\pgfsetstrokecolor{currentstroke}%
\pgfsetdash{}{0pt}%
\pgfpathmoveto{\pgfqpoint{1.294699in}{1.109954in}}%
\pgfpathcurveto{\pgfqpoint{1.300523in}{1.109954in}}{\pgfqpoint{1.306109in}{1.112268in}}{\pgfqpoint{1.310227in}{1.116386in}}%
\pgfpathcurveto{\pgfqpoint{1.314345in}{1.120505in}}{\pgfqpoint{1.316659in}{1.126091in}}{\pgfqpoint{1.316659in}{1.131915in}}%
\pgfpathcurveto{\pgfqpoint{1.316659in}{1.137739in}}{\pgfqpoint{1.314345in}{1.143325in}}{\pgfqpoint{1.310227in}{1.147443in}}%
\pgfpathcurveto{\pgfqpoint{1.306109in}{1.151561in}}{\pgfqpoint{1.300523in}{1.153875in}}{\pgfqpoint{1.294699in}{1.153875in}}%
\pgfpathcurveto{\pgfqpoint{1.288875in}{1.153875in}}{\pgfqpoint{1.283289in}{1.151561in}}{\pgfqpoint{1.279171in}{1.147443in}}%
\pgfpathcurveto{\pgfqpoint{1.275053in}{1.143325in}}{\pgfqpoint{1.272739in}{1.137739in}}{\pgfqpoint{1.272739in}{1.131915in}}%
\pgfpathcurveto{\pgfqpoint{1.272739in}{1.126091in}}{\pgfqpoint{1.275053in}{1.120505in}}{\pgfqpoint{1.279171in}{1.116386in}}%
\pgfpathcurveto{\pgfqpoint{1.283289in}{1.112268in}}{\pgfqpoint{1.288875in}{1.109954in}}{\pgfqpoint{1.294699in}{1.109954in}}%
\pgfpathclose%
\pgfusepath{stroke,fill}%
\end{pgfscope}%
\begin{pgfscope}%
\pgfpathrectangle{\pgfqpoint{0.506010in}{1.121191in}}{\pgfqpoint{2.325000in}{1.400000in}} %
\pgfusepath{clip}%
\pgfsetbuttcap%
\pgfsetroundjoin%
\definecolor{currentfill}{rgb}{0.000000,0.500000,0.000000}%
\pgfsetfillcolor{currentfill}%
\pgfsetlinewidth{1.003750pt}%
\definecolor{currentstroke}{rgb}{0.000000,0.500000,0.000000}%
\pgfsetstrokecolor{currentstroke}%
\pgfsetdash{}{0pt}%
\pgfpathmoveto{\pgfqpoint{1.390732in}{1.111102in}}%
\pgfpathcurveto{\pgfqpoint{1.396556in}{1.111102in}}{\pgfqpoint{1.402142in}{1.113416in}}{\pgfqpoint{1.406261in}{1.117534in}}%
\pgfpathcurveto{\pgfqpoint{1.410379in}{1.121653in}}{\pgfqpoint{1.412693in}{1.127239in}}{\pgfqpoint{1.412693in}{1.133063in}}%
\pgfpathcurveto{\pgfqpoint{1.412693in}{1.138887in}}{\pgfqpoint{1.410379in}{1.144473in}}{\pgfqpoint{1.406261in}{1.148591in}}%
\pgfpathcurveto{\pgfqpoint{1.402142in}{1.152709in}}{\pgfqpoint{1.396556in}{1.155023in}}{\pgfqpoint{1.390732in}{1.155023in}}%
\pgfpathcurveto{\pgfqpoint{1.384908in}{1.155023in}}{\pgfqpoint{1.379322in}{1.152709in}}{\pgfqpoint{1.375204in}{1.148591in}}%
\pgfpathcurveto{\pgfqpoint{1.371086in}{1.144473in}}{\pgfqpoint{1.368772in}{1.138887in}}{\pgfqpoint{1.368772in}{1.133063in}}%
\pgfpathcurveto{\pgfqpoint{1.368772in}{1.127239in}}{\pgfqpoint{1.371086in}{1.121653in}}{\pgfqpoint{1.375204in}{1.117534in}}%
\pgfpathcurveto{\pgfqpoint{1.379322in}{1.113416in}}{\pgfqpoint{1.384908in}{1.111102in}}{\pgfqpoint{1.390732in}{1.111102in}}%
\pgfpathclose%
\pgfusepath{stroke,fill}%
\end{pgfscope}%
\begin{pgfscope}%
\pgfpathrectangle{\pgfqpoint{0.506010in}{1.121191in}}{\pgfqpoint{2.325000in}{1.400000in}} %
\pgfusepath{clip}%
\pgfsetbuttcap%
\pgfsetroundjoin%
\definecolor{currentfill}{rgb}{0.000000,0.500000,0.000000}%
\pgfsetfillcolor{currentfill}%
\pgfsetlinewidth{1.003750pt}%
\definecolor{currentstroke}{rgb}{0.000000,0.500000,0.000000}%
\pgfsetstrokecolor{currentstroke}%
\pgfsetdash{}{0pt}%
\pgfpathmoveto{\pgfqpoint{1.305057in}{2.084354in}}%
\pgfpathcurveto{\pgfqpoint{1.310881in}{2.084354in}}{\pgfqpoint{1.316468in}{2.086668in}}{\pgfqpoint{1.320586in}{2.090786in}}%
\pgfpathcurveto{\pgfqpoint{1.324704in}{2.094905in}}{\pgfqpoint{1.327018in}{2.100491in}}{\pgfqpoint{1.327018in}{2.106315in}}%
\pgfpathcurveto{\pgfqpoint{1.327018in}{2.112139in}}{\pgfqpoint{1.324704in}{2.117725in}}{\pgfqpoint{1.320586in}{2.121843in}}%
\pgfpathcurveto{\pgfqpoint{1.316468in}{2.125961in}}{\pgfqpoint{1.310881in}{2.128275in}}{\pgfqpoint{1.305057in}{2.128275in}}%
\pgfpathcurveto{\pgfqpoint{1.299234in}{2.128275in}}{\pgfqpoint{1.293647in}{2.125961in}}{\pgfqpoint{1.289529in}{2.121843in}}%
\pgfpathcurveto{\pgfqpoint{1.285411in}{2.117725in}}{\pgfqpoint{1.283097in}{2.112139in}}{\pgfqpoint{1.283097in}{2.106315in}}%
\pgfpathcurveto{\pgfqpoint{1.283097in}{2.100491in}}{\pgfqpoint{1.285411in}{2.094905in}}{\pgfqpoint{1.289529in}{2.090786in}}%
\pgfpathcurveto{\pgfqpoint{1.293647in}{2.086668in}}{\pgfqpoint{1.299234in}{2.084354in}}{\pgfqpoint{1.305057in}{2.084354in}}%
\pgfpathclose%
\pgfusepath{stroke,fill}%
\end{pgfscope}%
\begin{pgfscope}%
\pgfpathrectangle{\pgfqpoint{0.506010in}{1.121191in}}{\pgfqpoint{2.325000in}{1.400000in}} %
\pgfusepath{clip}%
\pgfsetbuttcap%
\pgfsetroundjoin%
\definecolor{currentfill}{rgb}{0.000000,0.500000,0.000000}%
\pgfsetfillcolor{currentfill}%
\pgfsetlinewidth{1.003750pt}%
\definecolor{currentstroke}{rgb}{0.000000,0.500000,0.000000}%
\pgfsetstrokecolor{currentstroke}%
\pgfsetdash{}{0pt}%
\pgfpathmoveto{\pgfqpoint{1.081382in}{2.134726in}}%
\pgfpathcurveto{\pgfqpoint{1.087206in}{2.134726in}}{\pgfqpoint{1.092792in}{2.137040in}}{\pgfqpoint{1.096910in}{2.141158in}}%
\pgfpathcurveto{\pgfqpoint{1.101028in}{2.145277in}}{\pgfqpoint{1.103342in}{2.150863in}}{\pgfqpoint{1.103342in}{2.156687in}}%
\pgfpathcurveto{\pgfqpoint{1.103342in}{2.162511in}}{\pgfqpoint{1.101028in}{2.168097in}}{\pgfqpoint{1.096910in}{2.172215in}}%
\pgfpathcurveto{\pgfqpoint{1.092792in}{2.176333in}}{\pgfqpoint{1.087206in}{2.178647in}}{\pgfqpoint{1.081382in}{2.178647in}}%
\pgfpathcurveto{\pgfqpoint{1.075558in}{2.178647in}}{\pgfqpoint{1.069972in}{2.176333in}}{\pgfqpoint{1.065854in}{2.172215in}}%
\pgfpathcurveto{\pgfqpoint{1.061736in}{2.168097in}}{\pgfqpoint{1.059422in}{2.162511in}}{\pgfqpoint{1.059422in}{2.156687in}}%
\pgfpathcurveto{\pgfqpoint{1.059422in}{2.150863in}}{\pgfqpoint{1.061736in}{2.145277in}}{\pgfqpoint{1.065854in}{2.141158in}}%
\pgfpathcurveto{\pgfqpoint{1.069972in}{2.137040in}}{\pgfqpoint{1.075558in}{2.134726in}}{\pgfqpoint{1.081382in}{2.134726in}}%
\pgfpathclose%
\pgfusepath{stroke,fill}%
\end{pgfscope}%
\begin{pgfscope}%
\pgfpathrectangle{\pgfqpoint{0.506010in}{1.121191in}}{\pgfqpoint{2.325000in}{1.400000in}} %
\pgfusepath{clip}%
\pgfsetbuttcap%
\pgfsetroundjoin%
\definecolor{currentfill}{rgb}{0.000000,0.500000,0.000000}%
\pgfsetfillcolor{currentfill}%
\pgfsetlinewidth{1.003750pt}%
\definecolor{currentstroke}{rgb}{0.000000,0.500000,0.000000}%
\pgfsetstrokecolor{currentstroke}%
\pgfsetdash{}{0pt}%
\pgfpathmoveto{\pgfqpoint{0.810364in}{6.139230in}}%
\pgfpathcurveto{\pgfqpoint{0.816188in}{6.139230in}}{\pgfqpoint{0.821774in}{6.141544in}}{\pgfqpoint{0.825892in}{6.145662in}}%
\pgfpathcurveto{\pgfqpoint{0.830010in}{6.149781in}}{\pgfqpoint{0.832324in}{6.155367in}}{\pgfqpoint{0.832324in}{6.161191in}}%
\pgfpathcurveto{\pgfqpoint{0.832324in}{6.167015in}}{\pgfqpoint{0.830010in}{6.172601in}}{\pgfqpoint{0.825892in}{6.176719in}}%
\pgfpathcurveto{\pgfqpoint{0.821774in}{6.180837in}}{\pgfqpoint{0.816188in}{6.183151in}}{\pgfqpoint{0.810364in}{6.183151in}}%
\pgfpathcurveto{\pgfqpoint{0.804540in}{6.183151in}}{\pgfqpoint{0.798954in}{6.180837in}}{\pgfqpoint{0.794836in}{6.176719in}}%
\pgfpathcurveto{\pgfqpoint{0.790718in}{6.172601in}}{\pgfqpoint{0.788404in}{6.167015in}}{\pgfqpoint{0.788404in}{6.161191in}}%
\pgfpathcurveto{\pgfqpoint{0.788404in}{6.155367in}}{\pgfqpoint{0.790718in}{6.149781in}}{\pgfqpoint{0.794836in}{6.145662in}}%
\pgfpathcurveto{\pgfqpoint{0.798954in}{6.141544in}}{\pgfqpoint{0.804540in}{6.139230in}}{\pgfqpoint{0.810364in}{6.139230in}}%
\pgfpathclose%
\pgfusepath{stroke,fill}%
\end{pgfscope}%
\begin{pgfscope}%
\pgfpathrectangle{\pgfqpoint{0.506010in}{1.121191in}}{\pgfqpoint{2.325000in}{1.400000in}} %
\pgfusepath{clip}%
\pgfsetbuttcap%
\pgfsetroundjoin%
\definecolor{currentfill}{rgb}{0.000000,0.500000,0.000000}%
\pgfsetfillcolor{currentfill}%
\pgfsetlinewidth{1.003750pt}%
\definecolor{currentstroke}{rgb}{0.000000,0.500000,0.000000}%
\pgfsetstrokecolor{currentstroke}%
\pgfsetdash{}{0pt}%
\pgfpathmoveto{\pgfqpoint{1.366871in}{6.139230in}}%
\pgfpathcurveto{\pgfqpoint{1.372694in}{6.139230in}}{\pgfqpoint{1.378281in}{6.141544in}}{\pgfqpoint{1.382399in}{6.145662in}}%
\pgfpathcurveto{\pgfqpoint{1.386517in}{6.149781in}}{\pgfqpoint{1.388831in}{6.155367in}}{\pgfqpoint{1.388831in}{6.161191in}}%
\pgfpathcurveto{\pgfqpoint{1.388831in}{6.167015in}}{\pgfqpoint{1.386517in}{6.172601in}}{\pgfqpoint{1.382399in}{6.176719in}}%
\pgfpathcurveto{\pgfqpoint{1.378281in}{6.180837in}}{\pgfqpoint{1.372694in}{6.183151in}}{\pgfqpoint{1.366871in}{6.183151in}}%
\pgfpathcurveto{\pgfqpoint{1.361047in}{6.183151in}}{\pgfqpoint{1.355460in}{6.180837in}}{\pgfqpoint{1.351342in}{6.176719in}}%
\pgfpathcurveto{\pgfqpoint{1.347224in}{6.172601in}}{\pgfqpoint{1.344910in}{6.167015in}}{\pgfqpoint{1.344910in}{6.161191in}}%
\pgfpathcurveto{\pgfqpoint{1.344910in}{6.155367in}}{\pgfqpoint{1.347224in}{6.149781in}}{\pgfqpoint{1.351342in}{6.145662in}}%
\pgfpathcurveto{\pgfqpoint{1.355460in}{6.141544in}}{\pgfqpoint{1.361047in}{6.139230in}}{\pgfqpoint{1.366871in}{6.139230in}}%
\pgfpathclose%
\pgfusepath{stroke,fill}%
\end{pgfscope}%
\begin{pgfscope}%
\pgfpathrectangle{\pgfqpoint{0.506010in}{1.121191in}}{\pgfqpoint{2.325000in}{1.400000in}} %
\pgfusepath{clip}%
\pgfsetbuttcap%
\pgfsetroundjoin%
\definecolor{currentfill}{rgb}{0.000000,0.500000,0.000000}%
\pgfsetfillcolor{currentfill}%
\pgfsetlinewidth{1.003750pt}%
\definecolor{currentstroke}{rgb}{0.000000,0.500000,0.000000}%
\pgfsetstrokecolor{currentstroke}%
\pgfsetdash{}{0pt}%
\pgfpathmoveto{\pgfqpoint{1.443680in}{6.139230in}}%
\pgfpathcurveto{\pgfqpoint{1.449504in}{6.139230in}}{\pgfqpoint{1.455090in}{6.141544in}}{\pgfqpoint{1.459208in}{6.145662in}}%
\pgfpathcurveto{\pgfqpoint{1.463326in}{6.149781in}}{\pgfqpoint{1.465640in}{6.155367in}}{\pgfqpoint{1.465640in}{6.161191in}}%
\pgfpathcurveto{\pgfqpoint{1.465640in}{6.167015in}}{\pgfqpoint{1.463326in}{6.172601in}}{\pgfqpoint{1.459208in}{6.176719in}}%
\pgfpathcurveto{\pgfqpoint{1.455090in}{6.180837in}}{\pgfqpoint{1.449504in}{6.183151in}}{\pgfqpoint{1.443680in}{6.183151in}}%
\pgfpathcurveto{\pgfqpoint{1.437856in}{6.183151in}}{\pgfqpoint{1.432270in}{6.180837in}}{\pgfqpoint{1.428152in}{6.176719in}}%
\pgfpathcurveto{\pgfqpoint{1.424034in}{6.172601in}}{\pgfqpoint{1.421720in}{6.167015in}}{\pgfqpoint{1.421720in}{6.161191in}}%
\pgfpathcurveto{\pgfqpoint{1.421720in}{6.155367in}}{\pgfqpoint{1.424034in}{6.149781in}}{\pgfqpoint{1.428152in}{6.145662in}}%
\pgfpathcurveto{\pgfqpoint{1.432270in}{6.141544in}}{\pgfqpoint{1.437856in}{6.139230in}}{\pgfqpoint{1.443680in}{6.139230in}}%
\pgfpathclose%
\pgfusepath{stroke,fill}%
\end{pgfscope}%
\begin{pgfscope}%
\pgfpathrectangle{\pgfqpoint{0.506010in}{1.121191in}}{\pgfqpoint{2.325000in}{1.400000in}} %
\pgfusepath{clip}%
\pgfsetbuttcap%
\pgfsetroundjoin%
\definecolor{currentfill}{rgb}{0.000000,0.500000,0.000000}%
\pgfsetfillcolor{currentfill}%
\pgfsetlinewidth{1.003750pt}%
\definecolor{currentstroke}{rgb}{0.000000,0.500000,0.000000}%
\pgfsetstrokecolor{currentstroke}%
\pgfsetdash{}{0pt}%
\pgfpathmoveto{\pgfqpoint{1.406887in}{6.139230in}}%
\pgfpathcurveto{\pgfqpoint{1.412711in}{6.139230in}}{\pgfqpoint{1.418297in}{6.141544in}}{\pgfqpoint{1.422415in}{6.145662in}}%
\pgfpathcurveto{\pgfqpoint{1.426533in}{6.149781in}}{\pgfqpoint{1.428847in}{6.155367in}}{\pgfqpoint{1.428847in}{6.161191in}}%
\pgfpathcurveto{\pgfqpoint{1.428847in}{6.167015in}}{\pgfqpoint{1.426533in}{6.172601in}}{\pgfqpoint{1.422415in}{6.176719in}}%
\pgfpathcurveto{\pgfqpoint{1.418297in}{6.180837in}}{\pgfqpoint{1.412711in}{6.183151in}}{\pgfqpoint{1.406887in}{6.183151in}}%
\pgfpathcurveto{\pgfqpoint{1.401063in}{6.183151in}}{\pgfqpoint{1.395477in}{6.180837in}}{\pgfqpoint{1.391359in}{6.176719in}}%
\pgfpathcurveto{\pgfqpoint{1.387240in}{6.172601in}}{\pgfqpoint{1.384927in}{6.167015in}}{\pgfqpoint{1.384927in}{6.161191in}}%
\pgfpathcurveto{\pgfqpoint{1.384927in}{6.155367in}}{\pgfqpoint{1.387240in}{6.149781in}}{\pgfqpoint{1.391359in}{6.145662in}}%
\pgfpathcurveto{\pgfqpoint{1.395477in}{6.141544in}}{\pgfqpoint{1.401063in}{6.139230in}}{\pgfqpoint{1.406887in}{6.139230in}}%
\pgfpathclose%
\pgfusepath{stroke,fill}%
\end{pgfscope}%
\begin{pgfscope}%
\pgfpathrectangle{\pgfqpoint{0.506010in}{1.121191in}}{\pgfqpoint{2.325000in}{1.400000in}} %
\pgfusepath{clip}%
\pgfsetbuttcap%
\pgfsetroundjoin%
\definecolor{currentfill}{rgb}{0.000000,0.500000,0.000000}%
\pgfsetfillcolor{currentfill}%
\pgfsetlinewidth{1.003750pt}%
\definecolor{currentstroke}{rgb}{0.000000,0.500000,0.000000}%
\pgfsetstrokecolor{currentstroke}%
\pgfsetdash{}{0pt}%
\pgfpathmoveto{\pgfqpoint{1.108974in}{6.139230in}}%
\pgfpathcurveto{\pgfqpoint{1.114798in}{6.139230in}}{\pgfqpoint{1.120384in}{6.141544in}}{\pgfqpoint{1.124502in}{6.145662in}}%
\pgfpathcurveto{\pgfqpoint{1.128620in}{6.149781in}}{\pgfqpoint{1.130934in}{6.155367in}}{\pgfqpoint{1.130934in}{6.161191in}}%
\pgfpathcurveto{\pgfqpoint{1.130934in}{6.167015in}}{\pgfqpoint{1.128620in}{6.172601in}}{\pgfqpoint{1.124502in}{6.176719in}}%
\pgfpathcurveto{\pgfqpoint{1.120384in}{6.180837in}}{\pgfqpoint{1.114798in}{6.183151in}}{\pgfqpoint{1.108974in}{6.183151in}}%
\pgfpathcurveto{\pgfqpoint{1.103150in}{6.183151in}}{\pgfqpoint{1.097564in}{6.180837in}}{\pgfqpoint{1.093446in}{6.176719in}}%
\pgfpathcurveto{\pgfqpoint{1.089327in}{6.172601in}}{\pgfqpoint{1.087014in}{6.167015in}}{\pgfqpoint{1.087014in}{6.161191in}}%
\pgfpathcurveto{\pgfqpoint{1.087014in}{6.155367in}}{\pgfqpoint{1.089327in}{6.149781in}}{\pgfqpoint{1.093446in}{6.145662in}}%
\pgfpathcurveto{\pgfqpoint{1.097564in}{6.141544in}}{\pgfqpoint{1.103150in}{6.139230in}}{\pgfqpoint{1.108974in}{6.139230in}}%
\pgfpathclose%
\pgfusepath{stroke,fill}%
\end{pgfscope}%
\begin{pgfscope}%
\pgfpathrectangle{\pgfqpoint{0.506010in}{1.121191in}}{\pgfqpoint{2.325000in}{1.400000in}} %
\pgfusepath{clip}%
\pgfsetbuttcap%
\pgfsetroundjoin%
\definecolor{currentfill}{rgb}{0.000000,0.500000,0.000000}%
\pgfsetfillcolor{currentfill}%
\pgfsetlinewidth{1.003750pt}%
\definecolor{currentstroke}{rgb}{0.000000,0.500000,0.000000}%
\pgfsetstrokecolor{currentstroke}%
\pgfsetdash{}{0pt}%
\pgfpathmoveto{\pgfqpoint{0.821928in}{6.139230in}}%
\pgfpathcurveto{\pgfqpoint{0.827752in}{6.139230in}}{\pgfqpoint{0.833338in}{6.141544in}}{\pgfqpoint{0.837456in}{6.145662in}}%
\pgfpathcurveto{\pgfqpoint{0.841574in}{6.149781in}}{\pgfqpoint{0.843888in}{6.155367in}}{\pgfqpoint{0.843888in}{6.161191in}}%
\pgfpathcurveto{\pgfqpoint{0.843888in}{6.167015in}}{\pgfqpoint{0.841574in}{6.172601in}}{\pgfqpoint{0.837456in}{6.176719in}}%
\pgfpathcurveto{\pgfqpoint{0.833338in}{6.180837in}}{\pgfqpoint{0.827752in}{6.183151in}}{\pgfqpoint{0.821928in}{6.183151in}}%
\pgfpathcurveto{\pgfqpoint{0.816104in}{6.183151in}}{\pgfqpoint{0.810518in}{6.180837in}}{\pgfqpoint{0.806400in}{6.176719in}}%
\pgfpathcurveto{\pgfqpoint{0.802281in}{6.172601in}}{\pgfqpoint{0.799968in}{6.167015in}}{\pgfqpoint{0.799968in}{6.161191in}}%
\pgfpathcurveto{\pgfqpoint{0.799968in}{6.155367in}}{\pgfqpoint{0.802281in}{6.149781in}}{\pgfqpoint{0.806400in}{6.145662in}}%
\pgfpathcurveto{\pgfqpoint{0.810518in}{6.141544in}}{\pgfqpoint{0.816104in}{6.139230in}}{\pgfqpoint{0.821928in}{6.139230in}}%
\pgfpathclose%
\pgfusepath{stroke,fill}%
\end{pgfscope}%
\begin{pgfscope}%
\pgfpathrectangle{\pgfqpoint{0.506010in}{1.121191in}}{\pgfqpoint{2.325000in}{1.400000in}} %
\pgfusepath{clip}%
\pgfsetbuttcap%
\pgfsetroundjoin%
\definecolor{currentfill}{rgb}{0.000000,0.500000,0.000000}%
\pgfsetfillcolor{currentfill}%
\pgfsetlinewidth{1.003750pt}%
\definecolor{currentstroke}{rgb}{0.000000,0.500000,0.000000}%
\pgfsetstrokecolor{currentstroke}%
\pgfsetdash{}{0pt}%
\pgfpathmoveto{\pgfqpoint{1.443680in}{6.139230in}}%
\pgfpathcurveto{\pgfqpoint{1.449504in}{6.139230in}}{\pgfqpoint{1.455090in}{6.141544in}}{\pgfqpoint{1.459208in}{6.145662in}}%
\pgfpathcurveto{\pgfqpoint{1.463326in}{6.149781in}}{\pgfqpoint{1.465640in}{6.155367in}}{\pgfqpoint{1.465640in}{6.161191in}}%
\pgfpathcurveto{\pgfqpoint{1.465640in}{6.167015in}}{\pgfqpoint{1.463326in}{6.172601in}}{\pgfqpoint{1.459208in}{6.176719in}}%
\pgfpathcurveto{\pgfqpoint{1.455090in}{6.180837in}}{\pgfqpoint{1.449504in}{6.183151in}}{\pgfqpoint{1.443680in}{6.183151in}}%
\pgfpathcurveto{\pgfqpoint{1.437856in}{6.183151in}}{\pgfqpoint{1.432270in}{6.180837in}}{\pgfqpoint{1.428152in}{6.176719in}}%
\pgfpathcurveto{\pgfqpoint{1.424034in}{6.172601in}}{\pgfqpoint{1.421720in}{6.167015in}}{\pgfqpoint{1.421720in}{6.161191in}}%
\pgfpathcurveto{\pgfqpoint{1.421720in}{6.155367in}}{\pgfqpoint{1.424034in}{6.149781in}}{\pgfqpoint{1.428152in}{6.145662in}}%
\pgfpathcurveto{\pgfqpoint{1.432270in}{6.141544in}}{\pgfqpoint{1.437856in}{6.139230in}}{\pgfqpoint{1.443680in}{6.139230in}}%
\pgfpathclose%
\pgfusepath{stroke,fill}%
\end{pgfscope}%
\begin{pgfscope}%
\pgfpathrectangle{\pgfqpoint{0.506010in}{1.121191in}}{\pgfqpoint{2.325000in}{1.400000in}} %
\pgfusepath{clip}%
\pgfsetbuttcap%
\pgfsetroundjoin%
\definecolor{currentfill}{rgb}{0.000000,0.500000,0.000000}%
\pgfsetfillcolor{currentfill}%
\pgfsetlinewidth{1.003750pt}%
\definecolor{currentstroke}{rgb}{0.000000,0.500000,0.000000}%
\pgfsetstrokecolor{currentstroke}%
\pgfsetdash{}{0pt}%
\pgfpathmoveto{\pgfqpoint{1.379114in}{6.139230in}}%
\pgfpathcurveto{\pgfqpoint{1.384937in}{6.139230in}}{\pgfqpoint{1.390524in}{6.141544in}}{\pgfqpoint{1.394642in}{6.145662in}}%
\pgfpathcurveto{\pgfqpoint{1.398760in}{6.149781in}}{\pgfqpoint{1.401074in}{6.155367in}}{\pgfqpoint{1.401074in}{6.161191in}}%
\pgfpathcurveto{\pgfqpoint{1.401074in}{6.167015in}}{\pgfqpoint{1.398760in}{6.172601in}}{\pgfqpoint{1.394642in}{6.176719in}}%
\pgfpathcurveto{\pgfqpoint{1.390524in}{6.180837in}}{\pgfqpoint{1.384937in}{6.183151in}}{\pgfqpoint{1.379114in}{6.183151in}}%
\pgfpathcurveto{\pgfqpoint{1.373290in}{6.183151in}}{\pgfqpoint{1.367703in}{6.180837in}}{\pgfqpoint{1.363585in}{6.176719in}}%
\pgfpathcurveto{\pgfqpoint{1.359467in}{6.172601in}}{\pgfqpoint{1.357153in}{6.167015in}}{\pgfqpoint{1.357153in}{6.161191in}}%
\pgfpathcurveto{\pgfqpoint{1.357153in}{6.155367in}}{\pgfqpoint{1.359467in}{6.149781in}}{\pgfqpoint{1.363585in}{6.145662in}}%
\pgfpathcurveto{\pgfqpoint{1.367703in}{6.141544in}}{\pgfqpoint{1.373290in}{6.139230in}}{\pgfqpoint{1.379114in}{6.139230in}}%
\pgfpathclose%
\pgfusepath{stroke,fill}%
\end{pgfscope}%
\begin{pgfscope}%
\pgfpathrectangle{\pgfqpoint{0.506010in}{1.121191in}}{\pgfqpoint{2.325000in}{1.400000in}} %
\pgfusepath{clip}%
\pgfsetbuttcap%
\pgfsetroundjoin%
\definecolor{currentfill}{rgb}{0.000000,0.500000,0.000000}%
\pgfsetfillcolor{currentfill}%
\pgfsetlinewidth{1.003750pt}%
\definecolor{currentstroke}{rgb}{0.000000,0.500000,0.000000}%
\pgfsetstrokecolor{currentstroke}%
\pgfsetdash{}{0pt}%
\pgfpathmoveto{\pgfqpoint{1.081382in}{6.139230in}}%
\pgfpathcurveto{\pgfqpoint{1.087206in}{6.139230in}}{\pgfqpoint{1.092792in}{6.141544in}}{\pgfqpoint{1.096910in}{6.145662in}}%
\pgfpathcurveto{\pgfqpoint{1.101028in}{6.149781in}}{\pgfqpoint{1.103342in}{6.155367in}}{\pgfqpoint{1.103342in}{6.161191in}}%
\pgfpathcurveto{\pgfqpoint{1.103342in}{6.167015in}}{\pgfqpoint{1.101028in}{6.172601in}}{\pgfqpoint{1.096910in}{6.176719in}}%
\pgfpathcurveto{\pgfqpoint{1.092792in}{6.180837in}}{\pgfqpoint{1.087206in}{6.183151in}}{\pgfqpoint{1.081382in}{6.183151in}}%
\pgfpathcurveto{\pgfqpoint{1.075558in}{6.183151in}}{\pgfqpoint{1.069972in}{6.180837in}}{\pgfqpoint{1.065854in}{6.176719in}}%
\pgfpathcurveto{\pgfqpoint{1.061736in}{6.172601in}}{\pgfqpoint{1.059422in}{6.167015in}}{\pgfqpoint{1.059422in}{6.161191in}}%
\pgfpathcurveto{\pgfqpoint{1.059422in}{6.155367in}}{\pgfqpoint{1.061736in}{6.149781in}}{\pgfqpoint{1.065854in}{6.145662in}}%
\pgfpathcurveto{\pgfqpoint{1.069972in}{6.141544in}}{\pgfqpoint{1.075558in}{6.139230in}}{\pgfqpoint{1.081382in}{6.139230in}}%
\pgfpathclose%
\pgfusepath{stroke,fill}%
\end{pgfscope}%
\begin{pgfscope}%
\pgfpathrectangle{\pgfqpoint{0.506010in}{1.121191in}}{\pgfqpoint{2.325000in}{1.400000in}} %
\pgfusepath{clip}%
\pgfsetbuttcap%
\pgfsetroundjoin%
\definecolor{currentfill}{rgb}{0.000000,0.500000,0.000000}%
\pgfsetfillcolor{currentfill}%
\pgfsetlinewidth{1.003750pt}%
\definecolor{currentstroke}{rgb}{0.000000,0.500000,0.000000}%
\pgfsetstrokecolor{currentstroke}%
\pgfsetdash{}{0pt}%
\pgfpathmoveto{\pgfqpoint{1.108974in}{6.139230in}}%
\pgfpathcurveto{\pgfqpoint{1.114798in}{6.139230in}}{\pgfqpoint{1.120384in}{6.141544in}}{\pgfqpoint{1.124502in}{6.145662in}}%
\pgfpathcurveto{\pgfqpoint{1.128620in}{6.149781in}}{\pgfqpoint{1.130934in}{6.155367in}}{\pgfqpoint{1.130934in}{6.161191in}}%
\pgfpathcurveto{\pgfqpoint{1.130934in}{6.167015in}}{\pgfqpoint{1.128620in}{6.172601in}}{\pgfqpoint{1.124502in}{6.176719in}}%
\pgfpathcurveto{\pgfqpoint{1.120384in}{6.180837in}}{\pgfqpoint{1.114798in}{6.183151in}}{\pgfqpoint{1.108974in}{6.183151in}}%
\pgfpathcurveto{\pgfqpoint{1.103150in}{6.183151in}}{\pgfqpoint{1.097564in}{6.180837in}}{\pgfqpoint{1.093446in}{6.176719in}}%
\pgfpathcurveto{\pgfqpoint{1.089327in}{6.172601in}}{\pgfqpoint{1.087014in}{6.167015in}}{\pgfqpoint{1.087014in}{6.161191in}}%
\pgfpathcurveto{\pgfqpoint{1.087014in}{6.155367in}}{\pgfqpoint{1.089327in}{6.149781in}}{\pgfqpoint{1.093446in}{6.145662in}}%
\pgfpathcurveto{\pgfqpoint{1.097564in}{6.141544in}}{\pgfqpoint{1.103150in}{6.139230in}}{\pgfqpoint{1.108974in}{6.139230in}}%
\pgfpathclose%
\pgfusepath{stroke,fill}%
\end{pgfscope}%
\begin{pgfscope}%
\pgfpathrectangle{\pgfqpoint{0.506010in}{1.121191in}}{\pgfqpoint{2.325000in}{1.400000in}} %
\pgfusepath{clip}%
\pgfsetbuttcap%
\pgfsetroundjoin%
\definecolor{currentfill}{rgb}{0.000000,0.500000,0.000000}%
\pgfsetfillcolor{currentfill}%
\pgfsetlinewidth{1.003750pt}%
\definecolor{currentstroke}{rgb}{0.000000,0.500000,0.000000}%
\pgfsetstrokecolor{currentstroke}%
\pgfsetdash{}{0pt}%
\pgfpathmoveto{\pgfqpoint{0.681255in}{6.139230in}}%
\pgfpathcurveto{\pgfqpoint{0.687079in}{6.139230in}}{\pgfqpoint{0.692665in}{6.141544in}}{\pgfqpoint{0.696783in}{6.145662in}}%
\pgfpathcurveto{\pgfqpoint{0.700901in}{6.149781in}}{\pgfqpoint{0.703215in}{6.155367in}}{\pgfqpoint{0.703215in}{6.161191in}}%
\pgfpathcurveto{\pgfqpoint{0.703215in}{6.167015in}}{\pgfqpoint{0.700901in}{6.172601in}}{\pgfqpoint{0.696783in}{6.176719in}}%
\pgfpathcurveto{\pgfqpoint{0.692665in}{6.180837in}}{\pgfqpoint{0.687079in}{6.183151in}}{\pgfqpoint{0.681255in}{6.183151in}}%
\pgfpathcurveto{\pgfqpoint{0.675431in}{6.183151in}}{\pgfqpoint{0.669845in}{6.180837in}}{\pgfqpoint{0.665727in}{6.176719in}}%
\pgfpathcurveto{\pgfqpoint{0.661609in}{6.172601in}}{\pgfqpoint{0.659295in}{6.167015in}}{\pgfqpoint{0.659295in}{6.161191in}}%
\pgfpathcurveto{\pgfqpoint{0.659295in}{6.155367in}}{\pgfqpoint{0.661609in}{6.149781in}}{\pgfqpoint{0.665727in}{6.145662in}}%
\pgfpathcurveto{\pgfqpoint{0.669845in}{6.141544in}}{\pgfqpoint{0.675431in}{6.139230in}}{\pgfqpoint{0.681255in}{6.139230in}}%
\pgfpathclose%
\pgfusepath{stroke,fill}%
\end{pgfscope}%
\begin{pgfscope}%
\pgfpathrectangle{\pgfqpoint{0.506010in}{1.121191in}}{\pgfqpoint{2.325000in}{1.400000in}} %
\pgfusepath{clip}%
\pgfsetbuttcap%
\pgfsetroundjoin%
\definecolor{currentfill}{rgb}{0.000000,0.500000,0.000000}%
\pgfsetfillcolor{currentfill}%
\pgfsetlinewidth{1.003750pt}%
\definecolor{currentstroke}{rgb}{0.000000,0.500000,0.000000}%
\pgfsetstrokecolor{currentstroke}%
\pgfsetdash{}{0pt}%
\pgfpathmoveto{\pgfqpoint{1.081382in}{6.139230in}}%
\pgfpathcurveto{\pgfqpoint{1.087206in}{6.139230in}}{\pgfqpoint{1.092792in}{6.141544in}}{\pgfqpoint{1.096910in}{6.145662in}}%
\pgfpathcurveto{\pgfqpoint{1.101028in}{6.149781in}}{\pgfqpoint{1.103342in}{6.155367in}}{\pgfqpoint{1.103342in}{6.161191in}}%
\pgfpathcurveto{\pgfqpoint{1.103342in}{6.167015in}}{\pgfqpoint{1.101028in}{6.172601in}}{\pgfqpoint{1.096910in}{6.176719in}}%
\pgfpathcurveto{\pgfqpoint{1.092792in}{6.180837in}}{\pgfqpoint{1.087206in}{6.183151in}}{\pgfqpoint{1.081382in}{6.183151in}}%
\pgfpathcurveto{\pgfqpoint{1.075558in}{6.183151in}}{\pgfqpoint{1.069972in}{6.180837in}}{\pgfqpoint{1.065854in}{6.176719in}}%
\pgfpathcurveto{\pgfqpoint{1.061736in}{6.172601in}}{\pgfqpoint{1.059422in}{6.167015in}}{\pgfqpoint{1.059422in}{6.161191in}}%
\pgfpathcurveto{\pgfqpoint{1.059422in}{6.155367in}}{\pgfqpoint{1.061736in}{6.149781in}}{\pgfqpoint{1.065854in}{6.145662in}}%
\pgfpathcurveto{\pgfqpoint{1.069972in}{6.141544in}}{\pgfqpoint{1.075558in}{6.139230in}}{\pgfqpoint{1.081382in}{6.139230in}}%
\pgfpathclose%
\pgfusepath{stroke,fill}%
\end{pgfscope}%
\begin{pgfscope}%
\pgfpathrectangle{\pgfqpoint{0.506010in}{1.121191in}}{\pgfqpoint{2.325000in}{1.400000in}} %
\pgfusepath{clip}%
\pgfsetbuttcap%
\pgfsetroundjoin%
\definecolor{currentfill}{rgb}{0.000000,0.500000,0.000000}%
\pgfsetfillcolor{currentfill}%
\pgfsetlinewidth{1.003750pt}%
\definecolor{currentstroke}{rgb}{0.000000,0.500000,0.000000}%
\pgfsetstrokecolor{currentstroke}%
\pgfsetdash{}{0pt}%
\pgfpathmoveto{\pgfqpoint{1.081382in}{6.139230in}}%
\pgfpathcurveto{\pgfqpoint{1.087206in}{6.139230in}}{\pgfqpoint{1.092792in}{6.141544in}}{\pgfqpoint{1.096910in}{6.145662in}}%
\pgfpathcurveto{\pgfqpoint{1.101028in}{6.149781in}}{\pgfqpoint{1.103342in}{6.155367in}}{\pgfqpoint{1.103342in}{6.161191in}}%
\pgfpathcurveto{\pgfqpoint{1.103342in}{6.167015in}}{\pgfqpoint{1.101028in}{6.172601in}}{\pgfqpoint{1.096910in}{6.176719in}}%
\pgfpathcurveto{\pgfqpoint{1.092792in}{6.180837in}}{\pgfqpoint{1.087206in}{6.183151in}}{\pgfqpoint{1.081382in}{6.183151in}}%
\pgfpathcurveto{\pgfqpoint{1.075558in}{6.183151in}}{\pgfqpoint{1.069972in}{6.180837in}}{\pgfqpoint{1.065854in}{6.176719in}}%
\pgfpathcurveto{\pgfqpoint{1.061736in}{6.172601in}}{\pgfqpoint{1.059422in}{6.167015in}}{\pgfqpoint{1.059422in}{6.161191in}}%
\pgfpathcurveto{\pgfqpoint{1.059422in}{6.155367in}}{\pgfqpoint{1.061736in}{6.149781in}}{\pgfqpoint{1.065854in}{6.145662in}}%
\pgfpathcurveto{\pgfqpoint{1.069972in}{6.141544in}}{\pgfqpoint{1.075558in}{6.139230in}}{\pgfqpoint{1.081382in}{6.139230in}}%
\pgfpathclose%
\pgfusepath{stroke,fill}%
\end{pgfscope}%
\begin{pgfscope}%
\pgfpathrectangle{\pgfqpoint{0.506010in}{1.121191in}}{\pgfqpoint{2.325000in}{1.400000in}} %
\pgfusepath{clip}%
\pgfsetbuttcap%
\pgfsetroundjoin%
\definecolor{currentfill}{rgb}{0.000000,0.500000,0.000000}%
\pgfsetfillcolor{currentfill}%
\pgfsetlinewidth{1.003750pt}%
\definecolor{currentstroke}{rgb}{0.000000,0.500000,0.000000}%
\pgfsetstrokecolor{currentstroke}%
\pgfsetdash{}{0pt}%
\pgfpathmoveto{\pgfqpoint{1.007822in}{6.139230in}}%
\pgfpathcurveto{\pgfqpoint{1.013646in}{6.139230in}}{\pgfqpoint{1.019232in}{6.141544in}}{\pgfqpoint{1.023351in}{6.145662in}}%
\pgfpathcurveto{\pgfqpoint{1.027469in}{6.149781in}}{\pgfqpoint{1.029783in}{6.155367in}}{\pgfqpoint{1.029783in}{6.161191in}}%
\pgfpathcurveto{\pgfqpoint{1.029783in}{6.167015in}}{\pgfqpoint{1.027469in}{6.172601in}}{\pgfqpoint{1.023351in}{6.176719in}}%
\pgfpathcurveto{\pgfqpoint{1.019232in}{6.180837in}}{\pgfqpoint{1.013646in}{6.183151in}}{\pgfqpoint{1.007822in}{6.183151in}}%
\pgfpathcurveto{\pgfqpoint{1.001998in}{6.183151in}}{\pgfqpoint{0.996412in}{6.180837in}}{\pgfqpoint{0.992294in}{6.176719in}}%
\pgfpathcurveto{\pgfqpoint{0.988176in}{6.172601in}}{\pgfqpoint{0.985862in}{6.167015in}}{\pgfqpoint{0.985862in}{6.161191in}}%
\pgfpathcurveto{\pgfqpoint{0.985862in}{6.155367in}}{\pgfqpoint{0.988176in}{6.149781in}}{\pgfqpoint{0.992294in}{6.145662in}}%
\pgfpathcurveto{\pgfqpoint{0.996412in}{6.141544in}}{\pgfqpoint{1.001998in}{6.139230in}}{\pgfqpoint{1.007822in}{6.139230in}}%
\pgfpathclose%
\pgfusepath{stroke,fill}%
\end{pgfscope}%
\begin{pgfscope}%
\pgfpathrectangle{\pgfqpoint{0.506010in}{1.121191in}}{\pgfqpoint{2.325000in}{1.400000in}} %
\pgfusepath{clip}%
\pgfsetbuttcap%
\pgfsetroundjoin%
\definecolor{currentfill}{rgb}{0.000000,0.500000,0.000000}%
\pgfsetfillcolor{currentfill}%
\pgfsetlinewidth{1.003750pt}%
\definecolor{currentstroke}{rgb}{0.000000,0.500000,0.000000}%
\pgfsetstrokecolor{currentstroke}%
\pgfsetdash{}{0pt}%
\pgfpathmoveto{\pgfqpoint{0.986187in}{6.139230in}}%
\pgfpathcurveto{\pgfqpoint{0.992011in}{6.139230in}}{\pgfqpoint{0.997597in}{6.141544in}}{\pgfqpoint{1.001716in}{6.145662in}}%
\pgfpathcurveto{\pgfqpoint{1.005834in}{6.149781in}}{\pgfqpoint{1.008148in}{6.155367in}}{\pgfqpoint{1.008148in}{6.161191in}}%
\pgfpathcurveto{\pgfqpoint{1.008148in}{6.167015in}}{\pgfqpoint{1.005834in}{6.172601in}}{\pgfqpoint{1.001716in}{6.176719in}}%
\pgfpathcurveto{\pgfqpoint{0.997597in}{6.180837in}}{\pgfqpoint{0.992011in}{6.183151in}}{\pgfqpoint{0.986187in}{6.183151in}}%
\pgfpathcurveto{\pgfqpoint{0.980363in}{6.183151in}}{\pgfqpoint{0.974777in}{6.180837in}}{\pgfqpoint{0.970659in}{6.176719in}}%
\pgfpathcurveto{\pgfqpoint{0.966541in}{6.172601in}}{\pgfqpoint{0.964227in}{6.167015in}}{\pgfqpoint{0.964227in}{6.161191in}}%
\pgfpathcurveto{\pgfqpoint{0.964227in}{6.155367in}}{\pgfqpoint{0.966541in}{6.149781in}}{\pgfqpoint{0.970659in}{6.145662in}}%
\pgfpathcurveto{\pgfqpoint{0.974777in}{6.141544in}}{\pgfqpoint{0.980363in}{6.139230in}}{\pgfqpoint{0.986187in}{6.139230in}}%
\pgfpathclose%
\pgfusepath{stroke,fill}%
\end{pgfscope}%
\begin{pgfscope}%
\pgfpathrectangle{\pgfqpoint{0.506010in}{1.121191in}}{\pgfqpoint{2.325000in}{1.400000in}} %
\pgfusepath{clip}%
\pgfsetbuttcap%
\pgfsetroundjoin%
\definecolor{currentfill}{rgb}{0.000000,0.500000,0.000000}%
\pgfsetfillcolor{currentfill}%
\pgfsetlinewidth{1.003750pt}%
\definecolor{currentstroke}{rgb}{0.000000,0.500000,0.000000}%
\pgfsetstrokecolor{currentstroke}%
\pgfsetdash{}{0pt}%
\pgfpathmoveto{\pgfqpoint{1.108974in}{6.139230in}}%
\pgfpathcurveto{\pgfqpoint{1.114798in}{6.139230in}}{\pgfqpoint{1.120384in}{6.141544in}}{\pgfqpoint{1.124502in}{6.145662in}}%
\pgfpathcurveto{\pgfqpoint{1.128620in}{6.149781in}}{\pgfqpoint{1.130934in}{6.155367in}}{\pgfqpoint{1.130934in}{6.161191in}}%
\pgfpathcurveto{\pgfqpoint{1.130934in}{6.167015in}}{\pgfqpoint{1.128620in}{6.172601in}}{\pgfqpoint{1.124502in}{6.176719in}}%
\pgfpathcurveto{\pgfqpoint{1.120384in}{6.180837in}}{\pgfqpoint{1.114798in}{6.183151in}}{\pgfqpoint{1.108974in}{6.183151in}}%
\pgfpathcurveto{\pgfqpoint{1.103150in}{6.183151in}}{\pgfqpoint{1.097564in}{6.180837in}}{\pgfqpoint{1.093446in}{6.176719in}}%
\pgfpathcurveto{\pgfqpoint{1.089327in}{6.172601in}}{\pgfqpoint{1.087014in}{6.167015in}}{\pgfqpoint{1.087014in}{6.161191in}}%
\pgfpathcurveto{\pgfqpoint{1.087014in}{6.155367in}}{\pgfqpoint{1.089327in}{6.149781in}}{\pgfqpoint{1.093446in}{6.145662in}}%
\pgfpathcurveto{\pgfqpoint{1.097564in}{6.141544in}}{\pgfqpoint{1.103150in}{6.139230in}}{\pgfqpoint{1.108974in}{6.139230in}}%
\pgfpathclose%
\pgfusepath{stroke,fill}%
\end{pgfscope}%
\begin{pgfscope}%
\pgfpathrectangle{\pgfqpoint{0.506010in}{1.121191in}}{\pgfqpoint{2.325000in}{1.400000in}} %
\pgfusepath{clip}%
\pgfsetbuttcap%
\pgfsetroundjoin%
\definecolor{currentfill}{rgb}{0.000000,0.500000,0.000000}%
\pgfsetfillcolor{currentfill}%
\pgfsetlinewidth{1.003750pt}%
\definecolor{currentstroke}{rgb}{0.000000,0.500000,0.000000}%
\pgfsetstrokecolor{currentstroke}%
\pgfsetdash{}{0pt}%
\pgfpathmoveto{\pgfqpoint{1.081382in}{6.139230in}}%
\pgfpathcurveto{\pgfqpoint{1.087206in}{6.139230in}}{\pgfqpoint{1.092792in}{6.141544in}}{\pgfqpoint{1.096910in}{6.145662in}}%
\pgfpathcurveto{\pgfqpoint{1.101028in}{6.149781in}}{\pgfqpoint{1.103342in}{6.155367in}}{\pgfqpoint{1.103342in}{6.161191in}}%
\pgfpathcurveto{\pgfqpoint{1.103342in}{6.167015in}}{\pgfqpoint{1.101028in}{6.172601in}}{\pgfqpoint{1.096910in}{6.176719in}}%
\pgfpathcurveto{\pgfqpoint{1.092792in}{6.180837in}}{\pgfqpoint{1.087206in}{6.183151in}}{\pgfqpoint{1.081382in}{6.183151in}}%
\pgfpathcurveto{\pgfqpoint{1.075558in}{6.183151in}}{\pgfqpoint{1.069972in}{6.180837in}}{\pgfqpoint{1.065854in}{6.176719in}}%
\pgfpathcurveto{\pgfqpoint{1.061736in}{6.172601in}}{\pgfqpoint{1.059422in}{6.167015in}}{\pgfqpoint{1.059422in}{6.161191in}}%
\pgfpathcurveto{\pgfqpoint{1.059422in}{6.155367in}}{\pgfqpoint{1.061736in}{6.149781in}}{\pgfqpoint{1.065854in}{6.145662in}}%
\pgfpathcurveto{\pgfqpoint{1.069972in}{6.141544in}}{\pgfqpoint{1.075558in}{6.139230in}}{\pgfqpoint{1.081382in}{6.139230in}}%
\pgfpathclose%
\pgfusepath{stroke,fill}%
\end{pgfscope}%
\begin{pgfscope}%
\pgfpathrectangle{\pgfqpoint{0.506010in}{1.121191in}}{\pgfqpoint{2.325000in}{1.400000in}} %
\pgfusepath{clip}%
\pgfsetbuttcap%
\pgfsetroundjoin%
\definecolor{currentfill}{rgb}{0.000000,0.500000,0.000000}%
\pgfsetfillcolor{currentfill}%
\pgfsetlinewidth{1.003750pt}%
\definecolor{currentstroke}{rgb}{0.000000,0.500000,0.000000}%
\pgfsetstrokecolor{currentstroke}%
\pgfsetdash{}{0pt}%
\pgfpathmoveto{\pgfqpoint{1.474029in}{6.139230in}}%
\pgfpathcurveto{\pgfqpoint{1.479853in}{6.139230in}}{\pgfqpoint{1.485439in}{6.141544in}}{\pgfqpoint{1.489558in}{6.145662in}}%
\pgfpathcurveto{\pgfqpoint{1.493676in}{6.149781in}}{\pgfqpoint{1.495990in}{6.155367in}}{\pgfqpoint{1.495990in}{6.161191in}}%
\pgfpathcurveto{\pgfqpoint{1.495990in}{6.167015in}}{\pgfqpoint{1.493676in}{6.172601in}}{\pgfqpoint{1.489558in}{6.176719in}}%
\pgfpathcurveto{\pgfqpoint{1.485439in}{6.180837in}}{\pgfqpoint{1.479853in}{6.183151in}}{\pgfqpoint{1.474029in}{6.183151in}}%
\pgfpathcurveto{\pgfqpoint{1.468205in}{6.183151in}}{\pgfqpoint{1.462619in}{6.180837in}}{\pgfqpoint{1.458501in}{6.176719in}}%
\pgfpathcurveto{\pgfqpoint{1.454383in}{6.172601in}}{\pgfqpoint{1.452069in}{6.167015in}}{\pgfqpoint{1.452069in}{6.161191in}}%
\pgfpathcurveto{\pgfqpoint{1.452069in}{6.155367in}}{\pgfqpoint{1.454383in}{6.149781in}}{\pgfqpoint{1.458501in}{6.145662in}}%
\pgfpathcurveto{\pgfqpoint{1.462619in}{6.141544in}}{\pgfqpoint{1.468205in}{6.139230in}}{\pgfqpoint{1.474029in}{6.139230in}}%
\pgfpathclose%
\pgfusepath{stroke,fill}%
\end{pgfscope}%
\begin{pgfscope}%
\pgfpathrectangle{\pgfqpoint{0.506010in}{1.121191in}}{\pgfqpoint{2.325000in}{1.400000in}} %
\pgfusepath{clip}%
\pgfsetbuttcap%
\pgfsetroundjoin%
\definecolor{currentfill}{rgb}{0.000000,0.500000,0.000000}%
\pgfsetfillcolor{currentfill}%
\pgfsetlinewidth{1.003750pt}%
\definecolor{currentstroke}{rgb}{0.000000,0.500000,0.000000}%
\pgfsetstrokecolor{currentstroke}%
\pgfsetdash{}{0pt}%
\pgfpathmoveto{\pgfqpoint{1.007822in}{6.139230in}}%
\pgfpathcurveto{\pgfqpoint{1.013646in}{6.139230in}}{\pgfqpoint{1.019232in}{6.141544in}}{\pgfqpoint{1.023351in}{6.145662in}}%
\pgfpathcurveto{\pgfqpoint{1.027469in}{6.149781in}}{\pgfqpoint{1.029783in}{6.155367in}}{\pgfqpoint{1.029783in}{6.161191in}}%
\pgfpathcurveto{\pgfqpoint{1.029783in}{6.167015in}}{\pgfqpoint{1.027469in}{6.172601in}}{\pgfqpoint{1.023351in}{6.176719in}}%
\pgfpathcurveto{\pgfqpoint{1.019232in}{6.180837in}}{\pgfqpoint{1.013646in}{6.183151in}}{\pgfqpoint{1.007822in}{6.183151in}}%
\pgfpathcurveto{\pgfqpoint{1.001998in}{6.183151in}}{\pgfqpoint{0.996412in}{6.180837in}}{\pgfqpoint{0.992294in}{6.176719in}}%
\pgfpathcurveto{\pgfqpoint{0.988176in}{6.172601in}}{\pgfqpoint{0.985862in}{6.167015in}}{\pgfqpoint{0.985862in}{6.161191in}}%
\pgfpathcurveto{\pgfqpoint{0.985862in}{6.155367in}}{\pgfqpoint{0.988176in}{6.149781in}}{\pgfqpoint{0.992294in}{6.145662in}}%
\pgfpathcurveto{\pgfqpoint{0.996412in}{6.141544in}}{\pgfqpoint{1.001998in}{6.139230in}}{\pgfqpoint{1.007822in}{6.139230in}}%
\pgfpathclose%
\pgfusepath{stroke,fill}%
\end{pgfscope}%
\begin{pgfscope}%
\pgfpathrectangle{\pgfqpoint{0.506010in}{1.121191in}}{\pgfqpoint{2.325000in}{1.400000in}} %
\pgfusepath{clip}%
\pgfsetbuttcap%
\pgfsetroundjoin%
\definecolor{currentfill}{rgb}{0.000000,0.500000,0.000000}%
\pgfsetfillcolor{currentfill}%
\pgfsetlinewidth{1.003750pt}%
\definecolor{currentstroke}{rgb}{0.000000,0.500000,0.000000}%
\pgfsetstrokecolor{currentstroke}%
\pgfsetdash{}{0pt}%
\pgfpathmoveto{\pgfqpoint{1.081382in}{6.139230in}}%
\pgfpathcurveto{\pgfqpoint{1.087206in}{6.139230in}}{\pgfqpoint{1.092792in}{6.141544in}}{\pgfqpoint{1.096910in}{6.145662in}}%
\pgfpathcurveto{\pgfqpoint{1.101028in}{6.149781in}}{\pgfqpoint{1.103342in}{6.155367in}}{\pgfqpoint{1.103342in}{6.161191in}}%
\pgfpathcurveto{\pgfqpoint{1.103342in}{6.167015in}}{\pgfqpoint{1.101028in}{6.172601in}}{\pgfqpoint{1.096910in}{6.176719in}}%
\pgfpathcurveto{\pgfqpoint{1.092792in}{6.180837in}}{\pgfqpoint{1.087206in}{6.183151in}}{\pgfqpoint{1.081382in}{6.183151in}}%
\pgfpathcurveto{\pgfqpoint{1.075558in}{6.183151in}}{\pgfqpoint{1.069972in}{6.180837in}}{\pgfqpoint{1.065854in}{6.176719in}}%
\pgfpathcurveto{\pgfqpoint{1.061736in}{6.172601in}}{\pgfqpoint{1.059422in}{6.167015in}}{\pgfqpoint{1.059422in}{6.161191in}}%
\pgfpathcurveto{\pgfqpoint{1.059422in}{6.155367in}}{\pgfqpoint{1.061736in}{6.149781in}}{\pgfqpoint{1.065854in}{6.145662in}}%
\pgfpathcurveto{\pgfqpoint{1.069972in}{6.141544in}}{\pgfqpoint{1.075558in}{6.139230in}}{\pgfqpoint{1.081382in}{6.139230in}}%
\pgfpathclose%
\pgfusepath{stroke,fill}%
\end{pgfscope}%
\begin{pgfscope}%
\pgfpathrectangle{\pgfqpoint{0.506010in}{1.121191in}}{\pgfqpoint{2.325000in}{1.400000in}} %
\pgfusepath{clip}%
\pgfsetbuttcap%
\pgfsetroundjoin%
\definecolor{currentfill}{rgb}{0.000000,0.500000,0.000000}%
\pgfsetfillcolor{currentfill}%
\pgfsetlinewidth{1.003750pt}%
\definecolor{currentstroke}{rgb}{0.000000,0.500000,0.000000}%
\pgfsetstrokecolor{currentstroke}%
\pgfsetdash{}{0pt}%
\pgfpathmoveto{\pgfqpoint{1.081382in}{6.139230in}}%
\pgfpathcurveto{\pgfqpoint{1.087206in}{6.139230in}}{\pgfqpoint{1.092792in}{6.141544in}}{\pgfqpoint{1.096910in}{6.145662in}}%
\pgfpathcurveto{\pgfqpoint{1.101028in}{6.149781in}}{\pgfqpoint{1.103342in}{6.155367in}}{\pgfqpoint{1.103342in}{6.161191in}}%
\pgfpathcurveto{\pgfqpoint{1.103342in}{6.167015in}}{\pgfqpoint{1.101028in}{6.172601in}}{\pgfqpoint{1.096910in}{6.176719in}}%
\pgfpathcurveto{\pgfqpoint{1.092792in}{6.180837in}}{\pgfqpoint{1.087206in}{6.183151in}}{\pgfqpoint{1.081382in}{6.183151in}}%
\pgfpathcurveto{\pgfqpoint{1.075558in}{6.183151in}}{\pgfqpoint{1.069972in}{6.180837in}}{\pgfqpoint{1.065854in}{6.176719in}}%
\pgfpathcurveto{\pgfqpoint{1.061736in}{6.172601in}}{\pgfqpoint{1.059422in}{6.167015in}}{\pgfqpoint{1.059422in}{6.161191in}}%
\pgfpathcurveto{\pgfqpoint{1.059422in}{6.155367in}}{\pgfqpoint{1.061736in}{6.149781in}}{\pgfqpoint{1.065854in}{6.145662in}}%
\pgfpathcurveto{\pgfqpoint{1.069972in}{6.141544in}}{\pgfqpoint{1.075558in}{6.139230in}}{\pgfqpoint{1.081382in}{6.139230in}}%
\pgfpathclose%
\pgfusepath{stroke,fill}%
\end{pgfscope}%
\begin{pgfscope}%
\pgfpathrectangle{\pgfqpoint{0.506010in}{1.121191in}}{\pgfqpoint{2.325000in}{1.400000in}} %
\pgfusepath{clip}%
\pgfsetbuttcap%
\pgfsetroundjoin%
\definecolor{currentfill}{rgb}{0.000000,0.500000,0.000000}%
\pgfsetfillcolor{currentfill}%
\pgfsetlinewidth{1.003750pt}%
\definecolor{currentstroke}{rgb}{0.000000,0.500000,0.000000}%
\pgfsetstrokecolor{currentstroke}%
\pgfsetdash{}{0pt}%
\pgfpathmoveto{\pgfqpoint{1.081382in}{6.139230in}}%
\pgfpathcurveto{\pgfqpoint{1.087206in}{6.139230in}}{\pgfqpoint{1.092792in}{6.141544in}}{\pgfqpoint{1.096910in}{6.145662in}}%
\pgfpathcurveto{\pgfqpoint{1.101028in}{6.149781in}}{\pgfqpoint{1.103342in}{6.155367in}}{\pgfqpoint{1.103342in}{6.161191in}}%
\pgfpathcurveto{\pgfqpoint{1.103342in}{6.167015in}}{\pgfqpoint{1.101028in}{6.172601in}}{\pgfqpoint{1.096910in}{6.176719in}}%
\pgfpathcurveto{\pgfqpoint{1.092792in}{6.180837in}}{\pgfqpoint{1.087206in}{6.183151in}}{\pgfqpoint{1.081382in}{6.183151in}}%
\pgfpathcurveto{\pgfqpoint{1.075558in}{6.183151in}}{\pgfqpoint{1.069972in}{6.180837in}}{\pgfqpoint{1.065854in}{6.176719in}}%
\pgfpathcurveto{\pgfqpoint{1.061736in}{6.172601in}}{\pgfqpoint{1.059422in}{6.167015in}}{\pgfqpoint{1.059422in}{6.161191in}}%
\pgfpathcurveto{\pgfqpoint{1.059422in}{6.155367in}}{\pgfqpoint{1.061736in}{6.149781in}}{\pgfqpoint{1.065854in}{6.145662in}}%
\pgfpathcurveto{\pgfqpoint{1.069972in}{6.141544in}}{\pgfqpoint{1.075558in}{6.139230in}}{\pgfqpoint{1.081382in}{6.139230in}}%
\pgfpathclose%
\pgfusepath{stroke,fill}%
\end{pgfscope}%
\begin{pgfscope}%
\pgfpathrectangle{\pgfqpoint{0.506010in}{1.121191in}}{\pgfqpoint{2.325000in}{1.400000in}} %
\pgfusepath{clip}%
\pgfsetbuttcap%
\pgfsetroundjoin%
\definecolor{currentfill}{rgb}{0.000000,0.500000,0.000000}%
\pgfsetfillcolor{currentfill}%
\pgfsetlinewidth{1.003750pt}%
\definecolor{currentstroke}{rgb}{0.000000,0.500000,0.000000}%
\pgfsetstrokecolor{currentstroke}%
\pgfsetdash{}{0pt}%
\pgfpathmoveto{\pgfqpoint{1.007822in}{6.139230in}}%
\pgfpathcurveto{\pgfqpoint{1.013646in}{6.139230in}}{\pgfqpoint{1.019232in}{6.141544in}}{\pgfqpoint{1.023351in}{6.145662in}}%
\pgfpathcurveto{\pgfqpoint{1.027469in}{6.149781in}}{\pgfqpoint{1.029783in}{6.155367in}}{\pgfqpoint{1.029783in}{6.161191in}}%
\pgfpathcurveto{\pgfqpoint{1.029783in}{6.167015in}}{\pgfqpoint{1.027469in}{6.172601in}}{\pgfqpoint{1.023351in}{6.176719in}}%
\pgfpathcurveto{\pgfqpoint{1.019232in}{6.180837in}}{\pgfqpoint{1.013646in}{6.183151in}}{\pgfqpoint{1.007822in}{6.183151in}}%
\pgfpathcurveto{\pgfqpoint{1.001998in}{6.183151in}}{\pgfqpoint{0.996412in}{6.180837in}}{\pgfqpoint{0.992294in}{6.176719in}}%
\pgfpathcurveto{\pgfqpoint{0.988176in}{6.172601in}}{\pgfqpoint{0.985862in}{6.167015in}}{\pgfqpoint{0.985862in}{6.161191in}}%
\pgfpathcurveto{\pgfqpoint{0.985862in}{6.155367in}}{\pgfqpoint{0.988176in}{6.149781in}}{\pgfqpoint{0.992294in}{6.145662in}}%
\pgfpathcurveto{\pgfqpoint{0.996412in}{6.141544in}}{\pgfqpoint{1.001998in}{6.139230in}}{\pgfqpoint{1.007822in}{6.139230in}}%
\pgfpathclose%
\pgfusepath{stroke,fill}%
\end{pgfscope}%
\begin{pgfscope}%
\pgfpathrectangle{\pgfqpoint{0.506010in}{1.121191in}}{\pgfqpoint{2.325000in}{1.400000in}} %
\pgfusepath{clip}%
\pgfsetbuttcap%
\pgfsetroundjoin%
\definecolor{currentfill}{rgb}{0.000000,0.500000,0.000000}%
\pgfsetfillcolor{currentfill}%
\pgfsetlinewidth{1.003750pt}%
\definecolor{currentstroke}{rgb}{0.000000,0.500000,0.000000}%
\pgfsetstrokecolor{currentstroke}%
\pgfsetdash{}{0pt}%
\pgfpathmoveto{\pgfqpoint{1.007822in}{6.139230in}}%
\pgfpathcurveto{\pgfqpoint{1.013646in}{6.139230in}}{\pgfqpoint{1.019232in}{6.141544in}}{\pgfqpoint{1.023351in}{6.145662in}}%
\pgfpathcurveto{\pgfqpoint{1.027469in}{6.149781in}}{\pgfqpoint{1.029783in}{6.155367in}}{\pgfqpoint{1.029783in}{6.161191in}}%
\pgfpathcurveto{\pgfqpoint{1.029783in}{6.167015in}}{\pgfqpoint{1.027469in}{6.172601in}}{\pgfqpoint{1.023351in}{6.176719in}}%
\pgfpathcurveto{\pgfqpoint{1.019232in}{6.180837in}}{\pgfqpoint{1.013646in}{6.183151in}}{\pgfqpoint{1.007822in}{6.183151in}}%
\pgfpathcurveto{\pgfqpoint{1.001998in}{6.183151in}}{\pgfqpoint{0.996412in}{6.180837in}}{\pgfqpoint{0.992294in}{6.176719in}}%
\pgfpathcurveto{\pgfqpoint{0.988176in}{6.172601in}}{\pgfqpoint{0.985862in}{6.167015in}}{\pgfqpoint{0.985862in}{6.161191in}}%
\pgfpathcurveto{\pgfqpoint{0.985862in}{6.155367in}}{\pgfqpoint{0.988176in}{6.149781in}}{\pgfqpoint{0.992294in}{6.145662in}}%
\pgfpathcurveto{\pgfqpoint{0.996412in}{6.141544in}}{\pgfqpoint{1.001998in}{6.139230in}}{\pgfqpoint{1.007822in}{6.139230in}}%
\pgfpathclose%
\pgfusepath{stroke,fill}%
\end{pgfscope}%
\begin{pgfscope}%
\pgfpathrectangle{\pgfqpoint{0.506010in}{1.121191in}}{\pgfqpoint{2.325000in}{1.400000in}} %
\pgfusepath{clip}%
\pgfsetbuttcap%
\pgfsetroundjoin%
\definecolor{currentfill}{rgb}{0.000000,0.500000,0.000000}%
\pgfsetfillcolor{currentfill}%
\pgfsetlinewidth{1.003750pt}%
\definecolor{currentstroke}{rgb}{0.000000,0.500000,0.000000}%
\pgfsetstrokecolor{currentstroke}%
\pgfsetdash{}{0pt}%
\pgfpathmoveto{\pgfqpoint{1.108974in}{6.139230in}}%
\pgfpathcurveto{\pgfqpoint{1.114798in}{6.139230in}}{\pgfqpoint{1.120384in}{6.141544in}}{\pgfqpoint{1.124502in}{6.145662in}}%
\pgfpathcurveto{\pgfqpoint{1.128620in}{6.149781in}}{\pgfqpoint{1.130934in}{6.155367in}}{\pgfqpoint{1.130934in}{6.161191in}}%
\pgfpathcurveto{\pgfqpoint{1.130934in}{6.167015in}}{\pgfqpoint{1.128620in}{6.172601in}}{\pgfqpoint{1.124502in}{6.176719in}}%
\pgfpathcurveto{\pgfqpoint{1.120384in}{6.180837in}}{\pgfqpoint{1.114798in}{6.183151in}}{\pgfqpoint{1.108974in}{6.183151in}}%
\pgfpathcurveto{\pgfqpoint{1.103150in}{6.183151in}}{\pgfqpoint{1.097564in}{6.180837in}}{\pgfqpoint{1.093446in}{6.176719in}}%
\pgfpathcurveto{\pgfqpoint{1.089327in}{6.172601in}}{\pgfqpoint{1.087014in}{6.167015in}}{\pgfqpoint{1.087014in}{6.161191in}}%
\pgfpathcurveto{\pgfqpoint{1.087014in}{6.155367in}}{\pgfqpoint{1.089327in}{6.149781in}}{\pgfqpoint{1.093446in}{6.145662in}}%
\pgfpathcurveto{\pgfqpoint{1.097564in}{6.141544in}}{\pgfqpoint{1.103150in}{6.139230in}}{\pgfqpoint{1.108974in}{6.139230in}}%
\pgfpathclose%
\pgfusepath{stroke,fill}%
\end{pgfscope}%
\begin{pgfscope}%
\pgfpathrectangle{\pgfqpoint{0.506010in}{1.121191in}}{\pgfqpoint{2.325000in}{1.400000in}} %
\pgfusepath{clip}%
\pgfsetbuttcap%
\pgfsetroundjoin%
\definecolor{currentfill}{rgb}{0.000000,0.500000,0.000000}%
\pgfsetfillcolor{currentfill}%
\pgfsetlinewidth{1.003750pt}%
\definecolor{currentstroke}{rgb}{0.000000,0.500000,0.000000}%
\pgfsetstrokecolor{currentstroke}%
\pgfsetdash{}{0pt}%
\pgfpathmoveto{\pgfqpoint{1.474029in}{6.139230in}}%
\pgfpathcurveto{\pgfqpoint{1.479853in}{6.139230in}}{\pgfqpoint{1.485439in}{6.141544in}}{\pgfqpoint{1.489558in}{6.145662in}}%
\pgfpathcurveto{\pgfqpoint{1.493676in}{6.149781in}}{\pgfqpoint{1.495990in}{6.155367in}}{\pgfqpoint{1.495990in}{6.161191in}}%
\pgfpathcurveto{\pgfqpoint{1.495990in}{6.167015in}}{\pgfqpoint{1.493676in}{6.172601in}}{\pgfqpoint{1.489558in}{6.176719in}}%
\pgfpathcurveto{\pgfqpoint{1.485439in}{6.180837in}}{\pgfqpoint{1.479853in}{6.183151in}}{\pgfqpoint{1.474029in}{6.183151in}}%
\pgfpathcurveto{\pgfqpoint{1.468205in}{6.183151in}}{\pgfqpoint{1.462619in}{6.180837in}}{\pgfqpoint{1.458501in}{6.176719in}}%
\pgfpathcurveto{\pgfqpoint{1.454383in}{6.172601in}}{\pgfqpoint{1.452069in}{6.167015in}}{\pgfqpoint{1.452069in}{6.161191in}}%
\pgfpathcurveto{\pgfqpoint{1.452069in}{6.155367in}}{\pgfqpoint{1.454383in}{6.149781in}}{\pgfqpoint{1.458501in}{6.145662in}}%
\pgfpathcurveto{\pgfqpoint{1.462619in}{6.141544in}}{\pgfqpoint{1.468205in}{6.139230in}}{\pgfqpoint{1.474029in}{6.139230in}}%
\pgfpathclose%
\pgfusepath{stroke,fill}%
\end{pgfscope}%
\begin{pgfscope}%
\pgfpathrectangle{\pgfqpoint{0.506010in}{1.121191in}}{\pgfqpoint{2.325000in}{1.400000in}} %
\pgfusepath{clip}%
\pgfsetbuttcap%
\pgfsetroundjoin%
\definecolor{currentfill}{rgb}{0.000000,0.500000,0.000000}%
\pgfsetfillcolor{currentfill}%
\pgfsetlinewidth{1.003750pt}%
\definecolor{currentstroke}{rgb}{0.000000,0.500000,0.000000}%
\pgfsetstrokecolor{currentstroke}%
\pgfsetdash{}{0pt}%
\pgfpathmoveto{\pgfqpoint{1.081382in}{6.139230in}}%
\pgfpathcurveto{\pgfqpoint{1.087206in}{6.139230in}}{\pgfqpoint{1.092792in}{6.141544in}}{\pgfqpoint{1.096910in}{6.145662in}}%
\pgfpathcurveto{\pgfqpoint{1.101028in}{6.149781in}}{\pgfqpoint{1.103342in}{6.155367in}}{\pgfqpoint{1.103342in}{6.161191in}}%
\pgfpathcurveto{\pgfqpoint{1.103342in}{6.167015in}}{\pgfqpoint{1.101028in}{6.172601in}}{\pgfqpoint{1.096910in}{6.176719in}}%
\pgfpathcurveto{\pgfqpoint{1.092792in}{6.180837in}}{\pgfqpoint{1.087206in}{6.183151in}}{\pgfqpoint{1.081382in}{6.183151in}}%
\pgfpathcurveto{\pgfqpoint{1.075558in}{6.183151in}}{\pgfqpoint{1.069972in}{6.180837in}}{\pgfqpoint{1.065854in}{6.176719in}}%
\pgfpathcurveto{\pgfqpoint{1.061736in}{6.172601in}}{\pgfqpoint{1.059422in}{6.167015in}}{\pgfqpoint{1.059422in}{6.161191in}}%
\pgfpathcurveto{\pgfqpoint{1.059422in}{6.155367in}}{\pgfqpoint{1.061736in}{6.149781in}}{\pgfqpoint{1.065854in}{6.145662in}}%
\pgfpathcurveto{\pgfqpoint{1.069972in}{6.141544in}}{\pgfqpoint{1.075558in}{6.139230in}}{\pgfqpoint{1.081382in}{6.139230in}}%
\pgfpathclose%
\pgfusepath{stroke,fill}%
\end{pgfscope}%
\begin{pgfscope}%
\pgfpathrectangle{\pgfqpoint{0.506010in}{1.121191in}}{\pgfqpoint{2.325000in}{1.400000in}} %
\pgfusepath{clip}%
\pgfsetbuttcap%
\pgfsetroundjoin%
\definecolor{currentfill}{rgb}{0.000000,0.500000,0.000000}%
\pgfsetfillcolor{currentfill}%
\pgfsetlinewidth{1.003750pt}%
\definecolor{currentstroke}{rgb}{0.000000,0.500000,0.000000}%
\pgfsetstrokecolor{currentstroke}%
\pgfsetdash{}{0pt}%
\pgfpathmoveto{\pgfqpoint{1.081382in}{6.139230in}}%
\pgfpathcurveto{\pgfqpoint{1.087206in}{6.139230in}}{\pgfqpoint{1.092792in}{6.141544in}}{\pgfqpoint{1.096910in}{6.145662in}}%
\pgfpathcurveto{\pgfqpoint{1.101028in}{6.149781in}}{\pgfqpoint{1.103342in}{6.155367in}}{\pgfqpoint{1.103342in}{6.161191in}}%
\pgfpathcurveto{\pgfqpoint{1.103342in}{6.167015in}}{\pgfqpoint{1.101028in}{6.172601in}}{\pgfqpoint{1.096910in}{6.176719in}}%
\pgfpathcurveto{\pgfqpoint{1.092792in}{6.180837in}}{\pgfqpoint{1.087206in}{6.183151in}}{\pgfqpoint{1.081382in}{6.183151in}}%
\pgfpathcurveto{\pgfqpoint{1.075558in}{6.183151in}}{\pgfqpoint{1.069972in}{6.180837in}}{\pgfqpoint{1.065854in}{6.176719in}}%
\pgfpathcurveto{\pgfqpoint{1.061736in}{6.172601in}}{\pgfqpoint{1.059422in}{6.167015in}}{\pgfqpoint{1.059422in}{6.161191in}}%
\pgfpathcurveto{\pgfqpoint{1.059422in}{6.155367in}}{\pgfqpoint{1.061736in}{6.149781in}}{\pgfqpoint{1.065854in}{6.145662in}}%
\pgfpathcurveto{\pgfqpoint{1.069972in}{6.141544in}}{\pgfqpoint{1.075558in}{6.139230in}}{\pgfqpoint{1.081382in}{6.139230in}}%
\pgfpathclose%
\pgfusepath{stroke,fill}%
\end{pgfscope}%
\begin{pgfscope}%
\pgfpathrectangle{\pgfqpoint{0.506010in}{1.121191in}}{\pgfqpoint{2.325000in}{1.400000in}} %
\pgfusepath{clip}%
\pgfsetbuttcap%
\pgfsetroundjoin%
\definecolor{currentfill}{rgb}{0.000000,0.500000,0.000000}%
\pgfsetfillcolor{currentfill}%
\pgfsetlinewidth{1.003750pt}%
\definecolor{currentstroke}{rgb}{0.000000,0.500000,0.000000}%
\pgfsetstrokecolor{currentstroke}%
\pgfsetdash{}{0pt}%
\pgfpathmoveto{\pgfqpoint{1.007822in}{6.139230in}}%
\pgfpathcurveto{\pgfqpoint{1.013646in}{6.139230in}}{\pgfqpoint{1.019232in}{6.141544in}}{\pgfqpoint{1.023351in}{6.145662in}}%
\pgfpathcurveto{\pgfqpoint{1.027469in}{6.149781in}}{\pgfqpoint{1.029783in}{6.155367in}}{\pgfqpoint{1.029783in}{6.161191in}}%
\pgfpathcurveto{\pgfqpoint{1.029783in}{6.167015in}}{\pgfqpoint{1.027469in}{6.172601in}}{\pgfqpoint{1.023351in}{6.176719in}}%
\pgfpathcurveto{\pgfqpoint{1.019232in}{6.180837in}}{\pgfqpoint{1.013646in}{6.183151in}}{\pgfqpoint{1.007822in}{6.183151in}}%
\pgfpathcurveto{\pgfqpoint{1.001998in}{6.183151in}}{\pgfqpoint{0.996412in}{6.180837in}}{\pgfqpoint{0.992294in}{6.176719in}}%
\pgfpathcurveto{\pgfqpoint{0.988176in}{6.172601in}}{\pgfqpoint{0.985862in}{6.167015in}}{\pgfqpoint{0.985862in}{6.161191in}}%
\pgfpathcurveto{\pgfqpoint{0.985862in}{6.155367in}}{\pgfqpoint{0.988176in}{6.149781in}}{\pgfqpoint{0.992294in}{6.145662in}}%
\pgfpathcurveto{\pgfqpoint{0.996412in}{6.141544in}}{\pgfqpoint{1.001998in}{6.139230in}}{\pgfqpoint{1.007822in}{6.139230in}}%
\pgfpathclose%
\pgfusepath{stroke,fill}%
\end{pgfscope}%
\begin{pgfscope}%
\pgfpathrectangle{\pgfqpoint{0.506010in}{1.121191in}}{\pgfqpoint{2.325000in}{1.400000in}} %
\pgfusepath{clip}%
\pgfsetbuttcap%
\pgfsetroundjoin%
\definecolor{currentfill}{rgb}{0.000000,0.500000,0.000000}%
\pgfsetfillcolor{currentfill}%
\pgfsetlinewidth{1.003750pt}%
\definecolor{currentstroke}{rgb}{0.000000,0.500000,0.000000}%
\pgfsetstrokecolor{currentstroke}%
\pgfsetdash{}{0pt}%
\pgfpathmoveto{\pgfqpoint{1.108974in}{6.139230in}}%
\pgfpathcurveto{\pgfqpoint{1.114798in}{6.139230in}}{\pgfqpoint{1.120384in}{6.141544in}}{\pgfqpoint{1.124502in}{6.145662in}}%
\pgfpathcurveto{\pgfqpoint{1.128620in}{6.149781in}}{\pgfqpoint{1.130934in}{6.155367in}}{\pgfqpoint{1.130934in}{6.161191in}}%
\pgfpathcurveto{\pgfqpoint{1.130934in}{6.167015in}}{\pgfqpoint{1.128620in}{6.172601in}}{\pgfqpoint{1.124502in}{6.176719in}}%
\pgfpathcurveto{\pgfqpoint{1.120384in}{6.180837in}}{\pgfqpoint{1.114798in}{6.183151in}}{\pgfqpoint{1.108974in}{6.183151in}}%
\pgfpathcurveto{\pgfqpoint{1.103150in}{6.183151in}}{\pgfqpoint{1.097564in}{6.180837in}}{\pgfqpoint{1.093446in}{6.176719in}}%
\pgfpathcurveto{\pgfqpoint{1.089327in}{6.172601in}}{\pgfqpoint{1.087014in}{6.167015in}}{\pgfqpoint{1.087014in}{6.161191in}}%
\pgfpathcurveto{\pgfqpoint{1.087014in}{6.155367in}}{\pgfqpoint{1.089327in}{6.149781in}}{\pgfqpoint{1.093446in}{6.145662in}}%
\pgfpathcurveto{\pgfqpoint{1.097564in}{6.141544in}}{\pgfqpoint{1.103150in}{6.139230in}}{\pgfqpoint{1.108974in}{6.139230in}}%
\pgfpathclose%
\pgfusepath{stroke,fill}%
\end{pgfscope}%
\begin{pgfscope}%
\pgfpathrectangle{\pgfqpoint{0.506010in}{1.121191in}}{\pgfqpoint{2.325000in}{1.400000in}} %
\pgfusepath{clip}%
\pgfsetbuttcap%
\pgfsetroundjoin%
\definecolor{currentfill}{rgb}{0.000000,0.500000,0.000000}%
\pgfsetfillcolor{currentfill}%
\pgfsetlinewidth{1.003750pt}%
\definecolor{currentstroke}{rgb}{0.000000,0.500000,0.000000}%
\pgfsetstrokecolor{currentstroke}%
\pgfsetdash{}{0pt}%
\pgfpathmoveto{\pgfqpoint{1.081382in}{6.139230in}}%
\pgfpathcurveto{\pgfqpoint{1.087206in}{6.139230in}}{\pgfqpoint{1.092792in}{6.141544in}}{\pgfqpoint{1.096910in}{6.145662in}}%
\pgfpathcurveto{\pgfqpoint{1.101028in}{6.149781in}}{\pgfqpoint{1.103342in}{6.155367in}}{\pgfqpoint{1.103342in}{6.161191in}}%
\pgfpathcurveto{\pgfqpoint{1.103342in}{6.167015in}}{\pgfqpoint{1.101028in}{6.172601in}}{\pgfqpoint{1.096910in}{6.176719in}}%
\pgfpathcurveto{\pgfqpoint{1.092792in}{6.180837in}}{\pgfqpoint{1.087206in}{6.183151in}}{\pgfqpoint{1.081382in}{6.183151in}}%
\pgfpathcurveto{\pgfqpoint{1.075558in}{6.183151in}}{\pgfqpoint{1.069972in}{6.180837in}}{\pgfqpoint{1.065854in}{6.176719in}}%
\pgfpathcurveto{\pgfqpoint{1.061736in}{6.172601in}}{\pgfqpoint{1.059422in}{6.167015in}}{\pgfqpoint{1.059422in}{6.161191in}}%
\pgfpathcurveto{\pgfqpoint{1.059422in}{6.155367in}}{\pgfqpoint{1.061736in}{6.149781in}}{\pgfqpoint{1.065854in}{6.145662in}}%
\pgfpathcurveto{\pgfqpoint{1.069972in}{6.141544in}}{\pgfqpoint{1.075558in}{6.139230in}}{\pgfqpoint{1.081382in}{6.139230in}}%
\pgfpathclose%
\pgfusepath{stroke,fill}%
\end{pgfscope}%
\begin{pgfscope}%
\pgfpathrectangle{\pgfqpoint{0.506010in}{1.121191in}}{\pgfqpoint{2.325000in}{1.400000in}} %
\pgfusepath{clip}%
\pgfsetbuttcap%
\pgfsetroundjoin%
\definecolor{currentfill}{rgb}{0.000000,0.500000,0.000000}%
\pgfsetfillcolor{currentfill}%
\pgfsetlinewidth{1.003750pt}%
\definecolor{currentstroke}{rgb}{0.000000,0.500000,0.000000}%
\pgfsetstrokecolor{currentstroke}%
\pgfsetdash{}{0pt}%
\pgfpathmoveto{\pgfqpoint{1.007822in}{6.139230in}}%
\pgfpathcurveto{\pgfqpoint{1.013646in}{6.139230in}}{\pgfqpoint{1.019232in}{6.141544in}}{\pgfqpoint{1.023351in}{6.145662in}}%
\pgfpathcurveto{\pgfqpoint{1.027469in}{6.149781in}}{\pgfqpoint{1.029783in}{6.155367in}}{\pgfqpoint{1.029783in}{6.161191in}}%
\pgfpathcurveto{\pgfqpoint{1.029783in}{6.167015in}}{\pgfqpoint{1.027469in}{6.172601in}}{\pgfqpoint{1.023351in}{6.176719in}}%
\pgfpathcurveto{\pgfqpoint{1.019232in}{6.180837in}}{\pgfqpoint{1.013646in}{6.183151in}}{\pgfqpoint{1.007822in}{6.183151in}}%
\pgfpathcurveto{\pgfqpoint{1.001998in}{6.183151in}}{\pgfqpoint{0.996412in}{6.180837in}}{\pgfqpoint{0.992294in}{6.176719in}}%
\pgfpathcurveto{\pgfqpoint{0.988176in}{6.172601in}}{\pgfqpoint{0.985862in}{6.167015in}}{\pgfqpoint{0.985862in}{6.161191in}}%
\pgfpathcurveto{\pgfqpoint{0.985862in}{6.155367in}}{\pgfqpoint{0.988176in}{6.149781in}}{\pgfqpoint{0.992294in}{6.145662in}}%
\pgfpathcurveto{\pgfqpoint{0.996412in}{6.141544in}}{\pgfqpoint{1.001998in}{6.139230in}}{\pgfqpoint{1.007822in}{6.139230in}}%
\pgfpathclose%
\pgfusepath{stroke,fill}%
\end{pgfscope}%
\begin{pgfscope}%
\pgfpathrectangle{\pgfqpoint{0.506010in}{1.121191in}}{\pgfqpoint{2.325000in}{1.400000in}} %
\pgfusepath{clip}%
\pgfsetbuttcap%
\pgfsetroundjoin%
\definecolor{currentfill}{rgb}{0.000000,0.500000,0.000000}%
\pgfsetfillcolor{currentfill}%
\pgfsetlinewidth{1.003750pt}%
\definecolor{currentstroke}{rgb}{0.000000,0.500000,0.000000}%
\pgfsetstrokecolor{currentstroke}%
\pgfsetdash{}{0pt}%
\pgfpathmoveto{\pgfqpoint{1.081382in}{6.139230in}}%
\pgfpathcurveto{\pgfqpoint{1.087206in}{6.139230in}}{\pgfqpoint{1.092792in}{6.141544in}}{\pgfqpoint{1.096910in}{6.145662in}}%
\pgfpathcurveto{\pgfqpoint{1.101028in}{6.149781in}}{\pgfqpoint{1.103342in}{6.155367in}}{\pgfqpoint{1.103342in}{6.161191in}}%
\pgfpathcurveto{\pgfqpoint{1.103342in}{6.167015in}}{\pgfqpoint{1.101028in}{6.172601in}}{\pgfqpoint{1.096910in}{6.176719in}}%
\pgfpathcurveto{\pgfqpoint{1.092792in}{6.180837in}}{\pgfqpoint{1.087206in}{6.183151in}}{\pgfqpoint{1.081382in}{6.183151in}}%
\pgfpathcurveto{\pgfqpoint{1.075558in}{6.183151in}}{\pgfqpoint{1.069972in}{6.180837in}}{\pgfqpoint{1.065854in}{6.176719in}}%
\pgfpathcurveto{\pgfqpoint{1.061736in}{6.172601in}}{\pgfqpoint{1.059422in}{6.167015in}}{\pgfqpoint{1.059422in}{6.161191in}}%
\pgfpathcurveto{\pgfqpoint{1.059422in}{6.155367in}}{\pgfqpoint{1.061736in}{6.149781in}}{\pgfqpoint{1.065854in}{6.145662in}}%
\pgfpathcurveto{\pgfqpoint{1.069972in}{6.141544in}}{\pgfqpoint{1.075558in}{6.139230in}}{\pgfqpoint{1.081382in}{6.139230in}}%
\pgfpathclose%
\pgfusepath{stroke,fill}%
\end{pgfscope}%
\begin{pgfscope}%
\pgfpathrectangle{\pgfqpoint{0.506010in}{1.121191in}}{\pgfqpoint{2.325000in}{1.400000in}} %
\pgfusepath{clip}%
\pgfsetbuttcap%
\pgfsetroundjoin%
\definecolor{currentfill}{rgb}{0.000000,0.500000,0.000000}%
\pgfsetfillcolor{currentfill}%
\pgfsetlinewidth{1.003750pt}%
\definecolor{currentstroke}{rgb}{0.000000,0.500000,0.000000}%
\pgfsetstrokecolor{currentstroke}%
\pgfsetdash{}{0pt}%
\pgfpathmoveto{\pgfqpoint{1.081382in}{6.139230in}}%
\pgfpathcurveto{\pgfqpoint{1.087206in}{6.139230in}}{\pgfqpoint{1.092792in}{6.141544in}}{\pgfqpoint{1.096910in}{6.145662in}}%
\pgfpathcurveto{\pgfqpoint{1.101028in}{6.149781in}}{\pgfqpoint{1.103342in}{6.155367in}}{\pgfqpoint{1.103342in}{6.161191in}}%
\pgfpathcurveto{\pgfqpoint{1.103342in}{6.167015in}}{\pgfqpoint{1.101028in}{6.172601in}}{\pgfqpoint{1.096910in}{6.176719in}}%
\pgfpathcurveto{\pgfqpoint{1.092792in}{6.180837in}}{\pgfqpoint{1.087206in}{6.183151in}}{\pgfqpoint{1.081382in}{6.183151in}}%
\pgfpathcurveto{\pgfqpoint{1.075558in}{6.183151in}}{\pgfqpoint{1.069972in}{6.180837in}}{\pgfqpoint{1.065854in}{6.176719in}}%
\pgfpathcurveto{\pgfqpoint{1.061736in}{6.172601in}}{\pgfqpoint{1.059422in}{6.167015in}}{\pgfqpoint{1.059422in}{6.161191in}}%
\pgfpathcurveto{\pgfqpoint{1.059422in}{6.155367in}}{\pgfqpoint{1.061736in}{6.149781in}}{\pgfqpoint{1.065854in}{6.145662in}}%
\pgfpathcurveto{\pgfqpoint{1.069972in}{6.141544in}}{\pgfqpoint{1.075558in}{6.139230in}}{\pgfqpoint{1.081382in}{6.139230in}}%
\pgfpathclose%
\pgfusepath{stroke,fill}%
\end{pgfscope}%
\begin{pgfscope}%
\pgfpathrectangle{\pgfqpoint{0.506010in}{1.121191in}}{\pgfqpoint{2.325000in}{1.400000in}} %
\pgfusepath{clip}%
\pgfsetbuttcap%
\pgfsetroundjoin%
\definecolor{currentfill}{rgb}{0.000000,0.500000,0.000000}%
\pgfsetfillcolor{currentfill}%
\pgfsetlinewidth{1.003750pt}%
\definecolor{currentstroke}{rgb}{0.000000,0.500000,0.000000}%
\pgfsetstrokecolor{currentstroke}%
\pgfsetdash{}{0pt}%
\pgfpathmoveto{\pgfqpoint{1.081382in}{6.139230in}}%
\pgfpathcurveto{\pgfqpoint{1.087206in}{6.139230in}}{\pgfqpoint{1.092792in}{6.141544in}}{\pgfqpoint{1.096910in}{6.145662in}}%
\pgfpathcurveto{\pgfqpoint{1.101028in}{6.149781in}}{\pgfqpoint{1.103342in}{6.155367in}}{\pgfqpoint{1.103342in}{6.161191in}}%
\pgfpathcurveto{\pgfqpoint{1.103342in}{6.167015in}}{\pgfqpoint{1.101028in}{6.172601in}}{\pgfqpoint{1.096910in}{6.176719in}}%
\pgfpathcurveto{\pgfqpoint{1.092792in}{6.180837in}}{\pgfqpoint{1.087206in}{6.183151in}}{\pgfqpoint{1.081382in}{6.183151in}}%
\pgfpathcurveto{\pgfqpoint{1.075558in}{6.183151in}}{\pgfqpoint{1.069972in}{6.180837in}}{\pgfqpoint{1.065854in}{6.176719in}}%
\pgfpathcurveto{\pgfqpoint{1.061736in}{6.172601in}}{\pgfqpoint{1.059422in}{6.167015in}}{\pgfqpoint{1.059422in}{6.161191in}}%
\pgfpathcurveto{\pgfqpoint{1.059422in}{6.155367in}}{\pgfqpoint{1.061736in}{6.149781in}}{\pgfqpoint{1.065854in}{6.145662in}}%
\pgfpathcurveto{\pgfqpoint{1.069972in}{6.141544in}}{\pgfqpoint{1.075558in}{6.139230in}}{\pgfqpoint{1.081382in}{6.139230in}}%
\pgfpathclose%
\pgfusepath{stroke,fill}%
\end{pgfscope}%
\begin{pgfscope}%
\pgfpathrectangle{\pgfqpoint{0.506010in}{1.121191in}}{\pgfqpoint{2.325000in}{1.400000in}} %
\pgfusepath{clip}%
\pgfsetbuttcap%
\pgfsetroundjoin%
\definecolor{currentfill}{rgb}{0.000000,0.500000,0.000000}%
\pgfsetfillcolor{currentfill}%
\pgfsetlinewidth{1.003750pt}%
\definecolor{currentstroke}{rgb}{0.000000,0.500000,0.000000}%
\pgfsetstrokecolor{currentstroke}%
\pgfsetdash{}{0pt}%
\pgfpathmoveto{\pgfqpoint{1.081382in}{6.139230in}}%
\pgfpathcurveto{\pgfqpoint{1.087206in}{6.139230in}}{\pgfqpoint{1.092792in}{6.141544in}}{\pgfqpoint{1.096910in}{6.145662in}}%
\pgfpathcurveto{\pgfqpoint{1.101028in}{6.149781in}}{\pgfqpoint{1.103342in}{6.155367in}}{\pgfqpoint{1.103342in}{6.161191in}}%
\pgfpathcurveto{\pgfqpoint{1.103342in}{6.167015in}}{\pgfqpoint{1.101028in}{6.172601in}}{\pgfqpoint{1.096910in}{6.176719in}}%
\pgfpathcurveto{\pgfqpoint{1.092792in}{6.180837in}}{\pgfqpoint{1.087206in}{6.183151in}}{\pgfqpoint{1.081382in}{6.183151in}}%
\pgfpathcurveto{\pgfqpoint{1.075558in}{6.183151in}}{\pgfqpoint{1.069972in}{6.180837in}}{\pgfqpoint{1.065854in}{6.176719in}}%
\pgfpathcurveto{\pgfqpoint{1.061736in}{6.172601in}}{\pgfqpoint{1.059422in}{6.167015in}}{\pgfqpoint{1.059422in}{6.161191in}}%
\pgfpathcurveto{\pgfqpoint{1.059422in}{6.155367in}}{\pgfqpoint{1.061736in}{6.149781in}}{\pgfqpoint{1.065854in}{6.145662in}}%
\pgfpathcurveto{\pgfqpoint{1.069972in}{6.141544in}}{\pgfqpoint{1.075558in}{6.139230in}}{\pgfqpoint{1.081382in}{6.139230in}}%
\pgfpathclose%
\pgfusepath{stroke,fill}%
\end{pgfscope}%
\begin{pgfscope}%
\pgfpathrectangle{\pgfqpoint{0.506010in}{1.121191in}}{\pgfqpoint{2.325000in}{1.400000in}} %
\pgfusepath{clip}%
\pgfsetbuttcap%
\pgfsetroundjoin%
\definecolor{currentfill}{rgb}{0.000000,0.500000,0.000000}%
\pgfsetfillcolor{currentfill}%
\pgfsetlinewidth{1.003750pt}%
\definecolor{currentstroke}{rgb}{0.000000,0.500000,0.000000}%
\pgfsetstrokecolor{currentstroke}%
\pgfsetdash{}{0pt}%
\pgfpathmoveto{\pgfqpoint{1.007822in}{6.139230in}}%
\pgfpathcurveto{\pgfqpoint{1.013646in}{6.139230in}}{\pgfqpoint{1.019232in}{6.141544in}}{\pgfqpoint{1.023351in}{6.145662in}}%
\pgfpathcurveto{\pgfqpoint{1.027469in}{6.149781in}}{\pgfqpoint{1.029783in}{6.155367in}}{\pgfqpoint{1.029783in}{6.161191in}}%
\pgfpathcurveto{\pgfqpoint{1.029783in}{6.167015in}}{\pgfqpoint{1.027469in}{6.172601in}}{\pgfqpoint{1.023351in}{6.176719in}}%
\pgfpathcurveto{\pgfqpoint{1.019232in}{6.180837in}}{\pgfqpoint{1.013646in}{6.183151in}}{\pgfqpoint{1.007822in}{6.183151in}}%
\pgfpathcurveto{\pgfqpoint{1.001998in}{6.183151in}}{\pgfqpoint{0.996412in}{6.180837in}}{\pgfqpoint{0.992294in}{6.176719in}}%
\pgfpathcurveto{\pgfqpoint{0.988176in}{6.172601in}}{\pgfqpoint{0.985862in}{6.167015in}}{\pgfqpoint{0.985862in}{6.161191in}}%
\pgfpathcurveto{\pgfqpoint{0.985862in}{6.155367in}}{\pgfqpoint{0.988176in}{6.149781in}}{\pgfqpoint{0.992294in}{6.145662in}}%
\pgfpathcurveto{\pgfqpoint{0.996412in}{6.141544in}}{\pgfqpoint{1.001998in}{6.139230in}}{\pgfqpoint{1.007822in}{6.139230in}}%
\pgfpathclose%
\pgfusepath{stroke,fill}%
\end{pgfscope}%
\begin{pgfscope}%
\pgfpathrectangle{\pgfqpoint{0.506010in}{1.121191in}}{\pgfqpoint{2.325000in}{1.400000in}} %
\pgfusepath{clip}%
\pgfsetbuttcap%
\pgfsetroundjoin%
\definecolor{currentfill}{rgb}{0.000000,0.500000,0.000000}%
\pgfsetfillcolor{currentfill}%
\pgfsetlinewidth{1.003750pt}%
\definecolor{currentstroke}{rgb}{0.000000,0.500000,0.000000}%
\pgfsetstrokecolor{currentstroke}%
\pgfsetdash{}{0pt}%
\pgfpathmoveto{\pgfqpoint{0.788675in}{6.139230in}}%
\pgfpathcurveto{\pgfqpoint{0.794499in}{6.139230in}}{\pgfqpoint{0.800085in}{6.141544in}}{\pgfqpoint{0.804203in}{6.145662in}}%
\pgfpathcurveto{\pgfqpoint{0.808322in}{6.149781in}}{\pgfqpoint{0.810635in}{6.155367in}}{\pgfqpoint{0.810635in}{6.161191in}}%
\pgfpathcurveto{\pgfqpoint{0.810635in}{6.167015in}}{\pgfqpoint{0.808322in}{6.172601in}}{\pgfqpoint{0.804203in}{6.176719in}}%
\pgfpathcurveto{\pgfqpoint{0.800085in}{6.180837in}}{\pgfqpoint{0.794499in}{6.183151in}}{\pgfqpoint{0.788675in}{6.183151in}}%
\pgfpathcurveto{\pgfqpoint{0.782851in}{6.183151in}}{\pgfqpoint{0.777265in}{6.180837in}}{\pgfqpoint{0.773147in}{6.176719in}}%
\pgfpathcurveto{\pgfqpoint{0.769029in}{6.172601in}}{\pgfqpoint{0.766715in}{6.167015in}}{\pgfqpoint{0.766715in}{6.161191in}}%
\pgfpathcurveto{\pgfqpoint{0.766715in}{6.155367in}}{\pgfqpoint{0.769029in}{6.149781in}}{\pgfqpoint{0.773147in}{6.145662in}}%
\pgfpathcurveto{\pgfqpoint{0.777265in}{6.141544in}}{\pgfqpoint{0.782851in}{6.139230in}}{\pgfqpoint{0.788675in}{6.139230in}}%
\pgfpathclose%
\pgfusepath{stroke,fill}%
\end{pgfscope}%
\begin{pgfscope}%
\pgfpathrectangle{\pgfqpoint{0.506010in}{1.121191in}}{\pgfqpoint{2.325000in}{1.400000in}} %
\pgfusepath{clip}%
\pgfsetbuttcap%
\pgfsetroundjoin%
\definecolor{currentfill}{rgb}{0.000000,0.500000,0.000000}%
\pgfsetfillcolor{currentfill}%
\pgfsetlinewidth{1.003750pt}%
\definecolor{currentstroke}{rgb}{0.000000,0.500000,0.000000}%
\pgfsetstrokecolor{currentstroke}%
\pgfsetdash{}{0pt}%
\pgfpathmoveto{\pgfqpoint{2.480299in}{6.139230in}}%
\pgfpathcurveto{\pgfqpoint{2.486123in}{6.139230in}}{\pgfqpoint{2.491709in}{6.141544in}}{\pgfqpoint{2.495827in}{6.145662in}}%
\pgfpathcurveto{\pgfqpoint{2.499945in}{6.149781in}}{\pgfqpoint{2.502259in}{6.155367in}}{\pgfqpoint{2.502259in}{6.161191in}}%
\pgfpathcurveto{\pgfqpoint{2.502259in}{6.167015in}}{\pgfqpoint{2.499945in}{6.172601in}}{\pgfqpoint{2.495827in}{6.176719in}}%
\pgfpathcurveto{\pgfqpoint{2.491709in}{6.180837in}}{\pgfqpoint{2.486123in}{6.183151in}}{\pgfqpoint{2.480299in}{6.183151in}}%
\pgfpathcurveto{\pgfqpoint{2.474475in}{6.183151in}}{\pgfqpoint{2.468889in}{6.180837in}}{\pgfqpoint{2.464771in}{6.176719in}}%
\pgfpathcurveto{\pgfqpoint{2.460652in}{6.172601in}}{\pgfqpoint{2.458339in}{6.167015in}}{\pgfqpoint{2.458339in}{6.161191in}}%
\pgfpathcurveto{\pgfqpoint{2.458339in}{6.155367in}}{\pgfqpoint{2.460652in}{6.149781in}}{\pgfqpoint{2.464771in}{6.145662in}}%
\pgfpathcurveto{\pgfqpoint{2.468889in}{6.141544in}}{\pgfqpoint{2.474475in}{6.139230in}}{\pgfqpoint{2.480299in}{6.139230in}}%
\pgfpathclose%
\pgfusepath{stroke,fill}%
\end{pgfscope}%
\begin{pgfscope}%
\pgfpathrectangle{\pgfqpoint{0.506010in}{1.121191in}}{\pgfqpoint{2.325000in}{1.400000in}} %
\pgfusepath{clip}%
\pgfsetbuttcap%
\pgfsetroundjoin%
\definecolor{currentfill}{rgb}{0.000000,0.500000,0.000000}%
\pgfsetfillcolor{currentfill}%
\pgfsetlinewidth{1.003750pt}%
\definecolor{currentstroke}{rgb}{0.000000,0.500000,0.000000}%
\pgfsetstrokecolor{currentstroke}%
\pgfsetdash{}{0pt}%
\pgfpathmoveto{\pgfqpoint{1.139222in}{6.139230in}}%
\pgfpathcurveto{\pgfqpoint{1.145045in}{6.139230in}}{\pgfqpoint{1.150632in}{6.141544in}}{\pgfqpoint{1.154750in}{6.145662in}}%
\pgfpathcurveto{\pgfqpoint{1.158868in}{6.149781in}}{\pgfqpoint{1.161182in}{6.155367in}}{\pgfqpoint{1.161182in}{6.161191in}}%
\pgfpathcurveto{\pgfqpoint{1.161182in}{6.167015in}}{\pgfqpoint{1.158868in}{6.172601in}}{\pgfqpoint{1.154750in}{6.176719in}}%
\pgfpathcurveto{\pgfqpoint{1.150632in}{6.180837in}}{\pgfqpoint{1.145045in}{6.183151in}}{\pgfqpoint{1.139222in}{6.183151in}}%
\pgfpathcurveto{\pgfqpoint{1.133398in}{6.183151in}}{\pgfqpoint{1.127811in}{6.180837in}}{\pgfqpoint{1.123693in}{6.176719in}}%
\pgfpathcurveto{\pgfqpoint{1.119575in}{6.172601in}}{\pgfqpoint{1.117261in}{6.167015in}}{\pgfqpoint{1.117261in}{6.161191in}}%
\pgfpathcurveto{\pgfqpoint{1.117261in}{6.155367in}}{\pgfqpoint{1.119575in}{6.149781in}}{\pgfqpoint{1.123693in}{6.145662in}}%
\pgfpathcurveto{\pgfqpoint{1.127811in}{6.141544in}}{\pgfqpoint{1.133398in}{6.139230in}}{\pgfqpoint{1.139222in}{6.139230in}}%
\pgfpathclose%
\pgfusepath{stroke,fill}%
\end{pgfscope}%
\begin{pgfscope}%
\pgfpathrectangle{\pgfqpoint{0.506010in}{1.121191in}}{\pgfqpoint{2.325000in}{1.400000in}} %
\pgfusepath{clip}%
\pgfsetbuttcap%
\pgfsetroundjoin%
\definecolor{currentfill}{rgb}{0.000000,0.500000,0.000000}%
\pgfsetfillcolor{currentfill}%
\pgfsetlinewidth{1.003750pt}%
\definecolor{currentstroke}{rgb}{0.000000,0.500000,0.000000}%
\pgfsetstrokecolor{currentstroke}%
\pgfsetdash{}{0pt}%
\pgfpathmoveto{\pgfqpoint{1.007822in}{6.139230in}}%
\pgfpathcurveto{\pgfqpoint{1.013646in}{6.139230in}}{\pgfqpoint{1.019232in}{6.141544in}}{\pgfqpoint{1.023351in}{6.145662in}}%
\pgfpathcurveto{\pgfqpoint{1.027469in}{6.149781in}}{\pgfqpoint{1.029783in}{6.155367in}}{\pgfqpoint{1.029783in}{6.161191in}}%
\pgfpathcurveto{\pgfqpoint{1.029783in}{6.167015in}}{\pgfqpoint{1.027469in}{6.172601in}}{\pgfqpoint{1.023351in}{6.176719in}}%
\pgfpathcurveto{\pgfqpoint{1.019232in}{6.180837in}}{\pgfqpoint{1.013646in}{6.183151in}}{\pgfqpoint{1.007822in}{6.183151in}}%
\pgfpathcurveto{\pgfqpoint{1.001998in}{6.183151in}}{\pgfqpoint{0.996412in}{6.180837in}}{\pgfqpoint{0.992294in}{6.176719in}}%
\pgfpathcurveto{\pgfqpoint{0.988176in}{6.172601in}}{\pgfqpoint{0.985862in}{6.167015in}}{\pgfqpoint{0.985862in}{6.161191in}}%
\pgfpathcurveto{\pgfqpoint{0.985862in}{6.155367in}}{\pgfqpoint{0.988176in}{6.149781in}}{\pgfqpoint{0.992294in}{6.145662in}}%
\pgfpathcurveto{\pgfqpoint{0.996412in}{6.141544in}}{\pgfqpoint{1.001998in}{6.139230in}}{\pgfqpoint{1.007822in}{6.139230in}}%
\pgfpathclose%
\pgfusepath{stroke,fill}%
\end{pgfscope}%
\begin{pgfscope}%
\pgfpathrectangle{\pgfqpoint{0.506010in}{1.121191in}}{\pgfqpoint{2.325000in}{1.400000in}} %
\pgfusepath{clip}%
\pgfsetbuttcap%
\pgfsetroundjoin%
\definecolor{currentfill}{rgb}{0.000000,0.500000,0.000000}%
\pgfsetfillcolor{currentfill}%
\pgfsetlinewidth{1.003750pt}%
\definecolor{currentstroke}{rgb}{0.000000,0.500000,0.000000}%
\pgfsetstrokecolor{currentstroke}%
\pgfsetdash{}{0pt}%
\pgfpathmoveto{\pgfqpoint{1.007822in}{6.139230in}}%
\pgfpathcurveto{\pgfqpoint{1.013646in}{6.139230in}}{\pgfqpoint{1.019232in}{6.141544in}}{\pgfqpoint{1.023351in}{6.145662in}}%
\pgfpathcurveto{\pgfqpoint{1.027469in}{6.149781in}}{\pgfqpoint{1.029783in}{6.155367in}}{\pgfqpoint{1.029783in}{6.161191in}}%
\pgfpathcurveto{\pgfqpoint{1.029783in}{6.167015in}}{\pgfqpoint{1.027469in}{6.172601in}}{\pgfqpoint{1.023351in}{6.176719in}}%
\pgfpathcurveto{\pgfqpoint{1.019232in}{6.180837in}}{\pgfqpoint{1.013646in}{6.183151in}}{\pgfqpoint{1.007822in}{6.183151in}}%
\pgfpathcurveto{\pgfqpoint{1.001998in}{6.183151in}}{\pgfqpoint{0.996412in}{6.180837in}}{\pgfqpoint{0.992294in}{6.176719in}}%
\pgfpathcurveto{\pgfqpoint{0.988176in}{6.172601in}}{\pgfqpoint{0.985862in}{6.167015in}}{\pgfqpoint{0.985862in}{6.161191in}}%
\pgfpathcurveto{\pgfqpoint{0.985862in}{6.155367in}}{\pgfqpoint{0.988176in}{6.149781in}}{\pgfqpoint{0.992294in}{6.145662in}}%
\pgfpathcurveto{\pgfqpoint{0.996412in}{6.141544in}}{\pgfqpoint{1.001998in}{6.139230in}}{\pgfqpoint{1.007822in}{6.139230in}}%
\pgfpathclose%
\pgfusepath{stroke,fill}%
\end{pgfscope}%
\begin{pgfscope}%
\pgfpathrectangle{\pgfqpoint{0.506010in}{1.121191in}}{\pgfqpoint{2.325000in}{1.400000in}} %
\pgfusepath{clip}%
\pgfsetbuttcap%
\pgfsetroundjoin%
\definecolor{currentfill}{rgb}{0.000000,0.500000,0.000000}%
\pgfsetfillcolor{currentfill}%
\pgfsetlinewidth{1.003750pt}%
\definecolor{currentstroke}{rgb}{0.000000,0.500000,0.000000}%
\pgfsetstrokecolor{currentstroke}%
\pgfsetdash{}{0pt}%
\pgfpathmoveto{\pgfqpoint{1.007822in}{6.139230in}}%
\pgfpathcurveto{\pgfqpoint{1.013646in}{6.139230in}}{\pgfqpoint{1.019232in}{6.141544in}}{\pgfqpoint{1.023351in}{6.145662in}}%
\pgfpathcurveto{\pgfqpoint{1.027469in}{6.149781in}}{\pgfqpoint{1.029783in}{6.155367in}}{\pgfqpoint{1.029783in}{6.161191in}}%
\pgfpathcurveto{\pgfqpoint{1.029783in}{6.167015in}}{\pgfqpoint{1.027469in}{6.172601in}}{\pgfqpoint{1.023351in}{6.176719in}}%
\pgfpathcurveto{\pgfqpoint{1.019232in}{6.180837in}}{\pgfqpoint{1.013646in}{6.183151in}}{\pgfqpoint{1.007822in}{6.183151in}}%
\pgfpathcurveto{\pgfqpoint{1.001998in}{6.183151in}}{\pgfqpoint{0.996412in}{6.180837in}}{\pgfqpoint{0.992294in}{6.176719in}}%
\pgfpathcurveto{\pgfqpoint{0.988176in}{6.172601in}}{\pgfqpoint{0.985862in}{6.167015in}}{\pgfqpoint{0.985862in}{6.161191in}}%
\pgfpathcurveto{\pgfqpoint{0.985862in}{6.155367in}}{\pgfqpoint{0.988176in}{6.149781in}}{\pgfqpoint{0.992294in}{6.145662in}}%
\pgfpathcurveto{\pgfqpoint{0.996412in}{6.141544in}}{\pgfqpoint{1.001998in}{6.139230in}}{\pgfqpoint{1.007822in}{6.139230in}}%
\pgfpathclose%
\pgfusepath{stroke,fill}%
\end{pgfscope}%
\begin{pgfscope}%
\pgfpathrectangle{\pgfqpoint{0.506010in}{1.121191in}}{\pgfqpoint{2.325000in}{1.400000in}} %
\pgfusepath{clip}%
\pgfsetbuttcap%
\pgfsetroundjoin%
\definecolor{currentfill}{rgb}{0.000000,0.500000,0.000000}%
\pgfsetfillcolor{currentfill}%
\pgfsetlinewidth{1.003750pt}%
\definecolor{currentstroke}{rgb}{0.000000,0.500000,0.000000}%
\pgfsetstrokecolor{currentstroke}%
\pgfsetdash{}{0pt}%
\pgfpathmoveto{\pgfqpoint{1.108974in}{6.139230in}}%
\pgfpathcurveto{\pgfqpoint{1.114798in}{6.139230in}}{\pgfqpoint{1.120384in}{6.141544in}}{\pgfqpoint{1.124502in}{6.145662in}}%
\pgfpathcurveto{\pgfqpoint{1.128620in}{6.149781in}}{\pgfqpoint{1.130934in}{6.155367in}}{\pgfqpoint{1.130934in}{6.161191in}}%
\pgfpathcurveto{\pgfqpoint{1.130934in}{6.167015in}}{\pgfqpoint{1.128620in}{6.172601in}}{\pgfqpoint{1.124502in}{6.176719in}}%
\pgfpathcurveto{\pgfqpoint{1.120384in}{6.180837in}}{\pgfqpoint{1.114798in}{6.183151in}}{\pgfqpoint{1.108974in}{6.183151in}}%
\pgfpathcurveto{\pgfqpoint{1.103150in}{6.183151in}}{\pgfqpoint{1.097564in}{6.180837in}}{\pgfqpoint{1.093446in}{6.176719in}}%
\pgfpathcurveto{\pgfqpoint{1.089327in}{6.172601in}}{\pgfqpoint{1.087014in}{6.167015in}}{\pgfqpoint{1.087014in}{6.161191in}}%
\pgfpathcurveto{\pgfqpoint{1.087014in}{6.155367in}}{\pgfqpoint{1.089327in}{6.149781in}}{\pgfqpoint{1.093446in}{6.145662in}}%
\pgfpathcurveto{\pgfqpoint{1.097564in}{6.141544in}}{\pgfqpoint{1.103150in}{6.139230in}}{\pgfqpoint{1.108974in}{6.139230in}}%
\pgfpathclose%
\pgfusepath{stroke,fill}%
\end{pgfscope}%
\begin{pgfscope}%
\pgfpathrectangle{\pgfqpoint{0.506010in}{1.121191in}}{\pgfqpoint{2.325000in}{1.400000in}} %
\pgfusepath{clip}%
\pgfsetbuttcap%
\pgfsetroundjoin%
\definecolor{currentfill}{rgb}{0.000000,0.500000,0.000000}%
\pgfsetfillcolor{currentfill}%
\pgfsetlinewidth{1.003750pt}%
\definecolor{currentstroke}{rgb}{0.000000,0.500000,0.000000}%
\pgfsetstrokecolor{currentstroke}%
\pgfsetdash{}{0pt}%
\pgfpathmoveto{\pgfqpoint{1.108974in}{6.139230in}}%
\pgfpathcurveto{\pgfqpoint{1.114798in}{6.139230in}}{\pgfqpoint{1.120384in}{6.141544in}}{\pgfqpoint{1.124502in}{6.145662in}}%
\pgfpathcurveto{\pgfqpoint{1.128620in}{6.149781in}}{\pgfqpoint{1.130934in}{6.155367in}}{\pgfqpoint{1.130934in}{6.161191in}}%
\pgfpathcurveto{\pgfqpoint{1.130934in}{6.167015in}}{\pgfqpoint{1.128620in}{6.172601in}}{\pgfqpoint{1.124502in}{6.176719in}}%
\pgfpathcurveto{\pgfqpoint{1.120384in}{6.180837in}}{\pgfqpoint{1.114798in}{6.183151in}}{\pgfqpoint{1.108974in}{6.183151in}}%
\pgfpathcurveto{\pgfqpoint{1.103150in}{6.183151in}}{\pgfqpoint{1.097564in}{6.180837in}}{\pgfqpoint{1.093446in}{6.176719in}}%
\pgfpathcurveto{\pgfqpoint{1.089327in}{6.172601in}}{\pgfqpoint{1.087014in}{6.167015in}}{\pgfqpoint{1.087014in}{6.161191in}}%
\pgfpathcurveto{\pgfqpoint{1.087014in}{6.155367in}}{\pgfqpoint{1.089327in}{6.149781in}}{\pgfqpoint{1.093446in}{6.145662in}}%
\pgfpathcurveto{\pgfqpoint{1.097564in}{6.141544in}}{\pgfqpoint{1.103150in}{6.139230in}}{\pgfqpoint{1.108974in}{6.139230in}}%
\pgfpathclose%
\pgfusepath{stroke,fill}%
\end{pgfscope}%
\begin{pgfscope}%
\pgfpathrectangle{\pgfqpoint{0.506010in}{1.121191in}}{\pgfqpoint{2.325000in}{1.400000in}} %
\pgfusepath{clip}%
\pgfsetbuttcap%
\pgfsetroundjoin%
\definecolor{currentfill}{rgb}{0.000000,0.500000,0.000000}%
\pgfsetfillcolor{currentfill}%
\pgfsetlinewidth{1.003750pt}%
\definecolor{currentstroke}{rgb}{0.000000,0.500000,0.000000}%
\pgfsetstrokecolor{currentstroke}%
\pgfsetdash{}{0pt}%
\pgfpathmoveto{\pgfqpoint{1.081382in}{6.139230in}}%
\pgfpathcurveto{\pgfqpoint{1.087206in}{6.139230in}}{\pgfqpoint{1.092792in}{6.141544in}}{\pgfqpoint{1.096910in}{6.145662in}}%
\pgfpathcurveto{\pgfqpoint{1.101028in}{6.149781in}}{\pgfqpoint{1.103342in}{6.155367in}}{\pgfqpoint{1.103342in}{6.161191in}}%
\pgfpathcurveto{\pgfqpoint{1.103342in}{6.167015in}}{\pgfqpoint{1.101028in}{6.172601in}}{\pgfqpoint{1.096910in}{6.176719in}}%
\pgfpathcurveto{\pgfqpoint{1.092792in}{6.180837in}}{\pgfqpoint{1.087206in}{6.183151in}}{\pgfqpoint{1.081382in}{6.183151in}}%
\pgfpathcurveto{\pgfqpoint{1.075558in}{6.183151in}}{\pgfqpoint{1.069972in}{6.180837in}}{\pgfqpoint{1.065854in}{6.176719in}}%
\pgfpathcurveto{\pgfqpoint{1.061736in}{6.172601in}}{\pgfqpoint{1.059422in}{6.167015in}}{\pgfqpoint{1.059422in}{6.161191in}}%
\pgfpathcurveto{\pgfqpoint{1.059422in}{6.155367in}}{\pgfqpoint{1.061736in}{6.149781in}}{\pgfqpoint{1.065854in}{6.145662in}}%
\pgfpathcurveto{\pgfqpoint{1.069972in}{6.141544in}}{\pgfqpoint{1.075558in}{6.139230in}}{\pgfqpoint{1.081382in}{6.139230in}}%
\pgfpathclose%
\pgfusepath{stroke,fill}%
\end{pgfscope}%
\begin{pgfscope}%
\pgfpathrectangle{\pgfqpoint{0.506010in}{1.121191in}}{\pgfqpoint{2.325000in}{1.400000in}} %
\pgfusepath{clip}%
\pgfsetbuttcap%
\pgfsetroundjoin%
\definecolor{currentfill}{rgb}{0.000000,0.500000,0.000000}%
\pgfsetfillcolor{currentfill}%
\pgfsetlinewidth{1.003750pt}%
\definecolor{currentstroke}{rgb}{0.000000,0.500000,0.000000}%
\pgfsetstrokecolor{currentstroke}%
\pgfsetdash{}{0pt}%
\pgfpathmoveto{\pgfqpoint{0.820559in}{6.139230in}}%
\pgfpathcurveto{\pgfqpoint{0.826383in}{6.139230in}}{\pgfqpoint{0.831969in}{6.141544in}}{\pgfqpoint{0.836088in}{6.145662in}}%
\pgfpathcurveto{\pgfqpoint{0.840206in}{6.149781in}}{\pgfqpoint{0.842520in}{6.155367in}}{\pgfqpoint{0.842520in}{6.161191in}}%
\pgfpathcurveto{\pgfqpoint{0.842520in}{6.167015in}}{\pgfqpoint{0.840206in}{6.172601in}}{\pgfqpoint{0.836088in}{6.176719in}}%
\pgfpathcurveto{\pgfqpoint{0.831969in}{6.180837in}}{\pgfqpoint{0.826383in}{6.183151in}}{\pgfqpoint{0.820559in}{6.183151in}}%
\pgfpathcurveto{\pgfqpoint{0.814735in}{6.183151in}}{\pgfqpoint{0.809149in}{6.180837in}}{\pgfqpoint{0.805031in}{6.176719in}}%
\pgfpathcurveto{\pgfqpoint{0.800913in}{6.172601in}}{\pgfqpoint{0.798599in}{6.167015in}}{\pgfqpoint{0.798599in}{6.161191in}}%
\pgfpathcurveto{\pgfqpoint{0.798599in}{6.155367in}}{\pgfqpoint{0.800913in}{6.149781in}}{\pgfqpoint{0.805031in}{6.145662in}}%
\pgfpathcurveto{\pgfqpoint{0.809149in}{6.141544in}}{\pgfqpoint{0.814735in}{6.139230in}}{\pgfqpoint{0.820559in}{6.139230in}}%
\pgfpathclose%
\pgfusepath{stroke,fill}%
\end{pgfscope}%
\begin{pgfscope}%
\pgfpathrectangle{\pgfqpoint{0.506010in}{1.121191in}}{\pgfqpoint{2.325000in}{1.400000in}} %
\pgfusepath{clip}%
\pgfsetbuttcap%
\pgfsetroundjoin%
\definecolor{currentfill}{rgb}{0.000000,0.500000,0.000000}%
\pgfsetfillcolor{currentfill}%
\pgfsetlinewidth{1.003750pt}%
\definecolor{currentstroke}{rgb}{0.000000,0.500000,0.000000}%
\pgfsetstrokecolor{currentstroke}%
\pgfsetdash{}{0pt}%
\pgfpathmoveto{\pgfqpoint{1.167543in}{6.139230in}}%
\pgfpathcurveto{\pgfqpoint{1.173367in}{6.139230in}}{\pgfqpoint{1.178953in}{6.141544in}}{\pgfqpoint{1.183071in}{6.145662in}}%
\pgfpathcurveto{\pgfqpoint{1.187189in}{6.149781in}}{\pgfqpoint{1.189503in}{6.155367in}}{\pgfqpoint{1.189503in}{6.161191in}}%
\pgfpathcurveto{\pgfqpoint{1.189503in}{6.167015in}}{\pgfqpoint{1.187189in}{6.172601in}}{\pgfqpoint{1.183071in}{6.176719in}}%
\pgfpathcurveto{\pgfqpoint{1.178953in}{6.180837in}}{\pgfqpoint{1.173367in}{6.183151in}}{\pgfqpoint{1.167543in}{6.183151in}}%
\pgfpathcurveto{\pgfqpoint{1.161719in}{6.183151in}}{\pgfqpoint{1.156133in}{6.180837in}}{\pgfqpoint{1.152015in}{6.176719in}}%
\pgfpathcurveto{\pgfqpoint{1.147897in}{6.172601in}}{\pgfqpoint{1.145583in}{6.167015in}}{\pgfqpoint{1.145583in}{6.161191in}}%
\pgfpathcurveto{\pgfqpoint{1.145583in}{6.155367in}}{\pgfqpoint{1.147897in}{6.149781in}}{\pgfqpoint{1.152015in}{6.145662in}}%
\pgfpathcurveto{\pgfqpoint{1.156133in}{6.141544in}}{\pgfqpoint{1.161719in}{6.139230in}}{\pgfqpoint{1.167543in}{6.139230in}}%
\pgfpathclose%
\pgfusepath{stroke,fill}%
\end{pgfscope}%
\begin{pgfscope}%
\pgfpathrectangle{\pgfqpoint{0.506010in}{1.121191in}}{\pgfqpoint{2.325000in}{1.400000in}} %
\pgfusepath{clip}%
\pgfsetbuttcap%
\pgfsetroundjoin%
\definecolor{currentfill}{rgb}{0.000000,0.500000,0.000000}%
\pgfsetfillcolor{currentfill}%
\pgfsetlinewidth{1.003750pt}%
\definecolor{currentstroke}{rgb}{0.000000,0.500000,0.000000}%
\pgfsetstrokecolor{currentstroke}%
\pgfsetdash{}{0pt}%
\pgfpathmoveto{\pgfqpoint{1.139222in}{6.139230in}}%
\pgfpathcurveto{\pgfqpoint{1.145045in}{6.139230in}}{\pgfqpoint{1.150632in}{6.141544in}}{\pgfqpoint{1.154750in}{6.145662in}}%
\pgfpathcurveto{\pgfqpoint{1.158868in}{6.149781in}}{\pgfqpoint{1.161182in}{6.155367in}}{\pgfqpoint{1.161182in}{6.161191in}}%
\pgfpathcurveto{\pgfqpoint{1.161182in}{6.167015in}}{\pgfqpoint{1.158868in}{6.172601in}}{\pgfqpoint{1.154750in}{6.176719in}}%
\pgfpathcurveto{\pgfqpoint{1.150632in}{6.180837in}}{\pgfqpoint{1.145045in}{6.183151in}}{\pgfqpoint{1.139222in}{6.183151in}}%
\pgfpathcurveto{\pgfqpoint{1.133398in}{6.183151in}}{\pgfqpoint{1.127811in}{6.180837in}}{\pgfqpoint{1.123693in}{6.176719in}}%
\pgfpathcurveto{\pgfqpoint{1.119575in}{6.172601in}}{\pgfqpoint{1.117261in}{6.167015in}}{\pgfqpoint{1.117261in}{6.161191in}}%
\pgfpathcurveto{\pgfqpoint{1.117261in}{6.155367in}}{\pgfqpoint{1.119575in}{6.149781in}}{\pgfqpoint{1.123693in}{6.145662in}}%
\pgfpathcurveto{\pgfqpoint{1.127811in}{6.141544in}}{\pgfqpoint{1.133398in}{6.139230in}}{\pgfqpoint{1.139222in}{6.139230in}}%
\pgfpathclose%
\pgfusepath{stroke,fill}%
\end{pgfscope}%
\begin{pgfscope}%
\pgfpathrectangle{\pgfqpoint{0.506010in}{1.121191in}}{\pgfqpoint{2.325000in}{1.400000in}} %
\pgfusepath{clip}%
\pgfsetbuttcap%
\pgfsetroundjoin%
\definecolor{currentfill}{rgb}{0.000000,0.500000,0.000000}%
\pgfsetfillcolor{currentfill}%
\pgfsetlinewidth{1.003750pt}%
\definecolor{currentstroke}{rgb}{0.000000,0.500000,0.000000}%
\pgfsetstrokecolor{currentstroke}%
\pgfsetdash{}{0pt}%
\pgfpathmoveto{\pgfqpoint{2.191066in}{6.139230in}}%
\pgfpathcurveto{\pgfqpoint{2.196890in}{6.139230in}}{\pgfqpoint{2.202476in}{6.141544in}}{\pgfqpoint{2.206594in}{6.145662in}}%
\pgfpathcurveto{\pgfqpoint{2.210712in}{6.149781in}}{\pgfqpoint{2.213026in}{6.155367in}}{\pgfqpoint{2.213026in}{6.161191in}}%
\pgfpathcurveto{\pgfqpoint{2.213026in}{6.167015in}}{\pgfqpoint{2.210712in}{6.172601in}}{\pgfqpoint{2.206594in}{6.176719in}}%
\pgfpathcurveto{\pgfqpoint{2.202476in}{6.180837in}}{\pgfqpoint{2.196890in}{6.183151in}}{\pgfqpoint{2.191066in}{6.183151in}}%
\pgfpathcurveto{\pgfqpoint{2.185242in}{6.183151in}}{\pgfqpoint{2.179656in}{6.180837in}}{\pgfqpoint{2.175538in}{6.176719in}}%
\pgfpathcurveto{\pgfqpoint{2.171420in}{6.172601in}}{\pgfqpoint{2.169106in}{6.167015in}}{\pgfqpoint{2.169106in}{6.161191in}}%
\pgfpathcurveto{\pgfqpoint{2.169106in}{6.155367in}}{\pgfqpoint{2.171420in}{6.149781in}}{\pgfqpoint{2.175538in}{6.145662in}}%
\pgfpathcurveto{\pgfqpoint{2.179656in}{6.141544in}}{\pgfqpoint{2.185242in}{6.139230in}}{\pgfqpoint{2.191066in}{6.139230in}}%
\pgfpathclose%
\pgfusepath{stroke,fill}%
\end{pgfscope}%
\begin{pgfscope}%
\pgfpathrectangle{\pgfqpoint{0.506010in}{1.121191in}}{\pgfqpoint{2.325000in}{1.400000in}} %
\pgfusepath{clip}%
\pgfsetbuttcap%
\pgfsetroundjoin%
\definecolor{currentfill}{rgb}{0.000000,0.500000,0.000000}%
\pgfsetfillcolor{currentfill}%
\pgfsetlinewidth{1.003750pt}%
\definecolor{currentstroke}{rgb}{0.000000,0.500000,0.000000}%
\pgfsetstrokecolor{currentstroke}%
\pgfsetdash{}{0pt}%
\pgfpathmoveto{\pgfqpoint{2.265364in}{6.139230in}}%
\pgfpathcurveto{\pgfqpoint{2.271188in}{6.139230in}}{\pgfqpoint{2.276774in}{6.141544in}}{\pgfqpoint{2.280892in}{6.145662in}}%
\pgfpathcurveto{\pgfqpoint{2.285010in}{6.149781in}}{\pgfqpoint{2.287324in}{6.155367in}}{\pgfqpoint{2.287324in}{6.161191in}}%
\pgfpathcurveto{\pgfqpoint{2.287324in}{6.167015in}}{\pgfqpoint{2.285010in}{6.172601in}}{\pgfqpoint{2.280892in}{6.176719in}}%
\pgfpathcurveto{\pgfqpoint{2.276774in}{6.180837in}}{\pgfqpoint{2.271188in}{6.183151in}}{\pgfqpoint{2.265364in}{6.183151in}}%
\pgfpathcurveto{\pgfqpoint{2.259540in}{6.183151in}}{\pgfqpoint{2.253954in}{6.180837in}}{\pgfqpoint{2.249836in}{6.176719in}}%
\pgfpathcurveto{\pgfqpoint{2.245718in}{6.172601in}}{\pgfqpoint{2.243404in}{6.167015in}}{\pgfqpoint{2.243404in}{6.161191in}}%
\pgfpathcurveto{\pgfqpoint{2.243404in}{6.155367in}}{\pgfqpoint{2.245718in}{6.149781in}}{\pgfqpoint{2.249836in}{6.145662in}}%
\pgfpathcurveto{\pgfqpoint{2.253954in}{6.141544in}}{\pgfqpoint{2.259540in}{6.139230in}}{\pgfqpoint{2.265364in}{6.139230in}}%
\pgfpathclose%
\pgfusepath{stroke,fill}%
\end{pgfscope}%
\begin{pgfscope}%
\pgfpathrectangle{\pgfqpoint{0.506010in}{1.121191in}}{\pgfqpoint{2.325000in}{1.400000in}} %
\pgfusepath{clip}%
\pgfsetbuttcap%
\pgfsetroundjoin%
\definecolor{currentfill}{rgb}{0.000000,0.500000,0.000000}%
\pgfsetfillcolor{currentfill}%
\pgfsetlinewidth{1.003750pt}%
\definecolor{currentstroke}{rgb}{0.000000,0.500000,0.000000}%
\pgfsetstrokecolor{currentstroke}%
\pgfsetdash{}{0pt}%
\pgfpathmoveto{\pgfqpoint{2.185713in}{6.139230in}}%
\pgfpathcurveto{\pgfqpoint{2.191537in}{6.139230in}}{\pgfqpoint{2.197123in}{6.141544in}}{\pgfqpoint{2.201241in}{6.145662in}}%
\pgfpathcurveto{\pgfqpoint{2.205359in}{6.149781in}}{\pgfqpoint{2.207673in}{6.155367in}}{\pgfqpoint{2.207673in}{6.161191in}}%
\pgfpathcurveto{\pgfqpoint{2.207673in}{6.167015in}}{\pgfqpoint{2.205359in}{6.172601in}}{\pgfqpoint{2.201241in}{6.176719in}}%
\pgfpathcurveto{\pgfqpoint{2.197123in}{6.180837in}}{\pgfqpoint{2.191537in}{6.183151in}}{\pgfqpoint{2.185713in}{6.183151in}}%
\pgfpathcurveto{\pgfqpoint{2.179889in}{6.183151in}}{\pgfqpoint{2.174303in}{6.180837in}}{\pgfqpoint{2.170185in}{6.176719in}}%
\pgfpathcurveto{\pgfqpoint{2.166067in}{6.172601in}}{\pgfqpoint{2.163753in}{6.167015in}}{\pgfqpoint{2.163753in}{6.161191in}}%
\pgfpathcurveto{\pgfqpoint{2.163753in}{6.155367in}}{\pgfqpoint{2.166067in}{6.149781in}}{\pgfqpoint{2.170185in}{6.145662in}}%
\pgfpathcurveto{\pgfqpoint{2.174303in}{6.141544in}}{\pgfqpoint{2.179889in}{6.139230in}}{\pgfqpoint{2.185713in}{6.139230in}}%
\pgfpathclose%
\pgfusepath{stroke,fill}%
\end{pgfscope}%
\begin{pgfscope}%
\pgfpathrectangle{\pgfqpoint{0.506010in}{1.121191in}}{\pgfqpoint{2.325000in}{1.400000in}} %
\pgfusepath{clip}%
\pgfsetbuttcap%
\pgfsetroundjoin%
\definecolor{currentfill}{rgb}{0.000000,0.500000,0.000000}%
\pgfsetfillcolor{currentfill}%
\pgfsetlinewidth{1.003750pt}%
\definecolor{currentstroke}{rgb}{0.000000,0.500000,0.000000}%
\pgfsetstrokecolor{currentstroke}%
\pgfsetdash{}{0pt}%
\pgfpathmoveto{\pgfqpoint{2.217160in}{6.139230in}}%
\pgfpathcurveto{\pgfqpoint{2.222984in}{6.139230in}}{\pgfqpoint{2.228570in}{6.141544in}}{\pgfqpoint{2.232688in}{6.145662in}}%
\pgfpathcurveto{\pgfqpoint{2.236806in}{6.149781in}}{\pgfqpoint{2.239120in}{6.155367in}}{\pgfqpoint{2.239120in}{6.161191in}}%
\pgfpathcurveto{\pgfqpoint{2.239120in}{6.167015in}}{\pgfqpoint{2.236806in}{6.172601in}}{\pgfqpoint{2.232688in}{6.176719in}}%
\pgfpathcurveto{\pgfqpoint{2.228570in}{6.180837in}}{\pgfqpoint{2.222984in}{6.183151in}}{\pgfqpoint{2.217160in}{6.183151in}}%
\pgfpathcurveto{\pgfqpoint{2.211336in}{6.183151in}}{\pgfqpoint{2.205750in}{6.180837in}}{\pgfqpoint{2.201632in}{6.176719in}}%
\pgfpathcurveto{\pgfqpoint{2.197514in}{6.172601in}}{\pgfqpoint{2.195200in}{6.167015in}}{\pgfqpoint{2.195200in}{6.161191in}}%
\pgfpathcurveto{\pgfqpoint{2.195200in}{6.155367in}}{\pgfqpoint{2.197514in}{6.149781in}}{\pgfqpoint{2.201632in}{6.145662in}}%
\pgfpathcurveto{\pgfqpoint{2.205750in}{6.141544in}}{\pgfqpoint{2.211336in}{6.139230in}}{\pgfqpoint{2.217160in}{6.139230in}}%
\pgfpathclose%
\pgfusepath{stroke,fill}%
\end{pgfscope}%
\begin{pgfscope}%
\pgfpathrectangle{\pgfqpoint{0.506010in}{1.121191in}}{\pgfqpoint{2.325000in}{1.400000in}} %
\pgfusepath{clip}%
\pgfsetbuttcap%
\pgfsetroundjoin%
\definecolor{currentfill}{rgb}{0.000000,0.500000,0.000000}%
\pgfsetfillcolor{currentfill}%
\pgfsetlinewidth{1.003750pt}%
\definecolor{currentstroke}{rgb}{0.000000,0.500000,0.000000}%
\pgfsetstrokecolor{currentstroke}%
\pgfsetdash{}{0pt}%
\pgfpathmoveto{\pgfqpoint{2.262181in}{6.139230in}}%
\pgfpathcurveto{\pgfqpoint{2.268005in}{6.139230in}}{\pgfqpoint{2.273591in}{6.141544in}}{\pgfqpoint{2.277709in}{6.145662in}}%
\pgfpathcurveto{\pgfqpoint{2.281827in}{6.149781in}}{\pgfqpoint{2.284141in}{6.155367in}}{\pgfqpoint{2.284141in}{6.161191in}}%
\pgfpathcurveto{\pgfqpoint{2.284141in}{6.167015in}}{\pgfqpoint{2.281827in}{6.172601in}}{\pgfqpoint{2.277709in}{6.176719in}}%
\pgfpathcurveto{\pgfqpoint{2.273591in}{6.180837in}}{\pgfqpoint{2.268005in}{6.183151in}}{\pgfqpoint{2.262181in}{6.183151in}}%
\pgfpathcurveto{\pgfqpoint{2.256357in}{6.183151in}}{\pgfqpoint{2.250771in}{6.180837in}}{\pgfqpoint{2.246653in}{6.176719in}}%
\pgfpathcurveto{\pgfqpoint{2.242535in}{6.172601in}}{\pgfqpoint{2.240221in}{6.167015in}}{\pgfqpoint{2.240221in}{6.161191in}}%
\pgfpathcurveto{\pgfqpoint{2.240221in}{6.155367in}}{\pgfqpoint{2.242535in}{6.149781in}}{\pgfqpoint{2.246653in}{6.145662in}}%
\pgfpathcurveto{\pgfqpoint{2.250771in}{6.141544in}}{\pgfqpoint{2.256357in}{6.139230in}}{\pgfqpoint{2.262181in}{6.139230in}}%
\pgfpathclose%
\pgfusepath{stroke,fill}%
\end{pgfscope}%
\begin{pgfscope}%
\pgfpathrectangle{\pgfqpoint{0.506010in}{1.121191in}}{\pgfqpoint{2.325000in}{1.400000in}} %
\pgfusepath{clip}%
\pgfsetbuttcap%
\pgfsetroundjoin%
\definecolor{currentfill}{rgb}{0.000000,0.500000,0.000000}%
\pgfsetfillcolor{currentfill}%
\pgfsetlinewidth{1.003750pt}%
\definecolor{currentstroke}{rgb}{0.000000,0.500000,0.000000}%
\pgfsetstrokecolor{currentstroke}%
\pgfsetdash{}{0pt}%
\pgfpathmoveto{\pgfqpoint{2.216102in}{6.139230in}}%
\pgfpathcurveto{\pgfqpoint{2.221926in}{6.139230in}}{\pgfqpoint{2.227512in}{6.141544in}}{\pgfqpoint{2.231630in}{6.145662in}}%
\pgfpathcurveto{\pgfqpoint{2.235749in}{6.149781in}}{\pgfqpoint{2.238062in}{6.155367in}}{\pgfqpoint{2.238062in}{6.161191in}}%
\pgfpathcurveto{\pgfqpoint{2.238062in}{6.167015in}}{\pgfqpoint{2.235749in}{6.172601in}}{\pgfqpoint{2.231630in}{6.176719in}}%
\pgfpathcurveto{\pgfqpoint{2.227512in}{6.180837in}}{\pgfqpoint{2.221926in}{6.183151in}}{\pgfqpoint{2.216102in}{6.183151in}}%
\pgfpathcurveto{\pgfqpoint{2.210278in}{6.183151in}}{\pgfqpoint{2.204692in}{6.180837in}}{\pgfqpoint{2.200574in}{6.176719in}}%
\pgfpathcurveto{\pgfqpoint{2.196456in}{6.172601in}}{\pgfqpoint{2.194142in}{6.167015in}}{\pgfqpoint{2.194142in}{6.161191in}}%
\pgfpathcurveto{\pgfqpoint{2.194142in}{6.155367in}}{\pgfqpoint{2.196456in}{6.149781in}}{\pgfqpoint{2.200574in}{6.145662in}}%
\pgfpathcurveto{\pgfqpoint{2.204692in}{6.141544in}}{\pgfqpoint{2.210278in}{6.139230in}}{\pgfqpoint{2.216102in}{6.139230in}}%
\pgfpathclose%
\pgfusepath{stroke,fill}%
\end{pgfscope}%
\begin{pgfscope}%
\pgfpathrectangle{\pgfqpoint{0.506010in}{1.121191in}}{\pgfqpoint{2.325000in}{1.400000in}} %
\pgfusepath{clip}%
\pgfsetbuttcap%
\pgfsetroundjoin%
\definecolor{currentfill}{rgb}{0.000000,0.500000,0.000000}%
\pgfsetfillcolor{currentfill}%
\pgfsetlinewidth{1.003750pt}%
\definecolor{currentstroke}{rgb}{0.000000,0.500000,0.000000}%
\pgfsetstrokecolor{currentstroke}%
\pgfsetdash{}{0pt}%
\pgfpathmoveto{\pgfqpoint{2.234732in}{6.139230in}}%
\pgfpathcurveto{\pgfqpoint{2.240556in}{6.139230in}}{\pgfqpoint{2.246142in}{6.141544in}}{\pgfqpoint{2.250260in}{6.145662in}}%
\pgfpathcurveto{\pgfqpoint{2.254378in}{6.149781in}}{\pgfqpoint{2.256692in}{6.155367in}}{\pgfqpoint{2.256692in}{6.161191in}}%
\pgfpathcurveto{\pgfqpoint{2.256692in}{6.167015in}}{\pgfqpoint{2.254378in}{6.172601in}}{\pgfqpoint{2.250260in}{6.176719in}}%
\pgfpathcurveto{\pgfqpoint{2.246142in}{6.180837in}}{\pgfqpoint{2.240556in}{6.183151in}}{\pgfqpoint{2.234732in}{6.183151in}}%
\pgfpathcurveto{\pgfqpoint{2.228908in}{6.183151in}}{\pgfqpoint{2.223322in}{6.180837in}}{\pgfqpoint{2.219203in}{6.176719in}}%
\pgfpathcurveto{\pgfqpoint{2.215085in}{6.172601in}}{\pgfqpoint{2.212771in}{6.167015in}}{\pgfqpoint{2.212771in}{6.161191in}}%
\pgfpathcurveto{\pgfqpoint{2.212771in}{6.155367in}}{\pgfqpoint{2.215085in}{6.149781in}}{\pgfqpoint{2.219203in}{6.145662in}}%
\pgfpathcurveto{\pgfqpoint{2.223322in}{6.141544in}}{\pgfqpoint{2.228908in}{6.139230in}}{\pgfqpoint{2.234732in}{6.139230in}}%
\pgfpathclose%
\pgfusepath{stroke,fill}%
\end{pgfscope}%
\begin{pgfscope}%
\pgfpathrectangle{\pgfqpoint{0.506010in}{1.121191in}}{\pgfqpoint{2.325000in}{1.400000in}} %
\pgfusepath{clip}%
\pgfsetbuttcap%
\pgfsetroundjoin%
\definecolor{currentfill}{rgb}{0.000000,0.500000,0.000000}%
\pgfsetfillcolor{currentfill}%
\pgfsetlinewidth{1.003750pt}%
\definecolor{currentstroke}{rgb}{0.000000,0.500000,0.000000}%
\pgfsetstrokecolor{currentstroke}%
\pgfsetdash{}{0pt}%
\pgfpathmoveto{\pgfqpoint{2.201136in}{6.139230in}}%
\pgfpathcurveto{\pgfqpoint{2.206960in}{6.139230in}}{\pgfqpoint{2.212546in}{6.141544in}}{\pgfqpoint{2.216665in}{6.145662in}}%
\pgfpathcurveto{\pgfqpoint{2.220783in}{6.149781in}}{\pgfqpoint{2.223097in}{6.155367in}}{\pgfqpoint{2.223097in}{6.161191in}}%
\pgfpathcurveto{\pgfqpoint{2.223097in}{6.167015in}}{\pgfqpoint{2.220783in}{6.172601in}}{\pgfqpoint{2.216665in}{6.176719in}}%
\pgfpathcurveto{\pgfqpoint{2.212546in}{6.180837in}}{\pgfqpoint{2.206960in}{6.183151in}}{\pgfqpoint{2.201136in}{6.183151in}}%
\pgfpathcurveto{\pgfqpoint{2.195312in}{6.183151in}}{\pgfqpoint{2.189726in}{6.180837in}}{\pgfqpoint{2.185608in}{6.176719in}}%
\pgfpathcurveto{\pgfqpoint{2.181490in}{6.172601in}}{\pgfqpoint{2.179176in}{6.167015in}}{\pgfqpoint{2.179176in}{6.161191in}}%
\pgfpathcurveto{\pgfqpoint{2.179176in}{6.155367in}}{\pgfqpoint{2.181490in}{6.149781in}}{\pgfqpoint{2.185608in}{6.145662in}}%
\pgfpathcurveto{\pgfqpoint{2.189726in}{6.141544in}}{\pgfqpoint{2.195312in}{6.139230in}}{\pgfqpoint{2.201136in}{6.139230in}}%
\pgfpathclose%
\pgfusepath{stroke,fill}%
\end{pgfscope}%
\begin{pgfscope}%
\pgfpathrectangle{\pgfqpoint{0.506010in}{1.121191in}}{\pgfqpoint{2.325000in}{1.400000in}} %
\pgfusepath{clip}%
\pgfsetbuttcap%
\pgfsetroundjoin%
\definecolor{currentfill}{rgb}{0.000000,0.500000,0.000000}%
\pgfsetfillcolor{currentfill}%
\pgfsetlinewidth{1.003750pt}%
\definecolor{currentstroke}{rgb}{0.000000,0.500000,0.000000}%
\pgfsetstrokecolor{currentstroke}%
\pgfsetdash{}{0pt}%
\pgfpathmoveto{\pgfqpoint{2.233373in}{6.139230in}}%
\pgfpathcurveto{\pgfqpoint{2.239197in}{6.139230in}}{\pgfqpoint{2.244783in}{6.141544in}}{\pgfqpoint{2.248901in}{6.145662in}}%
\pgfpathcurveto{\pgfqpoint{2.253019in}{6.149781in}}{\pgfqpoint{2.255333in}{6.155367in}}{\pgfqpoint{2.255333in}{6.161191in}}%
\pgfpathcurveto{\pgfqpoint{2.255333in}{6.167015in}}{\pgfqpoint{2.253019in}{6.172601in}}{\pgfqpoint{2.248901in}{6.176719in}}%
\pgfpathcurveto{\pgfqpoint{2.244783in}{6.180837in}}{\pgfqpoint{2.239197in}{6.183151in}}{\pgfqpoint{2.233373in}{6.183151in}}%
\pgfpathcurveto{\pgfqpoint{2.227549in}{6.183151in}}{\pgfqpoint{2.221963in}{6.180837in}}{\pgfqpoint{2.217845in}{6.176719in}}%
\pgfpathcurveto{\pgfqpoint{2.213727in}{6.172601in}}{\pgfqpoint{2.211413in}{6.167015in}}{\pgfqpoint{2.211413in}{6.161191in}}%
\pgfpathcurveto{\pgfqpoint{2.211413in}{6.155367in}}{\pgfqpoint{2.213727in}{6.149781in}}{\pgfqpoint{2.217845in}{6.145662in}}%
\pgfpathcurveto{\pgfqpoint{2.221963in}{6.141544in}}{\pgfqpoint{2.227549in}{6.139230in}}{\pgfqpoint{2.233373in}{6.139230in}}%
\pgfpathclose%
\pgfusepath{stroke,fill}%
\end{pgfscope}%
\begin{pgfscope}%
\pgfpathrectangle{\pgfqpoint{0.506010in}{1.121191in}}{\pgfqpoint{2.325000in}{1.400000in}} %
\pgfusepath{clip}%
\pgfsetbuttcap%
\pgfsetroundjoin%
\definecolor{currentfill}{rgb}{0.000000,0.500000,0.000000}%
\pgfsetfillcolor{currentfill}%
\pgfsetlinewidth{1.003750pt}%
\definecolor{currentstroke}{rgb}{0.000000,0.500000,0.000000}%
\pgfsetstrokecolor{currentstroke}%
\pgfsetdash{}{0pt}%
\pgfpathmoveto{\pgfqpoint{2.222779in}{6.139230in}}%
\pgfpathcurveto{\pgfqpoint{2.228603in}{6.139230in}}{\pgfqpoint{2.234189in}{6.141544in}}{\pgfqpoint{2.238307in}{6.145662in}}%
\pgfpathcurveto{\pgfqpoint{2.242425in}{6.149781in}}{\pgfqpoint{2.244739in}{6.155367in}}{\pgfqpoint{2.244739in}{6.161191in}}%
\pgfpathcurveto{\pgfqpoint{2.244739in}{6.167015in}}{\pgfqpoint{2.242425in}{6.172601in}}{\pgfqpoint{2.238307in}{6.176719in}}%
\pgfpathcurveto{\pgfqpoint{2.234189in}{6.180837in}}{\pgfqpoint{2.228603in}{6.183151in}}{\pgfqpoint{2.222779in}{6.183151in}}%
\pgfpathcurveto{\pgfqpoint{2.216955in}{6.183151in}}{\pgfqpoint{2.211369in}{6.180837in}}{\pgfqpoint{2.207250in}{6.176719in}}%
\pgfpathcurveto{\pgfqpoint{2.203132in}{6.172601in}}{\pgfqpoint{2.200818in}{6.167015in}}{\pgfqpoint{2.200818in}{6.161191in}}%
\pgfpathcurveto{\pgfqpoint{2.200818in}{6.155367in}}{\pgfqpoint{2.203132in}{6.149781in}}{\pgfqpoint{2.207250in}{6.145662in}}%
\pgfpathcurveto{\pgfqpoint{2.211369in}{6.141544in}}{\pgfqpoint{2.216955in}{6.139230in}}{\pgfqpoint{2.222779in}{6.139230in}}%
\pgfpathclose%
\pgfusepath{stroke,fill}%
\end{pgfscope}%
\begin{pgfscope}%
\pgfpathrectangle{\pgfqpoint{0.506010in}{1.121191in}}{\pgfqpoint{2.325000in}{1.400000in}} %
\pgfusepath{clip}%
\pgfsetbuttcap%
\pgfsetroundjoin%
\definecolor{currentfill}{rgb}{0.000000,0.500000,0.000000}%
\pgfsetfillcolor{currentfill}%
\pgfsetlinewidth{1.003750pt}%
\definecolor{currentstroke}{rgb}{0.000000,0.500000,0.000000}%
\pgfsetstrokecolor{currentstroke}%
\pgfsetdash{}{0pt}%
\pgfpathmoveto{\pgfqpoint{2.248922in}{6.139230in}}%
\pgfpathcurveto{\pgfqpoint{2.254746in}{6.139230in}}{\pgfqpoint{2.260332in}{6.141544in}}{\pgfqpoint{2.264451in}{6.145662in}}%
\pgfpathcurveto{\pgfqpoint{2.268569in}{6.149781in}}{\pgfqpoint{2.270883in}{6.155367in}}{\pgfqpoint{2.270883in}{6.161191in}}%
\pgfpathcurveto{\pgfqpoint{2.270883in}{6.167015in}}{\pgfqpoint{2.268569in}{6.172601in}}{\pgfqpoint{2.264451in}{6.176719in}}%
\pgfpathcurveto{\pgfqpoint{2.260332in}{6.180837in}}{\pgfqpoint{2.254746in}{6.183151in}}{\pgfqpoint{2.248922in}{6.183151in}}%
\pgfpathcurveto{\pgfqpoint{2.243098in}{6.183151in}}{\pgfqpoint{2.237512in}{6.180837in}}{\pgfqpoint{2.233394in}{6.176719in}}%
\pgfpathcurveto{\pgfqpoint{2.229276in}{6.172601in}}{\pgfqpoint{2.226962in}{6.167015in}}{\pgfqpoint{2.226962in}{6.161191in}}%
\pgfpathcurveto{\pgfqpoint{2.226962in}{6.155367in}}{\pgfqpoint{2.229276in}{6.149781in}}{\pgfqpoint{2.233394in}{6.145662in}}%
\pgfpathcurveto{\pgfqpoint{2.237512in}{6.141544in}}{\pgfqpoint{2.243098in}{6.139230in}}{\pgfqpoint{2.248922in}{6.139230in}}%
\pgfpathclose%
\pgfusepath{stroke,fill}%
\end{pgfscope}%
\begin{pgfscope}%
\pgfpathrectangle{\pgfqpoint{0.506010in}{1.121191in}}{\pgfqpoint{2.325000in}{1.400000in}} %
\pgfusepath{clip}%
\pgfsetbuttcap%
\pgfsetroundjoin%
\definecolor{currentfill}{rgb}{0.000000,0.500000,0.000000}%
\pgfsetfillcolor{currentfill}%
\pgfsetlinewidth{1.003750pt}%
\definecolor{currentstroke}{rgb}{0.000000,0.500000,0.000000}%
\pgfsetstrokecolor{currentstroke}%
\pgfsetdash{}{0pt}%
\pgfpathmoveto{\pgfqpoint{2.246565in}{6.139230in}}%
\pgfpathcurveto{\pgfqpoint{2.252389in}{6.139230in}}{\pgfqpoint{2.257975in}{6.141544in}}{\pgfqpoint{2.262093in}{6.145662in}}%
\pgfpathcurveto{\pgfqpoint{2.266211in}{6.149781in}}{\pgfqpoint{2.268525in}{6.155367in}}{\pgfqpoint{2.268525in}{6.161191in}}%
\pgfpathcurveto{\pgfqpoint{2.268525in}{6.167015in}}{\pgfqpoint{2.266211in}{6.172601in}}{\pgfqpoint{2.262093in}{6.176719in}}%
\pgfpathcurveto{\pgfqpoint{2.257975in}{6.180837in}}{\pgfqpoint{2.252389in}{6.183151in}}{\pgfqpoint{2.246565in}{6.183151in}}%
\pgfpathcurveto{\pgfqpoint{2.240741in}{6.183151in}}{\pgfqpoint{2.235154in}{6.180837in}}{\pgfqpoint{2.231036in}{6.176719in}}%
\pgfpathcurveto{\pgfqpoint{2.226918in}{6.172601in}}{\pgfqpoint{2.224604in}{6.167015in}}{\pgfqpoint{2.224604in}{6.161191in}}%
\pgfpathcurveto{\pgfqpoint{2.224604in}{6.155367in}}{\pgfqpoint{2.226918in}{6.149781in}}{\pgfqpoint{2.231036in}{6.145662in}}%
\pgfpathcurveto{\pgfqpoint{2.235154in}{6.141544in}}{\pgfqpoint{2.240741in}{6.139230in}}{\pgfqpoint{2.246565in}{6.139230in}}%
\pgfpathclose%
\pgfusepath{stroke,fill}%
\end{pgfscope}%
\begin{pgfscope}%
\pgfpathrectangle{\pgfqpoint{0.506010in}{1.121191in}}{\pgfqpoint{2.325000in}{1.400000in}} %
\pgfusepath{clip}%
\pgfsetbuttcap%
\pgfsetroundjoin%
\definecolor{currentfill}{rgb}{0.000000,0.500000,0.000000}%
\pgfsetfillcolor{currentfill}%
\pgfsetlinewidth{1.003750pt}%
\definecolor{currentstroke}{rgb}{0.000000,0.500000,0.000000}%
\pgfsetstrokecolor{currentstroke}%
\pgfsetdash{}{0pt}%
\pgfpathmoveto{\pgfqpoint{2.233956in}{6.139230in}}%
\pgfpathcurveto{\pgfqpoint{2.239780in}{6.139230in}}{\pgfqpoint{2.245367in}{6.141544in}}{\pgfqpoint{2.249485in}{6.145662in}}%
\pgfpathcurveto{\pgfqpoint{2.253603in}{6.149781in}}{\pgfqpoint{2.255917in}{6.155367in}}{\pgfqpoint{2.255917in}{6.161191in}}%
\pgfpathcurveto{\pgfqpoint{2.255917in}{6.167015in}}{\pgfqpoint{2.253603in}{6.172601in}}{\pgfqpoint{2.249485in}{6.176719in}}%
\pgfpathcurveto{\pgfqpoint{2.245367in}{6.180837in}}{\pgfqpoint{2.239780in}{6.183151in}}{\pgfqpoint{2.233956in}{6.183151in}}%
\pgfpathcurveto{\pgfqpoint{2.228133in}{6.183151in}}{\pgfqpoint{2.222546in}{6.180837in}}{\pgfqpoint{2.218428in}{6.176719in}}%
\pgfpathcurveto{\pgfqpoint{2.214310in}{6.172601in}}{\pgfqpoint{2.211996in}{6.167015in}}{\pgfqpoint{2.211996in}{6.161191in}}%
\pgfpathcurveto{\pgfqpoint{2.211996in}{6.155367in}}{\pgfqpoint{2.214310in}{6.149781in}}{\pgfqpoint{2.218428in}{6.145662in}}%
\pgfpathcurveto{\pgfqpoint{2.222546in}{6.141544in}}{\pgfqpoint{2.228133in}{6.139230in}}{\pgfqpoint{2.233956in}{6.139230in}}%
\pgfpathclose%
\pgfusepath{stroke,fill}%
\end{pgfscope}%
\begin{pgfscope}%
\pgfpathrectangle{\pgfqpoint{0.506010in}{1.121191in}}{\pgfqpoint{2.325000in}{1.400000in}} %
\pgfusepath{clip}%
\pgfsetbuttcap%
\pgfsetroundjoin%
\definecolor{currentfill}{rgb}{0.000000,0.500000,0.000000}%
\pgfsetfillcolor{currentfill}%
\pgfsetlinewidth{1.003750pt}%
\definecolor{currentstroke}{rgb}{0.000000,0.500000,0.000000}%
\pgfsetstrokecolor{currentstroke}%
\pgfsetdash{}{0pt}%
\pgfpathmoveto{\pgfqpoint{2.251609in}{6.139230in}}%
\pgfpathcurveto{\pgfqpoint{2.257433in}{6.139230in}}{\pgfqpoint{2.263019in}{6.141544in}}{\pgfqpoint{2.267137in}{6.145662in}}%
\pgfpathcurveto{\pgfqpoint{2.271255in}{6.149781in}}{\pgfqpoint{2.273569in}{6.155367in}}{\pgfqpoint{2.273569in}{6.161191in}}%
\pgfpathcurveto{\pgfqpoint{2.273569in}{6.167015in}}{\pgfqpoint{2.271255in}{6.172601in}}{\pgfqpoint{2.267137in}{6.176719in}}%
\pgfpathcurveto{\pgfqpoint{2.263019in}{6.180837in}}{\pgfqpoint{2.257433in}{6.183151in}}{\pgfqpoint{2.251609in}{6.183151in}}%
\pgfpathcurveto{\pgfqpoint{2.245785in}{6.183151in}}{\pgfqpoint{2.240199in}{6.180837in}}{\pgfqpoint{2.236081in}{6.176719in}}%
\pgfpathcurveto{\pgfqpoint{2.231963in}{6.172601in}}{\pgfqpoint{2.229649in}{6.167015in}}{\pgfqpoint{2.229649in}{6.161191in}}%
\pgfpathcurveto{\pgfqpoint{2.229649in}{6.155367in}}{\pgfqpoint{2.231963in}{6.149781in}}{\pgfqpoint{2.236081in}{6.145662in}}%
\pgfpathcurveto{\pgfqpoint{2.240199in}{6.141544in}}{\pgfqpoint{2.245785in}{6.139230in}}{\pgfqpoint{2.251609in}{6.139230in}}%
\pgfpathclose%
\pgfusepath{stroke,fill}%
\end{pgfscope}%
\begin{pgfscope}%
\pgfpathrectangle{\pgfqpoint{0.506010in}{1.121191in}}{\pgfqpoint{2.325000in}{1.400000in}} %
\pgfusepath{clip}%
\pgfsetbuttcap%
\pgfsetroundjoin%
\definecolor{currentfill}{rgb}{0.000000,0.500000,0.000000}%
\pgfsetfillcolor{currentfill}%
\pgfsetlinewidth{1.003750pt}%
\definecolor{currentstroke}{rgb}{0.000000,0.500000,0.000000}%
\pgfsetstrokecolor{currentstroke}%
\pgfsetdash{}{0pt}%
\pgfpathmoveto{\pgfqpoint{2.238184in}{6.139230in}}%
\pgfpathcurveto{\pgfqpoint{2.244008in}{6.139230in}}{\pgfqpoint{2.249594in}{6.141544in}}{\pgfqpoint{2.253712in}{6.145662in}}%
\pgfpathcurveto{\pgfqpoint{2.257830in}{6.149781in}}{\pgfqpoint{2.260144in}{6.155367in}}{\pgfqpoint{2.260144in}{6.161191in}}%
\pgfpathcurveto{\pgfqpoint{2.260144in}{6.167015in}}{\pgfqpoint{2.257830in}{6.172601in}}{\pgfqpoint{2.253712in}{6.176719in}}%
\pgfpathcurveto{\pgfqpoint{2.249594in}{6.180837in}}{\pgfqpoint{2.244008in}{6.183151in}}{\pgfqpoint{2.238184in}{6.183151in}}%
\pgfpathcurveto{\pgfqpoint{2.232360in}{6.183151in}}{\pgfqpoint{2.226774in}{6.180837in}}{\pgfqpoint{2.222656in}{6.176719in}}%
\pgfpathcurveto{\pgfqpoint{2.218537in}{6.172601in}}{\pgfqpoint{2.216224in}{6.167015in}}{\pgfqpoint{2.216224in}{6.161191in}}%
\pgfpathcurveto{\pgfqpoint{2.216224in}{6.155367in}}{\pgfqpoint{2.218537in}{6.149781in}}{\pgfqpoint{2.222656in}{6.145662in}}%
\pgfpathcurveto{\pgfqpoint{2.226774in}{6.141544in}}{\pgfqpoint{2.232360in}{6.139230in}}{\pgfqpoint{2.238184in}{6.139230in}}%
\pgfpathclose%
\pgfusepath{stroke,fill}%
\end{pgfscope}%
\begin{pgfscope}%
\pgfpathrectangle{\pgfqpoint{0.506010in}{1.121191in}}{\pgfqpoint{2.325000in}{1.400000in}} %
\pgfusepath{clip}%
\pgfsetbuttcap%
\pgfsetroundjoin%
\definecolor{currentfill}{rgb}{0.000000,0.500000,0.000000}%
\pgfsetfillcolor{currentfill}%
\pgfsetlinewidth{1.003750pt}%
\definecolor{currentstroke}{rgb}{0.000000,0.500000,0.000000}%
\pgfsetstrokecolor{currentstroke}%
\pgfsetdash{}{0pt}%
\pgfpathmoveto{\pgfqpoint{2.248922in}{6.139230in}}%
\pgfpathcurveto{\pgfqpoint{2.254746in}{6.139230in}}{\pgfqpoint{2.260332in}{6.141544in}}{\pgfqpoint{2.264451in}{6.145662in}}%
\pgfpathcurveto{\pgfqpoint{2.268569in}{6.149781in}}{\pgfqpoint{2.270883in}{6.155367in}}{\pgfqpoint{2.270883in}{6.161191in}}%
\pgfpathcurveto{\pgfqpoint{2.270883in}{6.167015in}}{\pgfqpoint{2.268569in}{6.172601in}}{\pgfqpoint{2.264451in}{6.176719in}}%
\pgfpathcurveto{\pgfqpoint{2.260332in}{6.180837in}}{\pgfqpoint{2.254746in}{6.183151in}}{\pgfqpoint{2.248922in}{6.183151in}}%
\pgfpathcurveto{\pgfqpoint{2.243098in}{6.183151in}}{\pgfqpoint{2.237512in}{6.180837in}}{\pgfqpoint{2.233394in}{6.176719in}}%
\pgfpathcurveto{\pgfqpoint{2.229276in}{6.172601in}}{\pgfqpoint{2.226962in}{6.167015in}}{\pgfqpoint{2.226962in}{6.161191in}}%
\pgfpathcurveto{\pgfqpoint{2.226962in}{6.155367in}}{\pgfqpoint{2.229276in}{6.149781in}}{\pgfqpoint{2.233394in}{6.145662in}}%
\pgfpathcurveto{\pgfqpoint{2.237512in}{6.141544in}}{\pgfqpoint{2.243098in}{6.139230in}}{\pgfqpoint{2.248922in}{6.139230in}}%
\pgfpathclose%
\pgfusepath{stroke,fill}%
\end{pgfscope}%
\begin{pgfscope}%
\pgfpathrectangle{\pgfqpoint{0.506010in}{1.121191in}}{\pgfqpoint{2.325000in}{1.400000in}} %
\pgfusepath{clip}%
\pgfsetbuttcap%
\pgfsetroundjoin%
\definecolor{currentfill}{rgb}{0.000000,0.500000,0.000000}%
\pgfsetfillcolor{currentfill}%
\pgfsetlinewidth{1.003750pt}%
\definecolor{currentstroke}{rgb}{0.000000,0.500000,0.000000}%
\pgfsetstrokecolor{currentstroke}%
\pgfsetdash{}{0pt}%
\pgfpathmoveto{\pgfqpoint{2.262855in}{6.139230in}}%
\pgfpathcurveto{\pgfqpoint{2.268679in}{6.139230in}}{\pgfqpoint{2.274265in}{6.141544in}}{\pgfqpoint{2.278383in}{6.145662in}}%
\pgfpathcurveto{\pgfqpoint{2.282502in}{6.149781in}}{\pgfqpoint{2.284816in}{6.155367in}}{\pgfqpoint{2.284816in}{6.161191in}}%
\pgfpathcurveto{\pgfqpoint{2.284816in}{6.167015in}}{\pgfqpoint{2.282502in}{6.172601in}}{\pgfqpoint{2.278383in}{6.176719in}}%
\pgfpathcurveto{\pgfqpoint{2.274265in}{6.180837in}}{\pgfqpoint{2.268679in}{6.183151in}}{\pgfqpoint{2.262855in}{6.183151in}}%
\pgfpathcurveto{\pgfqpoint{2.257031in}{6.183151in}}{\pgfqpoint{2.251445in}{6.180837in}}{\pgfqpoint{2.247327in}{6.176719in}}%
\pgfpathcurveto{\pgfqpoint{2.243209in}{6.172601in}}{\pgfqpoint{2.240895in}{6.167015in}}{\pgfqpoint{2.240895in}{6.161191in}}%
\pgfpathcurveto{\pgfqpoint{2.240895in}{6.155367in}}{\pgfqpoint{2.243209in}{6.149781in}}{\pgfqpoint{2.247327in}{6.145662in}}%
\pgfpathcurveto{\pgfqpoint{2.251445in}{6.141544in}}{\pgfqpoint{2.257031in}{6.139230in}}{\pgfqpoint{2.262855in}{6.139230in}}%
\pgfpathclose%
\pgfusepath{stroke,fill}%
\end{pgfscope}%
\begin{pgfscope}%
\pgfpathrectangle{\pgfqpoint{0.506010in}{1.121191in}}{\pgfqpoint{2.325000in}{1.400000in}} %
\pgfusepath{clip}%
\pgfsetbuttcap%
\pgfsetroundjoin%
\definecolor{currentfill}{rgb}{0.000000,0.500000,0.000000}%
\pgfsetfillcolor{currentfill}%
\pgfsetlinewidth{1.003750pt}%
\definecolor{currentstroke}{rgb}{0.000000,0.500000,0.000000}%
\pgfsetstrokecolor{currentstroke}%
\pgfsetdash{}{0pt}%
\pgfpathmoveto{\pgfqpoint{2.233956in}{6.139230in}}%
\pgfpathcurveto{\pgfqpoint{2.239780in}{6.139230in}}{\pgfqpoint{2.245367in}{6.141544in}}{\pgfqpoint{2.249485in}{6.145662in}}%
\pgfpathcurveto{\pgfqpoint{2.253603in}{6.149781in}}{\pgfqpoint{2.255917in}{6.155367in}}{\pgfqpoint{2.255917in}{6.161191in}}%
\pgfpathcurveto{\pgfqpoint{2.255917in}{6.167015in}}{\pgfqpoint{2.253603in}{6.172601in}}{\pgfqpoint{2.249485in}{6.176719in}}%
\pgfpathcurveto{\pgfqpoint{2.245367in}{6.180837in}}{\pgfqpoint{2.239780in}{6.183151in}}{\pgfqpoint{2.233956in}{6.183151in}}%
\pgfpathcurveto{\pgfqpoint{2.228133in}{6.183151in}}{\pgfqpoint{2.222546in}{6.180837in}}{\pgfqpoint{2.218428in}{6.176719in}}%
\pgfpathcurveto{\pgfqpoint{2.214310in}{6.172601in}}{\pgfqpoint{2.211996in}{6.167015in}}{\pgfqpoint{2.211996in}{6.161191in}}%
\pgfpathcurveto{\pgfqpoint{2.211996in}{6.155367in}}{\pgfqpoint{2.214310in}{6.149781in}}{\pgfqpoint{2.218428in}{6.145662in}}%
\pgfpathcurveto{\pgfqpoint{2.222546in}{6.141544in}}{\pgfqpoint{2.228133in}{6.139230in}}{\pgfqpoint{2.233956in}{6.139230in}}%
\pgfpathclose%
\pgfusepath{stroke,fill}%
\end{pgfscope}%
\begin{pgfscope}%
\pgfpathrectangle{\pgfqpoint{0.506010in}{1.121191in}}{\pgfqpoint{2.325000in}{1.400000in}} %
\pgfusepath{clip}%
\pgfsetbuttcap%
\pgfsetroundjoin%
\definecolor{currentfill}{rgb}{0.000000,0.500000,0.000000}%
\pgfsetfillcolor{currentfill}%
\pgfsetlinewidth{1.003750pt}%
\definecolor{currentstroke}{rgb}{0.000000,0.500000,0.000000}%
\pgfsetstrokecolor{currentstroke}%
\pgfsetdash{}{0pt}%
\pgfpathmoveto{\pgfqpoint{2.217371in}{6.139230in}}%
\pgfpathcurveto{\pgfqpoint{2.223195in}{6.139230in}}{\pgfqpoint{2.228781in}{6.141544in}}{\pgfqpoint{2.232899in}{6.145662in}}%
\pgfpathcurveto{\pgfqpoint{2.237017in}{6.149781in}}{\pgfqpoint{2.239331in}{6.155367in}}{\pgfqpoint{2.239331in}{6.161191in}}%
\pgfpathcurveto{\pgfqpoint{2.239331in}{6.167015in}}{\pgfqpoint{2.237017in}{6.172601in}}{\pgfqpoint{2.232899in}{6.176719in}}%
\pgfpathcurveto{\pgfqpoint{2.228781in}{6.180837in}}{\pgfqpoint{2.223195in}{6.183151in}}{\pgfqpoint{2.217371in}{6.183151in}}%
\pgfpathcurveto{\pgfqpoint{2.211547in}{6.183151in}}{\pgfqpoint{2.205961in}{6.180837in}}{\pgfqpoint{2.201843in}{6.176719in}}%
\pgfpathcurveto{\pgfqpoint{2.197725in}{6.172601in}}{\pgfqpoint{2.195411in}{6.167015in}}{\pgfqpoint{2.195411in}{6.161191in}}%
\pgfpathcurveto{\pgfqpoint{2.195411in}{6.155367in}}{\pgfqpoint{2.197725in}{6.149781in}}{\pgfqpoint{2.201843in}{6.145662in}}%
\pgfpathcurveto{\pgfqpoint{2.205961in}{6.141544in}}{\pgfqpoint{2.211547in}{6.139230in}}{\pgfqpoint{2.217371in}{6.139230in}}%
\pgfpathclose%
\pgfusepath{stroke,fill}%
\end{pgfscope}%
\begin{pgfscope}%
\pgfpathrectangle{\pgfqpoint{0.506010in}{1.121191in}}{\pgfqpoint{2.325000in}{1.400000in}} %
\pgfusepath{clip}%
\pgfsetbuttcap%
\pgfsetroundjoin%
\definecolor{currentfill}{rgb}{0.000000,0.500000,0.000000}%
\pgfsetfillcolor{currentfill}%
\pgfsetlinewidth{1.003750pt}%
\definecolor{currentstroke}{rgb}{0.000000,0.500000,0.000000}%
\pgfsetstrokecolor{currentstroke}%
\pgfsetdash{}{0pt}%
\pgfpathmoveto{\pgfqpoint{2.184228in}{6.139230in}}%
\pgfpathcurveto{\pgfqpoint{2.190052in}{6.139230in}}{\pgfqpoint{2.195638in}{6.141544in}}{\pgfqpoint{2.199756in}{6.145662in}}%
\pgfpathcurveto{\pgfqpoint{2.203874in}{6.149781in}}{\pgfqpoint{2.206188in}{6.155367in}}{\pgfqpoint{2.206188in}{6.161191in}}%
\pgfpathcurveto{\pgfqpoint{2.206188in}{6.167015in}}{\pgfqpoint{2.203874in}{6.172601in}}{\pgfqpoint{2.199756in}{6.176719in}}%
\pgfpathcurveto{\pgfqpoint{2.195638in}{6.180837in}}{\pgfqpoint{2.190052in}{6.183151in}}{\pgfqpoint{2.184228in}{6.183151in}}%
\pgfpathcurveto{\pgfqpoint{2.178404in}{6.183151in}}{\pgfqpoint{2.172818in}{6.180837in}}{\pgfqpoint{2.168700in}{6.176719in}}%
\pgfpathcurveto{\pgfqpoint{2.164582in}{6.172601in}}{\pgfqpoint{2.162268in}{6.167015in}}{\pgfqpoint{2.162268in}{6.161191in}}%
\pgfpathcurveto{\pgfqpoint{2.162268in}{6.155367in}}{\pgfqpoint{2.164582in}{6.149781in}}{\pgfqpoint{2.168700in}{6.145662in}}%
\pgfpathcurveto{\pgfqpoint{2.172818in}{6.141544in}}{\pgfqpoint{2.178404in}{6.139230in}}{\pgfqpoint{2.184228in}{6.139230in}}%
\pgfpathclose%
\pgfusepath{stroke,fill}%
\end{pgfscope}%
\begin{pgfscope}%
\pgfpathrectangle{\pgfqpoint{0.506010in}{1.121191in}}{\pgfqpoint{2.325000in}{1.400000in}} %
\pgfusepath{clip}%
\pgfsetbuttcap%
\pgfsetroundjoin%
\definecolor{currentfill}{rgb}{0.000000,0.500000,0.000000}%
\pgfsetfillcolor{currentfill}%
\pgfsetlinewidth{1.003750pt}%
\definecolor{currentstroke}{rgb}{0.000000,0.500000,0.000000}%
\pgfsetstrokecolor{currentstroke}%
\pgfsetdash{}{0pt}%
\pgfpathmoveto{\pgfqpoint{2.201136in}{6.139230in}}%
\pgfpathcurveto{\pgfqpoint{2.206960in}{6.139230in}}{\pgfqpoint{2.212546in}{6.141544in}}{\pgfqpoint{2.216665in}{6.145662in}}%
\pgfpathcurveto{\pgfqpoint{2.220783in}{6.149781in}}{\pgfqpoint{2.223097in}{6.155367in}}{\pgfqpoint{2.223097in}{6.161191in}}%
\pgfpathcurveto{\pgfqpoint{2.223097in}{6.167015in}}{\pgfqpoint{2.220783in}{6.172601in}}{\pgfqpoint{2.216665in}{6.176719in}}%
\pgfpathcurveto{\pgfqpoint{2.212546in}{6.180837in}}{\pgfqpoint{2.206960in}{6.183151in}}{\pgfqpoint{2.201136in}{6.183151in}}%
\pgfpathcurveto{\pgfqpoint{2.195312in}{6.183151in}}{\pgfqpoint{2.189726in}{6.180837in}}{\pgfqpoint{2.185608in}{6.176719in}}%
\pgfpathcurveto{\pgfqpoint{2.181490in}{6.172601in}}{\pgfqpoint{2.179176in}{6.167015in}}{\pgfqpoint{2.179176in}{6.161191in}}%
\pgfpathcurveto{\pgfqpoint{2.179176in}{6.155367in}}{\pgfqpoint{2.181490in}{6.149781in}}{\pgfqpoint{2.185608in}{6.145662in}}%
\pgfpathcurveto{\pgfqpoint{2.189726in}{6.141544in}}{\pgfqpoint{2.195312in}{6.139230in}}{\pgfqpoint{2.201136in}{6.139230in}}%
\pgfpathclose%
\pgfusepath{stroke,fill}%
\end{pgfscope}%
\begin{pgfscope}%
\pgfpathrectangle{\pgfqpoint{0.506010in}{1.121191in}}{\pgfqpoint{2.325000in}{1.400000in}} %
\pgfusepath{clip}%
\pgfsetbuttcap%
\pgfsetroundjoin%
\definecolor{currentfill}{rgb}{1.000000,0.000000,0.000000}%
\pgfsetfillcolor{currentfill}%
\pgfsetlinewidth{1.003750pt}%
\definecolor{currentstroke}{rgb}{1.000000,0.000000,0.000000}%
\pgfsetstrokecolor{currentstroke}%
\pgfsetdash{}{0pt}%
\pgfpathmoveto{\pgfqpoint{1.151065in}{1.100322in}}%
\pgfpathmoveto{\pgfqpoint{1.156500in}{1.111191in}}%
\pgfpathlineto{\pgfqpoint{1.173026in}{1.144243in}}%
\pgfpathlineto{\pgfqpoint{1.129105in}{1.144243in}}%
\pgfpathlineto{\pgfqpoint{1.151065in}{1.100322in}}%
\pgfusepath{stroke,fill}%
\end{pgfscope}%
\begin{pgfscope}%
\pgfpathrectangle{\pgfqpoint{0.506010in}{1.121191in}}{\pgfqpoint{2.325000in}{1.400000in}} %
\pgfusepath{clip}%
\pgfsetbuttcap%
\pgfsetroundjoin%
\definecolor{currentfill}{rgb}{1.000000,0.000000,0.000000}%
\pgfsetfillcolor{currentfill}%
\pgfsetlinewidth{1.003750pt}%
\definecolor{currentstroke}{rgb}{1.000000,0.000000,0.000000}%
\pgfsetstrokecolor{currentstroke}%
\pgfsetdash{}{0pt}%
\pgfpathmoveto{\pgfqpoint{0.772689in}{1.100322in}}%
\pgfpathmoveto{\pgfqpoint{0.778123in}{1.111191in}}%
\pgfpathlineto{\pgfqpoint{0.794650in}{1.144243in}}%
\pgfpathlineto{\pgfqpoint{0.750729in}{1.144243in}}%
\pgfpathlineto{\pgfqpoint{0.772689in}{1.100322in}}%
\pgfusepath{stroke,fill}%
\end{pgfscope}%
\begin{pgfscope}%
\pgfpathrectangle{\pgfqpoint{0.506010in}{1.121191in}}{\pgfqpoint{2.325000in}{1.400000in}} %
\pgfusepath{clip}%
\pgfsetbuttcap%
\pgfsetroundjoin%
\definecolor{currentfill}{rgb}{1.000000,0.000000,0.000000}%
\pgfsetfillcolor{currentfill}%
\pgfsetlinewidth{1.003750pt}%
\definecolor{currentstroke}{rgb}{1.000000,0.000000,0.000000}%
\pgfsetstrokecolor{currentstroke}%
\pgfsetdash{}{0pt}%
\pgfpathmoveto{\pgfqpoint{1.222833in}{1.100350in}}%
\pgfpathmoveto{\pgfqpoint{1.228253in}{1.111191in}}%
\pgfpathlineto{\pgfqpoint{1.244793in}{1.144271in}}%
\pgfpathlineto{\pgfqpoint{1.200873in}{1.144271in}}%
\pgfpathlineto{\pgfqpoint{1.222833in}{1.100350in}}%
\pgfusepath{stroke,fill}%
\end{pgfscope}%
\begin{pgfscope}%
\pgfpathrectangle{\pgfqpoint{0.506010in}{1.121191in}}{\pgfqpoint{2.325000in}{1.400000in}} %
\pgfusepath{clip}%
\pgfsetbuttcap%
\pgfsetroundjoin%
\definecolor{currentfill}{rgb}{1.000000,0.000000,0.000000}%
\pgfsetfillcolor{currentfill}%
\pgfsetlinewidth{1.003750pt}%
\definecolor{currentstroke}{rgb}{1.000000,0.000000,0.000000}%
\pgfsetstrokecolor{currentstroke}%
\pgfsetdash{}{0pt}%
\pgfpathmoveto{\pgfqpoint{0.980706in}{1.100350in}}%
\pgfpathmoveto{\pgfqpoint{0.986126in}{1.111191in}}%
\pgfpathlineto{\pgfqpoint{1.002666in}{1.144271in}}%
\pgfpathlineto{\pgfqpoint{0.958745in}{1.144271in}}%
\pgfpathlineto{\pgfqpoint{0.980706in}{1.100350in}}%
\pgfusepath{stroke,fill}%
\end{pgfscope}%
\begin{pgfscope}%
\pgfpathrectangle{\pgfqpoint{0.506010in}{1.121191in}}{\pgfqpoint{2.325000in}{1.400000in}} %
\pgfusepath{clip}%
\pgfsetbuttcap%
\pgfsetroundjoin%
\definecolor{currentfill}{rgb}{1.000000,0.000000,0.000000}%
\pgfsetfillcolor{currentfill}%
\pgfsetlinewidth{1.003750pt}%
\definecolor{currentstroke}{rgb}{1.000000,0.000000,0.000000}%
\pgfsetstrokecolor{currentstroke}%
\pgfsetdash{}{0pt}%
\pgfpathmoveto{\pgfqpoint{0.968203in}{1.100350in}}%
\pgfpathmoveto{\pgfqpoint{0.973623in}{1.111191in}}%
\pgfpathlineto{\pgfqpoint{0.990163in}{1.144271in}}%
\pgfpathlineto{\pgfqpoint{0.946243in}{1.144271in}}%
\pgfpathlineto{\pgfqpoint{0.968203in}{1.100350in}}%
\pgfusepath{stroke,fill}%
\end{pgfscope}%
\begin{pgfscope}%
\pgfpathrectangle{\pgfqpoint{0.506010in}{1.121191in}}{\pgfqpoint{2.325000in}{1.400000in}} %
\pgfusepath{clip}%
\pgfsetbuttcap%
\pgfsetroundjoin%
\definecolor{currentfill}{rgb}{1.000000,0.000000,0.000000}%
\pgfsetfillcolor{currentfill}%
\pgfsetlinewidth{1.003750pt}%
\definecolor{currentstroke}{rgb}{1.000000,0.000000,0.000000}%
\pgfsetstrokecolor{currentstroke}%
\pgfsetdash{}{0pt}%
\pgfpathmoveto{\pgfqpoint{1.097148in}{1.100378in}}%
\pgfpathmoveto{\pgfqpoint{1.102554in}{1.111191in}}%
\pgfpathlineto{\pgfqpoint{1.119109in}{1.144299in}}%
\pgfpathlineto{\pgfqpoint{1.075188in}{1.144299in}}%
\pgfpathlineto{\pgfqpoint{1.097148in}{1.100378in}}%
\pgfusepath{stroke,fill}%
\end{pgfscope}%
\begin{pgfscope}%
\pgfpathrectangle{\pgfqpoint{0.506010in}{1.121191in}}{\pgfqpoint{2.325000in}{1.400000in}} %
\pgfusepath{clip}%
\pgfsetbuttcap%
\pgfsetroundjoin%
\definecolor{currentfill}{rgb}{1.000000,0.000000,0.000000}%
\pgfsetfillcolor{currentfill}%
\pgfsetlinewidth{1.003750pt}%
\definecolor{currentstroke}{rgb}{1.000000,0.000000,0.000000}%
\pgfsetstrokecolor{currentstroke}%
\pgfsetdash{}{0pt}%
\pgfpathmoveto{\pgfqpoint{0.813731in}{1.100378in}}%
\pgfpathmoveto{\pgfqpoint{0.819137in}{1.111191in}}%
\pgfpathlineto{\pgfqpoint{0.835691in}{1.144299in}}%
\pgfpathlineto{\pgfqpoint{0.791771in}{1.144299in}}%
\pgfpathlineto{\pgfqpoint{0.813731in}{1.100378in}}%
\pgfusepath{stroke,fill}%
\end{pgfscope}%
\begin{pgfscope}%
\pgfpathrectangle{\pgfqpoint{0.506010in}{1.121191in}}{\pgfqpoint{2.325000in}{1.400000in}} %
\pgfusepath{clip}%
\pgfsetbuttcap%
\pgfsetroundjoin%
\definecolor{currentfill}{rgb}{1.000000,0.000000,0.000000}%
\pgfsetfillcolor{currentfill}%
\pgfsetlinewidth{1.003750pt}%
\definecolor{currentstroke}{rgb}{1.000000,0.000000,0.000000}%
\pgfsetstrokecolor{currentstroke}%
\pgfsetdash{}{0pt}%
\pgfpathmoveto{\pgfqpoint{1.308740in}{1.100406in}}%
\pgfpathmoveto{\pgfqpoint{1.314132in}{1.111191in}}%
\pgfpathlineto{\pgfqpoint{1.330701in}{1.144327in}}%
\pgfpathlineto{\pgfqpoint{1.286780in}{1.144327in}}%
\pgfpathlineto{\pgfqpoint{1.308740in}{1.100406in}}%
\pgfusepath{stroke,fill}%
\end{pgfscope}%
\begin{pgfscope}%
\pgfpathrectangle{\pgfqpoint{0.506010in}{1.121191in}}{\pgfqpoint{2.325000in}{1.400000in}} %
\pgfusepath{clip}%
\pgfsetbuttcap%
\pgfsetroundjoin%
\definecolor{currentfill}{rgb}{1.000000,0.000000,0.000000}%
\pgfsetfillcolor{currentfill}%
\pgfsetlinewidth{1.003750pt}%
\definecolor{currentstroke}{rgb}{1.000000,0.000000,0.000000}%
\pgfsetstrokecolor{currentstroke}%
\pgfsetdash{}{0pt}%
\pgfpathmoveto{\pgfqpoint{0.846633in}{1.100406in}}%
\pgfpathmoveto{\pgfqpoint{0.852025in}{1.111191in}}%
\pgfpathlineto{\pgfqpoint{0.868593in}{1.144327in}}%
\pgfpathlineto{\pgfqpoint{0.824673in}{1.144327in}}%
\pgfpathlineto{\pgfqpoint{0.846633in}{1.100406in}}%
\pgfusepath{stroke,fill}%
\end{pgfscope}%
\begin{pgfscope}%
\pgfpathrectangle{\pgfqpoint{0.506010in}{1.121191in}}{\pgfqpoint{2.325000in}{1.400000in}} %
\pgfusepath{clip}%
\pgfsetbuttcap%
\pgfsetroundjoin%
\definecolor{currentfill}{rgb}{1.000000,0.000000,0.000000}%
\pgfsetfillcolor{currentfill}%
\pgfsetlinewidth{1.003750pt}%
\definecolor{currentstroke}{rgb}{1.000000,0.000000,0.000000}%
\pgfsetstrokecolor{currentstroke}%
\pgfsetdash{}{0pt}%
\pgfpathmoveto{\pgfqpoint{1.144617in}{1.100406in}}%
\pgfpathmoveto{\pgfqpoint{1.150009in}{1.111191in}}%
\pgfpathlineto{\pgfqpoint{1.166577in}{1.144327in}}%
\pgfpathlineto{\pgfqpoint{1.122657in}{1.144327in}}%
\pgfpathlineto{\pgfqpoint{1.144617in}{1.100406in}}%
\pgfusepath{stroke,fill}%
\end{pgfscope}%
\begin{pgfscope}%
\pgfpathrectangle{\pgfqpoint{0.506010in}{1.121191in}}{\pgfqpoint{2.325000in}{1.400000in}} %
\pgfusepath{clip}%
\pgfsetbuttcap%
\pgfsetroundjoin%
\definecolor{currentfill}{rgb}{1.000000,0.000000,0.000000}%
\pgfsetfillcolor{currentfill}%
\pgfsetlinewidth{1.003750pt}%
\definecolor{currentstroke}{rgb}{1.000000,0.000000,0.000000}%
\pgfsetstrokecolor{currentstroke}%
\pgfsetdash{}{0pt}%
\pgfpathmoveto{\pgfqpoint{1.165504in}{1.100434in}}%
\pgfpathmoveto{\pgfqpoint{1.170882in}{1.111191in}}%
\pgfpathlineto{\pgfqpoint{1.187464in}{1.144355in}}%
\pgfpathlineto{\pgfqpoint{1.143544in}{1.144355in}}%
\pgfpathlineto{\pgfqpoint{1.165504in}{1.100434in}}%
\pgfusepath{stroke,fill}%
\end{pgfscope}%
\begin{pgfscope}%
\pgfpathrectangle{\pgfqpoint{0.506010in}{1.121191in}}{\pgfqpoint{2.325000in}{1.400000in}} %
\pgfusepath{clip}%
\pgfsetbuttcap%
\pgfsetroundjoin%
\definecolor{currentfill}{rgb}{1.000000,0.000000,0.000000}%
\pgfsetfillcolor{currentfill}%
\pgfsetlinewidth{1.003750pt}%
\definecolor{currentstroke}{rgb}{1.000000,0.000000,0.000000}%
\pgfsetstrokecolor{currentstroke}%
\pgfsetdash{}{0pt}%
\pgfpathmoveto{\pgfqpoint{0.788475in}{1.100434in}}%
\pgfpathmoveto{\pgfqpoint{0.793853in}{1.111191in}}%
\pgfpathlineto{\pgfqpoint{0.810435in}{1.144355in}}%
\pgfpathlineto{\pgfqpoint{0.766515in}{1.144355in}}%
\pgfpathlineto{\pgfqpoint{0.788475in}{1.100434in}}%
\pgfusepath{stroke,fill}%
\end{pgfscope}%
\begin{pgfscope}%
\pgfpathrectangle{\pgfqpoint{0.506010in}{1.121191in}}{\pgfqpoint{2.325000in}{1.400000in}} %
\pgfusepath{clip}%
\pgfsetbuttcap%
\pgfsetroundjoin%
\definecolor{currentfill}{rgb}{1.000000,0.000000,0.000000}%
\pgfsetfillcolor{currentfill}%
\pgfsetlinewidth{1.003750pt}%
\definecolor{currentstroke}{rgb}{1.000000,0.000000,0.000000}%
\pgfsetstrokecolor{currentstroke}%
\pgfsetdash{}{0pt}%
\pgfpathmoveto{\pgfqpoint{1.587679in}{1.100462in}}%
\pgfpathmoveto{\pgfqpoint{1.593043in}{1.111191in}}%
\pgfpathlineto{\pgfqpoint{1.609639in}{1.144383in}}%
\pgfpathlineto{\pgfqpoint{1.565719in}{1.144383in}}%
\pgfpathlineto{\pgfqpoint{1.587679in}{1.100462in}}%
\pgfusepath{stroke,fill}%
\end{pgfscope}%
\begin{pgfscope}%
\pgfpathrectangle{\pgfqpoint{0.506010in}{1.121191in}}{\pgfqpoint{2.325000in}{1.400000in}} %
\pgfusepath{clip}%
\pgfsetbuttcap%
\pgfsetroundjoin%
\definecolor{currentfill}{rgb}{1.000000,0.000000,0.000000}%
\pgfsetfillcolor{currentfill}%
\pgfsetlinewidth{1.003750pt}%
\definecolor{currentstroke}{rgb}{1.000000,0.000000,0.000000}%
\pgfsetstrokecolor{currentstroke}%
\pgfsetdash{}{0pt}%
\pgfpathmoveto{\pgfqpoint{1.366286in}{1.100518in}}%
\pgfpathmoveto{\pgfqpoint{1.371622in}{1.111191in}}%
\pgfpathlineto{\pgfqpoint{1.388247in}{1.144439in}}%
\pgfpathlineto{\pgfqpoint{1.344326in}{1.144439in}}%
\pgfpathlineto{\pgfqpoint{1.366286in}{1.100518in}}%
\pgfusepath{stroke,fill}%
\end{pgfscope}%
\begin{pgfscope}%
\pgfpathrectangle{\pgfqpoint{0.506010in}{1.121191in}}{\pgfqpoint{2.325000in}{1.400000in}} %
\pgfusepath{clip}%
\pgfsetbuttcap%
\pgfsetroundjoin%
\definecolor{currentfill}{rgb}{1.000000,0.000000,0.000000}%
\pgfsetfillcolor{currentfill}%
\pgfsetlinewidth{1.003750pt}%
\definecolor{currentstroke}{rgb}{1.000000,0.000000,0.000000}%
\pgfsetstrokecolor{currentstroke}%
\pgfsetdash{}{0pt}%
\pgfpathmoveto{\pgfqpoint{1.279201in}{1.100602in}}%
\pgfpathmoveto{\pgfqpoint{1.284495in}{1.111191in}}%
\pgfpathlineto{\pgfqpoint{1.301161in}{1.144523in}}%
\pgfpathlineto{\pgfqpoint{1.257240in}{1.144523in}}%
\pgfpathlineto{\pgfqpoint{1.279201in}{1.100602in}}%
\pgfusepath{stroke,fill}%
\end{pgfscope}%
\begin{pgfscope}%
\pgfpathrectangle{\pgfqpoint{0.506010in}{1.121191in}}{\pgfqpoint{2.325000in}{1.400000in}} %
\pgfusepath{clip}%
\pgfsetbuttcap%
\pgfsetroundjoin%
\definecolor{currentfill}{rgb}{1.000000,0.000000,0.000000}%
\pgfsetfillcolor{currentfill}%
\pgfsetlinewidth{1.003750pt}%
\definecolor{currentstroke}{rgb}{1.000000,0.000000,0.000000}%
\pgfsetstrokecolor{currentstroke}%
\pgfsetdash{}{0pt}%
\pgfpathmoveto{\pgfqpoint{2.264030in}{1.100602in}}%
\pgfpathmoveto{\pgfqpoint{2.269324in}{1.111191in}}%
\pgfpathlineto{\pgfqpoint{2.285990in}{1.144523in}}%
\pgfpathlineto{\pgfqpoint{2.242070in}{1.144523in}}%
\pgfpathlineto{\pgfqpoint{2.264030in}{1.100602in}}%
\pgfusepath{stroke,fill}%
\end{pgfscope}%
\begin{pgfscope}%
\pgfpathrectangle{\pgfqpoint{0.506010in}{1.121191in}}{\pgfqpoint{2.325000in}{1.400000in}} %
\pgfusepath{clip}%
\pgfsetbuttcap%
\pgfsetroundjoin%
\definecolor{currentfill}{rgb}{1.000000,0.000000,0.000000}%
\pgfsetfillcolor{currentfill}%
\pgfsetlinewidth{1.003750pt}%
\definecolor{currentstroke}{rgb}{1.000000,0.000000,0.000000}%
\pgfsetstrokecolor{currentstroke}%
\pgfsetdash{}{0pt}%
\pgfpathmoveto{\pgfqpoint{2.191001in}{1.101022in}}%
\pgfpathmoveto{\pgfqpoint{2.196086in}{1.111191in}}%
\pgfpathlineto{\pgfqpoint{2.212962in}{1.144943in}}%
\pgfpathlineto{\pgfqpoint{2.169041in}{1.144943in}}%
\pgfpathlineto{\pgfqpoint{2.191001in}{1.101022in}}%
\pgfusepath{stroke,fill}%
\end{pgfscope}%
\begin{pgfscope}%
\pgfpathrectangle{\pgfqpoint{0.506010in}{1.121191in}}{\pgfqpoint{2.325000in}{1.400000in}} %
\pgfusepath{clip}%
\pgfsetbuttcap%
\pgfsetroundjoin%
\definecolor{currentfill}{rgb}{1.000000,0.000000,0.000000}%
\pgfsetfillcolor{currentfill}%
\pgfsetlinewidth{1.003750pt}%
\definecolor{currentstroke}{rgb}{1.000000,0.000000,0.000000}%
\pgfsetstrokecolor{currentstroke}%
\pgfsetdash{}{0pt}%
\pgfpathmoveto{\pgfqpoint{2.263023in}{1.101050in}}%
\pgfpathmoveto{\pgfqpoint{2.268094in}{1.111191in}}%
\pgfpathlineto{\pgfqpoint{2.284984in}{1.144971in}}%
\pgfpathlineto{\pgfqpoint{2.241063in}{1.144971in}}%
\pgfpathlineto{\pgfqpoint{2.263023in}{1.101050in}}%
\pgfusepath{stroke,fill}%
\end{pgfscope}%
\begin{pgfscope}%
\pgfpathrectangle{\pgfqpoint{0.506010in}{1.121191in}}{\pgfqpoint{2.325000in}{1.400000in}} %
\pgfusepath{clip}%
\pgfsetbuttcap%
\pgfsetroundjoin%
\definecolor{currentfill}{rgb}{1.000000,0.000000,0.000000}%
\pgfsetfillcolor{currentfill}%
\pgfsetlinewidth{1.003750pt}%
\definecolor{currentstroke}{rgb}{1.000000,0.000000,0.000000}%
\pgfsetstrokecolor{currentstroke}%
\pgfsetdash{}{0pt}%
\pgfpathmoveto{\pgfqpoint{2.251667in}{1.101050in}}%
\pgfpathmoveto{\pgfqpoint{2.256737in}{1.111191in}}%
\pgfpathlineto{\pgfqpoint{2.273627in}{1.144971in}}%
\pgfpathlineto{\pgfqpoint{2.229707in}{1.144971in}}%
\pgfpathlineto{\pgfqpoint{2.251667in}{1.101050in}}%
\pgfusepath{stroke,fill}%
\end{pgfscope}%
\begin{pgfscope}%
\pgfpathrectangle{\pgfqpoint{0.506010in}{1.121191in}}{\pgfqpoint{2.325000in}{1.400000in}} %
\pgfusepath{clip}%
\pgfsetbuttcap%
\pgfsetroundjoin%
\definecolor{currentfill}{rgb}{1.000000,0.000000,0.000000}%
\pgfsetfillcolor{currentfill}%
\pgfsetlinewidth{1.003750pt}%
\definecolor{currentstroke}{rgb}{1.000000,0.000000,0.000000}%
\pgfsetstrokecolor{currentstroke}%
\pgfsetdash{}{0pt}%
\pgfpathmoveto{\pgfqpoint{2.223598in}{1.101190in}}%
\pgfpathmoveto{\pgfqpoint{2.228598in}{1.111191in}}%
\pgfpathlineto{\pgfqpoint{2.245558in}{1.145111in}}%
\pgfpathlineto{\pgfqpoint{2.201638in}{1.145111in}}%
\pgfpathlineto{\pgfqpoint{2.223598in}{1.101190in}}%
\pgfusepath{stroke,fill}%
\end{pgfscope}%
\begin{pgfscope}%
\pgfpathrectangle{\pgfqpoint{0.506010in}{1.121191in}}{\pgfqpoint{2.325000in}{1.400000in}} %
\pgfusepath{clip}%
\pgfsetbuttcap%
\pgfsetroundjoin%
\definecolor{currentfill}{rgb}{1.000000,0.000000,0.000000}%
\pgfsetfillcolor{currentfill}%
\pgfsetlinewidth{1.003750pt}%
\definecolor{currentstroke}{rgb}{1.000000,0.000000,0.000000}%
\pgfsetstrokecolor{currentstroke}%
\pgfsetdash{}{0pt}%
\pgfpathmoveto{\pgfqpoint{2.223598in}{1.101442in}}%
\pgfpathmoveto{\pgfqpoint{2.228472in}{1.111191in}}%
\pgfpathlineto{\pgfqpoint{2.245558in}{1.145363in}}%
\pgfpathlineto{\pgfqpoint{2.201638in}{1.145363in}}%
\pgfpathlineto{\pgfqpoint{2.223598in}{1.101442in}}%
\pgfusepath{stroke,fill}%
\end{pgfscope}%
\begin{pgfscope}%
\pgfpathrectangle{\pgfqpoint{0.506010in}{1.121191in}}{\pgfqpoint{2.325000in}{1.400000in}} %
\pgfusepath{clip}%
\pgfsetbuttcap%
\pgfsetroundjoin%
\definecolor{currentfill}{rgb}{1.000000,0.000000,0.000000}%
\pgfsetfillcolor{currentfill}%
\pgfsetlinewidth{1.003750pt}%
\definecolor{currentstroke}{rgb}{1.000000,0.000000,0.000000}%
\pgfsetstrokecolor{currentstroke}%
\pgfsetdash{}{0pt}%
\pgfpathmoveto{\pgfqpoint{1.007822in}{1.101498in}}%
\pgfpathmoveto{\pgfqpoint{1.012668in}{1.111191in}}%
\pgfpathlineto{\pgfqpoint{1.029783in}{1.145419in}}%
\pgfpathlineto{\pgfqpoint{0.985862in}{1.145419in}}%
\pgfpathlineto{\pgfqpoint{1.007822in}{1.101498in}}%
\pgfusepath{stroke,fill}%
\end{pgfscope}%
\begin{pgfscope}%
\pgfpathrectangle{\pgfqpoint{0.506010in}{1.121191in}}{\pgfqpoint{2.325000in}{1.400000in}} %
\pgfusepath{clip}%
\pgfsetbuttcap%
\pgfsetroundjoin%
\definecolor{currentfill}{rgb}{1.000000,0.000000,0.000000}%
\pgfsetfillcolor{currentfill}%
\pgfsetlinewidth{1.003750pt}%
\definecolor{currentstroke}{rgb}{1.000000,0.000000,0.000000}%
\pgfsetstrokecolor{currentstroke}%
\pgfsetdash{}{0pt}%
\pgfpathmoveto{\pgfqpoint{1.007822in}{1.101526in}}%
\pgfpathmoveto{\pgfqpoint{1.012654in}{1.111191in}}%
\pgfpathlineto{\pgfqpoint{1.029783in}{1.145447in}}%
\pgfpathlineto{\pgfqpoint{0.985862in}{1.145447in}}%
\pgfpathlineto{\pgfqpoint{1.007822in}{1.101526in}}%
\pgfusepath{stroke,fill}%
\end{pgfscope}%
\begin{pgfscope}%
\pgfpathrectangle{\pgfqpoint{0.506010in}{1.121191in}}{\pgfqpoint{2.325000in}{1.400000in}} %
\pgfusepath{clip}%
\pgfsetbuttcap%
\pgfsetroundjoin%
\definecolor{currentfill}{rgb}{1.000000,0.000000,0.000000}%
\pgfsetfillcolor{currentfill}%
\pgfsetlinewidth{1.003750pt}%
\definecolor{currentstroke}{rgb}{1.000000,0.000000,0.000000}%
\pgfsetstrokecolor{currentstroke}%
\pgfsetdash{}{0pt}%
\pgfpathmoveto{\pgfqpoint{2.381458in}{1.101582in}}%
\pgfpathmoveto{\pgfqpoint{2.386262in}{1.111191in}}%
\pgfpathlineto{\pgfqpoint{2.403418in}{1.145503in}}%
\pgfpathlineto{\pgfqpoint{2.359497in}{1.145503in}}%
\pgfpathlineto{\pgfqpoint{2.381458in}{1.101582in}}%
\pgfusepath{stroke,fill}%
\end{pgfscope}%
\begin{pgfscope}%
\pgfpathrectangle{\pgfqpoint{0.506010in}{1.121191in}}{\pgfqpoint{2.325000in}{1.400000in}} %
\pgfusepath{clip}%
\pgfsetbuttcap%
\pgfsetroundjoin%
\definecolor{currentfill}{rgb}{1.000000,0.000000,0.000000}%
\pgfsetfillcolor{currentfill}%
\pgfsetlinewidth{1.003750pt}%
\definecolor{currentstroke}{rgb}{1.000000,0.000000,0.000000}%
\pgfsetstrokecolor{currentstroke}%
\pgfsetdash{}{0pt}%
\pgfpathmoveto{\pgfqpoint{2.380105in}{1.101638in}}%
\pgfpathmoveto{\pgfqpoint{2.384881in}{1.111191in}}%
\pgfpathlineto{\pgfqpoint{2.402065in}{1.145559in}}%
\pgfpathlineto{\pgfqpoint{2.358144in}{1.145559in}}%
\pgfpathlineto{\pgfqpoint{2.380105in}{1.101638in}}%
\pgfusepath{stroke,fill}%
\end{pgfscope}%
\begin{pgfscope}%
\pgfpathrectangle{\pgfqpoint{0.506010in}{1.121191in}}{\pgfqpoint{2.325000in}{1.400000in}} %
\pgfusepath{clip}%
\pgfsetbuttcap%
\pgfsetroundjoin%
\definecolor{currentfill}{rgb}{1.000000,0.000000,0.000000}%
\pgfsetfillcolor{currentfill}%
\pgfsetlinewidth{1.003750pt}%
\definecolor{currentstroke}{rgb}{1.000000,0.000000,0.000000}%
\pgfsetstrokecolor{currentstroke}%
\pgfsetdash{}{0pt}%
\pgfpathmoveto{\pgfqpoint{2.407155in}{1.101722in}}%
\pgfpathmoveto{\pgfqpoint{2.411889in}{1.111191in}}%
\pgfpathlineto{\pgfqpoint{2.429115in}{1.145643in}}%
\pgfpathlineto{\pgfqpoint{2.385195in}{1.145643in}}%
\pgfpathlineto{\pgfqpoint{2.407155in}{1.101722in}}%
\pgfusepath{stroke,fill}%
\end{pgfscope}%
\begin{pgfscope}%
\pgfpathrectangle{\pgfqpoint{0.506010in}{1.121191in}}{\pgfqpoint{2.325000in}{1.400000in}} %
\pgfusepath{clip}%
\pgfsetbuttcap%
\pgfsetroundjoin%
\definecolor{currentfill}{rgb}{1.000000,0.000000,0.000000}%
\pgfsetfillcolor{currentfill}%
\pgfsetlinewidth{1.003750pt}%
\definecolor{currentstroke}{rgb}{1.000000,0.000000,0.000000}%
\pgfsetstrokecolor{currentstroke}%
\pgfsetdash{}{0pt}%
\pgfpathmoveto{\pgfqpoint{1.294699in}{1.101862in}}%
\pgfpathmoveto{\pgfqpoint{1.299363in}{1.111191in}}%
\pgfpathlineto{\pgfqpoint{1.316659in}{1.145783in}}%
\pgfpathlineto{\pgfqpoint{1.272739in}{1.145783in}}%
\pgfpathlineto{\pgfqpoint{1.294699in}{1.101862in}}%
\pgfusepath{stroke,fill}%
\end{pgfscope}%
\begin{pgfscope}%
\pgfpathrectangle{\pgfqpoint{0.506010in}{1.121191in}}{\pgfqpoint{2.325000in}{1.400000in}} %
\pgfusepath{clip}%
\pgfsetbuttcap%
\pgfsetroundjoin%
\definecolor{currentfill}{rgb}{1.000000,0.000000,0.000000}%
\pgfsetfillcolor{currentfill}%
\pgfsetlinewidth{1.003750pt}%
\definecolor{currentstroke}{rgb}{1.000000,0.000000,0.000000}%
\pgfsetstrokecolor{currentstroke}%
\pgfsetdash{}{0pt}%
\pgfpathmoveto{\pgfqpoint{1.390732in}{1.101862in}}%
\pgfpathmoveto{\pgfqpoint{1.395397in}{1.111191in}}%
\pgfpathlineto{\pgfqpoint{1.412693in}{1.145783in}}%
\pgfpathlineto{\pgfqpoint{1.368772in}{1.145783in}}%
\pgfpathlineto{\pgfqpoint{1.390732in}{1.101862in}}%
\pgfusepath{stroke,fill}%
\end{pgfscope}%
\begin{pgfscope}%
\pgfpathrectangle{\pgfqpoint{0.506010in}{1.121191in}}{\pgfqpoint{2.325000in}{1.400000in}} %
\pgfusepath{clip}%
\pgfsetbuttcap%
\pgfsetroundjoin%
\definecolor{currentfill}{rgb}{1.000000,0.000000,0.000000}%
\pgfsetfillcolor{currentfill}%
\pgfsetlinewidth{1.003750pt}%
\definecolor{currentstroke}{rgb}{1.000000,0.000000,0.000000}%
\pgfsetstrokecolor{currentstroke}%
\pgfsetdash{}{0pt}%
\pgfpathmoveto{\pgfqpoint{1.305057in}{1.101974in}}%
\pgfpathmoveto{\pgfqpoint{1.309666in}{1.111191in}}%
\pgfpathlineto{\pgfqpoint{1.327018in}{1.145895in}}%
\pgfpathlineto{\pgfqpoint{1.283097in}{1.145895in}}%
\pgfpathlineto{\pgfqpoint{1.305057in}{1.101974in}}%
\pgfusepath{stroke,fill}%
\end{pgfscope}%
\begin{pgfscope}%
\pgfpathrectangle{\pgfqpoint{0.506010in}{1.121191in}}{\pgfqpoint{2.325000in}{1.400000in}} %
\pgfusepath{clip}%
\pgfsetbuttcap%
\pgfsetroundjoin%
\definecolor{currentfill}{rgb}{1.000000,0.000000,0.000000}%
\pgfsetfillcolor{currentfill}%
\pgfsetlinewidth{1.003750pt}%
\definecolor{currentstroke}{rgb}{1.000000,0.000000,0.000000}%
\pgfsetstrokecolor{currentstroke}%
\pgfsetdash{}{0pt}%
\pgfpathmoveto{\pgfqpoint{1.081382in}{1.101974in}}%
\pgfpathmoveto{\pgfqpoint{1.085990in}{1.111191in}}%
\pgfpathlineto{\pgfqpoint{1.103342in}{1.145895in}}%
\pgfpathlineto{\pgfqpoint{1.059422in}{1.145895in}}%
\pgfpathlineto{\pgfqpoint{1.081382in}{1.101974in}}%
\pgfusepath{stroke,fill}%
\end{pgfscope}%
\begin{pgfscope}%
\pgfpathrectangle{\pgfqpoint{0.506010in}{1.121191in}}{\pgfqpoint{2.325000in}{1.400000in}} %
\pgfusepath{clip}%
\pgfsetbuttcap%
\pgfsetroundjoin%
\definecolor{currentfill}{rgb}{1.000000,0.000000,0.000000}%
\pgfsetfillcolor{currentfill}%
\pgfsetlinewidth{1.003750pt}%
\definecolor{currentstroke}{rgb}{1.000000,0.000000,0.000000}%
\pgfsetstrokecolor{currentstroke}%
\pgfsetdash{}{0pt}%
\pgfpathmoveto{\pgfqpoint{0.810364in}{1.102030in}}%
\pgfpathmoveto{\pgfqpoint{0.814944in}{1.111191in}}%
\pgfpathlineto{\pgfqpoint{0.832324in}{1.145951in}}%
\pgfpathlineto{\pgfqpoint{0.788404in}{1.145951in}}%
\pgfpathlineto{\pgfqpoint{0.810364in}{1.102030in}}%
\pgfusepath{stroke,fill}%
\end{pgfscope}%
\begin{pgfscope}%
\pgfpathrectangle{\pgfqpoint{0.506010in}{1.121191in}}{\pgfqpoint{2.325000in}{1.400000in}} %
\pgfusepath{clip}%
\pgfsetbuttcap%
\pgfsetroundjoin%
\definecolor{currentfill}{rgb}{1.000000,0.000000,0.000000}%
\pgfsetfillcolor{currentfill}%
\pgfsetlinewidth{1.003750pt}%
\definecolor{currentstroke}{rgb}{1.000000,0.000000,0.000000}%
\pgfsetstrokecolor{currentstroke}%
\pgfsetdash{}{0pt}%
\pgfpathmoveto{\pgfqpoint{1.366871in}{1.102086in}}%
\pgfpathmoveto{\pgfqpoint{1.371423in}{1.111191in}}%
\pgfpathlineto{\pgfqpoint{1.388831in}{1.146007in}}%
\pgfpathlineto{\pgfqpoint{1.344910in}{1.146007in}}%
\pgfpathlineto{\pgfqpoint{1.366871in}{1.102086in}}%
\pgfusepath{stroke,fill}%
\end{pgfscope}%
\begin{pgfscope}%
\pgfpathrectangle{\pgfqpoint{0.506010in}{1.121191in}}{\pgfqpoint{2.325000in}{1.400000in}} %
\pgfusepath{clip}%
\pgfsetbuttcap%
\pgfsetroundjoin%
\definecolor{currentfill}{rgb}{1.000000,0.000000,0.000000}%
\pgfsetfillcolor{currentfill}%
\pgfsetlinewidth{1.003750pt}%
\definecolor{currentstroke}{rgb}{1.000000,0.000000,0.000000}%
\pgfsetstrokecolor{currentstroke}%
\pgfsetdash{}{0pt}%
\pgfpathmoveto{\pgfqpoint{1.443680in}{1.102114in}}%
\pgfpathmoveto{\pgfqpoint{1.448218in}{1.111191in}}%
\pgfpathlineto{\pgfqpoint{1.465640in}{1.146035in}}%
\pgfpathlineto{\pgfqpoint{1.421720in}{1.146035in}}%
\pgfpathlineto{\pgfqpoint{1.443680in}{1.102114in}}%
\pgfusepath{stroke,fill}%
\end{pgfscope}%
\begin{pgfscope}%
\pgfpathrectangle{\pgfqpoint{0.506010in}{1.121191in}}{\pgfqpoint{2.325000in}{1.400000in}} %
\pgfusepath{clip}%
\pgfsetbuttcap%
\pgfsetroundjoin%
\definecolor{currentfill}{rgb}{1.000000,0.000000,0.000000}%
\pgfsetfillcolor{currentfill}%
\pgfsetlinewidth{1.003750pt}%
\definecolor{currentstroke}{rgb}{1.000000,0.000000,0.000000}%
\pgfsetstrokecolor{currentstroke}%
\pgfsetdash{}{0pt}%
\pgfpathmoveto{\pgfqpoint{1.406887in}{1.102338in}}%
\pgfpathmoveto{\pgfqpoint{1.411313in}{1.111191in}}%
\pgfpathlineto{\pgfqpoint{1.428847in}{1.146259in}}%
\pgfpathlineto{\pgfqpoint{1.384927in}{1.146259in}}%
\pgfpathlineto{\pgfqpoint{1.406887in}{1.102338in}}%
\pgfusepath{stroke,fill}%
\end{pgfscope}%
\begin{pgfscope}%
\pgfpathrectangle{\pgfqpoint{0.506010in}{1.121191in}}{\pgfqpoint{2.325000in}{1.400000in}} %
\pgfusepath{clip}%
\pgfsetbuttcap%
\pgfsetroundjoin%
\definecolor{currentfill}{rgb}{1.000000,0.000000,0.000000}%
\pgfsetfillcolor{currentfill}%
\pgfsetlinewidth{1.003750pt}%
\definecolor{currentstroke}{rgb}{1.000000,0.000000,0.000000}%
\pgfsetstrokecolor{currentstroke}%
\pgfsetdash{}{0pt}%
\pgfpathmoveto{\pgfqpoint{1.108974in}{1.102814in}}%
\pgfpathmoveto{\pgfqpoint{1.113162in}{1.111191in}}%
\pgfpathlineto{\pgfqpoint{1.130934in}{1.146735in}}%
\pgfpathlineto{\pgfqpoint{1.087014in}{1.146735in}}%
\pgfpathlineto{\pgfqpoint{1.108974in}{1.102814in}}%
\pgfusepath{stroke,fill}%
\end{pgfscope}%
\begin{pgfscope}%
\pgfpathrectangle{\pgfqpoint{0.506010in}{1.121191in}}{\pgfqpoint{2.325000in}{1.400000in}} %
\pgfusepath{clip}%
\pgfsetbuttcap%
\pgfsetroundjoin%
\definecolor{currentfill}{rgb}{1.000000,0.000000,0.000000}%
\pgfsetfillcolor{currentfill}%
\pgfsetlinewidth{1.003750pt}%
\definecolor{currentstroke}{rgb}{1.000000,0.000000,0.000000}%
\pgfsetstrokecolor{currentstroke}%
\pgfsetdash{}{0pt}%
\pgfpathmoveto{\pgfqpoint{0.821928in}{1.103010in}}%
\pgfpathmoveto{\pgfqpoint{0.826018in}{1.111191in}}%
\pgfpathlineto{\pgfqpoint{0.843888in}{1.146931in}}%
\pgfpathlineto{\pgfqpoint{0.799968in}{1.146931in}}%
\pgfpathlineto{\pgfqpoint{0.821928in}{1.103010in}}%
\pgfusepath{stroke,fill}%
\end{pgfscope}%
\begin{pgfscope}%
\pgfpathrectangle{\pgfqpoint{0.506010in}{1.121191in}}{\pgfqpoint{2.325000in}{1.400000in}} %
\pgfusepath{clip}%
\pgfsetbuttcap%
\pgfsetroundjoin%
\definecolor{currentfill}{rgb}{1.000000,0.000000,0.000000}%
\pgfsetfillcolor{currentfill}%
\pgfsetlinewidth{1.003750pt}%
\definecolor{currentstroke}{rgb}{1.000000,0.000000,0.000000}%
\pgfsetstrokecolor{currentstroke}%
\pgfsetdash{}{0pt}%
\pgfpathmoveto{\pgfqpoint{1.443680in}{1.103290in}}%
\pgfpathmoveto{\pgfqpoint{1.447630in}{1.111191in}}%
\pgfpathlineto{\pgfqpoint{1.465640in}{1.147211in}}%
\pgfpathlineto{\pgfqpoint{1.421720in}{1.147211in}}%
\pgfpathlineto{\pgfqpoint{1.443680in}{1.103290in}}%
\pgfusepath{stroke,fill}%
\end{pgfscope}%
\begin{pgfscope}%
\pgfpathrectangle{\pgfqpoint{0.506010in}{1.121191in}}{\pgfqpoint{2.325000in}{1.400000in}} %
\pgfusepath{clip}%
\pgfsetbuttcap%
\pgfsetroundjoin%
\definecolor{currentfill}{rgb}{1.000000,0.000000,0.000000}%
\pgfsetfillcolor{currentfill}%
\pgfsetlinewidth{1.003750pt}%
\definecolor{currentstroke}{rgb}{1.000000,0.000000,0.000000}%
\pgfsetstrokecolor{currentstroke}%
\pgfsetdash{}{0pt}%
\pgfpathmoveto{\pgfqpoint{1.379114in}{1.103290in}}%
\pgfpathmoveto{\pgfqpoint{1.383064in}{1.111191in}}%
\pgfpathlineto{\pgfqpoint{1.401074in}{1.147211in}}%
\pgfpathlineto{\pgfqpoint{1.357153in}{1.147211in}}%
\pgfpathlineto{\pgfqpoint{1.379114in}{1.103290in}}%
\pgfusepath{stroke,fill}%
\end{pgfscope}%
\begin{pgfscope}%
\pgfpathrectangle{\pgfqpoint{0.506010in}{1.121191in}}{\pgfqpoint{2.325000in}{1.400000in}} %
\pgfusepath{clip}%
\pgfsetbuttcap%
\pgfsetroundjoin%
\definecolor{currentfill}{rgb}{1.000000,0.000000,0.000000}%
\pgfsetfillcolor{currentfill}%
\pgfsetlinewidth{1.003750pt}%
\definecolor{currentstroke}{rgb}{1.000000,0.000000,0.000000}%
\pgfsetstrokecolor{currentstroke}%
\pgfsetdash{}{0pt}%
\pgfpathmoveto{\pgfqpoint{1.081382in}{1.103430in}}%
\pgfpathmoveto{\pgfqpoint{1.085262in}{1.111191in}}%
\pgfpathlineto{\pgfqpoint{1.103342in}{1.147351in}}%
\pgfpathlineto{\pgfqpoint{1.059422in}{1.147351in}}%
\pgfpathlineto{\pgfqpoint{1.081382in}{1.103430in}}%
\pgfusepath{stroke,fill}%
\end{pgfscope}%
\begin{pgfscope}%
\pgfpathrectangle{\pgfqpoint{0.506010in}{1.121191in}}{\pgfqpoint{2.325000in}{1.400000in}} %
\pgfusepath{clip}%
\pgfsetbuttcap%
\pgfsetroundjoin%
\definecolor{currentfill}{rgb}{1.000000,0.000000,0.000000}%
\pgfsetfillcolor{currentfill}%
\pgfsetlinewidth{1.003750pt}%
\definecolor{currentstroke}{rgb}{1.000000,0.000000,0.000000}%
\pgfsetstrokecolor{currentstroke}%
\pgfsetdash{}{0pt}%
\pgfpathmoveto{\pgfqpoint{1.108974in}{1.104550in}}%
\pgfpathmoveto{\pgfqpoint{1.112294in}{1.111191in}}%
\pgfpathlineto{\pgfqpoint{1.130934in}{1.148471in}}%
\pgfpathlineto{\pgfqpoint{1.087014in}{1.148471in}}%
\pgfpathlineto{\pgfqpoint{1.108974in}{1.104550in}}%
\pgfusepath{stroke,fill}%
\end{pgfscope}%
\begin{pgfscope}%
\pgfpathrectangle{\pgfqpoint{0.506010in}{1.121191in}}{\pgfqpoint{2.325000in}{1.400000in}} %
\pgfusepath{clip}%
\pgfsetbuttcap%
\pgfsetroundjoin%
\definecolor{currentfill}{rgb}{1.000000,0.000000,0.000000}%
\pgfsetfillcolor{currentfill}%
\pgfsetlinewidth{1.003750pt}%
\definecolor{currentstroke}{rgb}{1.000000,0.000000,0.000000}%
\pgfsetstrokecolor{currentstroke}%
\pgfsetdash{}{0pt}%
\pgfpathmoveto{\pgfqpoint{0.681255in}{1.104774in}}%
\pgfpathmoveto{\pgfqpoint{0.684463in}{1.111191in}}%
\pgfpathlineto{\pgfqpoint{0.703215in}{1.148695in}}%
\pgfpathlineto{\pgfqpoint{0.659295in}{1.148695in}}%
\pgfpathlineto{\pgfqpoint{0.681255in}{1.104774in}}%
\pgfusepath{stroke,fill}%
\end{pgfscope}%
\begin{pgfscope}%
\pgfpathrectangle{\pgfqpoint{0.506010in}{1.121191in}}{\pgfqpoint{2.325000in}{1.400000in}} %
\pgfusepath{clip}%
\pgfsetbuttcap%
\pgfsetroundjoin%
\definecolor{currentfill}{rgb}{1.000000,0.000000,0.000000}%
\pgfsetfillcolor{currentfill}%
\pgfsetlinewidth{1.003750pt}%
\definecolor{currentstroke}{rgb}{1.000000,0.000000,0.000000}%
\pgfsetstrokecolor{currentstroke}%
\pgfsetdash{}{0pt}%
\pgfpathmoveto{\pgfqpoint{1.081382in}{1.104886in}}%
\pgfpathmoveto{\pgfqpoint{1.084534in}{1.111191in}}%
\pgfpathlineto{\pgfqpoint{1.103342in}{1.148807in}}%
\pgfpathlineto{\pgfqpoint{1.059422in}{1.148807in}}%
\pgfpathlineto{\pgfqpoint{1.081382in}{1.104886in}}%
\pgfusepath{stroke,fill}%
\end{pgfscope}%
\begin{pgfscope}%
\pgfpathrectangle{\pgfqpoint{0.506010in}{1.121191in}}{\pgfqpoint{2.325000in}{1.400000in}} %
\pgfusepath{clip}%
\pgfsetbuttcap%
\pgfsetroundjoin%
\definecolor{currentfill}{rgb}{1.000000,0.000000,0.000000}%
\pgfsetfillcolor{currentfill}%
\pgfsetlinewidth{1.003750pt}%
\definecolor{currentstroke}{rgb}{1.000000,0.000000,0.000000}%
\pgfsetstrokecolor{currentstroke}%
\pgfsetdash{}{0pt}%
\pgfpathmoveto{\pgfqpoint{1.081382in}{1.105782in}}%
\pgfpathmoveto{\pgfqpoint{1.084086in}{1.111191in}}%
\pgfpathlineto{\pgfqpoint{1.103342in}{1.149703in}}%
\pgfpathlineto{\pgfqpoint{1.059422in}{1.149703in}}%
\pgfpathlineto{\pgfqpoint{1.081382in}{1.105782in}}%
\pgfusepath{stroke,fill}%
\end{pgfscope}%
\begin{pgfscope}%
\pgfpathrectangle{\pgfqpoint{0.506010in}{1.121191in}}{\pgfqpoint{2.325000in}{1.400000in}} %
\pgfusepath{clip}%
\pgfsetbuttcap%
\pgfsetroundjoin%
\definecolor{currentfill}{rgb}{1.000000,0.000000,0.000000}%
\pgfsetfillcolor{currentfill}%
\pgfsetlinewidth{1.003750pt}%
\definecolor{currentstroke}{rgb}{1.000000,0.000000,0.000000}%
\pgfsetstrokecolor{currentstroke}%
\pgfsetdash{}{0pt}%
\pgfpathmoveto{\pgfqpoint{1.007822in}{1.105866in}}%
\pgfpathmoveto{\pgfqpoint{1.010484in}{1.111191in}}%
\pgfpathlineto{\pgfqpoint{1.029783in}{1.149787in}}%
\pgfpathlineto{\pgfqpoint{0.985862in}{1.149787in}}%
\pgfpathlineto{\pgfqpoint{1.007822in}{1.105866in}}%
\pgfusepath{stroke,fill}%
\end{pgfscope}%
\begin{pgfscope}%
\pgfpathrectangle{\pgfqpoint{0.506010in}{1.121191in}}{\pgfqpoint{2.325000in}{1.400000in}} %
\pgfusepath{clip}%
\pgfsetbuttcap%
\pgfsetroundjoin%
\definecolor{currentfill}{rgb}{1.000000,0.000000,0.000000}%
\pgfsetfillcolor{currentfill}%
\pgfsetlinewidth{1.003750pt}%
\definecolor{currentstroke}{rgb}{1.000000,0.000000,0.000000}%
\pgfsetstrokecolor{currentstroke}%
\pgfsetdash{}{0pt}%
\pgfpathmoveto{\pgfqpoint{0.986187in}{1.105894in}}%
\pgfpathmoveto{\pgfqpoint{0.988835in}{1.111191in}}%
\pgfpathlineto{\pgfqpoint{1.008148in}{1.149815in}}%
\pgfpathlineto{\pgfqpoint{0.964227in}{1.149815in}}%
\pgfpathlineto{\pgfqpoint{0.986187in}{1.105894in}}%
\pgfusepath{stroke,fill}%
\end{pgfscope}%
\begin{pgfscope}%
\pgfpathrectangle{\pgfqpoint{0.506010in}{1.121191in}}{\pgfqpoint{2.325000in}{1.400000in}} %
\pgfusepath{clip}%
\pgfsetbuttcap%
\pgfsetroundjoin%
\definecolor{currentfill}{rgb}{1.000000,0.000000,0.000000}%
\pgfsetfillcolor{currentfill}%
\pgfsetlinewidth{1.003750pt}%
\definecolor{currentstroke}{rgb}{1.000000,0.000000,0.000000}%
\pgfsetstrokecolor{currentstroke}%
\pgfsetdash{}{0pt}%
\pgfpathmoveto{\pgfqpoint{1.108974in}{1.106482in}}%
\pgfpathmoveto{\pgfqpoint{1.111328in}{1.111191in}}%
\pgfpathlineto{\pgfqpoint{1.130934in}{1.150403in}}%
\pgfpathlineto{\pgfqpoint{1.087014in}{1.150403in}}%
\pgfpathlineto{\pgfqpoint{1.108974in}{1.106482in}}%
\pgfusepath{stroke,fill}%
\end{pgfscope}%
\begin{pgfscope}%
\pgfpathrectangle{\pgfqpoint{0.506010in}{1.121191in}}{\pgfqpoint{2.325000in}{1.400000in}} %
\pgfusepath{clip}%
\pgfsetbuttcap%
\pgfsetroundjoin%
\definecolor{currentfill}{rgb}{1.000000,0.000000,0.000000}%
\pgfsetfillcolor{currentfill}%
\pgfsetlinewidth{1.003750pt}%
\definecolor{currentstroke}{rgb}{1.000000,0.000000,0.000000}%
\pgfsetstrokecolor{currentstroke}%
\pgfsetdash{}{0pt}%
\pgfpathmoveto{\pgfqpoint{1.081382in}{1.107210in}}%
\pgfpathmoveto{\pgfqpoint{1.083372in}{1.111191in}}%
\pgfpathlineto{\pgfqpoint{1.103342in}{1.151131in}}%
\pgfpathlineto{\pgfqpoint{1.059422in}{1.151131in}}%
\pgfpathlineto{\pgfqpoint{1.081382in}{1.107210in}}%
\pgfusepath{stroke,fill}%
\end{pgfscope}%
\begin{pgfscope}%
\pgfpathrectangle{\pgfqpoint{0.506010in}{1.121191in}}{\pgfqpoint{2.325000in}{1.400000in}} %
\pgfusepath{clip}%
\pgfsetbuttcap%
\pgfsetroundjoin%
\definecolor{currentfill}{rgb}{1.000000,0.000000,0.000000}%
\pgfsetfillcolor{currentfill}%
\pgfsetlinewidth{1.003750pt}%
\definecolor{currentstroke}{rgb}{1.000000,0.000000,0.000000}%
\pgfsetstrokecolor{currentstroke}%
\pgfsetdash{}{0pt}%
\pgfpathmoveto{\pgfqpoint{1.474029in}{1.107210in}}%
\pgfpathmoveto{\pgfqpoint{1.476019in}{1.111191in}}%
\pgfpathlineto{\pgfqpoint{1.495990in}{1.151131in}}%
\pgfpathlineto{\pgfqpoint{1.452069in}{1.151131in}}%
\pgfpathlineto{\pgfqpoint{1.474029in}{1.107210in}}%
\pgfusepath{stroke,fill}%
\end{pgfscope}%
\begin{pgfscope}%
\pgfpathrectangle{\pgfqpoint{0.506010in}{1.121191in}}{\pgfqpoint{2.325000in}{1.400000in}} %
\pgfusepath{clip}%
\pgfsetbuttcap%
\pgfsetroundjoin%
\definecolor{currentfill}{rgb}{1.000000,0.000000,0.000000}%
\pgfsetfillcolor{currentfill}%
\pgfsetlinewidth{1.003750pt}%
\definecolor{currentstroke}{rgb}{1.000000,0.000000,0.000000}%
\pgfsetstrokecolor{currentstroke}%
\pgfsetdash{}{0pt}%
\pgfpathmoveto{\pgfqpoint{1.007822in}{1.107546in}}%
\pgfpathmoveto{\pgfqpoint{1.009644in}{1.111191in}}%
\pgfpathlineto{\pgfqpoint{1.029783in}{1.151467in}}%
\pgfpathlineto{\pgfqpoint{0.985862in}{1.151467in}}%
\pgfpathlineto{\pgfqpoint{1.007822in}{1.107546in}}%
\pgfusepath{stroke,fill}%
\end{pgfscope}%
\begin{pgfscope}%
\pgfpathrectangle{\pgfqpoint{0.506010in}{1.121191in}}{\pgfqpoint{2.325000in}{1.400000in}} %
\pgfusepath{clip}%
\pgfsetbuttcap%
\pgfsetroundjoin%
\definecolor{currentfill}{rgb}{1.000000,0.000000,0.000000}%
\pgfsetfillcolor{currentfill}%
\pgfsetlinewidth{1.003750pt}%
\definecolor{currentstroke}{rgb}{1.000000,0.000000,0.000000}%
\pgfsetstrokecolor{currentstroke}%
\pgfsetdash{}{0pt}%
\pgfpathmoveto{\pgfqpoint{1.081382in}{1.107770in}}%
\pgfpathmoveto{\pgfqpoint{1.083092in}{1.111191in}}%
\pgfpathlineto{\pgfqpoint{1.103342in}{1.151691in}}%
\pgfpathlineto{\pgfqpoint{1.059422in}{1.151691in}}%
\pgfpathlineto{\pgfqpoint{1.081382in}{1.107770in}}%
\pgfusepath{stroke,fill}%
\end{pgfscope}%
\begin{pgfscope}%
\pgfpathrectangle{\pgfqpoint{0.506010in}{1.121191in}}{\pgfqpoint{2.325000in}{1.400000in}} %
\pgfusepath{clip}%
\pgfsetbuttcap%
\pgfsetroundjoin%
\definecolor{currentfill}{rgb}{1.000000,0.000000,0.000000}%
\pgfsetfillcolor{currentfill}%
\pgfsetlinewidth{1.003750pt}%
\definecolor{currentstroke}{rgb}{1.000000,0.000000,0.000000}%
\pgfsetstrokecolor{currentstroke}%
\pgfsetdash{}{0pt}%
\pgfpathmoveto{\pgfqpoint{1.081382in}{1.107854in}}%
\pgfpathmoveto{\pgfqpoint{1.083050in}{1.111191in}}%
\pgfpathlineto{\pgfqpoint{1.103342in}{1.151775in}}%
\pgfpathlineto{\pgfqpoint{1.059422in}{1.151775in}}%
\pgfpathlineto{\pgfqpoint{1.081382in}{1.107854in}}%
\pgfusepath{stroke,fill}%
\end{pgfscope}%
\begin{pgfscope}%
\pgfpathrectangle{\pgfqpoint{0.506010in}{1.121191in}}{\pgfqpoint{2.325000in}{1.400000in}} %
\pgfusepath{clip}%
\pgfsetbuttcap%
\pgfsetroundjoin%
\definecolor{currentfill}{rgb}{1.000000,0.000000,0.000000}%
\pgfsetfillcolor{currentfill}%
\pgfsetlinewidth{1.003750pt}%
\definecolor{currentstroke}{rgb}{1.000000,0.000000,0.000000}%
\pgfsetstrokecolor{currentstroke}%
\pgfsetdash{}{0pt}%
\pgfpathmoveto{\pgfqpoint{1.081382in}{1.108050in}}%
\pgfpathmoveto{\pgfqpoint{1.082952in}{1.111191in}}%
\pgfpathlineto{\pgfqpoint{1.103342in}{1.151971in}}%
\pgfpathlineto{\pgfqpoint{1.059422in}{1.151971in}}%
\pgfpathlineto{\pgfqpoint{1.081382in}{1.108050in}}%
\pgfusepath{stroke,fill}%
\end{pgfscope}%
\begin{pgfscope}%
\pgfpathrectangle{\pgfqpoint{0.506010in}{1.121191in}}{\pgfqpoint{2.325000in}{1.400000in}} %
\pgfusepath{clip}%
\pgfsetbuttcap%
\pgfsetroundjoin%
\definecolor{currentfill}{rgb}{1.000000,0.000000,0.000000}%
\pgfsetfillcolor{currentfill}%
\pgfsetlinewidth{1.003750pt}%
\definecolor{currentstroke}{rgb}{1.000000,0.000000,0.000000}%
\pgfsetstrokecolor{currentstroke}%
\pgfsetdash{}{0pt}%
\pgfpathmoveto{\pgfqpoint{1.007822in}{1.108778in}}%
\pgfpathmoveto{\pgfqpoint{1.009028in}{1.111191in}}%
\pgfpathlineto{\pgfqpoint{1.029783in}{1.152699in}}%
\pgfpathlineto{\pgfqpoint{0.985862in}{1.152699in}}%
\pgfpathlineto{\pgfqpoint{1.007822in}{1.108778in}}%
\pgfusepath{stroke,fill}%
\end{pgfscope}%
\begin{pgfscope}%
\pgfpathrectangle{\pgfqpoint{0.506010in}{1.121191in}}{\pgfqpoint{2.325000in}{1.400000in}} %
\pgfusepath{clip}%
\pgfsetbuttcap%
\pgfsetroundjoin%
\definecolor{currentfill}{rgb}{1.000000,0.000000,0.000000}%
\pgfsetfillcolor{currentfill}%
\pgfsetlinewidth{1.003750pt}%
\definecolor{currentstroke}{rgb}{1.000000,0.000000,0.000000}%
\pgfsetstrokecolor{currentstroke}%
\pgfsetdash{}{0pt}%
\pgfpathmoveto{\pgfqpoint{1.007822in}{1.108778in}}%
\pgfpathmoveto{\pgfqpoint{1.009028in}{1.111191in}}%
\pgfpathlineto{\pgfqpoint{1.029783in}{1.152699in}}%
\pgfpathlineto{\pgfqpoint{0.985862in}{1.152699in}}%
\pgfpathlineto{\pgfqpoint{1.007822in}{1.108778in}}%
\pgfusepath{stroke,fill}%
\end{pgfscope}%
\begin{pgfscope}%
\pgfpathrectangle{\pgfqpoint{0.506010in}{1.121191in}}{\pgfqpoint{2.325000in}{1.400000in}} %
\pgfusepath{clip}%
\pgfsetbuttcap%
\pgfsetroundjoin%
\definecolor{currentfill}{rgb}{1.000000,0.000000,0.000000}%
\pgfsetfillcolor{currentfill}%
\pgfsetlinewidth{1.003750pt}%
\definecolor{currentstroke}{rgb}{1.000000,0.000000,0.000000}%
\pgfsetstrokecolor{currentstroke}%
\pgfsetdash{}{0pt}%
\pgfpathmoveto{\pgfqpoint{1.108974in}{1.108890in}}%
\pgfpathmoveto{\pgfqpoint{1.110124in}{1.111191in}}%
\pgfpathlineto{\pgfqpoint{1.130934in}{1.152811in}}%
\pgfpathlineto{\pgfqpoint{1.087014in}{1.152811in}}%
\pgfpathlineto{\pgfqpoint{1.108974in}{1.108890in}}%
\pgfusepath{stroke,fill}%
\end{pgfscope}%
\begin{pgfscope}%
\pgfpathrectangle{\pgfqpoint{0.506010in}{1.121191in}}{\pgfqpoint{2.325000in}{1.400000in}} %
\pgfusepath{clip}%
\pgfsetbuttcap%
\pgfsetroundjoin%
\definecolor{currentfill}{rgb}{1.000000,0.000000,0.000000}%
\pgfsetfillcolor{currentfill}%
\pgfsetlinewidth{1.003750pt}%
\definecolor{currentstroke}{rgb}{1.000000,0.000000,0.000000}%
\pgfsetstrokecolor{currentstroke}%
\pgfsetdash{}{0pt}%
\pgfpathmoveto{\pgfqpoint{1.474029in}{1.109058in}}%
\pgfpathmoveto{\pgfqpoint{1.475095in}{1.111191in}}%
\pgfpathlineto{\pgfqpoint{1.495990in}{1.152979in}}%
\pgfpathlineto{\pgfqpoint{1.452069in}{1.152979in}}%
\pgfpathlineto{\pgfqpoint{1.474029in}{1.109058in}}%
\pgfusepath{stroke,fill}%
\end{pgfscope}%
\begin{pgfscope}%
\pgfpathrectangle{\pgfqpoint{0.506010in}{1.121191in}}{\pgfqpoint{2.325000in}{1.400000in}} %
\pgfusepath{clip}%
\pgfsetbuttcap%
\pgfsetroundjoin%
\definecolor{currentfill}{rgb}{1.000000,0.000000,0.000000}%
\pgfsetfillcolor{currentfill}%
\pgfsetlinewidth{1.003750pt}%
\definecolor{currentstroke}{rgb}{1.000000,0.000000,0.000000}%
\pgfsetstrokecolor{currentstroke}%
\pgfsetdash{}{0pt}%
\pgfpathmoveto{\pgfqpoint{1.081382in}{1.109198in}}%
\pgfpathmoveto{\pgfqpoint{1.082378in}{1.111191in}}%
\pgfpathlineto{\pgfqpoint{1.103342in}{1.153119in}}%
\pgfpathlineto{\pgfqpoint{1.059422in}{1.153119in}}%
\pgfpathlineto{\pgfqpoint{1.081382in}{1.109198in}}%
\pgfusepath{stroke,fill}%
\end{pgfscope}%
\begin{pgfscope}%
\pgfpathrectangle{\pgfqpoint{0.506010in}{1.121191in}}{\pgfqpoint{2.325000in}{1.400000in}} %
\pgfusepath{clip}%
\pgfsetbuttcap%
\pgfsetroundjoin%
\definecolor{currentfill}{rgb}{1.000000,0.000000,0.000000}%
\pgfsetfillcolor{currentfill}%
\pgfsetlinewidth{1.003750pt}%
\definecolor{currentstroke}{rgb}{1.000000,0.000000,0.000000}%
\pgfsetstrokecolor{currentstroke}%
\pgfsetdash{}{0pt}%
\pgfpathmoveto{\pgfqpoint{1.081382in}{1.109282in}}%
\pgfpathmoveto{\pgfqpoint{1.082336in}{1.111191in}}%
\pgfpathlineto{\pgfqpoint{1.103342in}{1.153203in}}%
\pgfpathlineto{\pgfqpoint{1.059422in}{1.153203in}}%
\pgfpathlineto{\pgfqpoint{1.081382in}{1.109282in}}%
\pgfusepath{stroke,fill}%
\end{pgfscope}%
\begin{pgfscope}%
\pgfpathrectangle{\pgfqpoint{0.506010in}{1.121191in}}{\pgfqpoint{2.325000in}{1.400000in}} %
\pgfusepath{clip}%
\pgfsetbuttcap%
\pgfsetroundjoin%
\definecolor{currentfill}{rgb}{1.000000,0.000000,0.000000}%
\pgfsetfillcolor{currentfill}%
\pgfsetlinewidth{1.003750pt}%
\definecolor{currentstroke}{rgb}{1.000000,0.000000,0.000000}%
\pgfsetstrokecolor{currentstroke}%
\pgfsetdash{}{0pt}%
\pgfpathmoveto{\pgfqpoint{1.007822in}{1.109450in}}%
\pgfpathmoveto{\pgfqpoint{1.008692in}{1.111191in}}%
\pgfpathlineto{\pgfqpoint{1.029783in}{1.153371in}}%
\pgfpathlineto{\pgfqpoint{0.985862in}{1.153371in}}%
\pgfpathlineto{\pgfqpoint{1.007822in}{1.109450in}}%
\pgfusepath{stroke,fill}%
\end{pgfscope}%
\begin{pgfscope}%
\pgfpathrectangle{\pgfqpoint{0.506010in}{1.121191in}}{\pgfqpoint{2.325000in}{1.400000in}} %
\pgfusepath{clip}%
\pgfsetbuttcap%
\pgfsetroundjoin%
\definecolor{currentfill}{rgb}{1.000000,0.000000,0.000000}%
\pgfsetfillcolor{currentfill}%
\pgfsetlinewidth{1.003750pt}%
\definecolor{currentstroke}{rgb}{1.000000,0.000000,0.000000}%
\pgfsetstrokecolor{currentstroke}%
\pgfsetdash{}{0pt}%
\pgfpathmoveto{\pgfqpoint{1.108974in}{1.109590in}}%
\pgfpathmoveto{\pgfqpoint{1.109774in}{1.111191in}}%
\pgfpathlineto{\pgfqpoint{1.130934in}{1.153511in}}%
\pgfpathlineto{\pgfqpoint{1.087014in}{1.153511in}}%
\pgfpathlineto{\pgfqpoint{1.108974in}{1.109590in}}%
\pgfusepath{stroke,fill}%
\end{pgfscope}%
\begin{pgfscope}%
\pgfpathrectangle{\pgfqpoint{0.506010in}{1.121191in}}{\pgfqpoint{2.325000in}{1.400000in}} %
\pgfusepath{clip}%
\pgfsetbuttcap%
\pgfsetroundjoin%
\definecolor{currentfill}{rgb}{1.000000,0.000000,0.000000}%
\pgfsetfillcolor{currentfill}%
\pgfsetlinewidth{1.003750pt}%
\definecolor{currentstroke}{rgb}{1.000000,0.000000,0.000000}%
\pgfsetstrokecolor{currentstroke}%
\pgfsetdash{}{0pt}%
\pgfpathmoveto{\pgfqpoint{1.081382in}{1.109814in}}%
\pgfpathmoveto{\pgfqpoint{1.082070in}{1.111191in}}%
\pgfpathlineto{\pgfqpoint{1.103342in}{1.153735in}}%
\pgfpathlineto{\pgfqpoint{1.059422in}{1.153735in}}%
\pgfpathlineto{\pgfqpoint{1.081382in}{1.109814in}}%
\pgfusepath{stroke,fill}%
\end{pgfscope}%
\begin{pgfscope}%
\pgfpathrectangle{\pgfqpoint{0.506010in}{1.121191in}}{\pgfqpoint{2.325000in}{1.400000in}} %
\pgfusepath{clip}%
\pgfsetbuttcap%
\pgfsetroundjoin%
\definecolor{currentfill}{rgb}{1.000000,0.000000,0.000000}%
\pgfsetfillcolor{currentfill}%
\pgfsetlinewidth{1.003750pt}%
\definecolor{currentstroke}{rgb}{1.000000,0.000000,0.000000}%
\pgfsetstrokecolor{currentstroke}%
\pgfsetdash{}{0pt}%
\pgfpathmoveto{\pgfqpoint{1.007822in}{1.110010in}}%
\pgfpathmoveto{\pgfqpoint{1.008412in}{1.111191in}}%
\pgfpathlineto{\pgfqpoint{1.029783in}{1.153931in}}%
\pgfpathlineto{\pgfqpoint{0.985862in}{1.153931in}}%
\pgfpathlineto{\pgfqpoint{1.007822in}{1.110010in}}%
\pgfusepath{stroke,fill}%
\end{pgfscope}%
\begin{pgfscope}%
\pgfpathrectangle{\pgfqpoint{0.506010in}{1.121191in}}{\pgfqpoint{2.325000in}{1.400000in}} %
\pgfusepath{clip}%
\pgfsetbuttcap%
\pgfsetroundjoin%
\definecolor{currentfill}{rgb}{1.000000,0.000000,0.000000}%
\pgfsetfillcolor{currentfill}%
\pgfsetlinewidth{1.003750pt}%
\definecolor{currentstroke}{rgb}{1.000000,0.000000,0.000000}%
\pgfsetstrokecolor{currentstroke}%
\pgfsetdash{}{0pt}%
\pgfpathmoveto{\pgfqpoint{1.081382in}{1.110318in}}%
\pgfpathmoveto{\pgfqpoint{1.081818in}{1.111191in}}%
\pgfpathlineto{\pgfqpoint{1.103342in}{1.154239in}}%
\pgfpathlineto{\pgfqpoint{1.059422in}{1.154239in}}%
\pgfpathlineto{\pgfqpoint{1.081382in}{1.110318in}}%
\pgfusepath{stroke,fill}%
\end{pgfscope}%
\begin{pgfscope}%
\pgfpathrectangle{\pgfqpoint{0.506010in}{1.121191in}}{\pgfqpoint{2.325000in}{1.400000in}} %
\pgfusepath{clip}%
\pgfsetbuttcap%
\pgfsetroundjoin%
\definecolor{currentfill}{rgb}{1.000000,0.000000,0.000000}%
\pgfsetfillcolor{currentfill}%
\pgfsetlinewidth{1.003750pt}%
\definecolor{currentstroke}{rgb}{1.000000,0.000000,0.000000}%
\pgfsetstrokecolor{currentstroke}%
\pgfsetdash{}{0pt}%
\pgfpathmoveto{\pgfqpoint{1.081382in}{1.110346in}}%
\pgfpathmoveto{\pgfqpoint{1.081804in}{1.111191in}}%
\pgfpathlineto{\pgfqpoint{1.103342in}{1.154267in}}%
\pgfpathlineto{\pgfqpoint{1.059422in}{1.154267in}}%
\pgfpathlineto{\pgfqpoint{1.081382in}{1.110346in}}%
\pgfusepath{stroke,fill}%
\end{pgfscope}%
\begin{pgfscope}%
\pgfpathrectangle{\pgfqpoint{0.506010in}{1.121191in}}{\pgfqpoint{2.325000in}{1.400000in}} %
\pgfusepath{clip}%
\pgfsetbuttcap%
\pgfsetroundjoin%
\definecolor{currentfill}{rgb}{1.000000,0.000000,0.000000}%
\pgfsetfillcolor{currentfill}%
\pgfsetlinewidth{1.003750pt}%
\definecolor{currentstroke}{rgb}{1.000000,0.000000,0.000000}%
\pgfsetstrokecolor{currentstroke}%
\pgfsetdash{}{0pt}%
\pgfpathmoveto{\pgfqpoint{1.081382in}{1.110402in}}%
\pgfpathmoveto{\pgfqpoint{1.081776in}{1.111191in}}%
\pgfpathlineto{\pgfqpoint{1.103342in}{1.154323in}}%
\pgfpathlineto{\pgfqpoint{1.059422in}{1.154323in}}%
\pgfpathlineto{\pgfqpoint{1.081382in}{1.110402in}}%
\pgfusepath{stroke,fill}%
\end{pgfscope}%
\begin{pgfscope}%
\pgfpathrectangle{\pgfqpoint{0.506010in}{1.121191in}}{\pgfqpoint{2.325000in}{1.400000in}} %
\pgfusepath{clip}%
\pgfsetbuttcap%
\pgfsetroundjoin%
\definecolor{currentfill}{rgb}{1.000000,0.000000,0.000000}%
\pgfsetfillcolor{currentfill}%
\pgfsetlinewidth{1.003750pt}%
\definecolor{currentstroke}{rgb}{1.000000,0.000000,0.000000}%
\pgfsetstrokecolor{currentstroke}%
\pgfsetdash{}{0pt}%
\pgfpathmoveto{\pgfqpoint{1.081382in}{1.110458in}}%
\pgfpathmoveto{\pgfqpoint{1.081748in}{1.111191in}}%
\pgfpathlineto{\pgfqpoint{1.103342in}{1.154379in}}%
\pgfpathlineto{\pgfqpoint{1.059422in}{1.154379in}}%
\pgfpathlineto{\pgfqpoint{1.081382in}{1.110458in}}%
\pgfusepath{stroke,fill}%
\end{pgfscope}%
\begin{pgfscope}%
\pgfpathrectangle{\pgfqpoint{0.506010in}{1.121191in}}{\pgfqpoint{2.325000in}{1.400000in}} %
\pgfusepath{clip}%
\pgfsetbuttcap%
\pgfsetroundjoin%
\definecolor{currentfill}{rgb}{1.000000,0.000000,0.000000}%
\pgfsetfillcolor{currentfill}%
\pgfsetlinewidth{1.003750pt}%
\definecolor{currentstroke}{rgb}{1.000000,0.000000,0.000000}%
\pgfsetstrokecolor{currentstroke}%
\pgfsetdash{}{0pt}%
\pgfpathmoveto{\pgfqpoint{1.007822in}{1.110906in}}%
\pgfpathmoveto{\pgfqpoint{1.007964in}{1.111191in}}%
\pgfpathlineto{\pgfqpoint{1.029783in}{1.154827in}}%
\pgfpathlineto{\pgfqpoint{0.985862in}{1.154827in}}%
\pgfpathlineto{\pgfqpoint{1.007822in}{1.110906in}}%
\pgfusepath{stroke,fill}%
\end{pgfscope}%
\begin{pgfscope}%
\pgfpathrectangle{\pgfqpoint{0.506010in}{1.121191in}}{\pgfqpoint{2.325000in}{1.400000in}} %
\pgfusepath{clip}%
\pgfsetbuttcap%
\pgfsetroundjoin%
\definecolor{currentfill}{rgb}{1.000000,0.000000,0.000000}%
\pgfsetfillcolor{currentfill}%
\pgfsetlinewidth{1.003750pt}%
\definecolor{currentstroke}{rgb}{1.000000,0.000000,0.000000}%
\pgfsetstrokecolor{currentstroke}%
\pgfsetdash{}{0pt}%
\pgfpathmoveto{\pgfqpoint{0.788675in}{1.112278in}}%
\pgfpathlineto{\pgfqpoint{0.810635in}{1.156199in}}%
\pgfpathlineto{\pgfqpoint{0.766715in}{1.156199in}}%
\pgfpathclose%
\pgfusepath{stroke,fill}%
\end{pgfscope}%
\begin{pgfscope}%
\pgfpathrectangle{\pgfqpoint{0.506010in}{1.121191in}}{\pgfqpoint{2.325000in}{1.400000in}} %
\pgfusepath{clip}%
\pgfsetbuttcap%
\pgfsetroundjoin%
\definecolor{currentfill}{rgb}{1.000000,0.000000,0.000000}%
\pgfsetfillcolor{currentfill}%
\pgfsetlinewidth{1.003750pt}%
\definecolor{currentstroke}{rgb}{1.000000,0.000000,0.000000}%
\pgfsetstrokecolor{currentstroke}%
\pgfsetdash{}{0pt}%
\pgfpathmoveto{\pgfqpoint{2.480299in}{6.139230in}}%
\pgfpathclose%
\pgfusepath{stroke,fill}%
\end{pgfscope}%
\begin{pgfscope}%
\pgfpathrectangle{\pgfqpoint{0.506010in}{1.121191in}}{\pgfqpoint{2.325000in}{1.400000in}} %
\pgfusepath{clip}%
\pgfsetbuttcap%
\pgfsetroundjoin%
\definecolor{currentfill}{rgb}{1.000000,0.000000,0.000000}%
\pgfsetfillcolor{currentfill}%
\pgfsetlinewidth{1.003750pt}%
\definecolor{currentstroke}{rgb}{1.000000,0.000000,0.000000}%
\pgfsetstrokecolor{currentstroke}%
\pgfsetdash{}{0pt}%
\pgfpathmoveto{\pgfqpoint{1.139222in}{6.139230in}}%
\pgfpathclose%
\pgfusepath{stroke,fill}%
\end{pgfscope}%
\begin{pgfscope}%
\pgfpathrectangle{\pgfqpoint{0.506010in}{1.121191in}}{\pgfqpoint{2.325000in}{1.400000in}} %
\pgfusepath{clip}%
\pgfsetbuttcap%
\pgfsetroundjoin%
\definecolor{currentfill}{rgb}{1.000000,0.000000,0.000000}%
\pgfsetfillcolor{currentfill}%
\pgfsetlinewidth{1.003750pt}%
\definecolor{currentstroke}{rgb}{1.000000,0.000000,0.000000}%
\pgfsetstrokecolor{currentstroke}%
\pgfsetdash{}{0pt}%
\pgfpathmoveto{\pgfqpoint{1.007822in}{6.139230in}}%
\pgfpathclose%
\pgfusepath{stroke,fill}%
\end{pgfscope}%
\begin{pgfscope}%
\pgfpathrectangle{\pgfqpoint{0.506010in}{1.121191in}}{\pgfqpoint{2.325000in}{1.400000in}} %
\pgfusepath{clip}%
\pgfsetbuttcap%
\pgfsetroundjoin%
\definecolor{currentfill}{rgb}{1.000000,0.000000,0.000000}%
\pgfsetfillcolor{currentfill}%
\pgfsetlinewidth{1.003750pt}%
\definecolor{currentstroke}{rgb}{1.000000,0.000000,0.000000}%
\pgfsetstrokecolor{currentstroke}%
\pgfsetdash{}{0pt}%
\pgfpathmoveto{\pgfqpoint{1.007822in}{6.139230in}}%
\pgfpathclose%
\pgfusepath{stroke,fill}%
\end{pgfscope}%
\begin{pgfscope}%
\pgfpathrectangle{\pgfqpoint{0.506010in}{1.121191in}}{\pgfqpoint{2.325000in}{1.400000in}} %
\pgfusepath{clip}%
\pgfsetbuttcap%
\pgfsetroundjoin%
\definecolor{currentfill}{rgb}{1.000000,0.000000,0.000000}%
\pgfsetfillcolor{currentfill}%
\pgfsetlinewidth{1.003750pt}%
\definecolor{currentstroke}{rgb}{1.000000,0.000000,0.000000}%
\pgfsetstrokecolor{currentstroke}%
\pgfsetdash{}{0pt}%
\pgfpathmoveto{\pgfqpoint{1.007822in}{6.139230in}}%
\pgfpathclose%
\pgfusepath{stroke,fill}%
\end{pgfscope}%
\begin{pgfscope}%
\pgfpathrectangle{\pgfqpoint{0.506010in}{1.121191in}}{\pgfqpoint{2.325000in}{1.400000in}} %
\pgfusepath{clip}%
\pgfsetbuttcap%
\pgfsetroundjoin%
\definecolor{currentfill}{rgb}{1.000000,0.000000,0.000000}%
\pgfsetfillcolor{currentfill}%
\pgfsetlinewidth{1.003750pt}%
\definecolor{currentstroke}{rgb}{1.000000,0.000000,0.000000}%
\pgfsetstrokecolor{currentstroke}%
\pgfsetdash{}{0pt}%
\pgfpathmoveto{\pgfqpoint{1.108974in}{6.139230in}}%
\pgfpathclose%
\pgfusepath{stroke,fill}%
\end{pgfscope}%
\begin{pgfscope}%
\pgfpathrectangle{\pgfqpoint{0.506010in}{1.121191in}}{\pgfqpoint{2.325000in}{1.400000in}} %
\pgfusepath{clip}%
\pgfsetbuttcap%
\pgfsetroundjoin%
\definecolor{currentfill}{rgb}{1.000000,0.000000,0.000000}%
\pgfsetfillcolor{currentfill}%
\pgfsetlinewidth{1.003750pt}%
\definecolor{currentstroke}{rgb}{1.000000,0.000000,0.000000}%
\pgfsetstrokecolor{currentstroke}%
\pgfsetdash{}{0pt}%
\pgfpathmoveto{\pgfqpoint{1.108974in}{6.139230in}}%
\pgfpathclose%
\pgfusepath{stroke,fill}%
\end{pgfscope}%
\begin{pgfscope}%
\pgfpathrectangle{\pgfqpoint{0.506010in}{1.121191in}}{\pgfqpoint{2.325000in}{1.400000in}} %
\pgfusepath{clip}%
\pgfsetbuttcap%
\pgfsetroundjoin%
\definecolor{currentfill}{rgb}{1.000000,0.000000,0.000000}%
\pgfsetfillcolor{currentfill}%
\pgfsetlinewidth{1.003750pt}%
\definecolor{currentstroke}{rgb}{1.000000,0.000000,0.000000}%
\pgfsetstrokecolor{currentstroke}%
\pgfsetdash{}{0pt}%
\pgfpathmoveto{\pgfqpoint{1.081382in}{6.139230in}}%
\pgfpathclose%
\pgfusepath{stroke,fill}%
\end{pgfscope}%
\begin{pgfscope}%
\pgfpathrectangle{\pgfqpoint{0.506010in}{1.121191in}}{\pgfqpoint{2.325000in}{1.400000in}} %
\pgfusepath{clip}%
\pgfsetbuttcap%
\pgfsetroundjoin%
\definecolor{currentfill}{rgb}{1.000000,0.000000,0.000000}%
\pgfsetfillcolor{currentfill}%
\pgfsetlinewidth{1.003750pt}%
\definecolor{currentstroke}{rgb}{1.000000,0.000000,0.000000}%
\pgfsetstrokecolor{currentstroke}%
\pgfsetdash{}{0pt}%
\pgfpathmoveto{\pgfqpoint{0.820559in}{6.139230in}}%
\pgfpathclose%
\pgfusepath{stroke,fill}%
\end{pgfscope}%
\begin{pgfscope}%
\pgfpathrectangle{\pgfqpoint{0.506010in}{1.121191in}}{\pgfqpoint{2.325000in}{1.400000in}} %
\pgfusepath{clip}%
\pgfsetbuttcap%
\pgfsetroundjoin%
\definecolor{currentfill}{rgb}{1.000000,0.000000,0.000000}%
\pgfsetfillcolor{currentfill}%
\pgfsetlinewidth{1.003750pt}%
\definecolor{currentstroke}{rgb}{1.000000,0.000000,0.000000}%
\pgfsetstrokecolor{currentstroke}%
\pgfsetdash{}{0pt}%
\pgfpathmoveto{\pgfqpoint{1.167543in}{6.139230in}}%
\pgfpathclose%
\pgfusepath{stroke,fill}%
\end{pgfscope}%
\begin{pgfscope}%
\pgfpathrectangle{\pgfqpoint{0.506010in}{1.121191in}}{\pgfqpoint{2.325000in}{1.400000in}} %
\pgfusepath{clip}%
\pgfsetbuttcap%
\pgfsetroundjoin%
\definecolor{currentfill}{rgb}{1.000000,0.000000,0.000000}%
\pgfsetfillcolor{currentfill}%
\pgfsetlinewidth{1.003750pt}%
\definecolor{currentstroke}{rgb}{1.000000,0.000000,0.000000}%
\pgfsetstrokecolor{currentstroke}%
\pgfsetdash{}{0pt}%
\pgfpathmoveto{\pgfqpoint{1.139222in}{6.139230in}}%
\pgfpathclose%
\pgfusepath{stroke,fill}%
\end{pgfscope}%
\begin{pgfscope}%
\pgfpathrectangle{\pgfqpoint{0.506010in}{1.121191in}}{\pgfqpoint{2.325000in}{1.400000in}} %
\pgfusepath{clip}%
\pgfsetbuttcap%
\pgfsetroundjoin%
\definecolor{currentfill}{rgb}{1.000000,0.000000,0.000000}%
\pgfsetfillcolor{currentfill}%
\pgfsetlinewidth{1.003750pt}%
\definecolor{currentstroke}{rgb}{1.000000,0.000000,0.000000}%
\pgfsetstrokecolor{currentstroke}%
\pgfsetdash{}{0pt}%
\pgfpathmoveto{\pgfqpoint{2.191066in}{6.139230in}}%
\pgfpathclose%
\pgfusepath{stroke,fill}%
\end{pgfscope}%
\begin{pgfscope}%
\pgfpathrectangle{\pgfqpoint{0.506010in}{1.121191in}}{\pgfqpoint{2.325000in}{1.400000in}} %
\pgfusepath{clip}%
\pgfsetbuttcap%
\pgfsetroundjoin%
\definecolor{currentfill}{rgb}{1.000000,0.000000,0.000000}%
\pgfsetfillcolor{currentfill}%
\pgfsetlinewidth{1.003750pt}%
\definecolor{currentstroke}{rgb}{1.000000,0.000000,0.000000}%
\pgfsetstrokecolor{currentstroke}%
\pgfsetdash{}{0pt}%
\pgfpathmoveto{\pgfqpoint{2.265364in}{6.139230in}}%
\pgfpathclose%
\pgfusepath{stroke,fill}%
\end{pgfscope}%
\begin{pgfscope}%
\pgfpathrectangle{\pgfqpoint{0.506010in}{1.121191in}}{\pgfqpoint{2.325000in}{1.400000in}} %
\pgfusepath{clip}%
\pgfsetbuttcap%
\pgfsetroundjoin%
\definecolor{currentfill}{rgb}{1.000000,0.000000,0.000000}%
\pgfsetfillcolor{currentfill}%
\pgfsetlinewidth{1.003750pt}%
\definecolor{currentstroke}{rgb}{1.000000,0.000000,0.000000}%
\pgfsetstrokecolor{currentstroke}%
\pgfsetdash{}{0pt}%
\pgfpathmoveto{\pgfqpoint{2.185713in}{6.139230in}}%
\pgfpathclose%
\pgfusepath{stroke,fill}%
\end{pgfscope}%
\begin{pgfscope}%
\pgfpathrectangle{\pgfqpoint{0.506010in}{1.121191in}}{\pgfqpoint{2.325000in}{1.400000in}} %
\pgfusepath{clip}%
\pgfsetbuttcap%
\pgfsetroundjoin%
\definecolor{currentfill}{rgb}{1.000000,0.000000,0.000000}%
\pgfsetfillcolor{currentfill}%
\pgfsetlinewidth{1.003750pt}%
\definecolor{currentstroke}{rgb}{1.000000,0.000000,0.000000}%
\pgfsetstrokecolor{currentstroke}%
\pgfsetdash{}{0pt}%
\pgfpathmoveto{\pgfqpoint{2.217160in}{6.139230in}}%
\pgfpathclose%
\pgfusepath{stroke,fill}%
\end{pgfscope}%
\begin{pgfscope}%
\pgfpathrectangle{\pgfqpoint{0.506010in}{1.121191in}}{\pgfqpoint{2.325000in}{1.400000in}} %
\pgfusepath{clip}%
\pgfsetbuttcap%
\pgfsetroundjoin%
\definecolor{currentfill}{rgb}{1.000000,0.000000,0.000000}%
\pgfsetfillcolor{currentfill}%
\pgfsetlinewidth{1.003750pt}%
\definecolor{currentstroke}{rgb}{1.000000,0.000000,0.000000}%
\pgfsetstrokecolor{currentstroke}%
\pgfsetdash{}{0pt}%
\pgfpathmoveto{\pgfqpoint{2.262181in}{6.139230in}}%
\pgfpathclose%
\pgfusepath{stroke,fill}%
\end{pgfscope}%
\begin{pgfscope}%
\pgfpathrectangle{\pgfqpoint{0.506010in}{1.121191in}}{\pgfqpoint{2.325000in}{1.400000in}} %
\pgfusepath{clip}%
\pgfsetbuttcap%
\pgfsetroundjoin%
\definecolor{currentfill}{rgb}{1.000000,0.000000,0.000000}%
\pgfsetfillcolor{currentfill}%
\pgfsetlinewidth{1.003750pt}%
\definecolor{currentstroke}{rgb}{1.000000,0.000000,0.000000}%
\pgfsetstrokecolor{currentstroke}%
\pgfsetdash{}{0pt}%
\pgfpathmoveto{\pgfqpoint{2.216102in}{6.139230in}}%
\pgfpathclose%
\pgfusepath{stroke,fill}%
\end{pgfscope}%
\begin{pgfscope}%
\pgfpathrectangle{\pgfqpoint{0.506010in}{1.121191in}}{\pgfqpoint{2.325000in}{1.400000in}} %
\pgfusepath{clip}%
\pgfsetbuttcap%
\pgfsetroundjoin%
\definecolor{currentfill}{rgb}{1.000000,0.000000,0.000000}%
\pgfsetfillcolor{currentfill}%
\pgfsetlinewidth{1.003750pt}%
\definecolor{currentstroke}{rgb}{1.000000,0.000000,0.000000}%
\pgfsetstrokecolor{currentstroke}%
\pgfsetdash{}{0pt}%
\pgfpathmoveto{\pgfqpoint{2.234732in}{6.139230in}}%
\pgfpathclose%
\pgfusepath{stroke,fill}%
\end{pgfscope}%
\begin{pgfscope}%
\pgfpathrectangle{\pgfqpoint{0.506010in}{1.121191in}}{\pgfqpoint{2.325000in}{1.400000in}} %
\pgfusepath{clip}%
\pgfsetbuttcap%
\pgfsetroundjoin%
\definecolor{currentfill}{rgb}{1.000000,0.000000,0.000000}%
\pgfsetfillcolor{currentfill}%
\pgfsetlinewidth{1.003750pt}%
\definecolor{currentstroke}{rgb}{1.000000,0.000000,0.000000}%
\pgfsetstrokecolor{currentstroke}%
\pgfsetdash{}{0pt}%
\pgfpathmoveto{\pgfqpoint{2.201136in}{6.139230in}}%
\pgfpathclose%
\pgfusepath{stroke,fill}%
\end{pgfscope}%
\begin{pgfscope}%
\pgfpathrectangle{\pgfqpoint{0.506010in}{1.121191in}}{\pgfqpoint{2.325000in}{1.400000in}} %
\pgfusepath{clip}%
\pgfsetbuttcap%
\pgfsetroundjoin%
\definecolor{currentfill}{rgb}{1.000000,0.000000,0.000000}%
\pgfsetfillcolor{currentfill}%
\pgfsetlinewidth{1.003750pt}%
\definecolor{currentstroke}{rgb}{1.000000,0.000000,0.000000}%
\pgfsetstrokecolor{currentstroke}%
\pgfsetdash{}{0pt}%
\pgfpathmoveto{\pgfqpoint{2.233373in}{6.139230in}}%
\pgfpathclose%
\pgfusepath{stroke,fill}%
\end{pgfscope}%
\begin{pgfscope}%
\pgfpathrectangle{\pgfqpoint{0.506010in}{1.121191in}}{\pgfqpoint{2.325000in}{1.400000in}} %
\pgfusepath{clip}%
\pgfsetbuttcap%
\pgfsetroundjoin%
\definecolor{currentfill}{rgb}{1.000000,0.000000,0.000000}%
\pgfsetfillcolor{currentfill}%
\pgfsetlinewidth{1.003750pt}%
\definecolor{currentstroke}{rgb}{1.000000,0.000000,0.000000}%
\pgfsetstrokecolor{currentstroke}%
\pgfsetdash{}{0pt}%
\pgfpathmoveto{\pgfqpoint{2.222779in}{6.139230in}}%
\pgfpathclose%
\pgfusepath{stroke,fill}%
\end{pgfscope}%
\begin{pgfscope}%
\pgfpathrectangle{\pgfqpoint{0.506010in}{1.121191in}}{\pgfqpoint{2.325000in}{1.400000in}} %
\pgfusepath{clip}%
\pgfsetbuttcap%
\pgfsetroundjoin%
\definecolor{currentfill}{rgb}{1.000000,0.000000,0.000000}%
\pgfsetfillcolor{currentfill}%
\pgfsetlinewidth{1.003750pt}%
\definecolor{currentstroke}{rgb}{1.000000,0.000000,0.000000}%
\pgfsetstrokecolor{currentstroke}%
\pgfsetdash{}{0pt}%
\pgfpathmoveto{\pgfqpoint{2.248922in}{6.139230in}}%
\pgfpathclose%
\pgfusepath{stroke,fill}%
\end{pgfscope}%
\begin{pgfscope}%
\pgfpathrectangle{\pgfqpoint{0.506010in}{1.121191in}}{\pgfqpoint{2.325000in}{1.400000in}} %
\pgfusepath{clip}%
\pgfsetbuttcap%
\pgfsetroundjoin%
\definecolor{currentfill}{rgb}{1.000000,0.000000,0.000000}%
\pgfsetfillcolor{currentfill}%
\pgfsetlinewidth{1.003750pt}%
\definecolor{currentstroke}{rgb}{1.000000,0.000000,0.000000}%
\pgfsetstrokecolor{currentstroke}%
\pgfsetdash{}{0pt}%
\pgfpathmoveto{\pgfqpoint{2.246565in}{6.139230in}}%
\pgfpathclose%
\pgfusepath{stroke,fill}%
\end{pgfscope}%
\begin{pgfscope}%
\pgfpathrectangle{\pgfqpoint{0.506010in}{1.121191in}}{\pgfqpoint{2.325000in}{1.400000in}} %
\pgfusepath{clip}%
\pgfsetbuttcap%
\pgfsetroundjoin%
\definecolor{currentfill}{rgb}{1.000000,0.000000,0.000000}%
\pgfsetfillcolor{currentfill}%
\pgfsetlinewidth{1.003750pt}%
\definecolor{currentstroke}{rgb}{1.000000,0.000000,0.000000}%
\pgfsetstrokecolor{currentstroke}%
\pgfsetdash{}{0pt}%
\pgfpathmoveto{\pgfqpoint{2.233956in}{6.139230in}}%
\pgfpathclose%
\pgfusepath{stroke,fill}%
\end{pgfscope}%
\begin{pgfscope}%
\pgfpathrectangle{\pgfqpoint{0.506010in}{1.121191in}}{\pgfqpoint{2.325000in}{1.400000in}} %
\pgfusepath{clip}%
\pgfsetbuttcap%
\pgfsetroundjoin%
\definecolor{currentfill}{rgb}{1.000000,0.000000,0.000000}%
\pgfsetfillcolor{currentfill}%
\pgfsetlinewidth{1.003750pt}%
\definecolor{currentstroke}{rgb}{1.000000,0.000000,0.000000}%
\pgfsetstrokecolor{currentstroke}%
\pgfsetdash{}{0pt}%
\pgfpathmoveto{\pgfqpoint{2.251609in}{6.139230in}}%
\pgfpathclose%
\pgfusepath{stroke,fill}%
\end{pgfscope}%
\begin{pgfscope}%
\pgfpathrectangle{\pgfqpoint{0.506010in}{1.121191in}}{\pgfqpoint{2.325000in}{1.400000in}} %
\pgfusepath{clip}%
\pgfsetbuttcap%
\pgfsetroundjoin%
\definecolor{currentfill}{rgb}{1.000000,0.000000,0.000000}%
\pgfsetfillcolor{currentfill}%
\pgfsetlinewidth{1.003750pt}%
\definecolor{currentstroke}{rgb}{1.000000,0.000000,0.000000}%
\pgfsetstrokecolor{currentstroke}%
\pgfsetdash{}{0pt}%
\pgfpathmoveto{\pgfqpoint{2.238184in}{6.139230in}}%
\pgfpathclose%
\pgfusepath{stroke,fill}%
\end{pgfscope}%
\begin{pgfscope}%
\pgfpathrectangle{\pgfqpoint{0.506010in}{1.121191in}}{\pgfqpoint{2.325000in}{1.400000in}} %
\pgfusepath{clip}%
\pgfsetbuttcap%
\pgfsetroundjoin%
\definecolor{currentfill}{rgb}{1.000000,0.000000,0.000000}%
\pgfsetfillcolor{currentfill}%
\pgfsetlinewidth{1.003750pt}%
\definecolor{currentstroke}{rgb}{1.000000,0.000000,0.000000}%
\pgfsetstrokecolor{currentstroke}%
\pgfsetdash{}{0pt}%
\pgfpathmoveto{\pgfqpoint{2.248922in}{6.139230in}}%
\pgfpathclose%
\pgfusepath{stroke,fill}%
\end{pgfscope}%
\begin{pgfscope}%
\pgfpathrectangle{\pgfqpoint{0.506010in}{1.121191in}}{\pgfqpoint{2.325000in}{1.400000in}} %
\pgfusepath{clip}%
\pgfsetbuttcap%
\pgfsetroundjoin%
\definecolor{currentfill}{rgb}{1.000000,0.000000,0.000000}%
\pgfsetfillcolor{currentfill}%
\pgfsetlinewidth{1.003750pt}%
\definecolor{currentstroke}{rgb}{1.000000,0.000000,0.000000}%
\pgfsetstrokecolor{currentstroke}%
\pgfsetdash{}{0pt}%
\pgfpathmoveto{\pgfqpoint{2.262855in}{6.139230in}}%
\pgfpathclose%
\pgfusepath{stroke,fill}%
\end{pgfscope}%
\begin{pgfscope}%
\pgfpathrectangle{\pgfqpoint{0.506010in}{1.121191in}}{\pgfqpoint{2.325000in}{1.400000in}} %
\pgfusepath{clip}%
\pgfsetbuttcap%
\pgfsetroundjoin%
\definecolor{currentfill}{rgb}{1.000000,0.000000,0.000000}%
\pgfsetfillcolor{currentfill}%
\pgfsetlinewidth{1.003750pt}%
\definecolor{currentstroke}{rgb}{1.000000,0.000000,0.000000}%
\pgfsetstrokecolor{currentstroke}%
\pgfsetdash{}{0pt}%
\pgfpathmoveto{\pgfqpoint{2.233956in}{6.139230in}}%
\pgfpathclose%
\pgfusepath{stroke,fill}%
\end{pgfscope}%
\begin{pgfscope}%
\pgfpathrectangle{\pgfqpoint{0.506010in}{1.121191in}}{\pgfqpoint{2.325000in}{1.400000in}} %
\pgfusepath{clip}%
\pgfsetbuttcap%
\pgfsetroundjoin%
\definecolor{currentfill}{rgb}{1.000000,0.000000,0.000000}%
\pgfsetfillcolor{currentfill}%
\pgfsetlinewidth{1.003750pt}%
\definecolor{currentstroke}{rgb}{1.000000,0.000000,0.000000}%
\pgfsetstrokecolor{currentstroke}%
\pgfsetdash{}{0pt}%
\pgfpathmoveto{\pgfqpoint{2.217371in}{6.139230in}}%
\pgfpathclose%
\pgfusepath{stroke,fill}%
\end{pgfscope}%
\begin{pgfscope}%
\pgfpathrectangle{\pgfqpoint{0.506010in}{1.121191in}}{\pgfqpoint{2.325000in}{1.400000in}} %
\pgfusepath{clip}%
\pgfsetbuttcap%
\pgfsetroundjoin%
\definecolor{currentfill}{rgb}{1.000000,0.000000,0.000000}%
\pgfsetfillcolor{currentfill}%
\pgfsetlinewidth{1.003750pt}%
\definecolor{currentstroke}{rgb}{1.000000,0.000000,0.000000}%
\pgfsetstrokecolor{currentstroke}%
\pgfsetdash{}{0pt}%
\pgfpathmoveto{\pgfqpoint{2.184228in}{6.139230in}}%
\pgfpathclose%
\pgfusepath{stroke,fill}%
\end{pgfscope}%
\begin{pgfscope}%
\pgfpathrectangle{\pgfqpoint{0.506010in}{1.121191in}}{\pgfqpoint{2.325000in}{1.400000in}} %
\pgfusepath{clip}%
\pgfsetbuttcap%
\pgfsetroundjoin%
\definecolor{currentfill}{rgb}{1.000000,0.000000,0.000000}%
\pgfsetfillcolor{currentfill}%
\pgfsetlinewidth{1.003750pt}%
\definecolor{currentstroke}{rgb}{1.000000,0.000000,0.000000}%
\pgfsetstrokecolor{currentstroke}%
\pgfsetdash{}{0pt}%
\pgfpathmoveto{\pgfqpoint{2.201136in}{6.139230in}}%
\pgfpathclose%
\pgfusepath{stroke,fill}%
\end{pgfscope}%
\begin{pgfscope}%
\pgfpathrectangle{\pgfqpoint{0.506010in}{1.121191in}}{\pgfqpoint{2.325000in}{1.400000in}} %
\pgfusepath{clip}%
\pgfsetbuttcap%
\pgfsetroundjoin%
\definecolor{currentfill}{rgb}{0.000000,0.000000,1.000000}%
\pgfsetfillcolor{currentfill}%
\pgfsetlinewidth{1.003750pt}%
\definecolor{currentstroke}{rgb}{0.000000,0.000000,1.000000}%
\pgfsetstrokecolor{currentstroke}%
\pgfsetdash{}{0pt}%
\pgfpathmoveto{\pgfqpoint{1.129105in}{1.099734in}}%
\pgfpathmoveto{\pgfqpoint{1.140561in}{1.111191in}}%
\pgfpathlineto{\pgfqpoint{1.173026in}{1.143655in}}%
\pgfpathmoveto{\pgfqpoint{1.129105in}{1.143655in}}%
\pgfpathlineto{\pgfqpoint{1.161569in}{1.111191in}}%
\pgfusepath{stroke,fill}%
\end{pgfscope}%
\begin{pgfscope}%
\pgfpathrectangle{\pgfqpoint{0.506010in}{1.121191in}}{\pgfqpoint{2.325000in}{1.400000in}} %
\pgfusepath{clip}%
\pgfsetbuttcap%
\pgfsetroundjoin%
\definecolor{currentfill}{rgb}{0.000000,0.000000,1.000000}%
\pgfsetfillcolor{currentfill}%
\pgfsetlinewidth{1.003750pt}%
\definecolor{currentstroke}{rgb}{0.000000,0.000000,1.000000}%
\pgfsetstrokecolor{currentstroke}%
\pgfsetdash{}{0pt}%
\pgfpathmoveto{\pgfqpoint{0.750729in}{1.099818in}}%
\pgfpathmoveto{\pgfqpoint{0.762101in}{1.111191in}}%
\pgfpathlineto{\pgfqpoint{0.794650in}{1.143739in}}%
\pgfpathmoveto{\pgfqpoint{0.750729in}{1.143739in}}%
\pgfpathlineto{\pgfqpoint{0.783277in}{1.111191in}}%
\pgfusepath{stroke,fill}%
\end{pgfscope}%
\begin{pgfscope}%
\pgfpathrectangle{\pgfqpoint{0.506010in}{1.121191in}}{\pgfqpoint{2.325000in}{1.400000in}} %
\pgfusepath{clip}%
\pgfsetbuttcap%
\pgfsetroundjoin%
\definecolor{currentfill}{rgb}{0.000000,0.000000,1.000000}%
\pgfsetfillcolor{currentfill}%
\pgfsetlinewidth{1.003750pt}%
\definecolor{currentstroke}{rgb}{0.000000,0.000000,1.000000}%
\pgfsetstrokecolor{currentstroke}%
\pgfsetdash{}{0pt}%
\pgfpathmoveto{\pgfqpoint{1.200873in}{1.099818in}}%
\pgfpathmoveto{\pgfqpoint{1.212245in}{1.111191in}}%
\pgfpathlineto{\pgfqpoint{1.244793in}{1.143739in}}%
\pgfpathmoveto{\pgfqpoint{1.200873in}{1.143739in}}%
\pgfpathlineto{\pgfqpoint{1.233421in}{1.111191in}}%
\pgfusepath{stroke,fill}%
\end{pgfscope}%
\begin{pgfscope}%
\pgfpathrectangle{\pgfqpoint{0.506010in}{1.121191in}}{\pgfqpoint{2.325000in}{1.400000in}} %
\pgfusepath{clip}%
\pgfsetbuttcap%
\pgfsetroundjoin%
\definecolor{currentfill}{rgb}{0.000000,0.000000,1.000000}%
\pgfsetfillcolor{currentfill}%
\pgfsetlinewidth{1.003750pt}%
\definecolor{currentstroke}{rgb}{0.000000,0.000000,1.000000}%
\pgfsetstrokecolor{currentstroke}%
\pgfsetdash{}{0pt}%
\pgfpathmoveto{\pgfqpoint{0.958745in}{1.099818in}}%
\pgfpathmoveto{\pgfqpoint{0.970118in}{1.111191in}}%
\pgfpathlineto{\pgfqpoint{1.002666in}{1.143739in}}%
\pgfpathmoveto{\pgfqpoint{0.958745in}{1.143739in}}%
\pgfpathlineto{\pgfqpoint{0.991294in}{1.111191in}}%
\pgfusepath{stroke,fill}%
\end{pgfscope}%
\begin{pgfscope}%
\pgfpathrectangle{\pgfqpoint{0.506010in}{1.121191in}}{\pgfqpoint{2.325000in}{1.400000in}} %
\pgfusepath{clip}%
\pgfsetbuttcap%
\pgfsetroundjoin%
\definecolor{currentfill}{rgb}{0.000000,0.000000,1.000000}%
\pgfsetfillcolor{currentfill}%
\pgfsetlinewidth{1.003750pt}%
\definecolor{currentstroke}{rgb}{0.000000,0.000000,1.000000}%
\pgfsetstrokecolor{currentstroke}%
\pgfsetdash{}{0pt}%
\pgfpathmoveto{\pgfqpoint{0.946243in}{1.099846in}}%
\pgfpathmoveto{\pgfqpoint{0.957587in}{1.111191in}}%
\pgfpathlineto{\pgfqpoint{0.990163in}{1.143767in}}%
\pgfpathmoveto{\pgfqpoint{0.946243in}{1.143767in}}%
\pgfpathlineto{\pgfqpoint{0.978819in}{1.111191in}}%
\pgfusepath{stroke,fill}%
\end{pgfscope}%
\begin{pgfscope}%
\pgfpathrectangle{\pgfqpoint{0.506010in}{1.121191in}}{\pgfqpoint{2.325000in}{1.400000in}} %
\pgfusepath{clip}%
\pgfsetbuttcap%
\pgfsetroundjoin%
\definecolor{currentfill}{rgb}{0.000000,0.000000,1.000000}%
\pgfsetfillcolor{currentfill}%
\pgfsetlinewidth{1.003750pt}%
\definecolor{currentstroke}{rgb}{0.000000,0.000000,1.000000}%
\pgfsetstrokecolor{currentstroke}%
\pgfsetdash{}{0pt}%
\pgfpathmoveto{\pgfqpoint{1.075188in}{1.099874in}}%
\pgfpathmoveto{\pgfqpoint{1.086504in}{1.111191in}}%
\pgfpathlineto{\pgfqpoint{1.119109in}{1.143795in}}%
\pgfpathmoveto{\pgfqpoint{1.075188in}{1.143795in}}%
\pgfpathlineto{\pgfqpoint{1.107792in}{1.111191in}}%
\pgfusepath{stroke,fill}%
\end{pgfscope}%
\begin{pgfscope}%
\pgfpathrectangle{\pgfqpoint{0.506010in}{1.121191in}}{\pgfqpoint{2.325000in}{1.400000in}} %
\pgfusepath{clip}%
\pgfsetbuttcap%
\pgfsetroundjoin%
\definecolor{currentfill}{rgb}{0.000000,0.000000,1.000000}%
\pgfsetfillcolor{currentfill}%
\pgfsetlinewidth{1.003750pt}%
\definecolor{currentstroke}{rgb}{0.000000,0.000000,1.000000}%
\pgfsetstrokecolor{currentstroke}%
\pgfsetdash{}{0pt}%
\pgfpathmoveto{\pgfqpoint{0.791771in}{1.099902in}}%
\pgfpathmoveto{\pgfqpoint{0.803059in}{1.111191in}}%
\pgfpathlineto{\pgfqpoint{0.835691in}{1.143823in}}%
\pgfpathmoveto{\pgfqpoint{0.791771in}{1.143823in}}%
\pgfpathlineto{\pgfqpoint{0.824403in}{1.111191in}}%
\pgfusepath{stroke,fill}%
\end{pgfscope}%
\begin{pgfscope}%
\pgfpathrectangle{\pgfqpoint{0.506010in}{1.121191in}}{\pgfqpoint{2.325000in}{1.400000in}} %
\pgfusepath{clip}%
\pgfsetbuttcap%
\pgfsetroundjoin%
\definecolor{currentfill}{rgb}{0.000000,0.000000,1.000000}%
\pgfsetfillcolor{currentfill}%
\pgfsetlinewidth{1.003750pt}%
\definecolor{currentstroke}{rgb}{0.000000,0.000000,1.000000}%
\pgfsetstrokecolor{currentstroke}%
\pgfsetdash{}{0pt}%
\pgfpathmoveto{\pgfqpoint{1.286780in}{1.099902in}}%
\pgfpathmoveto{\pgfqpoint{1.298068in}{1.111191in}}%
\pgfpathlineto{\pgfqpoint{1.330701in}{1.143823in}}%
\pgfpathmoveto{\pgfqpoint{1.286780in}{1.143823in}}%
\pgfpathlineto{\pgfqpoint{1.319412in}{1.111191in}}%
\pgfusepath{stroke,fill}%
\end{pgfscope}%
\begin{pgfscope}%
\pgfpathrectangle{\pgfqpoint{0.506010in}{1.121191in}}{\pgfqpoint{2.325000in}{1.400000in}} %
\pgfusepath{clip}%
\pgfsetbuttcap%
\pgfsetroundjoin%
\definecolor{currentfill}{rgb}{0.000000,0.000000,1.000000}%
\pgfsetfillcolor{currentfill}%
\pgfsetlinewidth{1.003750pt}%
\definecolor{currentstroke}{rgb}{0.000000,0.000000,1.000000}%
\pgfsetstrokecolor{currentstroke}%
\pgfsetdash{}{0pt}%
\pgfpathmoveto{\pgfqpoint{0.824673in}{1.099902in}}%
\pgfpathmoveto{\pgfqpoint{0.835961in}{1.111191in}}%
\pgfpathlineto{\pgfqpoint{0.868593in}{1.143823in}}%
\pgfpathmoveto{\pgfqpoint{0.824673in}{1.143823in}}%
\pgfpathlineto{\pgfqpoint{0.857305in}{1.111191in}}%
\pgfusepath{stroke,fill}%
\end{pgfscope}%
\begin{pgfscope}%
\pgfpathrectangle{\pgfqpoint{0.506010in}{1.121191in}}{\pgfqpoint{2.325000in}{1.400000in}} %
\pgfusepath{clip}%
\pgfsetbuttcap%
\pgfsetroundjoin%
\definecolor{currentfill}{rgb}{0.000000,0.000000,1.000000}%
\pgfsetfillcolor{currentfill}%
\pgfsetlinewidth{1.003750pt}%
\definecolor{currentstroke}{rgb}{0.000000,0.000000,1.000000}%
\pgfsetstrokecolor{currentstroke}%
\pgfsetdash{}{0pt}%
\pgfpathmoveto{\pgfqpoint{1.122657in}{1.099930in}}%
\pgfpathmoveto{\pgfqpoint{1.133917in}{1.111191in}}%
\pgfpathlineto{\pgfqpoint{1.166577in}{1.143851in}}%
\pgfpathmoveto{\pgfqpoint{1.122657in}{1.143851in}}%
\pgfpathlineto{\pgfqpoint{1.155317in}{1.111191in}}%
\pgfusepath{stroke,fill}%
\end{pgfscope}%
\begin{pgfscope}%
\pgfpathrectangle{\pgfqpoint{0.506010in}{1.121191in}}{\pgfqpoint{2.325000in}{1.400000in}} %
\pgfusepath{clip}%
\pgfsetbuttcap%
\pgfsetroundjoin%
\definecolor{currentfill}{rgb}{0.000000,0.000000,1.000000}%
\pgfsetfillcolor{currentfill}%
\pgfsetlinewidth{1.003750pt}%
\definecolor{currentstroke}{rgb}{0.000000,0.000000,1.000000}%
\pgfsetstrokecolor{currentstroke}%
\pgfsetdash{}{0pt}%
\pgfpathmoveto{\pgfqpoint{1.143544in}{1.099930in}}%
\pgfpathmoveto{\pgfqpoint{1.154804in}{1.111191in}}%
\pgfpathlineto{\pgfqpoint{1.187464in}{1.143851in}}%
\pgfpathmoveto{\pgfqpoint{1.143544in}{1.143851in}}%
\pgfpathlineto{\pgfqpoint{1.176204in}{1.111191in}}%
\pgfusepath{stroke,fill}%
\end{pgfscope}%
\begin{pgfscope}%
\pgfpathrectangle{\pgfqpoint{0.506010in}{1.121191in}}{\pgfqpoint{2.325000in}{1.400000in}} %
\pgfusepath{clip}%
\pgfsetbuttcap%
\pgfsetroundjoin%
\definecolor{currentfill}{rgb}{0.000000,0.000000,1.000000}%
\pgfsetfillcolor{currentfill}%
\pgfsetlinewidth{1.003750pt}%
\definecolor{currentstroke}{rgb}{0.000000,0.000000,1.000000}%
\pgfsetstrokecolor{currentstroke}%
\pgfsetdash{}{0pt}%
\pgfpathmoveto{\pgfqpoint{0.766515in}{1.099958in}}%
\pgfpathmoveto{\pgfqpoint{0.777747in}{1.111191in}}%
\pgfpathlineto{\pgfqpoint{0.810435in}{1.143879in}}%
\pgfpathmoveto{\pgfqpoint{0.766515in}{1.143879in}}%
\pgfpathlineto{\pgfqpoint{0.799203in}{1.111191in}}%
\pgfusepath{stroke,fill}%
\end{pgfscope}%
\begin{pgfscope}%
\pgfpathrectangle{\pgfqpoint{0.506010in}{1.121191in}}{\pgfqpoint{2.325000in}{1.400000in}} %
\pgfusepath{clip}%
\pgfsetbuttcap%
\pgfsetroundjoin%
\definecolor{currentfill}{rgb}{0.000000,0.000000,1.000000}%
\pgfsetfillcolor{currentfill}%
\pgfsetlinewidth{1.003750pt}%
\definecolor{currentstroke}{rgb}{0.000000,0.000000,1.000000}%
\pgfsetstrokecolor{currentstroke}%
\pgfsetdash{}{0pt}%
\pgfpathmoveto{\pgfqpoint{1.565719in}{1.100014in}}%
\pgfpathmoveto{\pgfqpoint{1.576895in}{1.111191in}}%
\pgfpathlineto{\pgfqpoint{1.609639in}{1.143935in}}%
\pgfpathmoveto{\pgfqpoint{1.565719in}{1.143935in}}%
\pgfpathlineto{\pgfqpoint{1.598463in}{1.111191in}}%
\pgfusepath{stroke,fill}%
\end{pgfscope}%
\begin{pgfscope}%
\pgfpathrectangle{\pgfqpoint{0.506010in}{1.121191in}}{\pgfqpoint{2.325000in}{1.400000in}} %
\pgfusepath{clip}%
\pgfsetbuttcap%
\pgfsetroundjoin%
\definecolor{currentfill}{rgb}{0.000000,0.000000,1.000000}%
\pgfsetfillcolor{currentfill}%
\pgfsetlinewidth{1.003750pt}%
\definecolor{currentstroke}{rgb}{0.000000,0.000000,1.000000}%
\pgfsetstrokecolor{currentstroke}%
\pgfsetdash{}{0pt}%
\pgfpathmoveto{\pgfqpoint{1.344326in}{1.100098in}}%
\pgfpathmoveto{\pgfqpoint{1.355418in}{1.111191in}}%
\pgfpathlineto{\pgfqpoint{1.388247in}{1.144019in}}%
\pgfpathmoveto{\pgfqpoint{1.344326in}{1.144019in}}%
\pgfpathlineto{\pgfqpoint{1.377154in}{1.111191in}}%
\pgfusepath{stroke,fill}%
\end{pgfscope}%
\begin{pgfscope}%
\pgfpathrectangle{\pgfqpoint{0.506010in}{1.121191in}}{\pgfqpoint{2.325000in}{1.400000in}} %
\pgfusepath{clip}%
\pgfsetbuttcap%
\pgfsetroundjoin%
\definecolor{currentfill}{rgb}{0.000000,0.000000,1.000000}%
\pgfsetfillcolor{currentfill}%
\pgfsetlinewidth{1.003750pt}%
\definecolor{currentstroke}{rgb}{0.000000,0.000000,1.000000}%
\pgfsetstrokecolor{currentstroke}%
\pgfsetdash{}{0pt}%
\pgfpathmoveto{\pgfqpoint{1.257240in}{1.100126in}}%
\pgfpathmoveto{\pgfqpoint{1.268305in}{1.111191in}}%
\pgfpathlineto{\pgfqpoint{1.301161in}{1.144047in}}%
\pgfpathmoveto{\pgfqpoint{1.257240in}{1.144047in}}%
\pgfpathlineto{\pgfqpoint{1.290097in}{1.111191in}}%
\pgfusepath{stroke,fill}%
\end{pgfscope}%
\begin{pgfscope}%
\pgfpathrectangle{\pgfqpoint{0.506010in}{1.121191in}}{\pgfqpoint{2.325000in}{1.400000in}} %
\pgfusepath{clip}%
\pgfsetbuttcap%
\pgfsetroundjoin%
\definecolor{currentfill}{rgb}{0.000000,0.000000,1.000000}%
\pgfsetfillcolor{currentfill}%
\pgfsetlinewidth{1.003750pt}%
\definecolor{currentstroke}{rgb}{0.000000,0.000000,1.000000}%
\pgfsetstrokecolor{currentstroke}%
\pgfsetdash{}{0pt}%
\pgfpathmoveto{\pgfqpoint{2.242070in}{1.100126in}}%
\pgfpathmoveto{\pgfqpoint{2.253134in}{1.111191in}}%
\pgfpathlineto{\pgfqpoint{2.285990in}{1.144047in}}%
\pgfpathmoveto{\pgfqpoint{2.242070in}{1.144047in}}%
\pgfpathlineto{\pgfqpoint{2.274926in}{1.111191in}}%
\pgfusepath{stroke,fill}%
\end{pgfscope}%
\begin{pgfscope}%
\pgfpathrectangle{\pgfqpoint{0.506010in}{1.121191in}}{\pgfqpoint{2.325000in}{1.400000in}} %
\pgfusepath{clip}%
\pgfsetbuttcap%
\pgfsetroundjoin%
\definecolor{currentfill}{rgb}{0.000000,0.000000,1.000000}%
\pgfsetfillcolor{currentfill}%
\pgfsetlinewidth{1.003750pt}%
\definecolor{currentstroke}{rgb}{0.000000,0.000000,1.000000}%
\pgfsetstrokecolor{currentstroke}%
\pgfsetdash{}{0pt}%
\pgfpathmoveto{\pgfqpoint{2.169041in}{1.100154in}}%
\pgfpathmoveto{\pgfqpoint{2.180077in}{1.111191in}}%
\pgfpathlineto{\pgfqpoint{2.212962in}{1.144075in}}%
\pgfpathmoveto{\pgfqpoint{2.169041in}{1.144075in}}%
\pgfpathlineto{\pgfqpoint{2.201925in}{1.111191in}}%
\pgfusepath{stroke,fill}%
\end{pgfscope}%
\begin{pgfscope}%
\pgfpathrectangle{\pgfqpoint{0.506010in}{1.121191in}}{\pgfqpoint{2.325000in}{1.400000in}} %
\pgfusepath{clip}%
\pgfsetbuttcap%
\pgfsetroundjoin%
\definecolor{currentfill}{rgb}{0.000000,0.000000,1.000000}%
\pgfsetfillcolor{currentfill}%
\pgfsetlinewidth{1.003750pt}%
\definecolor{currentstroke}{rgb}{0.000000,0.000000,1.000000}%
\pgfsetstrokecolor{currentstroke}%
\pgfsetdash{}{0pt}%
\pgfpathmoveto{\pgfqpoint{2.241063in}{1.100154in}}%
\pgfpathmoveto{\pgfqpoint{2.252099in}{1.111191in}}%
\pgfpathlineto{\pgfqpoint{2.284984in}{1.144075in}}%
\pgfpathmoveto{\pgfqpoint{2.241063in}{1.144075in}}%
\pgfpathlineto{\pgfqpoint{2.273947in}{1.111191in}}%
\pgfusepath{stroke,fill}%
\end{pgfscope}%
\begin{pgfscope}%
\pgfpathrectangle{\pgfqpoint{0.506010in}{1.121191in}}{\pgfqpoint{2.325000in}{1.400000in}} %
\pgfusepath{clip}%
\pgfsetbuttcap%
\pgfsetroundjoin%
\definecolor{currentfill}{rgb}{0.000000,0.000000,1.000000}%
\pgfsetfillcolor{currentfill}%
\pgfsetlinewidth{1.003750pt}%
\definecolor{currentstroke}{rgb}{0.000000,0.000000,1.000000}%
\pgfsetstrokecolor{currentstroke}%
\pgfsetdash{}{0pt}%
\pgfpathmoveto{\pgfqpoint{2.229707in}{1.100182in}}%
\pgfpathmoveto{\pgfqpoint{2.240715in}{1.111191in}}%
\pgfpathlineto{\pgfqpoint{2.273627in}{1.144103in}}%
\pgfpathmoveto{\pgfqpoint{2.229707in}{1.144103in}}%
\pgfpathlineto{\pgfqpoint{2.262619in}{1.111191in}}%
\pgfusepath{stroke,fill}%
\end{pgfscope}%
\begin{pgfscope}%
\pgfpathrectangle{\pgfqpoint{0.506010in}{1.121191in}}{\pgfqpoint{2.325000in}{1.400000in}} %
\pgfusepath{clip}%
\pgfsetbuttcap%
\pgfsetroundjoin%
\definecolor{currentfill}{rgb}{0.000000,0.000000,1.000000}%
\pgfsetfillcolor{currentfill}%
\pgfsetlinewidth{1.003750pt}%
\definecolor{currentstroke}{rgb}{0.000000,0.000000,1.000000}%
\pgfsetstrokecolor{currentstroke}%
\pgfsetdash{}{0pt}%
\pgfpathmoveto{\pgfqpoint{2.201638in}{1.100182in}}%
\pgfpathmoveto{\pgfqpoint{2.212646in}{1.111191in}}%
\pgfpathlineto{\pgfqpoint{2.245558in}{1.144103in}}%
\pgfpathmoveto{\pgfqpoint{2.201638in}{1.144103in}}%
\pgfpathlineto{\pgfqpoint{2.234550in}{1.111191in}}%
\pgfusepath{stroke,fill}%
\end{pgfscope}%
\begin{pgfscope}%
\pgfpathrectangle{\pgfqpoint{0.506010in}{1.121191in}}{\pgfqpoint{2.325000in}{1.400000in}} %
\pgfusepath{clip}%
\pgfsetbuttcap%
\pgfsetroundjoin%
\definecolor{currentfill}{rgb}{0.000000,0.000000,1.000000}%
\pgfsetfillcolor{currentfill}%
\pgfsetlinewidth{1.003750pt}%
\definecolor{currentstroke}{rgb}{0.000000,0.000000,1.000000}%
\pgfsetstrokecolor{currentstroke}%
\pgfsetdash{}{0pt}%
\pgfpathmoveto{\pgfqpoint{2.201638in}{1.100182in}}%
\pgfpathmoveto{\pgfqpoint{2.212646in}{1.111191in}}%
\pgfpathlineto{\pgfqpoint{2.245558in}{1.144103in}}%
\pgfpathmoveto{\pgfqpoint{2.201638in}{1.144103in}}%
\pgfpathlineto{\pgfqpoint{2.234550in}{1.111191in}}%
\pgfusepath{stroke,fill}%
\end{pgfscope}%
\begin{pgfscope}%
\pgfpathrectangle{\pgfqpoint{0.506010in}{1.121191in}}{\pgfqpoint{2.325000in}{1.400000in}} %
\pgfusepath{clip}%
\pgfsetbuttcap%
\pgfsetroundjoin%
\definecolor{currentfill}{rgb}{0.000000,0.000000,1.000000}%
\pgfsetfillcolor{currentfill}%
\pgfsetlinewidth{1.003750pt}%
\definecolor{currentstroke}{rgb}{0.000000,0.000000,1.000000}%
\pgfsetstrokecolor{currentstroke}%
\pgfsetdash{}{0pt}%
\pgfpathmoveto{\pgfqpoint{0.985862in}{1.100182in}}%
\pgfpathmoveto{\pgfqpoint{0.996870in}{1.111191in}}%
\pgfpathlineto{\pgfqpoint{1.029783in}{1.144103in}}%
\pgfpathmoveto{\pgfqpoint{0.985862in}{1.144103in}}%
\pgfpathlineto{\pgfqpoint{1.018774in}{1.111191in}}%
\pgfusepath{stroke,fill}%
\end{pgfscope}%
\begin{pgfscope}%
\pgfpathrectangle{\pgfqpoint{0.506010in}{1.121191in}}{\pgfqpoint{2.325000in}{1.400000in}} %
\pgfusepath{clip}%
\pgfsetbuttcap%
\pgfsetroundjoin%
\definecolor{currentfill}{rgb}{0.000000,0.000000,1.000000}%
\pgfsetfillcolor{currentfill}%
\pgfsetlinewidth{1.003750pt}%
\definecolor{currentstroke}{rgb}{0.000000,0.000000,1.000000}%
\pgfsetstrokecolor{currentstroke}%
\pgfsetdash{}{0pt}%
\pgfpathmoveto{\pgfqpoint{0.985862in}{1.100210in}}%
\pgfpathmoveto{\pgfqpoint{0.996842in}{1.111191in}}%
\pgfpathlineto{\pgfqpoint{1.029783in}{1.144131in}}%
\pgfpathmoveto{\pgfqpoint{0.985862in}{1.144131in}}%
\pgfpathlineto{\pgfqpoint{1.018802in}{1.111191in}}%
\pgfusepath{stroke,fill}%
\end{pgfscope}%
\begin{pgfscope}%
\pgfpathrectangle{\pgfqpoint{0.506010in}{1.121191in}}{\pgfqpoint{2.325000in}{1.400000in}} %
\pgfusepath{clip}%
\pgfsetbuttcap%
\pgfsetroundjoin%
\definecolor{currentfill}{rgb}{0.000000,0.000000,1.000000}%
\pgfsetfillcolor{currentfill}%
\pgfsetlinewidth{1.003750pt}%
\definecolor{currentstroke}{rgb}{0.000000,0.000000,1.000000}%
\pgfsetstrokecolor{currentstroke}%
\pgfsetdash{}{0pt}%
\pgfpathmoveto{\pgfqpoint{2.359497in}{1.100238in}}%
\pgfpathmoveto{\pgfqpoint{2.370450in}{1.111191in}}%
\pgfpathlineto{\pgfqpoint{2.403418in}{1.144159in}}%
\pgfpathmoveto{\pgfqpoint{2.359497in}{1.144159in}}%
\pgfpathlineto{\pgfqpoint{2.392466in}{1.111191in}}%
\pgfusepath{stroke,fill}%
\end{pgfscope}%
\begin{pgfscope}%
\pgfpathrectangle{\pgfqpoint{0.506010in}{1.121191in}}{\pgfqpoint{2.325000in}{1.400000in}} %
\pgfusepath{clip}%
\pgfsetbuttcap%
\pgfsetroundjoin%
\definecolor{currentfill}{rgb}{0.000000,0.000000,1.000000}%
\pgfsetfillcolor{currentfill}%
\pgfsetlinewidth{1.003750pt}%
\definecolor{currentstroke}{rgb}{0.000000,0.000000,1.000000}%
\pgfsetstrokecolor{currentstroke}%
\pgfsetdash{}{0pt}%
\pgfpathmoveto{\pgfqpoint{2.358144in}{1.100238in}}%
\pgfpathmoveto{\pgfqpoint{2.369097in}{1.111191in}}%
\pgfpathlineto{\pgfqpoint{2.402065in}{1.144159in}}%
\pgfpathmoveto{\pgfqpoint{2.358144in}{1.144159in}}%
\pgfpathlineto{\pgfqpoint{2.391113in}{1.111191in}}%
\pgfusepath{stroke,fill}%
\end{pgfscope}%
\begin{pgfscope}%
\pgfpathrectangle{\pgfqpoint{0.506010in}{1.121191in}}{\pgfqpoint{2.325000in}{1.400000in}} %
\pgfusepath{clip}%
\pgfsetbuttcap%
\pgfsetroundjoin%
\definecolor{currentfill}{rgb}{0.000000,0.000000,1.000000}%
\pgfsetfillcolor{currentfill}%
\pgfsetlinewidth{1.003750pt}%
\definecolor{currentstroke}{rgb}{0.000000,0.000000,1.000000}%
\pgfsetstrokecolor{currentstroke}%
\pgfsetdash{}{0pt}%
\pgfpathmoveto{\pgfqpoint{2.385195in}{1.100378in}}%
\pgfpathmoveto{\pgfqpoint{2.396007in}{1.111191in}}%
\pgfpathlineto{\pgfqpoint{2.429115in}{1.144299in}}%
\pgfpathmoveto{\pgfqpoint{2.385195in}{1.144299in}}%
\pgfpathlineto{\pgfqpoint{2.418303in}{1.111191in}}%
\pgfusepath{stroke,fill}%
\end{pgfscope}%
\begin{pgfscope}%
\pgfpathrectangle{\pgfqpoint{0.506010in}{1.121191in}}{\pgfqpoint{2.325000in}{1.400000in}} %
\pgfusepath{clip}%
\pgfsetbuttcap%
\pgfsetroundjoin%
\definecolor{currentfill}{rgb}{0.000000,0.000000,1.000000}%
\pgfsetfillcolor{currentfill}%
\pgfsetlinewidth{1.003750pt}%
\definecolor{currentstroke}{rgb}{0.000000,0.000000,1.000000}%
\pgfsetstrokecolor{currentstroke}%
\pgfsetdash{}{0pt}%
\pgfpathmoveto{\pgfqpoint{1.272739in}{1.102002in}}%
\pgfpathmoveto{\pgfqpoint{1.281927in}{1.111191in}}%
\pgfpathlineto{\pgfqpoint{1.316659in}{1.145923in}}%
\pgfpathmoveto{\pgfqpoint{1.272739in}{1.145923in}}%
\pgfpathlineto{\pgfqpoint{1.307471in}{1.111191in}}%
\pgfusepath{stroke,fill}%
\end{pgfscope}%
\begin{pgfscope}%
\pgfpathrectangle{\pgfqpoint{0.506010in}{1.121191in}}{\pgfqpoint{2.325000in}{1.400000in}} %
\pgfusepath{clip}%
\pgfsetbuttcap%
\pgfsetroundjoin%
\definecolor{currentfill}{rgb}{0.000000,0.000000,1.000000}%
\pgfsetfillcolor{currentfill}%
\pgfsetlinewidth{1.003750pt}%
\definecolor{currentstroke}{rgb}{0.000000,0.000000,1.000000}%
\pgfsetstrokecolor{currentstroke}%
\pgfsetdash{}{0pt}%
\pgfpathmoveto{\pgfqpoint{1.368772in}{1.105922in}}%
\pgfpathmoveto{\pgfqpoint{1.374040in}{1.111191in}}%
\pgfpathlineto{\pgfqpoint{1.412693in}{1.149843in}}%
\pgfpathmoveto{\pgfqpoint{1.368772in}{1.149843in}}%
\pgfpathlineto{\pgfqpoint{1.407424in}{1.111191in}}%
\pgfusepath{stroke,fill}%
\end{pgfscope}%
\begin{pgfscope}%
\pgfpathrectangle{\pgfqpoint{0.506010in}{1.121191in}}{\pgfqpoint{2.325000in}{1.400000in}} %
\pgfusepath{clip}%
\pgfsetbuttcap%
\pgfsetroundjoin%
\definecolor{currentfill}{rgb}{0.000000,0.000000,1.000000}%
\pgfsetfillcolor{currentfill}%
\pgfsetlinewidth{1.003750pt}%
\definecolor{currentstroke}{rgb}{0.000000,0.000000,1.000000}%
\pgfsetstrokecolor{currentstroke}%
\pgfsetdash{}{0pt}%
\pgfpathmoveto{\pgfqpoint{1.283097in}{1.109506in}}%
\pgfpathmoveto{\pgfqpoint{1.284781in}{1.111191in}}%
\pgfpathlineto{\pgfqpoint{1.327018in}{1.153427in}}%
\pgfpathmoveto{\pgfqpoint{1.283097in}{1.153427in}}%
\pgfpathlineto{\pgfqpoint{1.325333in}{1.111191in}}%
\pgfusepath{stroke,fill}%
\end{pgfscope}%
\begin{pgfscope}%
\pgfpathrectangle{\pgfqpoint{0.506010in}{1.121191in}}{\pgfqpoint{2.325000in}{1.400000in}} %
\pgfusepath{clip}%
\pgfsetbuttcap%
\pgfsetroundjoin%
\definecolor{currentfill}{rgb}{0.000000,0.000000,1.000000}%
\pgfsetfillcolor{currentfill}%
\pgfsetlinewidth{1.003750pt}%
\definecolor{currentstroke}{rgb}{0.000000,0.000000,1.000000}%
\pgfsetstrokecolor{currentstroke}%
\pgfsetdash{}{0pt}%
\pgfpathmoveto{\pgfqpoint{1.059422in}{1.120426in}}%
\pgfpathlineto{\pgfqpoint{1.103342in}{1.164347in}}%
\pgfpathmoveto{\pgfqpoint{1.059422in}{1.164347in}}%
\pgfpathlineto{\pgfqpoint{1.103342in}{1.120426in}}%
\pgfusepath{stroke,fill}%
\end{pgfscope}%
\begin{pgfscope}%
\pgfpathrectangle{\pgfqpoint{0.506010in}{1.121191in}}{\pgfqpoint{2.325000in}{1.400000in}} %
\pgfusepath{clip}%
\pgfsetbuttcap%
\pgfsetroundjoin%
\definecolor{currentfill}{rgb}{0.000000,0.000000,1.000000}%
\pgfsetfillcolor{currentfill}%
\pgfsetlinewidth{1.003750pt}%
\definecolor{currentstroke}{rgb}{0.000000,0.000000,1.000000}%
\pgfsetstrokecolor{currentstroke}%
\pgfsetdash{}{0pt}%
\pgfpathmoveto{\pgfqpoint{0.788404in}{1.123030in}}%
\pgfpathlineto{\pgfqpoint{0.832324in}{1.166951in}}%
\pgfpathmoveto{\pgfqpoint{0.788404in}{1.166951in}}%
\pgfpathlineto{\pgfqpoint{0.832324in}{1.123030in}}%
\pgfusepath{stroke,fill}%
\end{pgfscope}%
\begin{pgfscope}%
\pgfpathrectangle{\pgfqpoint{0.506010in}{1.121191in}}{\pgfqpoint{2.325000in}{1.400000in}} %
\pgfusepath{clip}%
\pgfsetbuttcap%
\pgfsetroundjoin%
\definecolor{currentfill}{rgb}{0.000000,0.000000,1.000000}%
\pgfsetfillcolor{currentfill}%
\pgfsetlinewidth{1.003750pt}%
\definecolor{currentstroke}{rgb}{0.000000,0.000000,1.000000}%
\pgfsetstrokecolor{currentstroke}%
\pgfsetdash{}{0pt}%
\pgfpathmoveto{\pgfqpoint{1.344910in}{1.131514in}}%
\pgfpathlineto{\pgfqpoint{1.388831in}{1.175435in}}%
\pgfpathmoveto{\pgfqpoint{1.344910in}{1.175435in}}%
\pgfpathlineto{\pgfqpoint{1.388831in}{1.131514in}}%
\pgfusepath{stroke,fill}%
\end{pgfscope}%
\begin{pgfscope}%
\pgfpathrectangle{\pgfqpoint{0.506010in}{1.121191in}}{\pgfqpoint{2.325000in}{1.400000in}} %
\pgfusepath{clip}%
\pgfsetbuttcap%
\pgfsetroundjoin%
\definecolor{currentfill}{rgb}{0.000000,0.000000,1.000000}%
\pgfsetfillcolor{currentfill}%
\pgfsetlinewidth{1.003750pt}%
\definecolor{currentstroke}{rgb}{0.000000,0.000000,1.000000}%
\pgfsetstrokecolor{currentstroke}%
\pgfsetdash{}{0pt}%
\pgfpathmoveto{\pgfqpoint{1.421720in}{1.139606in}}%
\pgfpathlineto{\pgfqpoint{1.465640in}{1.183527in}}%
\pgfpathmoveto{\pgfqpoint{1.421720in}{1.183527in}}%
\pgfpathlineto{\pgfqpoint{1.465640in}{1.139606in}}%
\pgfusepath{stroke,fill}%
\end{pgfscope}%
\begin{pgfscope}%
\pgfpathrectangle{\pgfqpoint{0.506010in}{1.121191in}}{\pgfqpoint{2.325000in}{1.400000in}} %
\pgfusepath{clip}%
\pgfsetbuttcap%
\pgfsetroundjoin%
\definecolor{currentfill}{rgb}{0.000000,0.000000,1.000000}%
\pgfsetfillcolor{currentfill}%
\pgfsetlinewidth{1.003750pt}%
\definecolor{currentstroke}{rgb}{0.000000,0.000000,1.000000}%
\pgfsetstrokecolor{currentstroke}%
\pgfsetdash{}{0pt}%
\pgfpathmoveto{\pgfqpoint{1.384927in}{1.142994in}}%
\pgfpathlineto{\pgfqpoint{1.428847in}{1.186915in}}%
\pgfpathmoveto{\pgfqpoint{1.384927in}{1.186915in}}%
\pgfpathlineto{\pgfqpoint{1.428847in}{1.142994in}}%
\pgfusepath{stroke,fill}%
\end{pgfscope}%
\begin{pgfscope}%
\pgfpathrectangle{\pgfqpoint{0.506010in}{1.121191in}}{\pgfqpoint{2.325000in}{1.400000in}} %
\pgfusepath{clip}%
\pgfsetbuttcap%
\pgfsetroundjoin%
\definecolor{currentfill}{rgb}{0.000000,0.000000,1.000000}%
\pgfsetfillcolor{currentfill}%
\pgfsetlinewidth{1.003750pt}%
\definecolor{currentstroke}{rgb}{0.000000,0.000000,1.000000}%
\pgfsetstrokecolor{currentstroke}%
\pgfsetdash{}{0pt}%
\pgfpathmoveto{\pgfqpoint{1.087014in}{1.158618in}}%
\pgfpathlineto{\pgfqpoint{1.130934in}{1.202539in}}%
\pgfpathmoveto{\pgfqpoint{1.087014in}{1.202539in}}%
\pgfpathlineto{\pgfqpoint{1.130934in}{1.158618in}}%
\pgfusepath{stroke,fill}%
\end{pgfscope}%
\begin{pgfscope}%
\pgfpathrectangle{\pgfqpoint{0.506010in}{1.121191in}}{\pgfqpoint{2.325000in}{1.400000in}} %
\pgfusepath{clip}%
\pgfsetbuttcap%
\pgfsetroundjoin%
\definecolor{currentfill}{rgb}{0.000000,0.000000,1.000000}%
\pgfsetfillcolor{currentfill}%
\pgfsetlinewidth{1.003750pt}%
\definecolor{currentstroke}{rgb}{0.000000,0.000000,1.000000}%
\pgfsetstrokecolor{currentstroke}%
\pgfsetdash{}{0pt}%
\pgfpathmoveto{\pgfqpoint{0.799968in}{1.161138in}}%
\pgfpathlineto{\pgfqpoint{0.843888in}{1.205059in}}%
\pgfpathmoveto{\pgfqpoint{0.799968in}{1.205059in}}%
\pgfpathlineto{\pgfqpoint{0.843888in}{1.161138in}}%
\pgfusepath{stroke,fill}%
\end{pgfscope}%
\begin{pgfscope}%
\pgfpathrectangle{\pgfqpoint{0.506010in}{1.121191in}}{\pgfqpoint{2.325000in}{1.400000in}} %
\pgfusepath{clip}%
\pgfsetbuttcap%
\pgfsetroundjoin%
\definecolor{currentfill}{rgb}{0.000000,0.000000,1.000000}%
\pgfsetfillcolor{currentfill}%
\pgfsetlinewidth{1.003750pt}%
\definecolor{currentstroke}{rgb}{0.000000,0.000000,1.000000}%
\pgfsetstrokecolor{currentstroke}%
\pgfsetdash{}{0pt}%
\pgfpathmoveto{\pgfqpoint{1.421720in}{1.179562in}}%
\pgfpathlineto{\pgfqpoint{1.465640in}{1.223483in}}%
\pgfpathmoveto{\pgfqpoint{1.421720in}{1.223483in}}%
\pgfpathlineto{\pgfqpoint{1.465640in}{1.179562in}}%
\pgfusepath{stroke,fill}%
\end{pgfscope}%
\begin{pgfscope}%
\pgfpathrectangle{\pgfqpoint{0.506010in}{1.121191in}}{\pgfqpoint{2.325000in}{1.400000in}} %
\pgfusepath{clip}%
\pgfsetbuttcap%
\pgfsetroundjoin%
\definecolor{currentfill}{rgb}{0.000000,0.000000,1.000000}%
\pgfsetfillcolor{currentfill}%
\pgfsetlinewidth{1.003750pt}%
\definecolor{currentstroke}{rgb}{0.000000,0.000000,1.000000}%
\pgfsetstrokecolor{currentstroke}%
\pgfsetdash{}{0pt}%
\pgfpathmoveto{\pgfqpoint{1.357153in}{1.206750in}}%
\pgfpathlineto{\pgfqpoint{1.401074in}{1.250671in}}%
\pgfpathmoveto{\pgfqpoint{1.357153in}{1.250671in}}%
\pgfpathlineto{\pgfqpoint{1.401074in}{1.206750in}}%
\pgfusepath{stroke,fill}%
\end{pgfscope}%
\begin{pgfscope}%
\pgfpathrectangle{\pgfqpoint{0.506010in}{1.121191in}}{\pgfqpoint{2.325000in}{1.400000in}} %
\pgfusepath{clip}%
\pgfsetbuttcap%
\pgfsetroundjoin%
\definecolor{currentfill}{rgb}{0.000000,0.000000,1.000000}%
\pgfsetfillcolor{currentfill}%
\pgfsetlinewidth{1.003750pt}%
\definecolor{currentstroke}{rgb}{0.000000,0.000000,1.000000}%
\pgfsetstrokecolor{currentstroke}%
\pgfsetdash{}{0pt}%
\pgfpathmoveto{\pgfqpoint{1.059422in}{1.208570in}}%
\pgfpathlineto{\pgfqpoint{1.103342in}{1.252491in}}%
\pgfpathmoveto{\pgfqpoint{1.059422in}{1.252491in}}%
\pgfpathlineto{\pgfqpoint{1.103342in}{1.208570in}}%
\pgfusepath{stroke,fill}%
\end{pgfscope}%
\begin{pgfscope}%
\pgfpathrectangle{\pgfqpoint{0.506010in}{1.121191in}}{\pgfqpoint{2.325000in}{1.400000in}} %
\pgfusepath{clip}%
\pgfsetbuttcap%
\pgfsetroundjoin%
\definecolor{currentfill}{rgb}{0.000000,0.000000,1.000000}%
\pgfsetfillcolor{currentfill}%
\pgfsetlinewidth{1.003750pt}%
\definecolor{currentstroke}{rgb}{0.000000,0.000000,1.000000}%
\pgfsetstrokecolor{currentstroke}%
\pgfsetdash{}{0pt}%
\pgfpathmoveto{\pgfqpoint{1.087014in}{1.229318in}}%
\pgfpathlineto{\pgfqpoint{1.130934in}{1.273239in}}%
\pgfpathmoveto{\pgfqpoint{1.087014in}{1.273239in}}%
\pgfpathlineto{\pgfqpoint{1.130934in}{1.229318in}}%
\pgfusepath{stroke,fill}%
\end{pgfscope}%
\begin{pgfscope}%
\pgfpathrectangle{\pgfqpoint{0.506010in}{1.121191in}}{\pgfqpoint{2.325000in}{1.400000in}} %
\pgfusepath{clip}%
\pgfsetbuttcap%
\pgfsetroundjoin%
\definecolor{currentfill}{rgb}{0.000000,0.000000,1.000000}%
\pgfsetfillcolor{currentfill}%
\pgfsetlinewidth{1.003750pt}%
\definecolor{currentstroke}{rgb}{0.000000,0.000000,1.000000}%
\pgfsetstrokecolor{currentstroke}%
\pgfsetdash{}{0pt}%
\pgfpathmoveto{\pgfqpoint{0.659295in}{1.231054in}}%
\pgfpathlineto{\pgfqpoint{0.703215in}{1.274975in}}%
\pgfpathmoveto{\pgfqpoint{0.659295in}{1.274975in}}%
\pgfpathlineto{\pgfqpoint{0.703215in}{1.231054in}}%
\pgfusepath{stroke,fill}%
\end{pgfscope}%
\begin{pgfscope}%
\pgfpathrectangle{\pgfqpoint{0.506010in}{1.121191in}}{\pgfqpoint{2.325000in}{1.400000in}} %
\pgfusepath{clip}%
\pgfsetbuttcap%
\pgfsetroundjoin%
\definecolor{currentfill}{rgb}{0.000000,0.000000,1.000000}%
\pgfsetfillcolor{currentfill}%
\pgfsetlinewidth{1.003750pt}%
\definecolor{currentstroke}{rgb}{0.000000,0.000000,1.000000}%
\pgfsetstrokecolor{currentstroke}%
\pgfsetdash{}{0pt}%
\pgfpathmoveto{\pgfqpoint{1.059422in}{1.237074in}}%
\pgfpathlineto{\pgfqpoint{1.103342in}{1.280995in}}%
\pgfpathmoveto{\pgfqpoint{1.059422in}{1.280995in}}%
\pgfpathlineto{\pgfqpoint{1.103342in}{1.237074in}}%
\pgfusepath{stroke,fill}%
\end{pgfscope}%
\begin{pgfscope}%
\pgfpathrectangle{\pgfqpoint{0.506010in}{1.121191in}}{\pgfqpoint{2.325000in}{1.400000in}} %
\pgfusepath{clip}%
\pgfsetbuttcap%
\pgfsetroundjoin%
\definecolor{currentfill}{rgb}{0.000000,0.000000,1.000000}%
\pgfsetfillcolor{currentfill}%
\pgfsetlinewidth{1.003750pt}%
\definecolor{currentstroke}{rgb}{0.000000,0.000000,1.000000}%
\pgfsetstrokecolor{currentstroke}%
\pgfsetdash{}{0pt}%
\pgfpathmoveto{\pgfqpoint{1.059422in}{1.270310in}}%
\pgfpathlineto{\pgfqpoint{1.103342in}{1.314231in}}%
\pgfpathmoveto{\pgfqpoint{1.059422in}{1.314231in}}%
\pgfpathlineto{\pgfqpoint{1.103342in}{1.270310in}}%
\pgfusepath{stroke,fill}%
\end{pgfscope}%
\begin{pgfscope}%
\pgfpathrectangle{\pgfqpoint{0.506010in}{1.121191in}}{\pgfqpoint{2.325000in}{1.400000in}} %
\pgfusepath{clip}%
\pgfsetbuttcap%
\pgfsetroundjoin%
\definecolor{currentfill}{rgb}{0.000000,0.000000,1.000000}%
\pgfsetfillcolor{currentfill}%
\pgfsetlinewidth{1.003750pt}%
\definecolor{currentstroke}{rgb}{0.000000,0.000000,1.000000}%
\pgfsetstrokecolor{currentstroke}%
\pgfsetdash{}{0pt}%
\pgfpathmoveto{\pgfqpoint{0.985862in}{1.274958in}}%
\pgfpathlineto{\pgfqpoint{1.029783in}{1.318879in}}%
\pgfpathmoveto{\pgfqpoint{0.985862in}{1.318879in}}%
\pgfpathlineto{\pgfqpoint{1.029783in}{1.274958in}}%
\pgfusepath{stroke,fill}%
\end{pgfscope}%
\begin{pgfscope}%
\pgfpathrectangle{\pgfqpoint{0.506010in}{1.121191in}}{\pgfqpoint{2.325000in}{1.400000in}} %
\pgfusepath{clip}%
\pgfsetbuttcap%
\pgfsetroundjoin%
\definecolor{currentfill}{rgb}{0.000000,0.000000,1.000000}%
\pgfsetfillcolor{currentfill}%
\pgfsetlinewidth{1.003750pt}%
\definecolor{currentstroke}{rgb}{0.000000,0.000000,1.000000}%
\pgfsetstrokecolor{currentstroke}%
\pgfsetdash{}{0pt}%
\pgfpathmoveto{\pgfqpoint{0.964227in}{1.275434in}}%
\pgfpathlineto{\pgfqpoint{1.008148in}{1.319355in}}%
\pgfpathmoveto{\pgfqpoint{0.964227in}{1.319355in}}%
\pgfpathlineto{\pgfqpoint{1.008148in}{1.275434in}}%
\pgfusepath{stroke,fill}%
\end{pgfscope}%
\begin{pgfscope}%
\pgfpathrectangle{\pgfqpoint{0.506010in}{1.121191in}}{\pgfqpoint{2.325000in}{1.400000in}} %
\pgfusepath{clip}%
\pgfsetbuttcap%
\pgfsetroundjoin%
\definecolor{currentfill}{rgb}{0.000000,0.000000,1.000000}%
\pgfsetfillcolor{currentfill}%
\pgfsetlinewidth{1.003750pt}%
\definecolor{currentstroke}{rgb}{0.000000,0.000000,1.000000}%
\pgfsetstrokecolor{currentstroke}%
\pgfsetdash{}{0pt}%
\pgfpathmoveto{\pgfqpoint{1.087014in}{1.279942in}}%
\pgfpathlineto{\pgfqpoint{1.130934in}{1.323863in}}%
\pgfpathmoveto{\pgfqpoint{1.087014in}{1.323863in}}%
\pgfpathlineto{\pgfqpoint{1.130934in}{1.279942in}}%
\pgfusepath{stroke,fill}%
\end{pgfscope}%
\begin{pgfscope}%
\pgfpathrectangle{\pgfqpoint{0.506010in}{1.121191in}}{\pgfqpoint{2.325000in}{1.400000in}} %
\pgfusepath{clip}%
\pgfsetbuttcap%
\pgfsetroundjoin%
\definecolor{currentfill}{rgb}{0.000000,0.000000,1.000000}%
\pgfsetfillcolor{currentfill}%
\pgfsetlinewidth{1.003750pt}%
\definecolor{currentstroke}{rgb}{0.000000,0.000000,1.000000}%
\pgfsetstrokecolor{currentstroke}%
\pgfsetdash{}{0pt}%
\pgfpathmoveto{\pgfqpoint{1.059422in}{1.295342in}}%
\pgfpathlineto{\pgfqpoint{1.103342in}{1.339263in}}%
\pgfpathmoveto{\pgfqpoint{1.059422in}{1.339263in}}%
\pgfpathlineto{\pgfqpoint{1.103342in}{1.295342in}}%
\pgfusepath{stroke,fill}%
\end{pgfscope}%
\begin{pgfscope}%
\pgfpathrectangle{\pgfqpoint{0.506010in}{1.121191in}}{\pgfqpoint{2.325000in}{1.400000in}} %
\pgfusepath{clip}%
\pgfsetbuttcap%
\pgfsetroundjoin%
\definecolor{currentfill}{rgb}{0.000000,0.000000,1.000000}%
\pgfsetfillcolor{currentfill}%
\pgfsetlinewidth{1.003750pt}%
\definecolor{currentstroke}{rgb}{0.000000,0.000000,1.000000}%
\pgfsetstrokecolor{currentstroke}%
\pgfsetdash{}{0pt}%
\pgfpathmoveto{\pgfqpoint{1.452069in}{1.296602in}}%
\pgfpathlineto{\pgfqpoint{1.495990in}{1.340523in}}%
\pgfpathmoveto{\pgfqpoint{1.452069in}{1.340523in}}%
\pgfpathlineto{\pgfqpoint{1.495990in}{1.296602in}}%
\pgfusepath{stroke,fill}%
\end{pgfscope}%
\begin{pgfscope}%
\pgfpathrectangle{\pgfqpoint{0.506010in}{1.121191in}}{\pgfqpoint{2.325000in}{1.400000in}} %
\pgfusepath{clip}%
\pgfsetbuttcap%
\pgfsetroundjoin%
\definecolor{currentfill}{rgb}{0.000000,0.000000,1.000000}%
\pgfsetfillcolor{currentfill}%
\pgfsetlinewidth{1.003750pt}%
\definecolor{currentstroke}{rgb}{0.000000,0.000000,1.000000}%
\pgfsetstrokecolor{currentstroke}%
\pgfsetdash{}{0pt}%
\pgfpathmoveto{\pgfqpoint{0.985862in}{1.361002in}}%
\pgfpathlineto{\pgfqpoint{1.029783in}{1.404923in}}%
\pgfpathmoveto{\pgfqpoint{0.985862in}{1.404923in}}%
\pgfpathlineto{\pgfqpoint{1.029783in}{1.361002in}}%
\pgfusepath{stroke,fill}%
\end{pgfscope}%
\begin{pgfscope}%
\pgfpathrectangle{\pgfqpoint{0.506010in}{1.121191in}}{\pgfqpoint{2.325000in}{1.400000in}} %
\pgfusepath{clip}%
\pgfsetbuttcap%
\pgfsetroundjoin%
\definecolor{currentfill}{rgb}{0.000000,0.000000,1.000000}%
\pgfsetfillcolor{currentfill}%
\pgfsetlinewidth{1.003750pt}%
\definecolor{currentstroke}{rgb}{0.000000,0.000000,1.000000}%
\pgfsetstrokecolor{currentstroke}%
\pgfsetdash{}{0pt}%
\pgfpathmoveto{\pgfqpoint{1.059422in}{1.404150in}}%
\pgfpathlineto{\pgfqpoint{1.103342in}{1.448071in}}%
\pgfpathmoveto{\pgfqpoint{1.059422in}{1.448071in}}%
\pgfpathlineto{\pgfqpoint{1.103342in}{1.404150in}}%
\pgfusepath{stroke,fill}%
\end{pgfscope}%
\begin{pgfscope}%
\pgfpathrectangle{\pgfqpoint{0.506010in}{1.121191in}}{\pgfqpoint{2.325000in}{1.400000in}} %
\pgfusepath{clip}%
\pgfsetbuttcap%
\pgfsetroundjoin%
\definecolor{currentfill}{rgb}{0.000000,0.000000,1.000000}%
\pgfsetfillcolor{currentfill}%
\pgfsetlinewidth{1.003750pt}%
\definecolor{currentstroke}{rgb}{0.000000,0.000000,1.000000}%
\pgfsetstrokecolor{currentstroke}%
\pgfsetdash{}{0pt}%
\pgfpathmoveto{\pgfqpoint{1.059422in}{1.460542in}}%
\pgfpathlineto{\pgfqpoint{1.103342in}{1.504463in}}%
\pgfpathmoveto{\pgfqpoint{1.059422in}{1.504463in}}%
\pgfpathlineto{\pgfqpoint{1.103342in}{1.460542in}}%
\pgfusepath{stroke,fill}%
\end{pgfscope}%
\begin{pgfscope}%
\pgfpathrectangle{\pgfqpoint{0.506010in}{1.121191in}}{\pgfqpoint{2.325000in}{1.400000in}} %
\pgfusepath{clip}%
\pgfsetbuttcap%
\pgfsetroundjoin%
\definecolor{currentfill}{rgb}{0.000000,0.000000,1.000000}%
\pgfsetfillcolor{currentfill}%
\pgfsetlinewidth{1.003750pt}%
\definecolor{currentstroke}{rgb}{0.000000,0.000000,1.000000}%
\pgfsetstrokecolor{currentstroke}%
\pgfsetdash{}{0pt}%
\pgfpathmoveto{\pgfqpoint{1.059422in}{1.505566in}}%
\pgfpathlineto{\pgfqpoint{1.103342in}{1.549487in}}%
\pgfpathmoveto{\pgfqpoint{1.059422in}{1.549487in}}%
\pgfpathlineto{\pgfqpoint{1.103342in}{1.505566in}}%
\pgfusepath{stroke,fill}%
\end{pgfscope}%
\begin{pgfscope}%
\pgfpathrectangle{\pgfqpoint{0.506010in}{1.121191in}}{\pgfqpoint{2.325000in}{1.400000in}} %
\pgfusepath{clip}%
\pgfsetbuttcap%
\pgfsetroundjoin%
\definecolor{currentfill}{rgb}{0.000000,0.000000,1.000000}%
\pgfsetfillcolor{currentfill}%
\pgfsetlinewidth{1.003750pt}%
\definecolor{currentstroke}{rgb}{0.000000,0.000000,1.000000}%
\pgfsetstrokecolor{currentstroke}%
\pgfsetdash{}{0pt}%
\pgfpathmoveto{\pgfqpoint{0.985862in}{1.614290in}}%
\pgfpathlineto{\pgfqpoint{1.029783in}{1.658211in}}%
\pgfpathmoveto{\pgfqpoint{0.985862in}{1.658211in}}%
\pgfpathlineto{\pgfqpoint{1.029783in}{1.614290in}}%
\pgfusepath{stroke,fill}%
\end{pgfscope}%
\begin{pgfscope}%
\pgfpathrectangle{\pgfqpoint{0.506010in}{1.121191in}}{\pgfqpoint{2.325000in}{1.400000in}} %
\pgfusepath{clip}%
\pgfsetbuttcap%
\pgfsetroundjoin%
\definecolor{currentfill}{rgb}{0.000000,0.000000,1.000000}%
\pgfsetfillcolor{currentfill}%
\pgfsetlinewidth{1.003750pt}%
\definecolor{currentstroke}{rgb}{0.000000,0.000000,1.000000}%
\pgfsetstrokecolor{currentstroke}%
\pgfsetdash{}{0pt}%
\pgfpathmoveto{\pgfqpoint{0.985862in}{1.623950in}}%
\pgfpathlineto{\pgfqpoint{1.029783in}{1.667871in}}%
\pgfpathmoveto{\pgfqpoint{0.985862in}{1.667871in}}%
\pgfpathlineto{\pgfqpoint{1.029783in}{1.623950in}}%
\pgfusepath{stroke,fill}%
\end{pgfscope}%
\begin{pgfscope}%
\pgfpathrectangle{\pgfqpoint{0.506010in}{1.121191in}}{\pgfqpoint{2.325000in}{1.400000in}} %
\pgfusepath{clip}%
\pgfsetbuttcap%
\pgfsetroundjoin%
\definecolor{currentfill}{rgb}{0.000000,0.000000,1.000000}%
\pgfsetfillcolor{currentfill}%
\pgfsetlinewidth{1.003750pt}%
\definecolor{currentstroke}{rgb}{0.000000,0.000000,1.000000}%
\pgfsetstrokecolor{currentstroke}%
\pgfsetdash{}{0pt}%
\pgfpathmoveto{\pgfqpoint{1.087014in}{1.744406in}}%
\pgfpathlineto{\pgfqpoint{1.130934in}{1.788327in}}%
\pgfpathmoveto{\pgfqpoint{1.087014in}{1.788327in}}%
\pgfpathlineto{\pgfqpoint{1.130934in}{1.744406in}}%
\pgfusepath{stroke,fill}%
\end{pgfscope}%
\begin{pgfscope}%
\pgfpathrectangle{\pgfqpoint{0.506010in}{1.121191in}}{\pgfqpoint{2.325000in}{1.400000in}} %
\pgfusepath{clip}%
\pgfsetbuttcap%
\pgfsetroundjoin%
\definecolor{currentfill}{rgb}{0.000000,0.000000,1.000000}%
\pgfsetfillcolor{currentfill}%
\pgfsetlinewidth{1.003750pt}%
\definecolor{currentstroke}{rgb}{0.000000,0.000000,1.000000}%
\pgfsetstrokecolor{currentstroke}%
\pgfsetdash{}{0pt}%
\pgfpathmoveto{\pgfqpoint{1.452069in}{6.142086in}}%
\pgfpathmoveto{\pgfqpoint{1.452069in}{6.186007in}}%
\pgfusepath{stroke,fill}%
\end{pgfscope}%
\begin{pgfscope}%
\pgfpathrectangle{\pgfqpoint{0.506010in}{1.121191in}}{\pgfqpoint{2.325000in}{1.400000in}} %
\pgfusepath{clip}%
\pgfsetbuttcap%
\pgfsetroundjoin%
\definecolor{currentfill}{rgb}{0.000000,0.000000,1.000000}%
\pgfsetfillcolor{currentfill}%
\pgfsetlinewidth{1.003750pt}%
\definecolor{currentstroke}{rgb}{0.000000,0.000000,1.000000}%
\pgfsetstrokecolor{currentstroke}%
\pgfsetdash{}{0pt}%
\pgfpathmoveto{\pgfqpoint{1.059422in}{6.142114in}}%
\pgfpathmoveto{\pgfqpoint{1.059422in}{6.186035in}}%
\pgfusepath{stroke,fill}%
\end{pgfscope}%
\begin{pgfscope}%
\pgfpathrectangle{\pgfqpoint{0.506010in}{1.121191in}}{\pgfqpoint{2.325000in}{1.400000in}} %
\pgfusepath{clip}%
\pgfsetbuttcap%
\pgfsetroundjoin%
\definecolor{currentfill}{rgb}{0.000000,0.000000,1.000000}%
\pgfsetfillcolor{currentfill}%
\pgfsetlinewidth{1.003750pt}%
\definecolor{currentstroke}{rgb}{0.000000,0.000000,1.000000}%
\pgfsetstrokecolor{currentstroke}%
\pgfsetdash{}{0pt}%
\pgfpathmoveto{\pgfqpoint{1.059422in}{6.142114in}}%
\pgfpathmoveto{\pgfqpoint{1.059422in}{6.186035in}}%
\pgfusepath{stroke,fill}%
\end{pgfscope}%
\begin{pgfscope}%
\pgfpathrectangle{\pgfqpoint{0.506010in}{1.121191in}}{\pgfqpoint{2.325000in}{1.400000in}} %
\pgfusepath{clip}%
\pgfsetbuttcap%
\pgfsetroundjoin%
\definecolor{currentfill}{rgb}{0.000000,0.000000,1.000000}%
\pgfsetfillcolor{currentfill}%
\pgfsetlinewidth{1.003750pt}%
\definecolor{currentstroke}{rgb}{0.000000,0.000000,1.000000}%
\pgfsetstrokecolor{currentstroke}%
\pgfsetdash{}{0pt}%
\pgfpathmoveto{\pgfqpoint{0.985862in}{6.142114in}}%
\pgfpathmoveto{\pgfqpoint{0.985862in}{6.186035in}}%
\pgfusepath{stroke,fill}%
\end{pgfscope}%
\begin{pgfscope}%
\pgfpathrectangle{\pgfqpoint{0.506010in}{1.121191in}}{\pgfqpoint{2.325000in}{1.400000in}} %
\pgfusepath{clip}%
\pgfsetbuttcap%
\pgfsetroundjoin%
\definecolor{currentfill}{rgb}{0.000000,0.000000,1.000000}%
\pgfsetfillcolor{currentfill}%
\pgfsetlinewidth{1.003750pt}%
\definecolor{currentstroke}{rgb}{0.000000,0.000000,1.000000}%
\pgfsetstrokecolor{currentstroke}%
\pgfsetdash{}{0pt}%
\pgfpathmoveto{\pgfqpoint{1.087014in}{6.142114in}}%
\pgfpathmoveto{\pgfqpoint{1.087014in}{6.186035in}}%
\pgfusepath{stroke,fill}%
\end{pgfscope}%
\begin{pgfscope}%
\pgfpathrectangle{\pgfqpoint{0.506010in}{1.121191in}}{\pgfqpoint{2.325000in}{1.400000in}} %
\pgfusepath{clip}%
\pgfsetbuttcap%
\pgfsetroundjoin%
\definecolor{currentfill}{rgb}{0.000000,0.000000,1.000000}%
\pgfsetfillcolor{currentfill}%
\pgfsetlinewidth{1.003750pt}%
\definecolor{currentstroke}{rgb}{0.000000,0.000000,1.000000}%
\pgfsetstrokecolor{currentstroke}%
\pgfsetdash{}{0pt}%
\pgfpathmoveto{\pgfqpoint{1.059422in}{6.142114in}}%
\pgfpathmoveto{\pgfqpoint{1.059422in}{6.186035in}}%
\pgfusepath{stroke,fill}%
\end{pgfscope}%
\begin{pgfscope}%
\pgfpathrectangle{\pgfqpoint{0.506010in}{1.121191in}}{\pgfqpoint{2.325000in}{1.400000in}} %
\pgfusepath{clip}%
\pgfsetbuttcap%
\pgfsetroundjoin%
\definecolor{currentfill}{rgb}{0.000000,0.000000,1.000000}%
\pgfsetfillcolor{currentfill}%
\pgfsetlinewidth{1.003750pt}%
\definecolor{currentstroke}{rgb}{0.000000,0.000000,1.000000}%
\pgfsetstrokecolor{currentstroke}%
\pgfsetdash{}{0pt}%
\pgfpathmoveto{\pgfqpoint{0.985862in}{6.142114in}}%
\pgfpathmoveto{\pgfqpoint{0.985862in}{6.186035in}}%
\pgfusepath{stroke,fill}%
\end{pgfscope}%
\begin{pgfscope}%
\pgfpathrectangle{\pgfqpoint{0.506010in}{1.121191in}}{\pgfqpoint{2.325000in}{1.400000in}} %
\pgfusepath{clip}%
\pgfsetbuttcap%
\pgfsetroundjoin%
\definecolor{currentfill}{rgb}{0.000000,0.000000,1.000000}%
\pgfsetfillcolor{currentfill}%
\pgfsetlinewidth{1.003750pt}%
\definecolor{currentstroke}{rgb}{0.000000,0.000000,1.000000}%
\pgfsetstrokecolor{currentstroke}%
\pgfsetdash{}{0pt}%
\pgfpathmoveto{\pgfqpoint{1.059422in}{6.142170in}}%
\pgfpathmoveto{\pgfqpoint{1.059422in}{6.186091in}}%
\pgfusepath{stroke,fill}%
\end{pgfscope}%
\begin{pgfscope}%
\pgfpathrectangle{\pgfqpoint{0.506010in}{1.121191in}}{\pgfqpoint{2.325000in}{1.400000in}} %
\pgfusepath{clip}%
\pgfsetbuttcap%
\pgfsetroundjoin%
\definecolor{currentfill}{rgb}{0.000000,0.000000,1.000000}%
\pgfsetfillcolor{currentfill}%
\pgfsetlinewidth{1.003750pt}%
\definecolor{currentstroke}{rgb}{0.000000,0.000000,1.000000}%
\pgfsetstrokecolor{currentstroke}%
\pgfsetdash{}{0pt}%
\pgfpathmoveto{\pgfqpoint{1.059422in}{6.142170in}}%
\pgfpathmoveto{\pgfqpoint{1.059422in}{6.186091in}}%
\pgfusepath{stroke,fill}%
\end{pgfscope}%
\begin{pgfscope}%
\pgfpathrectangle{\pgfqpoint{0.506010in}{1.121191in}}{\pgfqpoint{2.325000in}{1.400000in}} %
\pgfusepath{clip}%
\pgfsetbuttcap%
\pgfsetroundjoin%
\definecolor{currentfill}{rgb}{0.000000,0.000000,1.000000}%
\pgfsetfillcolor{currentfill}%
\pgfsetlinewidth{1.003750pt}%
\definecolor{currentstroke}{rgb}{0.000000,0.000000,1.000000}%
\pgfsetstrokecolor{currentstroke}%
\pgfsetdash{}{0pt}%
\pgfpathmoveto{\pgfqpoint{1.059422in}{6.142170in}}%
\pgfpathmoveto{\pgfqpoint{1.059422in}{6.186091in}}%
\pgfusepath{stroke,fill}%
\end{pgfscope}%
\begin{pgfscope}%
\pgfpathrectangle{\pgfqpoint{0.506010in}{1.121191in}}{\pgfqpoint{2.325000in}{1.400000in}} %
\pgfusepath{clip}%
\pgfsetbuttcap%
\pgfsetroundjoin%
\definecolor{currentfill}{rgb}{0.000000,0.000000,1.000000}%
\pgfsetfillcolor{currentfill}%
\pgfsetlinewidth{1.003750pt}%
\definecolor{currentstroke}{rgb}{0.000000,0.000000,1.000000}%
\pgfsetstrokecolor{currentstroke}%
\pgfsetdash{}{0pt}%
\pgfpathmoveto{\pgfqpoint{1.059422in}{6.142170in}}%
\pgfpathmoveto{\pgfqpoint{1.059422in}{6.186091in}}%
\pgfusepath{stroke,fill}%
\end{pgfscope}%
\begin{pgfscope}%
\pgfpathrectangle{\pgfqpoint{0.506010in}{1.121191in}}{\pgfqpoint{2.325000in}{1.400000in}} %
\pgfusepath{clip}%
\pgfsetbuttcap%
\pgfsetroundjoin%
\definecolor{currentfill}{rgb}{0.000000,0.000000,1.000000}%
\pgfsetfillcolor{currentfill}%
\pgfsetlinewidth{1.003750pt}%
\definecolor{currentstroke}{rgb}{0.000000,0.000000,1.000000}%
\pgfsetstrokecolor{currentstroke}%
\pgfsetdash{}{0pt}%
\pgfpathmoveto{\pgfqpoint{0.985862in}{6.142198in}}%
\pgfpathmoveto{\pgfqpoint{0.985862in}{6.186119in}}%
\pgfusepath{stroke,fill}%
\end{pgfscope}%
\begin{pgfscope}%
\pgfpathrectangle{\pgfqpoint{0.506010in}{1.121191in}}{\pgfqpoint{2.325000in}{1.400000in}} %
\pgfusepath{clip}%
\pgfsetbuttcap%
\pgfsetroundjoin%
\definecolor{currentfill}{rgb}{0.000000,0.000000,1.000000}%
\pgfsetfillcolor{currentfill}%
\pgfsetlinewidth{1.003750pt}%
\definecolor{currentstroke}{rgb}{0.000000,0.000000,1.000000}%
\pgfsetstrokecolor{currentstroke}%
\pgfsetdash{}{0pt}%
\pgfpathmoveto{\pgfqpoint{0.766715in}{6.142198in}}%
\pgfpathmoveto{\pgfqpoint{0.766715in}{6.186119in}}%
\pgfusepath{stroke,fill}%
\end{pgfscope}%
\begin{pgfscope}%
\pgfpathrectangle{\pgfqpoint{0.506010in}{1.121191in}}{\pgfqpoint{2.325000in}{1.400000in}} %
\pgfusepath{clip}%
\pgfsetbuttcap%
\pgfsetroundjoin%
\definecolor{currentfill}{rgb}{0.000000,0.000000,1.000000}%
\pgfsetfillcolor{currentfill}%
\pgfsetlinewidth{1.003750pt}%
\definecolor{currentstroke}{rgb}{0.000000,0.000000,1.000000}%
\pgfsetstrokecolor{currentstroke}%
\pgfsetdash{}{0pt}%
\pgfpathmoveto{\pgfqpoint{2.458339in}{6.142198in}}%
\pgfpathmoveto{\pgfqpoint{2.458339in}{6.186119in}}%
\pgfusepath{stroke,fill}%
\end{pgfscope}%
\begin{pgfscope}%
\pgfpathrectangle{\pgfqpoint{0.506010in}{1.121191in}}{\pgfqpoint{2.325000in}{1.400000in}} %
\pgfusepath{clip}%
\pgfsetbuttcap%
\pgfsetroundjoin%
\definecolor{currentfill}{rgb}{0.000000,0.000000,1.000000}%
\pgfsetfillcolor{currentfill}%
\pgfsetlinewidth{1.003750pt}%
\definecolor{currentstroke}{rgb}{0.000000,0.000000,1.000000}%
\pgfsetstrokecolor{currentstroke}%
\pgfsetdash{}{0pt}%
\pgfpathmoveto{\pgfqpoint{1.117261in}{6.142254in}}%
\pgfpathmoveto{\pgfqpoint{1.117261in}{6.186175in}}%
\pgfusepath{stroke,fill}%
\end{pgfscope}%
\begin{pgfscope}%
\pgfpathrectangle{\pgfqpoint{0.506010in}{1.121191in}}{\pgfqpoint{2.325000in}{1.400000in}} %
\pgfusepath{clip}%
\pgfsetbuttcap%
\pgfsetroundjoin%
\definecolor{currentfill}{rgb}{0.000000,0.000000,1.000000}%
\pgfsetfillcolor{currentfill}%
\pgfsetlinewidth{1.003750pt}%
\definecolor{currentstroke}{rgb}{0.000000,0.000000,1.000000}%
\pgfsetstrokecolor{currentstroke}%
\pgfsetdash{}{0pt}%
\pgfpathmoveto{\pgfqpoint{0.985862in}{6.142254in}}%
\pgfpathmoveto{\pgfqpoint{0.985862in}{6.186175in}}%
\pgfusepath{stroke,fill}%
\end{pgfscope}%
\begin{pgfscope}%
\pgfpathrectangle{\pgfqpoint{0.506010in}{1.121191in}}{\pgfqpoint{2.325000in}{1.400000in}} %
\pgfusepath{clip}%
\pgfsetbuttcap%
\pgfsetroundjoin%
\definecolor{currentfill}{rgb}{0.000000,0.000000,1.000000}%
\pgfsetfillcolor{currentfill}%
\pgfsetlinewidth{1.003750pt}%
\definecolor{currentstroke}{rgb}{0.000000,0.000000,1.000000}%
\pgfsetstrokecolor{currentstroke}%
\pgfsetdash{}{0pt}%
\pgfpathmoveto{\pgfqpoint{0.985862in}{6.142254in}}%
\pgfpathmoveto{\pgfqpoint{0.985862in}{6.186175in}}%
\pgfusepath{stroke,fill}%
\end{pgfscope}%
\begin{pgfscope}%
\pgfpathrectangle{\pgfqpoint{0.506010in}{1.121191in}}{\pgfqpoint{2.325000in}{1.400000in}} %
\pgfusepath{clip}%
\pgfsetbuttcap%
\pgfsetroundjoin%
\definecolor{currentfill}{rgb}{0.000000,0.000000,1.000000}%
\pgfsetfillcolor{currentfill}%
\pgfsetlinewidth{1.003750pt}%
\definecolor{currentstroke}{rgb}{0.000000,0.000000,1.000000}%
\pgfsetstrokecolor{currentstroke}%
\pgfsetdash{}{0pt}%
\pgfpathmoveto{\pgfqpoint{0.985862in}{6.142254in}}%
\pgfpathmoveto{\pgfqpoint{0.985862in}{6.186175in}}%
\pgfusepath{stroke,fill}%
\end{pgfscope}%
\begin{pgfscope}%
\pgfpathrectangle{\pgfqpoint{0.506010in}{1.121191in}}{\pgfqpoint{2.325000in}{1.400000in}} %
\pgfusepath{clip}%
\pgfsetbuttcap%
\pgfsetroundjoin%
\definecolor{currentfill}{rgb}{0.000000,0.000000,1.000000}%
\pgfsetfillcolor{currentfill}%
\pgfsetlinewidth{1.003750pt}%
\definecolor{currentstroke}{rgb}{0.000000,0.000000,1.000000}%
\pgfsetstrokecolor{currentstroke}%
\pgfsetdash{}{0pt}%
\pgfpathmoveto{\pgfqpoint{1.087014in}{6.142254in}}%
\pgfpathmoveto{\pgfqpoint{1.087014in}{6.186175in}}%
\pgfusepath{stroke,fill}%
\end{pgfscope}%
\begin{pgfscope}%
\pgfpathrectangle{\pgfqpoint{0.506010in}{1.121191in}}{\pgfqpoint{2.325000in}{1.400000in}} %
\pgfusepath{clip}%
\pgfsetbuttcap%
\pgfsetroundjoin%
\definecolor{currentfill}{rgb}{0.000000,0.000000,1.000000}%
\pgfsetfillcolor{currentfill}%
\pgfsetlinewidth{1.003750pt}%
\definecolor{currentstroke}{rgb}{0.000000,0.000000,1.000000}%
\pgfsetstrokecolor{currentstroke}%
\pgfsetdash{}{0pt}%
\pgfpathmoveto{\pgfqpoint{1.087014in}{6.142282in}}%
\pgfpathmoveto{\pgfqpoint{1.087014in}{6.186203in}}%
\pgfusepath{stroke,fill}%
\end{pgfscope}%
\begin{pgfscope}%
\pgfpathrectangle{\pgfqpoint{0.506010in}{1.121191in}}{\pgfqpoint{2.325000in}{1.400000in}} %
\pgfusepath{clip}%
\pgfsetbuttcap%
\pgfsetroundjoin%
\definecolor{currentfill}{rgb}{0.000000,0.000000,1.000000}%
\pgfsetfillcolor{currentfill}%
\pgfsetlinewidth{1.003750pt}%
\definecolor{currentstroke}{rgb}{0.000000,0.000000,1.000000}%
\pgfsetstrokecolor{currentstroke}%
\pgfsetdash{}{0pt}%
\pgfpathmoveto{\pgfqpoint{1.059422in}{6.142282in}}%
\pgfpathmoveto{\pgfqpoint{1.059422in}{6.186203in}}%
\pgfusepath{stroke,fill}%
\end{pgfscope}%
\begin{pgfscope}%
\pgfpathrectangle{\pgfqpoint{0.506010in}{1.121191in}}{\pgfqpoint{2.325000in}{1.400000in}} %
\pgfusepath{clip}%
\pgfsetbuttcap%
\pgfsetroundjoin%
\definecolor{currentfill}{rgb}{0.000000,0.000000,1.000000}%
\pgfsetfillcolor{currentfill}%
\pgfsetlinewidth{1.003750pt}%
\definecolor{currentstroke}{rgb}{0.000000,0.000000,1.000000}%
\pgfsetstrokecolor{currentstroke}%
\pgfsetdash{}{0pt}%
\pgfpathmoveto{\pgfqpoint{0.798599in}{6.142310in}}%
\pgfpathmoveto{\pgfqpoint{0.798599in}{6.186231in}}%
\pgfusepath{stroke,fill}%
\end{pgfscope}%
\begin{pgfscope}%
\pgfpathrectangle{\pgfqpoint{0.506010in}{1.121191in}}{\pgfqpoint{2.325000in}{1.400000in}} %
\pgfusepath{clip}%
\pgfsetbuttcap%
\pgfsetroundjoin%
\definecolor{currentfill}{rgb}{0.000000,0.000000,1.000000}%
\pgfsetfillcolor{currentfill}%
\pgfsetlinewidth{1.003750pt}%
\definecolor{currentstroke}{rgb}{0.000000,0.000000,1.000000}%
\pgfsetstrokecolor{currentstroke}%
\pgfsetdash{}{0pt}%
\pgfpathmoveto{\pgfqpoint{1.145583in}{6.142310in}}%
\pgfpathmoveto{\pgfqpoint{1.145583in}{6.186231in}}%
\pgfusepath{stroke,fill}%
\end{pgfscope}%
\begin{pgfscope}%
\pgfpathrectangle{\pgfqpoint{0.506010in}{1.121191in}}{\pgfqpoint{2.325000in}{1.400000in}} %
\pgfusepath{clip}%
\pgfsetbuttcap%
\pgfsetroundjoin%
\definecolor{currentfill}{rgb}{0.000000,0.000000,1.000000}%
\pgfsetfillcolor{currentfill}%
\pgfsetlinewidth{1.003750pt}%
\definecolor{currentstroke}{rgb}{0.000000,0.000000,1.000000}%
\pgfsetstrokecolor{currentstroke}%
\pgfsetdash{}{0pt}%
\pgfpathmoveto{\pgfqpoint{1.117261in}{6.142394in}}%
\pgfpathmoveto{\pgfqpoint{1.117261in}{6.186315in}}%
\pgfusepath{stroke,fill}%
\end{pgfscope}%
\begin{pgfscope}%
\pgfpathrectangle{\pgfqpoint{0.506010in}{1.121191in}}{\pgfqpoint{2.325000in}{1.400000in}} %
\pgfusepath{clip}%
\pgfsetbuttcap%
\pgfsetroundjoin%
\definecolor{currentfill}{rgb}{0.000000,0.000000,1.000000}%
\pgfsetfillcolor{currentfill}%
\pgfsetlinewidth{1.003750pt}%
\definecolor{currentstroke}{rgb}{0.000000,0.000000,1.000000}%
\pgfsetstrokecolor{currentstroke}%
\pgfsetdash{}{0pt}%
\pgfpathmoveto{\pgfqpoint{2.169106in}{6.142394in}}%
\pgfpathmoveto{\pgfqpoint{2.169106in}{6.186315in}}%
\pgfusepath{stroke,fill}%
\end{pgfscope}%
\begin{pgfscope}%
\pgfpathrectangle{\pgfqpoint{0.506010in}{1.121191in}}{\pgfqpoint{2.325000in}{1.400000in}} %
\pgfusepath{clip}%
\pgfsetbuttcap%
\pgfsetroundjoin%
\definecolor{currentfill}{rgb}{0.000000,0.000000,1.000000}%
\pgfsetfillcolor{currentfill}%
\pgfsetlinewidth{1.003750pt}%
\definecolor{currentstroke}{rgb}{0.000000,0.000000,1.000000}%
\pgfsetstrokecolor{currentstroke}%
\pgfsetdash{}{0pt}%
\pgfpathmoveto{\pgfqpoint{2.243404in}{6.142394in}}%
\pgfpathmoveto{\pgfqpoint{2.243404in}{6.186315in}}%
\pgfusepath{stroke,fill}%
\end{pgfscope}%
\begin{pgfscope}%
\pgfpathrectangle{\pgfqpoint{0.506010in}{1.121191in}}{\pgfqpoint{2.325000in}{1.400000in}} %
\pgfusepath{clip}%
\pgfsetbuttcap%
\pgfsetroundjoin%
\definecolor{currentfill}{rgb}{0.000000,0.000000,1.000000}%
\pgfsetfillcolor{currentfill}%
\pgfsetlinewidth{1.003750pt}%
\definecolor{currentstroke}{rgb}{0.000000,0.000000,1.000000}%
\pgfsetstrokecolor{currentstroke}%
\pgfsetdash{}{0pt}%
\pgfpathmoveto{\pgfqpoint{2.163753in}{6.142394in}}%
\pgfpathmoveto{\pgfqpoint{2.163753in}{6.186315in}}%
\pgfusepath{stroke,fill}%
\end{pgfscope}%
\begin{pgfscope}%
\pgfpathrectangle{\pgfqpoint{0.506010in}{1.121191in}}{\pgfqpoint{2.325000in}{1.400000in}} %
\pgfusepath{clip}%
\pgfsetbuttcap%
\pgfsetroundjoin%
\definecolor{currentfill}{rgb}{0.000000,0.000000,1.000000}%
\pgfsetfillcolor{currentfill}%
\pgfsetlinewidth{1.003750pt}%
\definecolor{currentstroke}{rgb}{0.000000,0.000000,1.000000}%
\pgfsetstrokecolor{currentstroke}%
\pgfsetdash{}{0pt}%
\pgfpathmoveto{\pgfqpoint{2.195200in}{6.142394in}}%
\pgfpathmoveto{\pgfqpoint{2.195200in}{6.186315in}}%
\pgfusepath{stroke,fill}%
\end{pgfscope}%
\begin{pgfscope}%
\pgfpathrectangle{\pgfqpoint{0.506010in}{1.121191in}}{\pgfqpoint{2.325000in}{1.400000in}} %
\pgfusepath{clip}%
\pgfsetbuttcap%
\pgfsetroundjoin%
\definecolor{currentfill}{rgb}{0.000000,0.000000,1.000000}%
\pgfsetfillcolor{currentfill}%
\pgfsetlinewidth{1.003750pt}%
\definecolor{currentstroke}{rgb}{0.000000,0.000000,1.000000}%
\pgfsetstrokecolor{currentstroke}%
\pgfsetdash{}{0pt}%
\pgfpathmoveto{\pgfqpoint{2.240221in}{6.142422in}}%
\pgfpathmoveto{\pgfqpoint{2.240221in}{6.186343in}}%
\pgfusepath{stroke,fill}%
\end{pgfscope}%
\begin{pgfscope}%
\pgfpathrectangle{\pgfqpoint{0.506010in}{1.121191in}}{\pgfqpoint{2.325000in}{1.400000in}} %
\pgfusepath{clip}%
\pgfsetbuttcap%
\pgfsetroundjoin%
\definecolor{currentfill}{rgb}{0.000000,0.000000,1.000000}%
\pgfsetfillcolor{currentfill}%
\pgfsetlinewidth{1.003750pt}%
\definecolor{currentstroke}{rgb}{0.000000,0.000000,1.000000}%
\pgfsetstrokecolor{currentstroke}%
\pgfsetdash{}{0pt}%
\pgfpathmoveto{\pgfqpoint{2.194142in}{6.142422in}}%
\pgfpathmoveto{\pgfqpoint{2.194142in}{6.186343in}}%
\pgfusepath{stroke,fill}%
\end{pgfscope}%
\begin{pgfscope}%
\pgfpathrectangle{\pgfqpoint{0.506010in}{1.121191in}}{\pgfqpoint{2.325000in}{1.400000in}} %
\pgfusepath{clip}%
\pgfsetbuttcap%
\pgfsetroundjoin%
\definecolor{currentfill}{rgb}{0.000000,0.000000,1.000000}%
\pgfsetfillcolor{currentfill}%
\pgfsetlinewidth{1.003750pt}%
\definecolor{currentstroke}{rgb}{0.000000,0.000000,1.000000}%
\pgfsetstrokecolor{currentstroke}%
\pgfsetdash{}{0pt}%
\pgfpathmoveto{\pgfqpoint{2.212771in}{6.142422in}}%
\pgfpathmoveto{\pgfqpoint{2.212771in}{6.186343in}}%
\pgfusepath{stroke,fill}%
\end{pgfscope}%
\begin{pgfscope}%
\pgfpathrectangle{\pgfqpoint{0.506010in}{1.121191in}}{\pgfqpoint{2.325000in}{1.400000in}} %
\pgfusepath{clip}%
\pgfsetbuttcap%
\pgfsetroundjoin%
\definecolor{currentfill}{rgb}{0.000000,0.000000,1.000000}%
\pgfsetfillcolor{currentfill}%
\pgfsetlinewidth{1.003750pt}%
\definecolor{currentstroke}{rgb}{0.000000,0.000000,1.000000}%
\pgfsetstrokecolor{currentstroke}%
\pgfsetdash{}{0pt}%
\pgfpathmoveto{\pgfqpoint{2.179176in}{6.142422in}}%
\pgfpathmoveto{\pgfqpoint{2.179176in}{6.186343in}}%
\pgfusepath{stroke,fill}%
\end{pgfscope}%
\begin{pgfscope}%
\pgfpathrectangle{\pgfqpoint{0.506010in}{1.121191in}}{\pgfqpoint{2.325000in}{1.400000in}} %
\pgfusepath{clip}%
\pgfsetbuttcap%
\pgfsetroundjoin%
\definecolor{currentfill}{rgb}{0.000000,0.000000,1.000000}%
\pgfsetfillcolor{currentfill}%
\pgfsetlinewidth{1.003750pt}%
\definecolor{currentstroke}{rgb}{0.000000,0.000000,1.000000}%
\pgfsetstrokecolor{currentstroke}%
\pgfsetdash{}{0pt}%
\pgfpathmoveto{\pgfqpoint{2.211413in}{6.142450in}}%
\pgfpathmoveto{\pgfqpoint{2.211413in}{6.186371in}}%
\pgfusepath{stroke,fill}%
\end{pgfscope}%
\begin{pgfscope}%
\pgfpathrectangle{\pgfqpoint{0.506010in}{1.121191in}}{\pgfqpoint{2.325000in}{1.400000in}} %
\pgfusepath{clip}%
\pgfsetbuttcap%
\pgfsetroundjoin%
\definecolor{currentfill}{rgb}{0.000000,0.000000,1.000000}%
\pgfsetfillcolor{currentfill}%
\pgfsetlinewidth{1.003750pt}%
\definecolor{currentstroke}{rgb}{0.000000,0.000000,1.000000}%
\pgfsetstrokecolor{currentstroke}%
\pgfsetdash{}{0pt}%
\pgfpathmoveto{\pgfqpoint{2.200818in}{6.142450in}}%
\pgfpathmoveto{\pgfqpoint{2.200818in}{6.186371in}}%
\pgfusepath{stroke,fill}%
\end{pgfscope}%
\begin{pgfscope}%
\pgfpathrectangle{\pgfqpoint{0.506010in}{1.121191in}}{\pgfqpoint{2.325000in}{1.400000in}} %
\pgfusepath{clip}%
\pgfsetbuttcap%
\pgfsetroundjoin%
\definecolor{currentfill}{rgb}{0.000000,0.000000,1.000000}%
\pgfsetfillcolor{currentfill}%
\pgfsetlinewidth{1.003750pt}%
\definecolor{currentstroke}{rgb}{0.000000,0.000000,1.000000}%
\pgfsetstrokecolor{currentstroke}%
\pgfsetdash{}{0pt}%
\pgfpathmoveto{\pgfqpoint{2.226962in}{6.142478in}}%
\pgfpathmoveto{\pgfqpoint{2.226962in}{6.186399in}}%
\pgfusepath{stroke,fill}%
\end{pgfscope}%
\begin{pgfscope}%
\pgfpathrectangle{\pgfqpoint{0.506010in}{1.121191in}}{\pgfqpoint{2.325000in}{1.400000in}} %
\pgfusepath{clip}%
\pgfsetbuttcap%
\pgfsetroundjoin%
\definecolor{currentfill}{rgb}{0.000000,0.000000,1.000000}%
\pgfsetfillcolor{currentfill}%
\pgfsetlinewidth{1.003750pt}%
\definecolor{currentstroke}{rgb}{0.000000,0.000000,1.000000}%
\pgfsetstrokecolor{currentstroke}%
\pgfsetdash{}{0pt}%
\pgfpathmoveto{\pgfqpoint{2.224604in}{6.142478in}}%
\pgfpathmoveto{\pgfqpoint{2.224604in}{6.186399in}}%
\pgfusepath{stroke,fill}%
\end{pgfscope}%
\begin{pgfscope}%
\pgfpathrectangle{\pgfqpoint{0.506010in}{1.121191in}}{\pgfqpoint{2.325000in}{1.400000in}} %
\pgfusepath{clip}%
\pgfsetbuttcap%
\pgfsetroundjoin%
\definecolor{currentfill}{rgb}{0.000000,0.000000,1.000000}%
\pgfsetfillcolor{currentfill}%
\pgfsetlinewidth{1.003750pt}%
\definecolor{currentstroke}{rgb}{0.000000,0.000000,1.000000}%
\pgfsetstrokecolor{currentstroke}%
\pgfsetdash{}{0pt}%
\pgfpathmoveto{\pgfqpoint{2.211996in}{6.142478in}}%
\pgfpathmoveto{\pgfqpoint{2.211996in}{6.186399in}}%
\pgfusepath{stroke,fill}%
\end{pgfscope}%
\begin{pgfscope}%
\pgfpathrectangle{\pgfqpoint{0.506010in}{1.121191in}}{\pgfqpoint{2.325000in}{1.400000in}} %
\pgfusepath{clip}%
\pgfsetbuttcap%
\pgfsetroundjoin%
\definecolor{currentfill}{rgb}{0.000000,0.000000,1.000000}%
\pgfsetfillcolor{currentfill}%
\pgfsetlinewidth{1.003750pt}%
\definecolor{currentstroke}{rgb}{0.000000,0.000000,1.000000}%
\pgfsetstrokecolor{currentstroke}%
\pgfsetdash{}{0pt}%
\pgfpathmoveto{\pgfqpoint{2.229649in}{6.142478in}}%
\pgfpathmoveto{\pgfqpoint{2.229649in}{6.186399in}}%
\pgfusepath{stroke,fill}%
\end{pgfscope}%
\begin{pgfscope}%
\pgfpathrectangle{\pgfqpoint{0.506010in}{1.121191in}}{\pgfqpoint{2.325000in}{1.400000in}} %
\pgfusepath{clip}%
\pgfsetbuttcap%
\pgfsetroundjoin%
\definecolor{currentfill}{rgb}{0.000000,0.000000,1.000000}%
\pgfsetfillcolor{currentfill}%
\pgfsetlinewidth{1.003750pt}%
\definecolor{currentstroke}{rgb}{0.000000,0.000000,1.000000}%
\pgfsetstrokecolor{currentstroke}%
\pgfsetdash{}{0pt}%
\pgfpathmoveto{\pgfqpoint{2.216224in}{6.142506in}}%
\pgfpathmoveto{\pgfqpoint{2.216224in}{6.186427in}}%
\pgfusepath{stroke,fill}%
\end{pgfscope}%
\begin{pgfscope}%
\pgfpathrectangle{\pgfqpoint{0.506010in}{1.121191in}}{\pgfqpoint{2.325000in}{1.400000in}} %
\pgfusepath{clip}%
\pgfsetbuttcap%
\pgfsetroundjoin%
\definecolor{currentfill}{rgb}{0.000000,0.000000,1.000000}%
\pgfsetfillcolor{currentfill}%
\pgfsetlinewidth{1.003750pt}%
\definecolor{currentstroke}{rgb}{0.000000,0.000000,1.000000}%
\pgfsetstrokecolor{currentstroke}%
\pgfsetdash{}{0pt}%
\pgfpathmoveto{\pgfqpoint{2.226962in}{6.142506in}}%
\pgfpathmoveto{\pgfqpoint{2.226962in}{6.186427in}}%
\pgfusepath{stroke,fill}%
\end{pgfscope}%
\begin{pgfscope}%
\pgfpathrectangle{\pgfqpoint{0.506010in}{1.121191in}}{\pgfqpoint{2.325000in}{1.400000in}} %
\pgfusepath{clip}%
\pgfsetbuttcap%
\pgfsetroundjoin%
\definecolor{currentfill}{rgb}{0.000000,0.000000,1.000000}%
\pgfsetfillcolor{currentfill}%
\pgfsetlinewidth{1.003750pt}%
\definecolor{currentstroke}{rgb}{0.000000,0.000000,1.000000}%
\pgfsetstrokecolor{currentstroke}%
\pgfsetdash{}{0pt}%
\pgfpathmoveto{\pgfqpoint{2.240895in}{6.142534in}}%
\pgfpathmoveto{\pgfqpoint{2.240895in}{6.186455in}}%
\pgfusepath{stroke,fill}%
\end{pgfscope}%
\begin{pgfscope}%
\pgfpathrectangle{\pgfqpoint{0.506010in}{1.121191in}}{\pgfqpoint{2.325000in}{1.400000in}} %
\pgfusepath{clip}%
\pgfsetbuttcap%
\pgfsetroundjoin%
\definecolor{currentfill}{rgb}{0.000000,0.000000,1.000000}%
\pgfsetfillcolor{currentfill}%
\pgfsetlinewidth{1.003750pt}%
\definecolor{currentstroke}{rgb}{0.000000,0.000000,1.000000}%
\pgfsetstrokecolor{currentstroke}%
\pgfsetdash{}{0pt}%
\pgfpathmoveto{\pgfqpoint{2.211996in}{6.142534in}}%
\pgfpathmoveto{\pgfqpoint{2.211996in}{6.186455in}}%
\pgfusepath{stroke,fill}%
\end{pgfscope}%
\begin{pgfscope}%
\pgfpathrectangle{\pgfqpoint{0.506010in}{1.121191in}}{\pgfqpoint{2.325000in}{1.400000in}} %
\pgfusepath{clip}%
\pgfsetbuttcap%
\pgfsetroundjoin%
\definecolor{currentfill}{rgb}{0.000000,0.000000,1.000000}%
\pgfsetfillcolor{currentfill}%
\pgfsetlinewidth{1.003750pt}%
\definecolor{currentstroke}{rgb}{0.000000,0.000000,1.000000}%
\pgfsetstrokecolor{currentstroke}%
\pgfsetdash{}{0pt}%
\pgfpathmoveto{\pgfqpoint{2.195411in}{6.142562in}}%
\pgfpathmoveto{\pgfqpoint{2.195411in}{6.186483in}}%
\pgfusepath{stroke,fill}%
\end{pgfscope}%
\begin{pgfscope}%
\pgfpathrectangle{\pgfqpoint{0.506010in}{1.121191in}}{\pgfqpoint{2.325000in}{1.400000in}} %
\pgfusepath{clip}%
\pgfsetbuttcap%
\pgfsetroundjoin%
\definecolor{currentfill}{rgb}{0.000000,0.000000,1.000000}%
\pgfsetfillcolor{currentfill}%
\pgfsetlinewidth{1.003750pt}%
\definecolor{currentstroke}{rgb}{0.000000,0.000000,1.000000}%
\pgfsetstrokecolor{currentstroke}%
\pgfsetdash{}{0pt}%
\pgfpathmoveto{\pgfqpoint{2.162268in}{6.142562in}}%
\pgfpathmoveto{\pgfqpoint{2.162268in}{6.186483in}}%
\pgfusepath{stroke,fill}%
\end{pgfscope}%
\begin{pgfscope}%
\pgfpathrectangle{\pgfqpoint{0.506010in}{1.121191in}}{\pgfqpoint{2.325000in}{1.400000in}} %
\pgfusepath{clip}%
\pgfsetbuttcap%
\pgfsetroundjoin%
\definecolor{currentfill}{rgb}{0.000000,0.000000,1.000000}%
\pgfsetfillcolor{currentfill}%
\pgfsetlinewidth{1.003750pt}%
\definecolor{currentstroke}{rgb}{0.000000,0.000000,1.000000}%
\pgfsetstrokecolor{currentstroke}%
\pgfsetdash{}{0pt}%
\pgfpathmoveto{\pgfqpoint{2.179176in}{6.142562in}}%
\pgfpathmoveto{\pgfqpoint{2.179176in}{6.186483in}}%
\pgfusepath{stroke,fill}%
\end{pgfscope}%
\begin{pgfscope}%
\pgfpathrectangle{\pgfqpoint{0.506010in}{1.121191in}}{\pgfqpoint{2.325000in}{1.400000in}} %
\pgfusepath{clip}%
\pgfsetbuttcap%
\pgfsetroundjoin%
\definecolor{currentfill}{rgb}{0.750000,0.750000,0.000000}%
\pgfsetfillcolor{currentfill}%
\pgfsetlinewidth{1.003750pt}%
\definecolor{currentstroke}{rgb}{0.750000,0.750000,0.000000}%
\pgfsetstrokecolor{currentstroke}%
\pgfsetdash{}{0pt}%
\pgfpathmoveto{\pgfqpoint{1.151065in}{1.091170in}}%
\pgfpathmoveto{\pgfqpoint{1.163078in}{1.111191in}}%
\pgfpathlineto{\pgfqpoint{1.169699in}{1.122227in}}%
\pgfpathlineto{\pgfqpoint{1.151065in}{1.153283in}}%
\pgfpathlineto{\pgfqpoint{1.132432in}{1.122227in}}%
\pgfpathlineto{\pgfqpoint{1.151065in}{1.091170in}}%
\pgfusepath{stroke,fill}%
\end{pgfscope}%
\begin{pgfscope}%
\pgfpathrectangle{\pgfqpoint{0.506010in}{1.121191in}}{\pgfqpoint{2.325000in}{1.400000in}} %
\pgfusepath{clip}%
\pgfsetbuttcap%
\pgfsetroundjoin%
\definecolor{currentfill}{rgb}{0.750000,0.750000,0.000000}%
\pgfsetfillcolor{currentfill}%
\pgfsetlinewidth{1.003750pt}%
\definecolor{currentstroke}{rgb}{0.750000,0.750000,0.000000}%
\pgfsetstrokecolor{currentstroke}%
\pgfsetdash{}{0pt}%
\pgfpathmoveto{\pgfqpoint{0.772689in}{1.091198in}}%
\pgfpathmoveto{\pgfqpoint{0.784685in}{1.111191in}}%
\pgfpathlineto{\pgfqpoint{0.791323in}{1.122255in}}%
\pgfpathlineto{\pgfqpoint{0.772689in}{1.153311in}}%
\pgfpathlineto{\pgfqpoint{0.754055in}{1.122255in}}%
\pgfpathlineto{\pgfqpoint{0.772689in}{1.091198in}}%
\pgfusepath{stroke,fill}%
\end{pgfscope}%
\begin{pgfscope}%
\pgfpathrectangle{\pgfqpoint{0.506010in}{1.121191in}}{\pgfqpoint{2.325000in}{1.400000in}} %
\pgfusepath{clip}%
\pgfsetbuttcap%
\pgfsetroundjoin%
\definecolor{currentfill}{rgb}{0.750000,0.750000,0.000000}%
\pgfsetfillcolor{currentfill}%
\pgfsetlinewidth{1.003750pt}%
\definecolor{currentstroke}{rgb}{0.750000,0.750000,0.000000}%
\pgfsetstrokecolor{currentstroke}%
\pgfsetdash{}{0pt}%
\pgfpathmoveto{\pgfqpoint{1.222833in}{1.091198in}}%
\pgfpathmoveto{\pgfqpoint{1.234829in}{1.111191in}}%
\pgfpathlineto{\pgfqpoint{1.241467in}{1.122255in}}%
\pgfpathlineto{\pgfqpoint{1.222833in}{1.153311in}}%
\pgfpathlineto{\pgfqpoint{1.204199in}{1.122255in}}%
\pgfpathlineto{\pgfqpoint{1.222833in}{1.091198in}}%
\pgfusepath{stroke,fill}%
\end{pgfscope}%
\begin{pgfscope}%
\pgfpathrectangle{\pgfqpoint{0.506010in}{1.121191in}}{\pgfqpoint{2.325000in}{1.400000in}} %
\pgfusepath{clip}%
\pgfsetbuttcap%
\pgfsetroundjoin%
\definecolor{currentfill}{rgb}{0.750000,0.750000,0.000000}%
\pgfsetfillcolor{currentfill}%
\pgfsetlinewidth{1.003750pt}%
\definecolor{currentstroke}{rgb}{0.750000,0.750000,0.000000}%
\pgfsetstrokecolor{currentstroke}%
\pgfsetdash{}{0pt}%
\pgfpathmoveto{\pgfqpoint{0.980706in}{1.091198in}}%
\pgfpathmoveto{\pgfqpoint{0.992701in}{1.111191in}}%
\pgfpathlineto{\pgfqpoint{0.999340in}{1.122255in}}%
\pgfpathlineto{\pgfqpoint{0.980706in}{1.153311in}}%
\pgfpathlineto{\pgfqpoint{0.962072in}{1.122255in}}%
\pgfpathlineto{\pgfqpoint{0.980706in}{1.091198in}}%
\pgfusepath{stroke,fill}%
\end{pgfscope}%
\begin{pgfscope}%
\pgfpathrectangle{\pgfqpoint{0.506010in}{1.121191in}}{\pgfqpoint{2.325000in}{1.400000in}} %
\pgfusepath{clip}%
\pgfsetbuttcap%
\pgfsetroundjoin%
\definecolor{currentfill}{rgb}{0.750000,0.750000,0.000000}%
\pgfsetfillcolor{currentfill}%
\pgfsetlinewidth{1.003750pt}%
\definecolor{currentstroke}{rgb}{0.750000,0.750000,0.000000}%
\pgfsetstrokecolor{currentstroke}%
\pgfsetdash{}{0pt}%
\pgfpathmoveto{\pgfqpoint{0.968203in}{1.091198in}}%
\pgfpathmoveto{\pgfqpoint{0.980199in}{1.111191in}}%
\pgfpathlineto{\pgfqpoint{0.986837in}{1.122255in}}%
\pgfpathlineto{\pgfqpoint{0.968203in}{1.153311in}}%
\pgfpathlineto{\pgfqpoint{0.949569in}{1.122255in}}%
\pgfpathlineto{\pgfqpoint{0.968203in}{1.091198in}}%
\pgfusepath{stroke,fill}%
\end{pgfscope}%
\begin{pgfscope}%
\pgfpathrectangle{\pgfqpoint{0.506010in}{1.121191in}}{\pgfqpoint{2.325000in}{1.400000in}} %
\pgfusepath{clip}%
\pgfsetbuttcap%
\pgfsetroundjoin%
\definecolor{currentfill}{rgb}{0.750000,0.750000,0.000000}%
\pgfsetfillcolor{currentfill}%
\pgfsetlinewidth{1.003750pt}%
\definecolor{currentstroke}{rgb}{0.750000,0.750000,0.000000}%
\pgfsetstrokecolor{currentstroke}%
\pgfsetdash{}{0pt}%
\pgfpathmoveto{\pgfqpoint{1.097148in}{1.091226in}}%
\pgfpathmoveto{\pgfqpoint{1.109127in}{1.111191in}}%
\pgfpathlineto{\pgfqpoint{1.115782in}{1.122283in}}%
\pgfpathlineto{\pgfqpoint{1.097148in}{1.153339in}}%
\pgfpathlineto{\pgfqpoint{1.078514in}{1.122283in}}%
\pgfpathlineto{\pgfqpoint{1.097148in}{1.091226in}}%
\pgfusepath{stroke,fill}%
\end{pgfscope}%
\begin{pgfscope}%
\pgfpathrectangle{\pgfqpoint{0.506010in}{1.121191in}}{\pgfqpoint{2.325000in}{1.400000in}} %
\pgfusepath{clip}%
\pgfsetbuttcap%
\pgfsetroundjoin%
\definecolor{currentfill}{rgb}{0.750000,0.750000,0.000000}%
\pgfsetfillcolor{currentfill}%
\pgfsetlinewidth{1.003750pt}%
\definecolor{currentstroke}{rgb}{0.750000,0.750000,0.000000}%
\pgfsetstrokecolor{currentstroke}%
\pgfsetdash{}{0pt}%
\pgfpathmoveto{\pgfqpoint{0.813731in}{1.091254in}}%
\pgfpathmoveto{\pgfqpoint{0.825693in}{1.111191in}}%
\pgfpathlineto{\pgfqpoint{0.832365in}{1.122311in}}%
\pgfpathlineto{\pgfqpoint{0.813731in}{1.153367in}}%
\pgfpathlineto{\pgfqpoint{0.795097in}{1.122311in}}%
\pgfpathlineto{\pgfqpoint{0.813731in}{1.091254in}}%
\pgfusepath{stroke,fill}%
\end{pgfscope}%
\begin{pgfscope}%
\pgfpathrectangle{\pgfqpoint{0.506010in}{1.121191in}}{\pgfqpoint{2.325000in}{1.400000in}} %
\pgfusepath{clip}%
\pgfsetbuttcap%
\pgfsetroundjoin%
\definecolor{currentfill}{rgb}{0.750000,0.750000,0.000000}%
\pgfsetfillcolor{currentfill}%
\pgfsetlinewidth{1.003750pt}%
\definecolor{currentstroke}{rgb}{0.750000,0.750000,0.000000}%
\pgfsetstrokecolor{currentstroke}%
\pgfsetdash{}{0pt}%
\pgfpathmoveto{\pgfqpoint{1.308740in}{1.091254in}}%
\pgfpathmoveto{\pgfqpoint{1.320702in}{1.111191in}}%
\pgfpathlineto{\pgfqpoint{1.327374in}{1.122311in}}%
\pgfpathlineto{\pgfqpoint{1.308740in}{1.153367in}}%
\pgfpathlineto{\pgfqpoint{1.290106in}{1.122311in}}%
\pgfpathlineto{\pgfqpoint{1.308740in}{1.091254in}}%
\pgfusepath{stroke,fill}%
\end{pgfscope}%
\begin{pgfscope}%
\pgfpathrectangle{\pgfqpoint{0.506010in}{1.121191in}}{\pgfqpoint{2.325000in}{1.400000in}} %
\pgfusepath{clip}%
\pgfsetbuttcap%
\pgfsetroundjoin%
\definecolor{currentfill}{rgb}{0.750000,0.750000,0.000000}%
\pgfsetfillcolor{currentfill}%
\pgfsetlinewidth{1.003750pt}%
\definecolor{currentstroke}{rgb}{0.750000,0.750000,0.000000}%
\pgfsetstrokecolor{currentstroke}%
\pgfsetdash{}{0pt}%
\pgfpathmoveto{\pgfqpoint{0.846633in}{1.091254in}}%
\pgfpathmoveto{\pgfqpoint{0.858595in}{1.111191in}}%
\pgfpathlineto{\pgfqpoint{0.865267in}{1.122311in}}%
\pgfpathlineto{\pgfqpoint{0.846633in}{1.153367in}}%
\pgfpathlineto{\pgfqpoint{0.827999in}{1.122311in}}%
\pgfpathlineto{\pgfqpoint{0.846633in}{1.091254in}}%
\pgfusepath{stroke,fill}%
\end{pgfscope}%
\begin{pgfscope}%
\pgfpathrectangle{\pgfqpoint{0.506010in}{1.121191in}}{\pgfqpoint{2.325000in}{1.400000in}} %
\pgfusepath{clip}%
\pgfsetbuttcap%
\pgfsetroundjoin%
\definecolor{currentfill}{rgb}{0.750000,0.750000,0.000000}%
\pgfsetfillcolor{currentfill}%
\pgfsetlinewidth{1.003750pt}%
\definecolor{currentstroke}{rgb}{0.750000,0.750000,0.000000}%
\pgfsetstrokecolor{currentstroke}%
\pgfsetdash{}{0pt}%
\pgfpathmoveto{\pgfqpoint{1.144617in}{1.091282in}}%
\pgfpathmoveto{\pgfqpoint{1.156562in}{1.111191in}}%
\pgfpathlineto{\pgfqpoint{1.163251in}{1.122339in}}%
\pgfpathlineto{\pgfqpoint{1.144617in}{1.153395in}}%
\pgfpathlineto{\pgfqpoint{1.125983in}{1.122339in}}%
\pgfpathlineto{\pgfqpoint{1.144617in}{1.091282in}}%
\pgfusepath{stroke,fill}%
\end{pgfscope}%
\begin{pgfscope}%
\pgfpathrectangle{\pgfqpoint{0.506010in}{1.121191in}}{\pgfqpoint{2.325000in}{1.400000in}} %
\pgfusepath{clip}%
\pgfsetbuttcap%
\pgfsetroundjoin%
\definecolor{currentfill}{rgb}{0.750000,0.750000,0.000000}%
\pgfsetfillcolor{currentfill}%
\pgfsetlinewidth{1.003750pt}%
\definecolor{currentstroke}{rgb}{0.750000,0.750000,0.000000}%
\pgfsetstrokecolor{currentstroke}%
\pgfsetdash{}{0pt}%
\pgfpathmoveto{\pgfqpoint{1.165504in}{1.091282in}}%
\pgfpathmoveto{\pgfqpoint{1.177449in}{1.111191in}}%
\pgfpathlineto{\pgfqpoint{1.184138in}{1.122339in}}%
\pgfpathlineto{\pgfqpoint{1.165504in}{1.153395in}}%
\pgfpathlineto{\pgfqpoint{1.146870in}{1.122339in}}%
\pgfpathlineto{\pgfqpoint{1.165504in}{1.091282in}}%
\pgfusepath{stroke,fill}%
\end{pgfscope}%
\begin{pgfscope}%
\pgfpathrectangle{\pgfqpoint{0.506010in}{1.121191in}}{\pgfqpoint{2.325000in}{1.400000in}} %
\pgfusepath{clip}%
\pgfsetbuttcap%
\pgfsetroundjoin%
\definecolor{currentfill}{rgb}{0.750000,0.750000,0.000000}%
\pgfsetfillcolor{currentfill}%
\pgfsetlinewidth{1.003750pt}%
\definecolor{currentstroke}{rgb}{0.750000,0.750000,0.000000}%
\pgfsetstrokecolor{currentstroke}%
\pgfsetdash{}{0pt}%
\pgfpathmoveto{\pgfqpoint{0.788475in}{1.091310in}}%
\pgfpathmoveto{\pgfqpoint{0.800403in}{1.111191in}}%
\pgfpathlineto{\pgfqpoint{0.807109in}{1.122367in}}%
\pgfpathlineto{\pgfqpoint{0.788475in}{1.153423in}}%
\pgfpathlineto{\pgfqpoint{0.769841in}{1.122367in}}%
\pgfpathlineto{\pgfqpoint{0.788475in}{1.091310in}}%
\pgfusepath{stroke,fill}%
\end{pgfscope}%
\begin{pgfscope}%
\pgfpathrectangle{\pgfqpoint{0.506010in}{1.121191in}}{\pgfqpoint{2.325000in}{1.400000in}} %
\pgfusepath{clip}%
\pgfsetbuttcap%
\pgfsetroundjoin%
\definecolor{currentfill}{rgb}{0.750000,0.750000,0.000000}%
\pgfsetfillcolor{currentfill}%
\pgfsetlinewidth{1.003750pt}%
\definecolor{currentstroke}{rgb}{0.750000,0.750000,0.000000}%
\pgfsetstrokecolor{currentstroke}%
\pgfsetdash{}{0pt}%
\pgfpathmoveto{\pgfqpoint{1.587679in}{1.091310in}}%
\pgfpathmoveto{\pgfqpoint{1.599607in}{1.111191in}}%
\pgfpathlineto{\pgfqpoint{1.606313in}{1.122367in}}%
\pgfpathlineto{\pgfqpoint{1.587679in}{1.153423in}}%
\pgfpathlineto{\pgfqpoint{1.569045in}{1.122367in}}%
\pgfpathlineto{\pgfqpoint{1.587679in}{1.091310in}}%
\pgfusepath{stroke,fill}%
\end{pgfscope}%
\begin{pgfscope}%
\pgfpathrectangle{\pgfqpoint{0.506010in}{1.121191in}}{\pgfqpoint{2.325000in}{1.400000in}} %
\pgfusepath{clip}%
\pgfsetbuttcap%
\pgfsetroundjoin%
\definecolor{currentfill}{rgb}{0.750000,0.750000,0.000000}%
\pgfsetfillcolor{currentfill}%
\pgfsetlinewidth{1.003750pt}%
\definecolor{currentstroke}{rgb}{0.750000,0.750000,0.000000}%
\pgfsetstrokecolor{currentstroke}%
\pgfsetdash{}{0pt}%
\pgfpathmoveto{\pgfqpoint{1.366286in}{1.092850in}}%
\pgfpathmoveto{\pgfqpoint{1.377291in}{1.111191in}}%
\pgfpathlineto{\pgfqpoint{1.384920in}{1.123907in}}%
\pgfpathlineto{\pgfqpoint{1.366286in}{1.154963in}}%
\pgfpathlineto{\pgfqpoint{1.347652in}{1.123907in}}%
\pgfpathlineto{\pgfqpoint{1.366286in}{1.092850in}}%
\pgfusepath{stroke,fill}%
\end{pgfscope}%
\begin{pgfscope}%
\pgfpathrectangle{\pgfqpoint{0.506010in}{1.121191in}}{\pgfqpoint{2.325000in}{1.400000in}} %
\pgfusepath{clip}%
\pgfsetbuttcap%
\pgfsetroundjoin%
\definecolor{currentfill}{rgb}{0.750000,0.750000,0.000000}%
\pgfsetfillcolor{currentfill}%
\pgfsetlinewidth{1.003750pt}%
\definecolor{currentstroke}{rgb}{0.750000,0.750000,0.000000}%
\pgfsetstrokecolor{currentstroke}%
\pgfsetdash{}{0pt}%
\pgfpathmoveto{\pgfqpoint{1.279201in}{1.096826in}}%
\pgfpathmoveto{\pgfqpoint{1.287819in}{1.111191in}}%
\pgfpathlineto{\pgfqpoint{1.297834in}{1.127883in}}%
\pgfpathlineto{\pgfqpoint{1.279201in}{1.158939in}}%
\pgfpathlineto{\pgfqpoint{1.260567in}{1.127883in}}%
\pgfpathlineto{\pgfqpoint{1.279201in}{1.096826in}}%
\pgfusepath{stroke,fill}%
\end{pgfscope}%
\begin{pgfscope}%
\pgfpathrectangle{\pgfqpoint{0.506010in}{1.121191in}}{\pgfqpoint{2.325000in}{1.400000in}} %
\pgfusepath{clip}%
\pgfsetbuttcap%
\pgfsetroundjoin%
\definecolor{currentfill}{rgb}{0.750000,0.750000,0.000000}%
\pgfsetfillcolor{currentfill}%
\pgfsetlinewidth{1.003750pt}%
\definecolor{currentstroke}{rgb}{0.750000,0.750000,0.000000}%
\pgfsetstrokecolor{currentstroke}%
\pgfsetdash{}{0pt}%
\pgfpathmoveto{\pgfqpoint{2.264030in}{1.100858in}}%
\pgfpathmoveto{\pgfqpoint{2.270229in}{1.111191in}}%
\pgfpathlineto{\pgfqpoint{2.282664in}{1.131915in}}%
\pgfpathlineto{\pgfqpoint{2.264030in}{1.162971in}}%
\pgfpathlineto{\pgfqpoint{2.245396in}{1.131915in}}%
\pgfpathlineto{\pgfqpoint{2.264030in}{1.100858in}}%
\pgfusepath{stroke,fill}%
\end{pgfscope}%
\begin{pgfscope}%
\pgfpathrectangle{\pgfqpoint{0.506010in}{1.121191in}}{\pgfqpoint{2.325000in}{1.400000in}} %
\pgfusepath{clip}%
\pgfsetbuttcap%
\pgfsetroundjoin%
\definecolor{currentfill}{rgb}{0.750000,0.750000,0.000000}%
\pgfsetfillcolor{currentfill}%
\pgfsetlinewidth{1.003750pt}%
\definecolor{currentstroke}{rgb}{0.750000,0.750000,0.000000}%
\pgfsetstrokecolor{currentstroke}%
\pgfsetdash{}{0pt}%
\pgfpathmoveto{\pgfqpoint{2.191001in}{1.113262in}}%
\pgfpathlineto{\pgfqpoint{2.209635in}{1.144319in}}%
\pgfpathlineto{\pgfqpoint{2.191001in}{1.175375in}}%
\pgfpathlineto{\pgfqpoint{2.172368in}{1.144319in}}%
\pgfpathclose%
\pgfusepath{stroke,fill}%
\end{pgfscope}%
\begin{pgfscope}%
\pgfpathrectangle{\pgfqpoint{0.506010in}{1.121191in}}{\pgfqpoint{2.325000in}{1.400000in}} %
\pgfusepath{clip}%
\pgfsetbuttcap%
\pgfsetroundjoin%
\definecolor{currentfill}{rgb}{0.750000,0.750000,0.000000}%
\pgfsetfillcolor{currentfill}%
\pgfsetlinewidth{1.003750pt}%
\definecolor{currentstroke}{rgb}{0.750000,0.750000,0.000000}%
\pgfsetstrokecolor{currentstroke}%
\pgfsetdash{}{0pt}%
\pgfpathmoveto{\pgfqpoint{2.263023in}{1.116230in}}%
\pgfpathlineto{\pgfqpoint{2.281657in}{1.147287in}}%
\pgfpathlineto{\pgfqpoint{2.263023in}{1.178343in}}%
\pgfpathlineto{\pgfqpoint{2.244390in}{1.147287in}}%
\pgfpathclose%
\pgfusepath{stroke,fill}%
\end{pgfscope}%
\begin{pgfscope}%
\pgfpathrectangle{\pgfqpoint{0.506010in}{1.121191in}}{\pgfqpoint{2.325000in}{1.400000in}} %
\pgfusepath{clip}%
\pgfsetbuttcap%
\pgfsetroundjoin%
\definecolor{currentfill}{rgb}{0.750000,0.750000,0.000000}%
\pgfsetfillcolor{currentfill}%
\pgfsetlinewidth{1.003750pt}%
\definecolor{currentstroke}{rgb}{0.750000,0.750000,0.000000}%
\pgfsetstrokecolor{currentstroke}%
\pgfsetdash{}{0pt}%
\pgfpathmoveto{\pgfqpoint{2.251667in}{1.124742in}}%
\pgfpathlineto{\pgfqpoint{2.270301in}{1.155799in}}%
\pgfpathlineto{\pgfqpoint{2.251667in}{1.186855in}}%
\pgfpathlineto{\pgfqpoint{2.233033in}{1.155799in}}%
\pgfpathclose%
\pgfusepath{stroke,fill}%
\end{pgfscope}%
\begin{pgfscope}%
\pgfpathrectangle{\pgfqpoint{0.506010in}{1.121191in}}{\pgfqpoint{2.325000in}{1.400000in}} %
\pgfusepath{clip}%
\pgfsetbuttcap%
\pgfsetroundjoin%
\definecolor{currentfill}{rgb}{0.750000,0.750000,0.000000}%
\pgfsetfillcolor{currentfill}%
\pgfsetlinewidth{1.003750pt}%
\definecolor{currentstroke}{rgb}{0.750000,0.750000,0.000000}%
\pgfsetstrokecolor{currentstroke}%
\pgfsetdash{}{0pt}%
\pgfpathmoveto{\pgfqpoint{2.223598in}{1.135046in}}%
\pgfpathlineto{\pgfqpoint{2.242232in}{1.166103in}}%
\pgfpathlineto{\pgfqpoint{2.223598in}{1.197159in}}%
\pgfpathlineto{\pgfqpoint{2.204964in}{1.166103in}}%
\pgfpathclose%
\pgfusepath{stroke,fill}%
\end{pgfscope}%
\begin{pgfscope}%
\pgfpathrectangle{\pgfqpoint{0.506010in}{1.121191in}}{\pgfqpoint{2.325000in}{1.400000in}} %
\pgfusepath{clip}%
\pgfsetbuttcap%
\pgfsetroundjoin%
\definecolor{currentfill}{rgb}{0.750000,0.750000,0.000000}%
\pgfsetfillcolor{currentfill}%
\pgfsetlinewidth{1.003750pt}%
\definecolor{currentstroke}{rgb}{0.750000,0.750000,0.000000}%
\pgfsetstrokecolor{currentstroke}%
\pgfsetdash{}{0pt}%
\pgfpathmoveto{\pgfqpoint{2.223598in}{1.135130in}}%
\pgfpathlineto{\pgfqpoint{2.242232in}{1.166187in}}%
\pgfpathlineto{\pgfqpoint{2.223598in}{1.197243in}}%
\pgfpathlineto{\pgfqpoint{2.204964in}{1.166187in}}%
\pgfpathclose%
\pgfusepath{stroke,fill}%
\end{pgfscope}%
\begin{pgfscope}%
\pgfpathrectangle{\pgfqpoint{0.506010in}{1.121191in}}{\pgfqpoint{2.325000in}{1.400000in}} %
\pgfusepath{clip}%
\pgfsetbuttcap%
\pgfsetroundjoin%
\definecolor{currentfill}{rgb}{0.750000,0.750000,0.000000}%
\pgfsetfillcolor{currentfill}%
\pgfsetlinewidth{1.003750pt}%
\definecolor{currentstroke}{rgb}{0.750000,0.750000,0.000000}%
\pgfsetstrokecolor{currentstroke}%
\pgfsetdash{}{0pt}%
\pgfpathmoveto{\pgfqpoint{1.007822in}{1.153722in}}%
\pgfpathlineto{\pgfqpoint{1.026456in}{1.184779in}}%
\pgfpathlineto{\pgfqpoint{1.007822in}{1.215835in}}%
\pgfpathlineto{\pgfqpoint{0.989188in}{1.184779in}}%
\pgfpathclose%
\pgfusepath{stroke,fill}%
\end{pgfscope}%
\begin{pgfscope}%
\pgfpathrectangle{\pgfqpoint{0.506010in}{1.121191in}}{\pgfqpoint{2.325000in}{1.400000in}} %
\pgfusepath{clip}%
\pgfsetbuttcap%
\pgfsetroundjoin%
\definecolor{currentfill}{rgb}{0.750000,0.750000,0.000000}%
\pgfsetfillcolor{currentfill}%
\pgfsetlinewidth{1.003750pt}%
\definecolor{currentstroke}{rgb}{0.750000,0.750000,0.000000}%
\pgfsetstrokecolor{currentstroke}%
\pgfsetdash{}{0pt}%
\pgfpathmoveto{\pgfqpoint{1.007822in}{1.159014in}}%
\pgfpathlineto{\pgfqpoint{1.026456in}{1.190071in}}%
\pgfpathlineto{\pgfqpoint{1.007822in}{1.221127in}}%
\pgfpathlineto{\pgfqpoint{0.989188in}{1.190071in}}%
\pgfpathclose%
\pgfusepath{stroke,fill}%
\end{pgfscope}%
\begin{pgfscope}%
\pgfpathrectangle{\pgfqpoint{0.506010in}{1.121191in}}{\pgfqpoint{2.325000in}{1.400000in}} %
\pgfusepath{clip}%
\pgfsetbuttcap%
\pgfsetroundjoin%
\definecolor{currentfill}{rgb}{0.750000,0.750000,0.000000}%
\pgfsetfillcolor{currentfill}%
\pgfsetlinewidth{1.003750pt}%
\definecolor{currentstroke}{rgb}{0.750000,0.750000,0.000000}%
\pgfsetstrokecolor{currentstroke}%
\pgfsetdash{}{0pt}%
\pgfpathmoveto{\pgfqpoint{2.381458in}{1.172734in}}%
\pgfpathlineto{\pgfqpoint{2.400091in}{1.203791in}}%
\pgfpathlineto{\pgfqpoint{2.381458in}{1.234847in}}%
\pgfpathlineto{\pgfqpoint{2.362824in}{1.203791in}}%
\pgfpathclose%
\pgfusepath{stroke,fill}%
\end{pgfscope}%
\begin{pgfscope}%
\pgfpathrectangle{\pgfqpoint{0.506010in}{1.121191in}}{\pgfqpoint{2.325000in}{1.400000in}} %
\pgfusepath{clip}%
\pgfsetbuttcap%
\pgfsetroundjoin%
\definecolor{currentfill}{rgb}{0.750000,0.750000,0.000000}%
\pgfsetfillcolor{currentfill}%
\pgfsetlinewidth{1.003750pt}%
\definecolor{currentstroke}{rgb}{0.750000,0.750000,0.000000}%
\pgfsetstrokecolor{currentstroke}%
\pgfsetdash{}{0pt}%
\pgfpathmoveto{\pgfqpoint{2.380105in}{1.207230in}}%
\pgfpathlineto{\pgfqpoint{2.398739in}{1.238287in}}%
\pgfpathlineto{\pgfqpoint{2.380105in}{1.269343in}}%
\pgfpathlineto{\pgfqpoint{2.361471in}{1.238287in}}%
\pgfpathclose%
\pgfusepath{stroke,fill}%
\end{pgfscope}%
\begin{pgfscope}%
\pgfpathrectangle{\pgfqpoint{0.506010in}{1.121191in}}{\pgfqpoint{2.325000in}{1.400000in}} %
\pgfusepath{clip}%
\pgfsetbuttcap%
\pgfsetroundjoin%
\definecolor{currentfill}{rgb}{0.750000,0.750000,0.000000}%
\pgfsetfillcolor{currentfill}%
\pgfsetlinewidth{1.003750pt}%
\definecolor{currentstroke}{rgb}{0.750000,0.750000,0.000000}%
\pgfsetstrokecolor{currentstroke}%
\pgfsetdash{}{0pt}%
\pgfpathmoveto{\pgfqpoint{2.407155in}{1.209974in}}%
\pgfpathlineto{\pgfqpoint{2.425789in}{1.241031in}}%
\pgfpathlineto{\pgfqpoint{2.407155in}{1.272087in}}%
\pgfpathlineto{\pgfqpoint{2.388521in}{1.241031in}}%
\pgfpathclose%
\pgfusepath{stroke,fill}%
\end{pgfscope}%
\begin{pgfscope}%
\pgfpathrectangle{\pgfqpoint{0.506010in}{1.121191in}}{\pgfqpoint{2.325000in}{1.400000in}} %
\pgfusepath{clip}%
\pgfsetbuttcap%
\pgfsetroundjoin%
\definecolor{currentfill}{rgb}{0.750000,0.750000,0.000000}%
\pgfsetfillcolor{currentfill}%
\pgfsetlinewidth{1.003750pt}%
\definecolor{currentstroke}{rgb}{0.750000,0.750000,0.000000}%
\pgfsetstrokecolor{currentstroke}%
\pgfsetdash{}{0pt}%
\pgfpathmoveto{\pgfqpoint{1.294699in}{1.229686in}}%
\pgfpathlineto{\pgfqpoint{1.313333in}{1.260743in}}%
\pgfpathlineto{\pgfqpoint{1.294699in}{1.291799in}}%
\pgfpathlineto{\pgfqpoint{1.276065in}{1.260743in}}%
\pgfpathclose%
\pgfusepath{stroke,fill}%
\end{pgfscope}%
\begin{pgfscope}%
\pgfpathrectangle{\pgfqpoint{0.506010in}{1.121191in}}{\pgfqpoint{2.325000in}{1.400000in}} %
\pgfusepath{clip}%
\pgfsetbuttcap%
\pgfsetroundjoin%
\definecolor{currentfill}{rgb}{0.750000,0.750000,0.000000}%
\pgfsetfillcolor{currentfill}%
\pgfsetlinewidth{1.003750pt}%
\definecolor{currentstroke}{rgb}{0.750000,0.750000,0.000000}%
\pgfsetstrokecolor{currentstroke}%
\pgfsetdash{}{0pt}%
\pgfpathmoveto{\pgfqpoint{1.390732in}{1.230610in}}%
\pgfpathlineto{\pgfqpoint{1.409366in}{1.261667in}}%
\pgfpathlineto{\pgfqpoint{1.390732in}{1.292723in}}%
\pgfpathlineto{\pgfqpoint{1.372098in}{1.261667in}}%
\pgfpathclose%
\pgfusepath{stroke,fill}%
\end{pgfscope}%
\begin{pgfscope}%
\pgfpathrectangle{\pgfqpoint{0.506010in}{1.121191in}}{\pgfqpoint{2.325000in}{1.400000in}} %
\pgfusepath{clip}%
\pgfsetbuttcap%
\pgfsetroundjoin%
\definecolor{currentfill}{rgb}{0.750000,0.750000,0.000000}%
\pgfsetfillcolor{currentfill}%
\pgfsetlinewidth{1.003750pt}%
\definecolor{currentstroke}{rgb}{0.750000,0.750000,0.000000}%
\pgfsetstrokecolor{currentstroke}%
\pgfsetdash{}{0pt}%
\pgfpathmoveto{\pgfqpoint{1.305057in}{1.254214in}}%
\pgfpathlineto{\pgfqpoint{1.323691in}{1.285271in}}%
\pgfpathlineto{\pgfqpoint{1.305057in}{1.316327in}}%
\pgfpathlineto{\pgfqpoint{1.286424in}{1.285271in}}%
\pgfpathclose%
\pgfusepath{stroke,fill}%
\end{pgfscope}%
\begin{pgfscope}%
\pgfpathrectangle{\pgfqpoint{0.506010in}{1.121191in}}{\pgfqpoint{2.325000in}{1.400000in}} %
\pgfusepath{clip}%
\pgfsetbuttcap%
\pgfsetroundjoin%
\definecolor{currentfill}{rgb}{0.750000,0.750000,0.000000}%
\pgfsetfillcolor{currentfill}%
\pgfsetlinewidth{1.003750pt}%
\definecolor{currentstroke}{rgb}{0.750000,0.750000,0.000000}%
\pgfsetstrokecolor{currentstroke}%
\pgfsetdash{}{0pt}%
\pgfpathmoveto{\pgfqpoint{1.081382in}{1.269502in}}%
\pgfpathlineto{\pgfqpoint{1.100016in}{1.300559in}}%
\pgfpathlineto{\pgfqpoint{1.081382in}{1.331615in}}%
\pgfpathlineto{\pgfqpoint{1.062748in}{1.300559in}}%
\pgfpathclose%
\pgfusepath{stroke,fill}%
\end{pgfscope}%
\begin{pgfscope}%
\pgfpathrectangle{\pgfqpoint{0.506010in}{1.121191in}}{\pgfqpoint{2.325000in}{1.400000in}} %
\pgfusepath{clip}%
\pgfsetbuttcap%
\pgfsetroundjoin%
\definecolor{currentfill}{rgb}{0.750000,0.750000,0.000000}%
\pgfsetfillcolor{currentfill}%
\pgfsetlinewidth{1.003750pt}%
\definecolor{currentstroke}{rgb}{0.750000,0.750000,0.000000}%
\pgfsetstrokecolor{currentstroke}%
\pgfsetdash{}{0pt}%
\pgfpathmoveto{\pgfqpoint{0.810364in}{1.278294in}}%
\pgfpathlineto{\pgfqpoint{0.828998in}{1.309351in}}%
\pgfpathlineto{\pgfqpoint{0.810364in}{1.340407in}}%
\pgfpathlineto{\pgfqpoint{0.791730in}{1.309351in}}%
\pgfpathclose%
\pgfusepath{stroke,fill}%
\end{pgfscope}%
\begin{pgfscope}%
\pgfpathrectangle{\pgfqpoint{0.506010in}{1.121191in}}{\pgfqpoint{2.325000in}{1.400000in}} %
\pgfusepath{clip}%
\pgfsetbuttcap%
\pgfsetroundjoin%
\definecolor{currentfill}{rgb}{0.750000,0.750000,0.000000}%
\pgfsetfillcolor{currentfill}%
\pgfsetlinewidth{1.003750pt}%
\definecolor{currentstroke}{rgb}{0.750000,0.750000,0.000000}%
\pgfsetstrokecolor{currentstroke}%
\pgfsetdash{}{0pt}%
\pgfpathmoveto{\pgfqpoint{1.366871in}{1.283222in}}%
\pgfpathlineto{\pgfqpoint{1.385504in}{1.314279in}}%
\pgfpathlineto{\pgfqpoint{1.366871in}{1.345335in}}%
\pgfpathlineto{\pgfqpoint{1.348237in}{1.314279in}}%
\pgfpathclose%
\pgfusepath{stroke,fill}%
\end{pgfscope}%
\begin{pgfscope}%
\pgfpathrectangle{\pgfqpoint{0.506010in}{1.121191in}}{\pgfqpoint{2.325000in}{1.400000in}} %
\pgfusepath{clip}%
\pgfsetbuttcap%
\pgfsetroundjoin%
\definecolor{currentfill}{rgb}{0.750000,0.750000,0.000000}%
\pgfsetfillcolor{currentfill}%
\pgfsetlinewidth{1.003750pt}%
\definecolor{currentstroke}{rgb}{0.750000,0.750000,0.000000}%
\pgfsetstrokecolor{currentstroke}%
\pgfsetdash{}{0pt}%
\pgfpathmoveto{\pgfqpoint{1.443680in}{1.302458in}}%
\pgfpathlineto{\pgfqpoint{1.462314in}{1.333515in}}%
\pgfpathlineto{\pgfqpoint{1.443680in}{1.364571in}}%
\pgfpathlineto{\pgfqpoint{1.425046in}{1.333515in}}%
\pgfpathclose%
\pgfusepath{stroke,fill}%
\end{pgfscope}%
\begin{pgfscope}%
\pgfpathrectangle{\pgfqpoint{0.506010in}{1.121191in}}{\pgfqpoint{2.325000in}{1.400000in}} %
\pgfusepath{clip}%
\pgfsetbuttcap%
\pgfsetroundjoin%
\definecolor{currentfill}{rgb}{0.750000,0.750000,0.000000}%
\pgfsetfillcolor{currentfill}%
\pgfsetlinewidth{1.003750pt}%
\definecolor{currentstroke}{rgb}{0.750000,0.750000,0.000000}%
\pgfsetstrokecolor{currentstroke}%
\pgfsetdash{}{0pt}%
\pgfpathmoveto{\pgfqpoint{1.406887in}{1.305930in}}%
\pgfpathlineto{\pgfqpoint{1.425521in}{1.336987in}}%
\pgfpathlineto{\pgfqpoint{1.406887in}{1.368043in}}%
\pgfpathlineto{\pgfqpoint{1.388253in}{1.336987in}}%
\pgfpathclose%
\pgfusepath{stroke,fill}%
\end{pgfscope}%
\begin{pgfscope}%
\pgfpathrectangle{\pgfqpoint{0.506010in}{1.121191in}}{\pgfqpoint{2.325000in}{1.400000in}} %
\pgfusepath{clip}%
\pgfsetbuttcap%
\pgfsetroundjoin%
\definecolor{currentfill}{rgb}{0.750000,0.750000,0.000000}%
\pgfsetfillcolor{currentfill}%
\pgfsetlinewidth{1.003750pt}%
\definecolor{currentstroke}{rgb}{0.750000,0.750000,0.000000}%
\pgfsetstrokecolor{currentstroke}%
\pgfsetdash{}{0pt}%
\pgfpathmoveto{\pgfqpoint{1.108974in}{1.319342in}}%
\pgfpathlineto{\pgfqpoint{1.127608in}{1.350399in}}%
\pgfpathlineto{\pgfqpoint{1.108974in}{1.381455in}}%
\pgfpathlineto{\pgfqpoint{1.090340in}{1.350399in}}%
\pgfpathclose%
\pgfusepath{stroke,fill}%
\end{pgfscope}%
\begin{pgfscope}%
\pgfpathrectangle{\pgfqpoint{0.506010in}{1.121191in}}{\pgfqpoint{2.325000in}{1.400000in}} %
\pgfusepath{clip}%
\pgfsetbuttcap%
\pgfsetroundjoin%
\definecolor{currentfill}{rgb}{0.750000,0.750000,0.000000}%
\pgfsetfillcolor{currentfill}%
\pgfsetlinewidth{1.003750pt}%
\definecolor{currentstroke}{rgb}{0.750000,0.750000,0.000000}%
\pgfsetstrokecolor{currentstroke}%
\pgfsetdash{}{0pt}%
\pgfpathmoveto{\pgfqpoint{0.821928in}{1.372542in}}%
\pgfpathlineto{\pgfqpoint{0.840562in}{1.403599in}}%
\pgfpathlineto{\pgfqpoint{0.821928in}{1.434655in}}%
\pgfpathlineto{\pgfqpoint{0.803294in}{1.403599in}}%
\pgfpathclose%
\pgfusepath{stroke,fill}%
\end{pgfscope}%
\begin{pgfscope}%
\pgfpathrectangle{\pgfqpoint{0.506010in}{1.121191in}}{\pgfqpoint{2.325000in}{1.400000in}} %
\pgfusepath{clip}%
\pgfsetbuttcap%
\pgfsetroundjoin%
\definecolor{currentfill}{rgb}{0.750000,0.750000,0.000000}%
\pgfsetfillcolor{currentfill}%
\pgfsetlinewidth{1.003750pt}%
\definecolor{currentstroke}{rgb}{0.750000,0.750000,0.000000}%
\pgfsetstrokecolor{currentstroke}%
\pgfsetdash{}{0pt}%
\pgfpathmoveto{\pgfqpoint{1.443680in}{1.409698in}}%
\pgfpathlineto{\pgfqpoint{1.462314in}{1.440755in}}%
\pgfpathlineto{\pgfqpoint{1.443680in}{1.471811in}}%
\pgfpathlineto{\pgfqpoint{1.425046in}{1.440755in}}%
\pgfpathclose%
\pgfusepath{stroke,fill}%
\end{pgfscope}%
\begin{pgfscope}%
\pgfpathrectangle{\pgfqpoint{0.506010in}{1.121191in}}{\pgfqpoint{2.325000in}{1.400000in}} %
\pgfusepath{clip}%
\pgfsetbuttcap%
\pgfsetroundjoin%
\definecolor{currentfill}{rgb}{0.750000,0.750000,0.000000}%
\pgfsetfillcolor{currentfill}%
\pgfsetlinewidth{1.003750pt}%
\definecolor{currentstroke}{rgb}{0.750000,0.750000,0.000000}%
\pgfsetstrokecolor{currentstroke}%
\pgfsetdash{}{0pt}%
\pgfpathmoveto{\pgfqpoint{1.379114in}{1.464970in}}%
\pgfpathlineto{\pgfqpoint{1.397747in}{1.496027in}}%
\pgfpathlineto{\pgfqpoint{1.379114in}{1.527083in}}%
\pgfpathlineto{\pgfqpoint{1.360480in}{1.496027in}}%
\pgfpathclose%
\pgfusepath{stroke,fill}%
\end{pgfscope}%
\begin{pgfscope}%
\pgfpathrectangle{\pgfqpoint{0.506010in}{1.121191in}}{\pgfqpoint{2.325000in}{1.400000in}} %
\pgfusepath{clip}%
\pgfsetbuttcap%
\pgfsetroundjoin%
\definecolor{currentfill}{rgb}{0.750000,0.750000,0.000000}%
\pgfsetfillcolor{currentfill}%
\pgfsetlinewidth{1.003750pt}%
\definecolor{currentstroke}{rgb}{0.750000,0.750000,0.000000}%
\pgfsetstrokecolor{currentstroke}%
\pgfsetdash{}{0pt}%
\pgfpathmoveto{\pgfqpoint{1.081382in}{1.523770in}}%
\pgfpathlineto{\pgfqpoint{1.100016in}{1.554827in}}%
\pgfpathlineto{\pgfqpoint{1.081382in}{1.585883in}}%
\pgfpathlineto{\pgfqpoint{1.062748in}{1.554827in}}%
\pgfpathclose%
\pgfusepath{stroke,fill}%
\end{pgfscope}%
\begin{pgfscope}%
\pgfpathrectangle{\pgfqpoint{0.506010in}{1.121191in}}{\pgfqpoint{2.325000in}{1.400000in}} %
\pgfusepath{clip}%
\pgfsetbuttcap%
\pgfsetroundjoin%
\definecolor{currentfill}{rgb}{0.750000,0.750000,0.000000}%
\pgfsetfillcolor{currentfill}%
\pgfsetlinewidth{1.003750pt}%
\definecolor{currentstroke}{rgb}{0.750000,0.750000,0.000000}%
\pgfsetstrokecolor{currentstroke}%
\pgfsetdash{}{0pt}%
\pgfpathmoveto{\pgfqpoint{1.108974in}{1.616758in}}%
\pgfpathlineto{\pgfqpoint{1.127608in}{1.647815in}}%
\pgfpathlineto{\pgfqpoint{1.108974in}{1.678871in}}%
\pgfpathlineto{\pgfqpoint{1.090340in}{1.647815in}}%
\pgfpathclose%
\pgfusepath{stroke,fill}%
\end{pgfscope}%
\begin{pgfscope}%
\pgfpathrectangle{\pgfqpoint{0.506010in}{1.121191in}}{\pgfqpoint{2.325000in}{1.400000in}} %
\pgfusepath{clip}%
\pgfsetbuttcap%
\pgfsetroundjoin%
\definecolor{currentfill}{rgb}{0.750000,0.750000,0.000000}%
\pgfsetfillcolor{currentfill}%
\pgfsetlinewidth{1.003750pt}%
\definecolor{currentstroke}{rgb}{0.750000,0.750000,0.000000}%
\pgfsetstrokecolor{currentstroke}%
\pgfsetdash{}{0pt}%
\pgfpathmoveto{\pgfqpoint{0.681255in}{1.637702in}}%
\pgfpathlineto{\pgfqpoint{0.699889in}{1.668759in}}%
\pgfpathlineto{\pgfqpoint{0.681255in}{1.699815in}}%
\pgfpathlineto{\pgfqpoint{0.662621in}{1.668759in}}%
\pgfpathclose%
\pgfusepath{stroke,fill}%
\end{pgfscope}%
\begin{pgfscope}%
\pgfpathrectangle{\pgfqpoint{0.506010in}{1.121191in}}{\pgfqpoint{2.325000in}{1.400000in}} %
\pgfusepath{clip}%
\pgfsetbuttcap%
\pgfsetroundjoin%
\definecolor{currentfill}{rgb}{0.750000,0.750000,0.000000}%
\pgfsetfillcolor{currentfill}%
\pgfsetlinewidth{1.003750pt}%
\definecolor{currentstroke}{rgb}{0.750000,0.750000,0.000000}%
\pgfsetstrokecolor{currentstroke}%
\pgfsetdash{}{0pt}%
\pgfpathmoveto{\pgfqpoint{1.081382in}{1.757458in}}%
\pgfpathlineto{\pgfqpoint{1.100016in}{1.788515in}}%
\pgfpathlineto{\pgfqpoint{1.081382in}{1.819571in}}%
\pgfpathlineto{\pgfqpoint{1.062748in}{1.788515in}}%
\pgfpathclose%
\pgfusepath{stroke,fill}%
\end{pgfscope}%
\begin{pgfscope}%
\pgfpathrectangle{\pgfqpoint{0.506010in}{1.121191in}}{\pgfqpoint{2.325000in}{1.400000in}} %
\pgfusepath{clip}%
\pgfsetbuttcap%
\pgfsetroundjoin%
\definecolor{currentfill}{rgb}{0.750000,0.750000,0.000000}%
\pgfsetfillcolor{currentfill}%
\pgfsetlinewidth{1.003750pt}%
\definecolor{currentstroke}{rgb}{0.750000,0.750000,0.000000}%
\pgfsetstrokecolor{currentstroke}%
\pgfsetdash{}{0pt}%
\pgfpathmoveto{\pgfqpoint{1.081382in}{6.131646in}}%
\pgfpathclose%
\pgfusepath{stroke,fill}%
\end{pgfscope}%
\begin{pgfscope}%
\pgfpathrectangle{\pgfqpoint{0.506010in}{1.121191in}}{\pgfqpoint{2.325000in}{1.400000in}} %
\pgfusepath{clip}%
\pgfsetbuttcap%
\pgfsetroundjoin%
\definecolor{currentfill}{rgb}{0.750000,0.750000,0.000000}%
\pgfsetfillcolor{currentfill}%
\pgfsetlinewidth{1.003750pt}%
\definecolor{currentstroke}{rgb}{0.750000,0.750000,0.000000}%
\pgfsetstrokecolor{currentstroke}%
\pgfsetdash{}{0pt}%
\pgfpathmoveto{\pgfqpoint{1.007822in}{6.131674in}}%
\pgfpathclose%
\pgfusepath{stroke,fill}%
\end{pgfscope}%
\begin{pgfscope}%
\pgfpathrectangle{\pgfqpoint{0.506010in}{1.121191in}}{\pgfqpoint{2.325000in}{1.400000in}} %
\pgfusepath{clip}%
\pgfsetbuttcap%
\pgfsetroundjoin%
\definecolor{currentfill}{rgb}{0.750000,0.750000,0.000000}%
\pgfsetfillcolor{currentfill}%
\pgfsetlinewidth{1.003750pt}%
\definecolor{currentstroke}{rgb}{0.750000,0.750000,0.000000}%
\pgfsetstrokecolor{currentstroke}%
\pgfsetdash{}{0pt}%
\pgfpathmoveto{\pgfqpoint{0.986187in}{6.134166in}}%
\pgfpathclose%
\pgfusepath{stroke,fill}%
\end{pgfscope}%
\begin{pgfscope}%
\pgfpathrectangle{\pgfqpoint{0.506010in}{1.121191in}}{\pgfqpoint{2.325000in}{1.400000in}} %
\pgfusepath{clip}%
\pgfsetbuttcap%
\pgfsetroundjoin%
\definecolor{currentfill}{rgb}{0.750000,0.750000,0.000000}%
\pgfsetfillcolor{currentfill}%
\pgfsetlinewidth{1.003750pt}%
\definecolor{currentstroke}{rgb}{0.750000,0.750000,0.000000}%
\pgfsetstrokecolor{currentstroke}%
\pgfsetdash{}{0pt}%
\pgfpathmoveto{\pgfqpoint{1.108974in}{6.134166in}}%
\pgfpathclose%
\pgfusepath{stroke,fill}%
\end{pgfscope}%
\begin{pgfscope}%
\pgfpathrectangle{\pgfqpoint{0.506010in}{1.121191in}}{\pgfqpoint{2.325000in}{1.400000in}} %
\pgfusepath{clip}%
\pgfsetbuttcap%
\pgfsetroundjoin%
\definecolor{currentfill}{rgb}{0.750000,0.750000,0.000000}%
\pgfsetfillcolor{currentfill}%
\pgfsetlinewidth{1.003750pt}%
\definecolor{currentstroke}{rgb}{0.750000,0.750000,0.000000}%
\pgfsetstrokecolor{currentstroke}%
\pgfsetdash{}{0pt}%
\pgfpathmoveto{\pgfqpoint{1.081382in}{6.134194in}}%
\pgfpathclose%
\pgfusepath{stroke,fill}%
\end{pgfscope}%
\begin{pgfscope}%
\pgfpathrectangle{\pgfqpoint{0.506010in}{1.121191in}}{\pgfqpoint{2.325000in}{1.400000in}} %
\pgfusepath{clip}%
\pgfsetbuttcap%
\pgfsetroundjoin%
\definecolor{currentfill}{rgb}{0.750000,0.750000,0.000000}%
\pgfsetfillcolor{currentfill}%
\pgfsetlinewidth{1.003750pt}%
\definecolor{currentstroke}{rgb}{0.750000,0.750000,0.000000}%
\pgfsetstrokecolor{currentstroke}%
\pgfsetdash{}{0pt}%
\pgfpathmoveto{\pgfqpoint{1.474029in}{6.134250in}}%
\pgfpathclose%
\pgfusepath{stroke,fill}%
\end{pgfscope}%
\begin{pgfscope}%
\pgfpathrectangle{\pgfqpoint{0.506010in}{1.121191in}}{\pgfqpoint{2.325000in}{1.400000in}} %
\pgfusepath{clip}%
\pgfsetbuttcap%
\pgfsetroundjoin%
\definecolor{currentfill}{rgb}{0.750000,0.750000,0.000000}%
\pgfsetfillcolor{currentfill}%
\pgfsetlinewidth{1.003750pt}%
\definecolor{currentstroke}{rgb}{0.750000,0.750000,0.000000}%
\pgfsetstrokecolor{currentstroke}%
\pgfsetdash{}{0pt}%
\pgfpathmoveto{\pgfqpoint{1.007822in}{6.134250in}}%
\pgfpathclose%
\pgfusepath{stroke,fill}%
\end{pgfscope}%
\begin{pgfscope}%
\pgfpathrectangle{\pgfqpoint{0.506010in}{1.121191in}}{\pgfqpoint{2.325000in}{1.400000in}} %
\pgfusepath{clip}%
\pgfsetbuttcap%
\pgfsetroundjoin%
\definecolor{currentfill}{rgb}{0.750000,0.750000,0.000000}%
\pgfsetfillcolor{currentfill}%
\pgfsetlinewidth{1.003750pt}%
\definecolor{currentstroke}{rgb}{0.750000,0.750000,0.000000}%
\pgfsetstrokecolor{currentstroke}%
\pgfsetdash{}{0pt}%
\pgfpathmoveto{\pgfqpoint{1.081382in}{6.134250in}}%
\pgfpathclose%
\pgfusepath{stroke,fill}%
\end{pgfscope}%
\begin{pgfscope}%
\pgfpathrectangle{\pgfqpoint{0.506010in}{1.121191in}}{\pgfqpoint{2.325000in}{1.400000in}} %
\pgfusepath{clip}%
\pgfsetbuttcap%
\pgfsetroundjoin%
\definecolor{currentfill}{rgb}{0.750000,0.750000,0.000000}%
\pgfsetfillcolor{currentfill}%
\pgfsetlinewidth{1.003750pt}%
\definecolor{currentstroke}{rgb}{0.750000,0.750000,0.000000}%
\pgfsetstrokecolor{currentstroke}%
\pgfsetdash{}{0pt}%
\pgfpathmoveto{\pgfqpoint{1.081382in}{6.134278in}}%
\pgfpathclose%
\pgfusepath{stroke,fill}%
\end{pgfscope}%
\begin{pgfscope}%
\pgfpathrectangle{\pgfqpoint{0.506010in}{1.121191in}}{\pgfqpoint{2.325000in}{1.400000in}} %
\pgfusepath{clip}%
\pgfsetbuttcap%
\pgfsetroundjoin%
\definecolor{currentfill}{rgb}{0.750000,0.750000,0.000000}%
\pgfsetfillcolor{currentfill}%
\pgfsetlinewidth{1.003750pt}%
\definecolor{currentstroke}{rgb}{0.750000,0.750000,0.000000}%
\pgfsetstrokecolor{currentstroke}%
\pgfsetdash{}{0pt}%
\pgfpathmoveto{\pgfqpoint{1.081382in}{6.134278in}}%
\pgfpathclose%
\pgfusepath{stroke,fill}%
\end{pgfscope}%
\begin{pgfscope}%
\pgfpathrectangle{\pgfqpoint{0.506010in}{1.121191in}}{\pgfqpoint{2.325000in}{1.400000in}} %
\pgfusepath{clip}%
\pgfsetbuttcap%
\pgfsetroundjoin%
\definecolor{currentfill}{rgb}{0.750000,0.750000,0.000000}%
\pgfsetfillcolor{currentfill}%
\pgfsetlinewidth{1.003750pt}%
\definecolor{currentstroke}{rgb}{0.750000,0.750000,0.000000}%
\pgfsetstrokecolor{currentstroke}%
\pgfsetdash{}{0pt}%
\pgfpathmoveto{\pgfqpoint{1.007822in}{6.134278in}}%
\pgfpathclose%
\pgfusepath{stroke,fill}%
\end{pgfscope}%
\begin{pgfscope}%
\pgfpathrectangle{\pgfqpoint{0.506010in}{1.121191in}}{\pgfqpoint{2.325000in}{1.400000in}} %
\pgfusepath{clip}%
\pgfsetbuttcap%
\pgfsetroundjoin%
\definecolor{currentfill}{rgb}{0.750000,0.750000,0.000000}%
\pgfsetfillcolor{currentfill}%
\pgfsetlinewidth{1.003750pt}%
\definecolor{currentstroke}{rgb}{0.750000,0.750000,0.000000}%
\pgfsetstrokecolor{currentstroke}%
\pgfsetdash{}{0pt}%
\pgfpathmoveto{\pgfqpoint{1.007822in}{6.134278in}}%
\pgfpathclose%
\pgfusepath{stroke,fill}%
\end{pgfscope}%
\begin{pgfscope}%
\pgfpathrectangle{\pgfqpoint{0.506010in}{1.121191in}}{\pgfqpoint{2.325000in}{1.400000in}} %
\pgfusepath{clip}%
\pgfsetbuttcap%
\pgfsetroundjoin%
\definecolor{currentfill}{rgb}{0.750000,0.750000,0.000000}%
\pgfsetfillcolor{currentfill}%
\pgfsetlinewidth{1.003750pt}%
\definecolor{currentstroke}{rgb}{0.750000,0.750000,0.000000}%
\pgfsetstrokecolor{currentstroke}%
\pgfsetdash{}{0pt}%
\pgfpathmoveto{\pgfqpoint{1.108974in}{6.134278in}}%
\pgfpathclose%
\pgfusepath{stroke,fill}%
\end{pgfscope}%
\begin{pgfscope}%
\pgfpathrectangle{\pgfqpoint{0.506010in}{1.121191in}}{\pgfqpoint{2.325000in}{1.400000in}} %
\pgfusepath{clip}%
\pgfsetbuttcap%
\pgfsetroundjoin%
\definecolor{currentfill}{rgb}{0.750000,0.750000,0.000000}%
\pgfsetfillcolor{currentfill}%
\pgfsetlinewidth{1.003750pt}%
\definecolor{currentstroke}{rgb}{0.750000,0.750000,0.000000}%
\pgfsetstrokecolor{currentstroke}%
\pgfsetdash{}{0pt}%
\pgfpathmoveto{\pgfqpoint{1.474029in}{6.134278in}}%
\pgfpathclose%
\pgfusepath{stroke,fill}%
\end{pgfscope}%
\begin{pgfscope}%
\pgfpathrectangle{\pgfqpoint{0.506010in}{1.121191in}}{\pgfqpoint{2.325000in}{1.400000in}} %
\pgfusepath{clip}%
\pgfsetbuttcap%
\pgfsetroundjoin%
\definecolor{currentfill}{rgb}{0.750000,0.750000,0.000000}%
\pgfsetfillcolor{currentfill}%
\pgfsetlinewidth{1.003750pt}%
\definecolor{currentstroke}{rgb}{0.750000,0.750000,0.000000}%
\pgfsetstrokecolor{currentstroke}%
\pgfsetdash{}{0pt}%
\pgfpathmoveto{\pgfqpoint{1.081382in}{6.134306in}}%
\pgfpathclose%
\pgfusepath{stroke,fill}%
\end{pgfscope}%
\begin{pgfscope}%
\pgfpathrectangle{\pgfqpoint{0.506010in}{1.121191in}}{\pgfqpoint{2.325000in}{1.400000in}} %
\pgfusepath{clip}%
\pgfsetbuttcap%
\pgfsetroundjoin%
\definecolor{currentfill}{rgb}{0.750000,0.750000,0.000000}%
\pgfsetfillcolor{currentfill}%
\pgfsetlinewidth{1.003750pt}%
\definecolor{currentstroke}{rgb}{0.750000,0.750000,0.000000}%
\pgfsetstrokecolor{currentstroke}%
\pgfsetdash{}{0pt}%
\pgfpathmoveto{\pgfqpoint{1.081382in}{6.134306in}}%
\pgfpathclose%
\pgfusepath{stroke,fill}%
\end{pgfscope}%
\begin{pgfscope}%
\pgfpathrectangle{\pgfqpoint{0.506010in}{1.121191in}}{\pgfqpoint{2.325000in}{1.400000in}} %
\pgfusepath{clip}%
\pgfsetbuttcap%
\pgfsetroundjoin%
\definecolor{currentfill}{rgb}{0.750000,0.750000,0.000000}%
\pgfsetfillcolor{currentfill}%
\pgfsetlinewidth{1.003750pt}%
\definecolor{currentstroke}{rgb}{0.750000,0.750000,0.000000}%
\pgfsetstrokecolor{currentstroke}%
\pgfsetdash{}{0pt}%
\pgfpathmoveto{\pgfqpoint{1.007822in}{6.134306in}}%
\pgfpathclose%
\pgfusepath{stroke,fill}%
\end{pgfscope}%
\begin{pgfscope}%
\pgfpathrectangle{\pgfqpoint{0.506010in}{1.121191in}}{\pgfqpoint{2.325000in}{1.400000in}} %
\pgfusepath{clip}%
\pgfsetbuttcap%
\pgfsetroundjoin%
\definecolor{currentfill}{rgb}{0.750000,0.750000,0.000000}%
\pgfsetfillcolor{currentfill}%
\pgfsetlinewidth{1.003750pt}%
\definecolor{currentstroke}{rgb}{0.750000,0.750000,0.000000}%
\pgfsetstrokecolor{currentstroke}%
\pgfsetdash{}{0pt}%
\pgfpathmoveto{\pgfqpoint{1.108974in}{6.134306in}}%
\pgfpathclose%
\pgfusepath{stroke,fill}%
\end{pgfscope}%
\begin{pgfscope}%
\pgfpathrectangle{\pgfqpoint{0.506010in}{1.121191in}}{\pgfqpoint{2.325000in}{1.400000in}} %
\pgfusepath{clip}%
\pgfsetbuttcap%
\pgfsetroundjoin%
\definecolor{currentfill}{rgb}{0.750000,0.750000,0.000000}%
\pgfsetfillcolor{currentfill}%
\pgfsetlinewidth{1.003750pt}%
\definecolor{currentstroke}{rgb}{0.750000,0.750000,0.000000}%
\pgfsetstrokecolor{currentstroke}%
\pgfsetdash{}{0pt}%
\pgfpathmoveto{\pgfqpoint{1.081382in}{6.134334in}}%
\pgfpathclose%
\pgfusepath{stroke,fill}%
\end{pgfscope}%
\begin{pgfscope}%
\pgfpathrectangle{\pgfqpoint{0.506010in}{1.121191in}}{\pgfqpoint{2.325000in}{1.400000in}} %
\pgfusepath{clip}%
\pgfsetbuttcap%
\pgfsetroundjoin%
\definecolor{currentfill}{rgb}{0.750000,0.750000,0.000000}%
\pgfsetfillcolor{currentfill}%
\pgfsetlinewidth{1.003750pt}%
\definecolor{currentstroke}{rgb}{0.750000,0.750000,0.000000}%
\pgfsetstrokecolor{currentstroke}%
\pgfsetdash{}{0pt}%
\pgfpathmoveto{\pgfqpoint{1.007822in}{6.134362in}}%
\pgfpathclose%
\pgfusepath{stroke,fill}%
\end{pgfscope}%
\begin{pgfscope}%
\pgfpathrectangle{\pgfqpoint{0.506010in}{1.121191in}}{\pgfqpoint{2.325000in}{1.400000in}} %
\pgfusepath{clip}%
\pgfsetbuttcap%
\pgfsetroundjoin%
\definecolor{currentfill}{rgb}{0.750000,0.750000,0.000000}%
\pgfsetfillcolor{currentfill}%
\pgfsetlinewidth{1.003750pt}%
\definecolor{currentstroke}{rgb}{0.750000,0.750000,0.000000}%
\pgfsetstrokecolor{currentstroke}%
\pgfsetdash{}{0pt}%
\pgfpathmoveto{\pgfqpoint{1.081382in}{6.134362in}}%
\pgfpathclose%
\pgfusepath{stroke,fill}%
\end{pgfscope}%
\begin{pgfscope}%
\pgfpathrectangle{\pgfqpoint{0.506010in}{1.121191in}}{\pgfqpoint{2.325000in}{1.400000in}} %
\pgfusepath{clip}%
\pgfsetbuttcap%
\pgfsetroundjoin%
\definecolor{currentfill}{rgb}{0.750000,0.750000,0.000000}%
\pgfsetfillcolor{currentfill}%
\pgfsetlinewidth{1.003750pt}%
\definecolor{currentstroke}{rgb}{0.750000,0.750000,0.000000}%
\pgfsetstrokecolor{currentstroke}%
\pgfsetdash{}{0pt}%
\pgfpathmoveto{\pgfqpoint{1.081382in}{6.134362in}}%
\pgfpathclose%
\pgfusepath{stroke,fill}%
\end{pgfscope}%
\begin{pgfscope}%
\pgfpathrectangle{\pgfqpoint{0.506010in}{1.121191in}}{\pgfqpoint{2.325000in}{1.400000in}} %
\pgfusepath{clip}%
\pgfsetbuttcap%
\pgfsetroundjoin%
\definecolor{currentfill}{rgb}{0.750000,0.750000,0.000000}%
\pgfsetfillcolor{currentfill}%
\pgfsetlinewidth{1.003750pt}%
\definecolor{currentstroke}{rgb}{0.750000,0.750000,0.000000}%
\pgfsetstrokecolor{currentstroke}%
\pgfsetdash{}{0pt}%
\pgfpathmoveto{\pgfqpoint{1.081382in}{6.134390in}}%
\pgfpathclose%
\pgfusepath{stroke,fill}%
\end{pgfscope}%
\begin{pgfscope}%
\pgfpathrectangle{\pgfqpoint{0.506010in}{1.121191in}}{\pgfqpoint{2.325000in}{1.400000in}} %
\pgfusepath{clip}%
\pgfsetbuttcap%
\pgfsetroundjoin%
\definecolor{currentfill}{rgb}{0.750000,0.750000,0.000000}%
\pgfsetfillcolor{currentfill}%
\pgfsetlinewidth{1.003750pt}%
\definecolor{currentstroke}{rgb}{0.750000,0.750000,0.000000}%
\pgfsetstrokecolor{currentstroke}%
\pgfsetdash{}{0pt}%
\pgfpathmoveto{\pgfqpoint{1.081382in}{6.134390in}}%
\pgfpathclose%
\pgfusepath{stroke,fill}%
\end{pgfscope}%
\begin{pgfscope}%
\pgfpathrectangle{\pgfqpoint{0.506010in}{1.121191in}}{\pgfqpoint{2.325000in}{1.400000in}} %
\pgfusepath{clip}%
\pgfsetbuttcap%
\pgfsetroundjoin%
\definecolor{currentfill}{rgb}{0.750000,0.750000,0.000000}%
\pgfsetfillcolor{currentfill}%
\pgfsetlinewidth{1.003750pt}%
\definecolor{currentstroke}{rgb}{0.750000,0.750000,0.000000}%
\pgfsetstrokecolor{currentstroke}%
\pgfsetdash{}{0pt}%
\pgfpathmoveto{\pgfqpoint{1.007822in}{6.134390in}}%
\pgfpathclose%
\pgfusepath{stroke,fill}%
\end{pgfscope}%
\begin{pgfscope}%
\pgfpathrectangle{\pgfqpoint{0.506010in}{1.121191in}}{\pgfqpoint{2.325000in}{1.400000in}} %
\pgfusepath{clip}%
\pgfsetbuttcap%
\pgfsetroundjoin%
\definecolor{currentfill}{rgb}{0.750000,0.750000,0.000000}%
\pgfsetfillcolor{currentfill}%
\pgfsetlinewidth{1.003750pt}%
\definecolor{currentstroke}{rgb}{0.750000,0.750000,0.000000}%
\pgfsetstrokecolor{currentstroke}%
\pgfsetdash{}{0pt}%
\pgfpathmoveto{\pgfqpoint{0.788675in}{6.134390in}}%
\pgfpathclose%
\pgfusepath{stroke,fill}%
\end{pgfscope}%
\begin{pgfscope}%
\pgfpathrectangle{\pgfqpoint{0.506010in}{1.121191in}}{\pgfqpoint{2.325000in}{1.400000in}} %
\pgfusepath{clip}%
\pgfsetbuttcap%
\pgfsetroundjoin%
\definecolor{currentfill}{rgb}{0.750000,0.750000,0.000000}%
\pgfsetfillcolor{currentfill}%
\pgfsetlinewidth{1.003750pt}%
\definecolor{currentstroke}{rgb}{0.750000,0.750000,0.000000}%
\pgfsetstrokecolor{currentstroke}%
\pgfsetdash{}{0pt}%
\pgfpathmoveto{\pgfqpoint{2.480299in}{6.134390in}}%
\pgfpathclose%
\pgfusepath{stroke,fill}%
\end{pgfscope}%
\begin{pgfscope}%
\pgfpathrectangle{\pgfqpoint{0.506010in}{1.121191in}}{\pgfqpoint{2.325000in}{1.400000in}} %
\pgfusepath{clip}%
\pgfsetbuttcap%
\pgfsetroundjoin%
\definecolor{currentfill}{rgb}{0.750000,0.750000,0.000000}%
\pgfsetfillcolor{currentfill}%
\pgfsetlinewidth{1.003750pt}%
\definecolor{currentstroke}{rgb}{0.750000,0.750000,0.000000}%
\pgfsetstrokecolor{currentstroke}%
\pgfsetdash{}{0pt}%
\pgfpathmoveto{\pgfqpoint{1.139222in}{6.134418in}}%
\pgfpathclose%
\pgfusepath{stroke,fill}%
\end{pgfscope}%
\begin{pgfscope}%
\pgfpathrectangle{\pgfqpoint{0.506010in}{1.121191in}}{\pgfqpoint{2.325000in}{1.400000in}} %
\pgfusepath{clip}%
\pgfsetbuttcap%
\pgfsetroundjoin%
\definecolor{currentfill}{rgb}{0.750000,0.750000,0.000000}%
\pgfsetfillcolor{currentfill}%
\pgfsetlinewidth{1.003750pt}%
\definecolor{currentstroke}{rgb}{0.750000,0.750000,0.000000}%
\pgfsetstrokecolor{currentstroke}%
\pgfsetdash{}{0pt}%
\pgfpathmoveto{\pgfqpoint{1.007822in}{6.134418in}}%
\pgfpathclose%
\pgfusepath{stroke,fill}%
\end{pgfscope}%
\begin{pgfscope}%
\pgfpathrectangle{\pgfqpoint{0.506010in}{1.121191in}}{\pgfqpoint{2.325000in}{1.400000in}} %
\pgfusepath{clip}%
\pgfsetbuttcap%
\pgfsetroundjoin%
\definecolor{currentfill}{rgb}{0.750000,0.750000,0.000000}%
\pgfsetfillcolor{currentfill}%
\pgfsetlinewidth{1.003750pt}%
\definecolor{currentstroke}{rgb}{0.750000,0.750000,0.000000}%
\pgfsetstrokecolor{currentstroke}%
\pgfsetdash{}{0pt}%
\pgfpathmoveto{\pgfqpoint{1.007822in}{6.134418in}}%
\pgfpathclose%
\pgfusepath{stroke,fill}%
\end{pgfscope}%
\begin{pgfscope}%
\pgfpathrectangle{\pgfqpoint{0.506010in}{1.121191in}}{\pgfqpoint{2.325000in}{1.400000in}} %
\pgfusepath{clip}%
\pgfsetbuttcap%
\pgfsetroundjoin%
\definecolor{currentfill}{rgb}{0.750000,0.750000,0.000000}%
\pgfsetfillcolor{currentfill}%
\pgfsetlinewidth{1.003750pt}%
\definecolor{currentstroke}{rgb}{0.750000,0.750000,0.000000}%
\pgfsetstrokecolor{currentstroke}%
\pgfsetdash{}{0pt}%
\pgfpathmoveto{\pgfqpoint{1.007822in}{6.134418in}}%
\pgfpathclose%
\pgfusepath{stroke,fill}%
\end{pgfscope}%
\begin{pgfscope}%
\pgfpathrectangle{\pgfqpoint{0.506010in}{1.121191in}}{\pgfqpoint{2.325000in}{1.400000in}} %
\pgfusepath{clip}%
\pgfsetbuttcap%
\pgfsetroundjoin%
\definecolor{currentfill}{rgb}{0.750000,0.750000,0.000000}%
\pgfsetfillcolor{currentfill}%
\pgfsetlinewidth{1.003750pt}%
\definecolor{currentstroke}{rgb}{0.750000,0.750000,0.000000}%
\pgfsetstrokecolor{currentstroke}%
\pgfsetdash{}{0pt}%
\pgfpathmoveto{\pgfqpoint{1.108974in}{6.134446in}}%
\pgfpathclose%
\pgfusepath{stroke,fill}%
\end{pgfscope}%
\begin{pgfscope}%
\pgfpathrectangle{\pgfqpoint{0.506010in}{1.121191in}}{\pgfqpoint{2.325000in}{1.400000in}} %
\pgfusepath{clip}%
\pgfsetbuttcap%
\pgfsetroundjoin%
\definecolor{currentfill}{rgb}{0.750000,0.750000,0.000000}%
\pgfsetfillcolor{currentfill}%
\pgfsetlinewidth{1.003750pt}%
\definecolor{currentstroke}{rgb}{0.750000,0.750000,0.000000}%
\pgfsetstrokecolor{currentstroke}%
\pgfsetdash{}{0pt}%
\pgfpathmoveto{\pgfqpoint{1.108974in}{6.134446in}}%
\pgfpathclose%
\pgfusepath{stroke,fill}%
\end{pgfscope}%
\begin{pgfscope}%
\pgfpathrectangle{\pgfqpoint{0.506010in}{1.121191in}}{\pgfqpoint{2.325000in}{1.400000in}} %
\pgfusepath{clip}%
\pgfsetbuttcap%
\pgfsetroundjoin%
\definecolor{currentfill}{rgb}{0.750000,0.750000,0.000000}%
\pgfsetfillcolor{currentfill}%
\pgfsetlinewidth{1.003750pt}%
\definecolor{currentstroke}{rgb}{0.750000,0.750000,0.000000}%
\pgfsetstrokecolor{currentstroke}%
\pgfsetdash{}{0pt}%
\pgfpathmoveto{\pgfqpoint{1.081382in}{6.134446in}}%
\pgfpathclose%
\pgfusepath{stroke,fill}%
\end{pgfscope}%
\begin{pgfscope}%
\pgfpathrectangle{\pgfqpoint{0.506010in}{1.121191in}}{\pgfqpoint{2.325000in}{1.400000in}} %
\pgfusepath{clip}%
\pgfsetbuttcap%
\pgfsetroundjoin%
\definecolor{currentfill}{rgb}{0.750000,0.750000,0.000000}%
\pgfsetfillcolor{currentfill}%
\pgfsetlinewidth{1.003750pt}%
\definecolor{currentstroke}{rgb}{0.750000,0.750000,0.000000}%
\pgfsetstrokecolor{currentstroke}%
\pgfsetdash{}{0pt}%
\pgfpathmoveto{\pgfqpoint{0.820559in}{6.134446in}}%
\pgfpathclose%
\pgfusepath{stroke,fill}%
\end{pgfscope}%
\begin{pgfscope}%
\pgfpathrectangle{\pgfqpoint{0.506010in}{1.121191in}}{\pgfqpoint{2.325000in}{1.400000in}} %
\pgfusepath{clip}%
\pgfsetbuttcap%
\pgfsetroundjoin%
\definecolor{currentfill}{rgb}{0.750000,0.750000,0.000000}%
\pgfsetfillcolor{currentfill}%
\pgfsetlinewidth{1.003750pt}%
\definecolor{currentstroke}{rgb}{0.750000,0.750000,0.000000}%
\pgfsetstrokecolor{currentstroke}%
\pgfsetdash{}{0pt}%
\pgfpathmoveto{\pgfqpoint{1.167543in}{6.134446in}}%
\pgfpathclose%
\pgfusepath{stroke,fill}%
\end{pgfscope}%
\begin{pgfscope}%
\pgfpathrectangle{\pgfqpoint{0.506010in}{1.121191in}}{\pgfqpoint{2.325000in}{1.400000in}} %
\pgfusepath{clip}%
\pgfsetbuttcap%
\pgfsetroundjoin%
\definecolor{currentfill}{rgb}{0.750000,0.750000,0.000000}%
\pgfsetfillcolor{currentfill}%
\pgfsetlinewidth{1.003750pt}%
\definecolor{currentstroke}{rgb}{0.750000,0.750000,0.000000}%
\pgfsetstrokecolor{currentstroke}%
\pgfsetdash{}{0pt}%
\pgfpathmoveto{\pgfqpoint{1.139222in}{6.134446in}}%
\pgfpathclose%
\pgfusepath{stroke,fill}%
\end{pgfscope}%
\begin{pgfscope}%
\pgfpathrectangle{\pgfqpoint{0.506010in}{1.121191in}}{\pgfqpoint{2.325000in}{1.400000in}} %
\pgfusepath{clip}%
\pgfsetbuttcap%
\pgfsetroundjoin%
\definecolor{currentfill}{rgb}{0.750000,0.750000,0.000000}%
\pgfsetfillcolor{currentfill}%
\pgfsetlinewidth{1.003750pt}%
\definecolor{currentstroke}{rgb}{0.750000,0.750000,0.000000}%
\pgfsetstrokecolor{currentstroke}%
\pgfsetdash{}{0pt}%
\pgfpathmoveto{\pgfqpoint{2.191066in}{6.134474in}}%
\pgfpathclose%
\pgfusepath{stroke,fill}%
\end{pgfscope}%
\begin{pgfscope}%
\pgfpathrectangle{\pgfqpoint{0.506010in}{1.121191in}}{\pgfqpoint{2.325000in}{1.400000in}} %
\pgfusepath{clip}%
\pgfsetbuttcap%
\pgfsetroundjoin%
\definecolor{currentfill}{rgb}{0.750000,0.750000,0.000000}%
\pgfsetfillcolor{currentfill}%
\pgfsetlinewidth{1.003750pt}%
\definecolor{currentstroke}{rgb}{0.750000,0.750000,0.000000}%
\pgfsetstrokecolor{currentstroke}%
\pgfsetdash{}{0pt}%
\pgfpathmoveto{\pgfqpoint{2.265364in}{6.134474in}}%
\pgfpathclose%
\pgfusepath{stroke,fill}%
\end{pgfscope}%
\begin{pgfscope}%
\pgfpathrectangle{\pgfqpoint{0.506010in}{1.121191in}}{\pgfqpoint{2.325000in}{1.400000in}} %
\pgfusepath{clip}%
\pgfsetbuttcap%
\pgfsetroundjoin%
\definecolor{currentfill}{rgb}{0.750000,0.750000,0.000000}%
\pgfsetfillcolor{currentfill}%
\pgfsetlinewidth{1.003750pt}%
\definecolor{currentstroke}{rgb}{0.750000,0.750000,0.000000}%
\pgfsetstrokecolor{currentstroke}%
\pgfsetdash{}{0pt}%
\pgfpathmoveto{\pgfqpoint{2.185713in}{6.134474in}}%
\pgfpathclose%
\pgfusepath{stroke,fill}%
\end{pgfscope}%
\begin{pgfscope}%
\pgfpathrectangle{\pgfqpoint{0.506010in}{1.121191in}}{\pgfqpoint{2.325000in}{1.400000in}} %
\pgfusepath{clip}%
\pgfsetbuttcap%
\pgfsetroundjoin%
\definecolor{currentfill}{rgb}{0.750000,0.750000,0.000000}%
\pgfsetfillcolor{currentfill}%
\pgfsetlinewidth{1.003750pt}%
\definecolor{currentstroke}{rgb}{0.750000,0.750000,0.000000}%
\pgfsetstrokecolor{currentstroke}%
\pgfsetdash{}{0pt}%
\pgfpathmoveto{\pgfqpoint{2.217160in}{6.134502in}}%
\pgfpathclose%
\pgfusepath{stroke,fill}%
\end{pgfscope}%
\begin{pgfscope}%
\pgfpathrectangle{\pgfqpoint{0.506010in}{1.121191in}}{\pgfqpoint{2.325000in}{1.400000in}} %
\pgfusepath{clip}%
\pgfsetbuttcap%
\pgfsetroundjoin%
\definecolor{currentfill}{rgb}{0.750000,0.750000,0.000000}%
\pgfsetfillcolor{currentfill}%
\pgfsetlinewidth{1.003750pt}%
\definecolor{currentstroke}{rgb}{0.750000,0.750000,0.000000}%
\pgfsetstrokecolor{currentstroke}%
\pgfsetdash{}{0pt}%
\pgfpathmoveto{\pgfqpoint{2.262181in}{6.134502in}}%
\pgfpathclose%
\pgfusepath{stroke,fill}%
\end{pgfscope}%
\begin{pgfscope}%
\pgfpathrectangle{\pgfqpoint{0.506010in}{1.121191in}}{\pgfqpoint{2.325000in}{1.400000in}} %
\pgfusepath{clip}%
\pgfsetbuttcap%
\pgfsetroundjoin%
\definecolor{currentfill}{rgb}{0.750000,0.750000,0.000000}%
\pgfsetfillcolor{currentfill}%
\pgfsetlinewidth{1.003750pt}%
\definecolor{currentstroke}{rgb}{0.750000,0.750000,0.000000}%
\pgfsetstrokecolor{currentstroke}%
\pgfsetdash{}{0pt}%
\pgfpathmoveto{\pgfqpoint{2.216102in}{6.134530in}}%
\pgfpathclose%
\pgfusepath{stroke,fill}%
\end{pgfscope}%
\begin{pgfscope}%
\pgfpathrectangle{\pgfqpoint{0.506010in}{1.121191in}}{\pgfqpoint{2.325000in}{1.400000in}} %
\pgfusepath{clip}%
\pgfsetbuttcap%
\pgfsetroundjoin%
\definecolor{currentfill}{rgb}{0.750000,0.750000,0.000000}%
\pgfsetfillcolor{currentfill}%
\pgfsetlinewidth{1.003750pt}%
\definecolor{currentstroke}{rgb}{0.750000,0.750000,0.000000}%
\pgfsetstrokecolor{currentstroke}%
\pgfsetdash{}{0pt}%
\pgfpathmoveto{\pgfqpoint{2.234732in}{6.134530in}}%
\pgfpathclose%
\pgfusepath{stroke,fill}%
\end{pgfscope}%
\begin{pgfscope}%
\pgfpathrectangle{\pgfqpoint{0.506010in}{1.121191in}}{\pgfqpoint{2.325000in}{1.400000in}} %
\pgfusepath{clip}%
\pgfsetbuttcap%
\pgfsetroundjoin%
\definecolor{currentfill}{rgb}{0.750000,0.750000,0.000000}%
\pgfsetfillcolor{currentfill}%
\pgfsetlinewidth{1.003750pt}%
\definecolor{currentstroke}{rgb}{0.750000,0.750000,0.000000}%
\pgfsetstrokecolor{currentstroke}%
\pgfsetdash{}{0pt}%
\pgfpathmoveto{\pgfqpoint{2.201136in}{6.134530in}}%
\pgfpathclose%
\pgfusepath{stroke,fill}%
\end{pgfscope}%
\begin{pgfscope}%
\pgfpathrectangle{\pgfqpoint{0.506010in}{1.121191in}}{\pgfqpoint{2.325000in}{1.400000in}} %
\pgfusepath{clip}%
\pgfsetbuttcap%
\pgfsetroundjoin%
\definecolor{currentfill}{rgb}{0.750000,0.750000,0.000000}%
\pgfsetfillcolor{currentfill}%
\pgfsetlinewidth{1.003750pt}%
\definecolor{currentstroke}{rgb}{0.750000,0.750000,0.000000}%
\pgfsetstrokecolor{currentstroke}%
\pgfsetdash{}{0pt}%
\pgfpathmoveto{\pgfqpoint{2.233373in}{6.134558in}}%
\pgfpathclose%
\pgfusepath{stroke,fill}%
\end{pgfscope}%
\begin{pgfscope}%
\pgfpathrectangle{\pgfqpoint{0.506010in}{1.121191in}}{\pgfqpoint{2.325000in}{1.400000in}} %
\pgfusepath{clip}%
\pgfsetbuttcap%
\pgfsetroundjoin%
\definecolor{currentfill}{rgb}{0.750000,0.750000,0.000000}%
\pgfsetfillcolor{currentfill}%
\pgfsetlinewidth{1.003750pt}%
\definecolor{currentstroke}{rgb}{0.750000,0.750000,0.000000}%
\pgfsetstrokecolor{currentstroke}%
\pgfsetdash{}{0pt}%
\pgfpathmoveto{\pgfqpoint{2.222779in}{6.134586in}}%
\pgfpathclose%
\pgfusepath{stroke,fill}%
\end{pgfscope}%
\begin{pgfscope}%
\pgfpathrectangle{\pgfqpoint{0.506010in}{1.121191in}}{\pgfqpoint{2.325000in}{1.400000in}} %
\pgfusepath{clip}%
\pgfsetbuttcap%
\pgfsetroundjoin%
\definecolor{currentfill}{rgb}{0.750000,0.750000,0.000000}%
\pgfsetfillcolor{currentfill}%
\pgfsetlinewidth{1.003750pt}%
\definecolor{currentstroke}{rgb}{0.750000,0.750000,0.000000}%
\pgfsetstrokecolor{currentstroke}%
\pgfsetdash{}{0pt}%
\pgfpathmoveto{\pgfqpoint{2.248922in}{6.134586in}}%
\pgfpathclose%
\pgfusepath{stroke,fill}%
\end{pgfscope}%
\begin{pgfscope}%
\pgfpathrectangle{\pgfqpoint{0.506010in}{1.121191in}}{\pgfqpoint{2.325000in}{1.400000in}} %
\pgfusepath{clip}%
\pgfsetbuttcap%
\pgfsetroundjoin%
\definecolor{currentfill}{rgb}{0.750000,0.750000,0.000000}%
\pgfsetfillcolor{currentfill}%
\pgfsetlinewidth{1.003750pt}%
\definecolor{currentstroke}{rgb}{0.750000,0.750000,0.000000}%
\pgfsetstrokecolor{currentstroke}%
\pgfsetdash{}{0pt}%
\pgfpathmoveto{\pgfqpoint{2.246565in}{6.134586in}}%
\pgfpathclose%
\pgfusepath{stroke,fill}%
\end{pgfscope}%
\begin{pgfscope}%
\pgfpathrectangle{\pgfqpoint{0.506010in}{1.121191in}}{\pgfqpoint{2.325000in}{1.400000in}} %
\pgfusepath{clip}%
\pgfsetbuttcap%
\pgfsetroundjoin%
\definecolor{currentfill}{rgb}{0.750000,0.750000,0.000000}%
\pgfsetfillcolor{currentfill}%
\pgfsetlinewidth{1.003750pt}%
\definecolor{currentstroke}{rgb}{0.750000,0.750000,0.000000}%
\pgfsetstrokecolor{currentstroke}%
\pgfsetdash{}{0pt}%
\pgfpathmoveto{\pgfqpoint{2.233956in}{6.134614in}}%
\pgfpathclose%
\pgfusepath{stroke,fill}%
\end{pgfscope}%
\begin{pgfscope}%
\pgfpathrectangle{\pgfqpoint{0.506010in}{1.121191in}}{\pgfqpoint{2.325000in}{1.400000in}} %
\pgfusepath{clip}%
\pgfsetbuttcap%
\pgfsetroundjoin%
\definecolor{currentfill}{rgb}{0.750000,0.750000,0.000000}%
\pgfsetfillcolor{currentfill}%
\pgfsetlinewidth{1.003750pt}%
\definecolor{currentstroke}{rgb}{0.750000,0.750000,0.000000}%
\pgfsetstrokecolor{currentstroke}%
\pgfsetdash{}{0pt}%
\pgfpathmoveto{\pgfqpoint{2.251609in}{6.134614in}}%
\pgfpathclose%
\pgfusepath{stroke,fill}%
\end{pgfscope}%
\begin{pgfscope}%
\pgfpathrectangle{\pgfqpoint{0.506010in}{1.121191in}}{\pgfqpoint{2.325000in}{1.400000in}} %
\pgfusepath{clip}%
\pgfsetbuttcap%
\pgfsetroundjoin%
\definecolor{currentfill}{rgb}{0.750000,0.750000,0.000000}%
\pgfsetfillcolor{currentfill}%
\pgfsetlinewidth{1.003750pt}%
\definecolor{currentstroke}{rgb}{0.750000,0.750000,0.000000}%
\pgfsetstrokecolor{currentstroke}%
\pgfsetdash{}{0pt}%
\pgfpathmoveto{\pgfqpoint{2.238184in}{6.134614in}}%
\pgfpathclose%
\pgfusepath{stroke,fill}%
\end{pgfscope}%
\begin{pgfscope}%
\pgfpathrectangle{\pgfqpoint{0.506010in}{1.121191in}}{\pgfqpoint{2.325000in}{1.400000in}} %
\pgfusepath{clip}%
\pgfsetbuttcap%
\pgfsetroundjoin%
\definecolor{currentfill}{rgb}{0.750000,0.750000,0.000000}%
\pgfsetfillcolor{currentfill}%
\pgfsetlinewidth{1.003750pt}%
\definecolor{currentstroke}{rgb}{0.750000,0.750000,0.000000}%
\pgfsetstrokecolor{currentstroke}%
\pgfsetdash{}{0pt}%
\pgfpathmoveto{\pgfqpoint{2.248922in}{6.134614in}}%
\pgfpathclose%
\pgfusepath{stroke,fill}%
\end{pgfscope}%
\begin{pgfscope}%
\pgfpathrectangle{\pgfqpoint{0.506010in}{1.121191in}}{\pgfqpoint{2.325000in}{1.400000in}} %
\pgfusepath{clip}%
\pgfsetbuttcap%
\pgfsetroundjoin%
\definecolor{currentfill}{rgb}{0.750000,0.750000,0.000000}%
\pgfsetfillcolor{currentfill}%
\pgfsetlinewidth{1.003750pt}%
\definecolor{currentstroke}{rgb}{0.750000,0.750000,0.000000}%
\pgfsetstrokecolor{currentstroke}%
\pgfsetdash{}{0pt}%
\pgfpathmoveto{\pgfqpoint{2.262855in}{6.134642in}}%
\pgfpathclose%
\pgfusepath{stroke,fill}%
\end{pgfscope}%
\begin{pgfscope}%
\pgfpathrectangle{\pgfqpoint{0.506010in}{1.121191in}}{\pgfqpoint{2.325000in}{1.400000in}} %
\pgfusepath{clip}%
\pgfsetbuttcap%
\pgfsetroundjoin%
\definecolor{currentfill}{rgb}{0.750000,0.750000,0.000000}%
\pgfsetfillcolor{currentfill}%
\pgfsetlinewidth{1.003750pt}%
\definecolor{currentstroke}{rgb}{0.750000,0.750000,0.000000}%
\pgfsetstrokecolor{currentstroke}%
\pgfsetdash{}{0pt}%
\pgfpathmoveto{\pgfqpoint{2.233956in}{6.134642in}}%
\pgfpathclose%
\pgfusepath{stroke,fill}%
\end{pgfscope}%
\begin{pgfscope}%
\pgfpathrectangle{\pgfqpoint{0.506010in}{1.121191in}}{\pgfqpoint{2.325000in}{1.400000in}} %
\pgfusepath{clip}%
\pgfsetbuttcap%
\pgfsetroundjoin%
\definecolor{currentfill}{rgb}{0.750000,0.750000,0.000000}%
\pgfsetfillcolor{currentfill}%
\pgfsetlinewidth{1.003750pt}%
\definecolor{currentstroke}{rgb}{0.750000,0.750000,0.000000}%
\pgfsetstrokecolor{currentstroke}%
\pgfsetdash{}{0pt}%
\pgfpathmoveto{\pgfqpoint{2.217371in}{6.134670in}}%
\pgfpathclose%
\pgfusepath{stroke,fill}%
\end{pgfscope}%
\begin{pgfscope}%
\pgfpathrectangle{\pgfqpoint{0.506010in}{1.121191in}}{\pgfqpoint{2.325000in}{1.400000in}} %
\pgfusepath{clip}%
\pgfsetbuttcap%
\pgfsetroundjoin%
\definecolor{currentfill}{rgb}{0.750000,0.750000,0.000000}%
\pgfsetfillcolor{currentfill}%
\pgfsetlinewidth{1.003750pt}%
\definecolor{currentstroke}{rgb}{0.750000,0.750000,0.000000}%
\pgfsetstrokecolor{currentstroke}%
\pgfsetdash{}{0pt}%
\pgfpathmoveto{\pgfqpoint{2.184228in}{6.134726in}}%
\pgfpathclose%
\pgfusepath{stroke,fill}%
\end{pgfscope}%
\begin{pgfscope}%
\pgfpathrectangle{\pgfqpoint{0.506010in}{1.121191in}}{\pgfqpoint{2.325000in}{1.400000in}} %
\pgfusepath{clip}%
\pgfsetbuttcap%
\pgfsetroundjoin%
\definecolor{currentfill}{rgb}{0.750000,0.750000,0.000000}%
\pgfsetfillcolor{currentfill}%
\pgfsetlinewidth{1.003750pt}%
\definecolor{currentstroke}{rgb}{0.750000,0.750000,0.000000}%
\pgfsetstrokecolor{currentstroke}%
\pgfsetdash{}{0pt}%
\pgfpathmoveto{\pgfqpoint{2.201136in}{6.134726in}}%
\pgfpathclose%
\pgfusepath{stroke,fill}%
\end{pgfscope}%
\begin{pgfscope}%
\pgfpathrectangle{\pgfqpoint{0.506010in}{1.121191in}}{\pgfqpoint{2.325000in}{1.400000in}} %
\pgfusepath{clip}%
\pgfsetbuttcap%
\pgfsetroundjoin%
\definecolor{currentfill}{rgb}{0.000000,0.750000,0.750000}%
\pgfsetfillcolor{currentfill}%
\pgfsetlinewidth{1.003750pt}%
\definecolor{currentstroke}{rgb}{0.000000,0.750000,0.750000}%
\pgfsetstrokecolor{currentstroke}%
\pgfsetdash{}{0pt}%
\pgfpathmoveto{\pgfqpoint{1.151065in}{1.143655in}}%
\pgfpathlineto{\pgfqpoint{1.134833in}{1.111191in}}%
\pgfpathclose%
\pgfusepath{stroke,fill}%
\end{pgfscope}%
\begin{pgfscope}%
\pgfpathrectangle{\pgfqpoint{0.506010in}{1.121191in}}{\pgfqpoint{2.325000in}{1.400000in}} %
\pgfusepath{clip}%
\pgfsetbuttcap%
\pgfsetroundjoin%
\definecolor{currentfill}{rgb}{0.000000,0.750000,0.750000}%
\pgfsetfillcolor{currentfill}%
\pgfsetlinewidth{1.003750pt}%
\definecolor{currentstroke}{rgb}{0.000000,0.750000,0.750000}%
\pgfsetstrokecolor{currentstroke}%
\pgfsetdash{}{0pt}%
\pgfpathmoveto{\pgfqpoint{0.772689in}{1.143711in}}%
\pgfpathlineto{\pgfqpoint{0.756429in}{1.111191in}}%
\pgfpathclose%
\pgfusepath{stroke,fill}%
\end{pgfscope}%
\begin{pgfscope}%
\pgfpathrectangle{\pgfqpoint{0.506010in}{1.121191in}}{\pgfqpoint{2.325000in}{1.400000in}} %
\pgfusepath{clip}%
\pgfsetbuttcap%
\pgfsetroundjoin%
\definecolor{currentfill}{rgb}{0.000000,0.750000,0.750000}%
\pgfsetfillcolor{currentfill}%
\pgfsetlinewidth{1.003750pt}%
\definecolor{currentstroke}{rgb}{0.000000,0.750000,0.750000}%
\pgfsetstrokecolor{currentstroke}%
\pgfsetdash{}{0pt}%
\pgfpathmoveto{\pgfqpoint{1.222833in}{1.143711in}}%
\pgfpathlineto{\pgfqpoint{1.206573in}{1.111191in}}%
\pgfpathclose%
\pgfusepath{stroke,fill}%
\end{pgfscope}%
\begin{pgfscope}%
\pgfpathrectangle{\pgfqpoint{0.506010in}{1.121191in}}{\pgfqpoint{2.325000in}{1.400000in}} %
\pgfusepath{clip}%
\pgfsetbuttcap%
\pgfsetroundjoin%
\definecolor{currentfill}{rgb}{0.000000,0.750000,0.750000}%
\pgfsetfillcolor{currentfill}%
\pgfsetlinewidth{1.003750pt}%
\definecolor{currentstroke}{rgb}{0.000000,0.750000,0.750000}%
\pgfsetstrokecolor{currentstroke}%
\pgfsetdash{}{0pt}%
\pgfpathmoveto{\pgfqpoint{0.980706in}{1.143739in}}%
\pgfpathlineto{\pgfqpoint{0.964431in}{1.111191in}}%
\pgfpathclose%
\pgfusepath{stroke,fill}%
\end{pgfscope}%
\begin{pgfscope}%
\pgfpathrectangle{\pgfqpoint{0.506010in}{1.121191in}}{\pgfqpoint{2.325000in}{1.400000in}} %
\pgfusepath{clip}%
\pgfsetbuttcap%
\pgfsetroundjoin%
\definecolor{currentfill}{rgb}{0.000000,0.750000,0.750000}%
\pgfsetfillcolor{currentfill}%
\pgfsetlinewidth{1.003750pt}%
\definecolor{currentstroke}{rgb}{0.000000,0.750000,0.750000}%
\pgfsetstrokecolor{currentstroke}%
\pgfsetdash{}{0pt}%
\pgfpathmoveto{\pgfqpoint{0.968203in}{1.143739in}}%
\pgfpathlineto{\pgfqpoint{0.951929in}{1.111191in}}%
\pgfpathclose%
\pgfusepath{stroke,fill}%
\end{pgfscope}%
\begin{pgfscope}%
\pgfpathrectangle{\pgfqpoint{0.506010in}{1.121191in}}{\pgfqpoint{2.325000in}{1.400000in}} %
\pgfusepath{clip}%
\pgfsetbuttcap%
\pgfsetroundjoin%
\definecolor{currentfill}{rgb}{0.000000,0.750000,0.750000}%
\pgfsetfillcolor{currentfill}%
\pgfsetlinewidth{1.003750pt}%
\definecolor{currentstroke}{rgb}{0.000000,0.750000,0.750000}%
\pgfsetstrokecolor{currentstroke}%
\pgfsetdash{}{0pt}%
\pgfpathmoveto{\pgfqpoint{1.097148in}{1.143739in}}%
\pgfpathlineto{\pgfqpoint{1.080874in}{1.111191in}}%
\pgfpathclose%
\pgfusepath{stroke,fill}%
\end{pgfscope}%
\begin{pgfscope}%
\pgfpathrectangle{\pgfqpoint{0.506010in}{1.121191in}}{\pgfqpoint{2.325000in}{1.400000in}} %
\pgfusepath{clip}%
\pgfsetbuttcap%
\pgfsetroundjoin%
\definecolor{currentfill}{rgb}{0.000000,0.750000,0.750000}%
\pgfsetfillcolor{currentfill}%
\pgfsetlinewidth{1.003750pt}%
\definecolor{currentstroke}{rgb}{0.000000,0.750000,0.750000}%
\pgfsetstrokecolor{currentstroke}%
\pgfsetdash{}{0pt}%
\pgfpathmoveto{\pgfqpoint{0.813731in}{1.143767in}}%
\pgfpathlineto{\pgfqpoint{0.797443in}{1.111191in}}%
\pgfpathclose%
\pgfusepath{stroke,fill}%
\end{pgfscope}%
\begin{pgfscope}%
\pgfpathrectangle{\pgfqpoint{0.506010in}{1.121191in}}{\pgfqpoint{2.325000in}{1.400000in}} %
\pgfusepath{clip}%
\pgfsetbuttcap%
\pgfsetroundjoin%
\definecolor{currentfill}{rgb}{0.000000,0.750000,0.750000}%
\pgfsetfillcolor{currentfill}%
\pgfsetlinewidth{1.003750pt}%
\definecolor{currentstroke}{rgb}{0.000000,0.750000,0.750000}%
\pgfsetstrokecolor{currentstroke}%
\pgfsetdash{}{0pt}%
\pgfpathmoveto{\pgfqpoint{1.308740in}{1.143851in}}%
\pgfpathlineto{\pgfqpoint{1.292410in}{1.111191in}}%
\pgfpathclose%
\pgfusepath{stroke,fill}%
\end{pgfscope}%
\begin{pgfscope}%
\pgfpathrectangle{\pgfqpoint{0.506010in}{1.121191in}}{\pgfqpoint{2.325000in}{1.400000in}} %
\pgfusepath{clip}%
\pgfsetbuttcap%
\pgfsetroundjoin%
\definecolor{currentfill}{rgb}{0.000000,0.750000,0.750000}%
\pgfsetfillcolor{currentfill}%
\pgfsetlinewidth{1.003750pt}%
\definecolor{currentstroke}{rgb}{0.000000,0.750000,0.750000}%
\pgfsetstrokecolor{currentstroke}%
\pgfsetdash{}{0pt}%
\pgfpathmoveto{\pgfqpoint{0.846633in}{1.143879in}}%
\pgfpathlineto{\pgfqpoint{0.830289in}{1.111191in}}%
\pgfpathclose%
\pgfusepath{stroke,fill}%
\end{pgfscope}%
\begin{pgfscope}%
\pgfpathrectangle{\pgfqpoint{0.506010in}{1.121191in}}{\pgfqpoint{2.325000in}{1.400000in}} %
\pgfusepath{clip}%
\pgfsetbuttcap%
\pgfsetroundjoin%
\definecolor{currentfill}{rgb}{0.000000,0.750000,0.750000}%
\pgfsetfillcolor{currentfill}%
\pgfsetlinewidth{1.003750pt}%
\definecolor{currentstroke}{rgb}{0.000000,0.750000,0.750000}%
\pgfsetstrokecolor{currentstroke}%
\pgfsetdash{}{0pt}%
\pgfpathmoveto{\pgfqpoint{1.144617in}{1.143879in}}%
\pgfpathlineto{\pgfqpoint{1.128273in}{1.111191in}}%
\pgfpathclose%
\pgfusepath{stroke,fill}%
\end{pgfscope}%
\begin{pgfscope}%
\pgfpathrectangle{\pgfqpoint{0.506010in}{1.121191in}}{\pgfqpoint{2.325000in}{1.400000in}} %
\pgfusepath{clip}%
\pgfsetbuttcap%
\pgfsetroundjoin%
\definecolor{currentfill}{rgb}{0.000000,0.750000,0.750000}%
\pgfsetfillcolor{currentfill}%
\pgfsetlinewidth{1.003750pt}%
\definecolor{currentstroke}{rgb}{0.000000,0.750000,0.750000}%
\pgfsetstrokecolor{currentstroke}%
\pgfsetdash{}{0pt}%
\pgfpathmoveto{\pgfqpoint{1.165504in}{1.143879in}}%
\pgfpathlineto{\pgfqpoint{1.149160in}{1.111191in}}%
\pgfpathclose%
\pgfusepath{stroke,fill}%
\end{pgfscope}%
\begin{pgfscope}%
\pgfpathrectangle{\pgfqpoint{0.506010in}{1.121191in}}{\pgfqpoint{2.325000in}{1.400000in}} %
\pgfusepath{clip}%
\pgfsetbuttcap%
\pgfsetroundjoin%
\definecolor{currentfill}{rgb}{0.000000,0.750000,0.750000}%
\pgfsetfillcolor{currentfill}%
\pgfsetlinewidth{1.003750pt}%
\definecolor{currentstroke}{rgb}{0.000000,0.750000,0.750000}%
\pgfsetstrokecolor{currentstroke}%
\pgfsetdash{}{0pt}%
\pgfpathmoveto{\pgfqpoint{0.788475in}{1.143879in}}%
\pgfpathlineto{\pgfqpoint{0.772131in}{1.111191in}}%
\pgfpathclose%
\pgfusepath{stroke,fill}%
\end{pgfscope}%
\begin{pgfscope}%
\pgfpathrectangle{\pgfqpoint{0.506010in}{1.121191in}}{\pgfqpoint{2.325000in}{1.400000in}} %
\pgfusepath{clip}%
\pgfsetbuttcap%
\pgfsetroundjoin%
\definecolor{currentfill}{rgb}{0.000000,0.750000,0.750000}%
\pgfsetfillcolor{currentfill}%
\pgfsetlinewidth{1.003750pt}%
\definecolor{currentstroke}{rgb}{0.000000,0.750000,0.750000}%
\pgfsetstrokecolor{currentstroke}%
\pgfsetdash{}{0pt}%
\pgfpathmoveto{\pgfqpoint{1.587679in}{1.143907in}}%
\pgfpathlineto{\pgfqpoint{1.571321in}{1.111191in}}%
\pgfpathclose%
\pgfusepath{stroke,fill}%
\end{pgfscope}%
\begin{pgfscope}%
\pgfpathrectangle{\pgfqpoint{0.506010in}{1.121191in}}{\pgfqpoint{2.325000in}{1.400000in}} %
\pgfusepath{clip}%
\pgfsetbuttcap%
\pgfsetroundjoin%
\definecolor{currentfill}{rgb}{0.000000,0.750000,0.750000}%
\pgfsetfillcolor{currentfill}%
\pgfsetlinewidth{1.003750pt}%
\definecolor{currentstroke}{rgb}{0.000000,0.750000,0.750000}%
\pgfsetstrokecolor{currentstroke}%
\pgfsetdash{}{0pt}%
\pgfpathmoveto{\pgfqpoint{1.366286in}{1.144159in}}%
\pgfpathlineto{\pgfqpoint{1.349802in}{1.111191in}}%
\pgfpathclose%
\pgfusepath{stroke,fill}%
\end{pgfscope}%
\begin{pgfscope}%
\pgfpathrectangle{\pgfqpoint{0.506010in}{1.121191in}}{\pgfqpoint{2.325000in}{1.400000in}} %
\pgfusepath{clip}%
\pgfsetbuttcap%
\pgfsetroundjoin%
\definecolor{currentfill}{rgb}{0.000000,0.750000,0.750000}%
\pgfsetfillcolor{currentfill}%
\pgfsetlinewidth{1.003750pt}%
\definecolor{currentstroke}{rgb}{0.000000,0.750000,0.750000}%
\pgfsetstrokecolor{currentstroke}%
\pgfsetdash{}{0pt}%
\pgfpathmoveto{\pgfqpoint{1.279201in}{1.144187in}}%
\pgfpathlineto{\pgfqpoint{1.262702in}{1.111191in}}%
\pgfpathclose%
\pgfusepath{stroke,fill}%
\end{pgfscope}%
\begin{pgfscope}%
\pgfpathrectangle{\pgfqpoint{0.506010in}{1.121191in}}{\pgfqpoint{2.325000in}{1.400000in}} %
\pgfusepath{clip}%
\pgfsetbuttcap%
\pgfsetroundjoin%
\definecolor{currentfill}{rgb}{0.000000,0.750000,0.750000}%
\pgfsetfillcolor{currentfill}%
\pgfsetlinewidth{1.003750pt}%
\definecolor{currentstroke}{rgb}{0.000000,0.750000,0.750000}%
\pgfsetstrokecolor{currentstroke}%
\pgfsetdash{}{0pt}%
\pgfpathmoveto{\pgfqpoint{2.264030in}{1.144215in}}%
\pgfpathlineto{\pgfqpoint{2.247518in}{1.111191in}}%
\pgfpathclose%
\pgfusepath{stroke,fill}%
\end{pgfscope}%
\begin{pgfscope}%
\pgfpathrectangle{\pgfqpoint{0.506010in}{1.121191in}}{\pgfqpoint{2.325000in}{1.400000in}} %
\pgfusepath{clip}%
\pgfsetbuttcap%
\pgfsetroundjoin%
\definecolor{currentfill}{rgb}{0.000000,0.750000,0.750000}%
\pgfsetfillcolor{currentfill}%
\pgfsetlinewidth{1.003750pt}%
\definecolor{currentstroke}{rgb}{0.000000,0.750000,0.750000}%
\pgfsetstrokecolor{currentstroke}%
\pgfsetdash{}{0pt}%
\pgfpathmoveto{\pgfqpoint{2.191001in}{1.144215in}}%
\pgfpathlineto{\pgfqpoint{2.174489in}{1.111191in}}%
\pgfpathclose%
\pgfusepath{stroke,fill}%
\end{pgfscope}%
\begin{pgfscope}%
\pgfpathrectangle{\pgfqpoint{0.506010in}{1.121191in}}{\pgfqpoint{2.325000in}{1.400000in}} %
\pgfusepath{clip}%
\pgfsetbuttcap%
\pgfsetroundjoin%
\definecolor{currentfill}{rgb}{0.000000,0.750000,0.750000}%
\pgfsetfillcolor{currentfill}%
\pgfsetlinewidth{1.003750pt}%
\definecolor{currentstroke}{rgb}{0.000000,0.750000,0.750000}%
\pgfsetstrokecolor{currentstroke}%
\pgfsetdash{}{0pt}%
\pgfpathmoveto{\pgfqpoint{2.263023in}{1.144215in}}%
\pgfpathlineto{\pgfqpoint{2.246511in}{1.111191in}}%
\pgfpathclose%
\pgfusepath{stroke,fill}%
\end{pgfscope}%
\begin{pgfscope}%
\pgfpathrectangle{\pgfqpoint{0.506010in}{1.121191in}}{\pgfqpoint{2.325000in}{1.400000in}} %
\pgfusepath{clip}%
\pgfsetbuttcap%
\pgfsetroundjoin%
\definecolor{currentfill}{rgb}{0.000000,0.750000,0.750000}%
\pgfsetfillcolor{currentfill}%
\pgfsetlinewidth{1.003750pt}%
\definecolor{currentstroke}{rgb}{0.000000,0.750000,0.750000}%
\pgfsetstrokecolor{currentstroke}%
\pgfsetdash{}{0pt}%
\pgfpathmoveto{\pgfqpoint{2.251667in}{1.144243in}}%
\pgfpathlineto{\pgfqpoint{2.235141in}{1.111191in}}%
\pgfpathclose%
\pgfusepath{stroke,fill}%
\end{pgfscope}%
\begin{pgfscope}%
\pgfpathrectangle{\pgfqpoint{0.506010in}{1.121191in}}{\pgfqpoint{2.325000in}{1.400000in}} %
\pgfusepath{clip}%
\pgfsetbuttcap%
\pgfsetroundjoin%
\definecolor{currentfill}{rgb}{0.000000,0.750000,0.750000}%
\pgfsetfillcolor{currentfill}%
\pgfsetlinewidth{1.003750pt}%
\definecolor{currentstroke}{rgb}{0.000000,0.750000,0.750000}%
\pgfsetstrokecolor{currentstroke}%
\pgfsetdash{}{0pt}%
\pgfpathmoveto{\pgfqpoint{2.223598in}{1.144243in}}%
\pgfpathlineto{\pgfqpoint{2.207072in}{1.111191in}}%
\pgfpathclose%
\pgfusepath{stroke,fill}%
\end{pgfscope}%
\begin{pgfscope}%
\pgfpathrectangle{\pgfqpoint{0.506010in}{1.121191in}}{\pgfqpoint{2.325000in}{1.400000in}} %
\pgfusepath{clip}%
\pgfsetbuttcap%
\pgfsetroundjoin%
\definecolor{currentfill}{rgb}{0.000000,0.750000,0.750000}%
\pgfsetfillcolor{currentfill}%
\pgfsetlinewidth{1.003750pt}%
\definecolor{currentstroke}{rgb}{0.000000,0.750000,0.750000}%
\pgfsetstrokecolor{currentstroke}%
\pgfsetdash{}{0pt}%
\pgfpathmoveto{\pgfqpoint{2.223598in}{1.144271in}}%
\pgfpathlineto{\pgfqpoint{2.207058in}{1.111191in}}%
\pgfpathclose%
\pgfusepath{stroke,fill}%
\end{pgfscope}%
\begin{pgfscope}%
\pgfpathrectangle{\pgfqpoint{0.506010in}{1.121191in}}{\pgfqpoint{2.325000in}{1.400000in}} %
\pgfusepath{clip}%
\pgfsetbuttcap%
\pgfsetroundjoin%
\definecolor{currentfill}{rgb}{0.000000,0.750000,0.750000}%
\pgfsetfillcolor{currentfill}%
\pgfsetlinewidth{1.003750pt}%
\definecolor{currentstroke}{rgb}{0.000000,0.750000,0.750000}%
\pgfsetstrokecolor{currentstroke}%
\pgfsetdash{}{0pt}%
\pgfpathmoveto{\pgfqpoint{1.007822in}{1.144271in}}%
\pgfpathlineto{\pgfqpoint{0.991282in}{1.111191in}}%
\pgfpathclose%
\pgfusepath{stroke,fill}%
\end{pgfscope}%
\begin{pgfscope}%
\pgfpathrectangle{\pgfqpoint{0.506010in}{1.121191in}}{\pgfqpoint{2.325000in}{1.400000in}} %
\pgfusepath{clip}%
\pgfsetbuttcap%
\pgfsetroundjoin%
\definecolor{currentfill}{rgb}{0.000000,0.750000,0.750000}%
\pgfsetfillcolor{currentfill}%
\pgfsetlinewidth{1.003750pt}%
\definecolor{currentstroke}{rgb}{0.000000,0.750000,0.750000}%
\pgfsetstrokecolor{currentstroke}%
\pgfsetdash{}{0pt}%
\pgfpathmoveto{\pgfqpoint{1.007822in}{1.144299in}}%
\pgfpathlineto{\pgfqpoint{0.991268in}{1.111191in}}%
\pgfpathclose%
\pgfusepath{stroke,fill}%
\end{pgfscope}%
\begin{pgfscope}%
\pgfpathrectangle{\pgfqpoint{0.506010in}{1.121191in}}{\pgfqpoint{2.325000in}{1.400000in}} %
\pgfusepath{clip}%
\pgfsetbuttcap%
\pgfsetroundjoin%
\definecolor{currentfill}{rgb}{0.000000,0.750000,0.750000}%
\pgfsetfillcolor{currentfill}%
\pgfsetlinewidth{1.003750pt}%
\definecolor{currentstroke}{rgb}{0.000000,0.750000,0.750000}%
\pgfsetstrokecolor{currentstroke}%
\pgfsetdash{}{0pt}%
\pgfpathmoveto{\pgfqpoint{2.381458in}{1.144299in}}%
\pgfpathlineto{\pgfqpoint{2.364903in}{1.111191in}}%
\pgfpathclose%
\pgfusepath{stroke,fill}%
\end{pgfscope}%
\begin{pgfscope}%
\pgfpathrectangle{\pgfqpoint{0.506010in}{1.121191in}}{\pgfqpoint{2.325000in}{1.400000in}} %
\pgfusepath{clip}%
\pgfsetbuttcap%
\pgfsetroundjoin%
\definecolor{currentfill}{rgb}{0.000000,0.750000,0.750000}%
\pgfsetfillcolor{currentfill}%
\pgfsetlinewidth{1.003750pt}%
\definecolor{currentstroke}{rgb}{0.000000,0.750000,0.750000}%
\pgfsetstrokecolor{currentstroke}%
\pgfsetdash{}{0pt}%
\pgfpathmoveto{\pgfqpoint{2.380105in}{1.144327in}}%
\pgfpathlineto{\pgfqpoint{2.363537in}{1.111191in}}%
\pgfpathclose%
\pgfusepath{stroke,fill}%
\end{pgfscope}%
\begin{pgfscope}%
\pgfpathrectangle{\pgfqpoint{0.506010in}{1.121191in}}{\pgfqpoint{2.325000in}{1.400000in}} %
\pgfusepath{clip}%
\pgfsetbuttcap%
\pgfsetroundjoin%
\definecolor{currentfill}{rgb}{0.000000,0.750000,0.750000}%
\pgfsetfillcolor{currentfill}%
\pgfsetlinewidth{1.003750pt}%
\definecolor{currentstroke}{rgb}{0.000000,0.750000,0.750000}%
\pgfsetstrokecolor{currentstroke}%
\pgfsetdash{}{0pt}%
\pgfpathmoveto{\pgfqpoint{2.407155in}{1.144355in}}%
\pgfpathlineto{\pgfqpoint{2.390573in}{1.111191in}}%
\pgfpathclose%
\pgfusepath{stroke,fill}%
\end{pgfscope}%
\begin{pgfscope}%
\pgfpathrectangle{\pgfqpoint{0.506010in}{1.121191in}}{\pgfqpoint{2.325000in}{1.400000in}} %
\pgfusepath{clip}%
\pgfsetbuttcap%
\pgfsetroundjoin%
\definecolor{currentfill}{rgb}{0.000000,0.750000,0.750000}%
\pgfsetfillcolor{currentfill}%
\pgfsetlinewidth{1.003750pt}%
\definecolor{currentstroke}{rgb}{0.000000,0.750000,0.750000}%
\pgfsetstrokecolor{currentstroke}%
\pgfsetdash{}{0pt}%
\pgfpathmoveto{\pgfqpoint{1.294699in}{1.146035in}}%
\pgfpathlineto{\pgfqpoint{1.277277in}{1.111191in}}%
\pgfpathclose%
\pgfusepath{stroke,fill}%
\end{pgfscope}%
\begin{pgfscope}%
\pgfpathrectangle{\pgfqpoint{0.506010in}{1.121191in}}{\pgfqpoint{2.325000in}{1.400000in}} %
\pgfusepath{clip}%
\pgfsetbuttcap%
\pgfsetroundjoin%
\definecolor{currentfill}{rgb}{0.000000,0.750000,0.750000}%
\pgfsetfillcolor{currentfill}%
\pgfsetlinewidth{1.003750pt}%
\definecolor{currentstroke}{rgb}{0.000000,0.750000,0.750000}%
\pgfsetstrokecolor{currentstroke}%
\pgfsetdash{}{0pt}%
\pgfpathmoveto{\pgfqpoint{1.390732in}{1.147351in}}%
\pgfpathlineto{\pgfqpoint{1.372652in}{1.111191in}}%
\pgfpathclose%
\pgfusepath{stroke,fill}%
\end{pgfscope}%
\begin{pgfscope}%
\pgfpathrectangle{\pgfqpoint{0.506010in}{1.121191in}}{\pgfqpoint{2.325000in}{1.400000in}} %
\pgfusepath{clip}%
\pgfsetbuttcap%
\pgfsetroundjoin%
\definecolor{currentfill}{rgb}{0.000000,0.750000,0.750000}%
\pgfsetfillcolor{currentfill}%
\pgfsetlinewidth{1.003750pt}%
\definecolor{currentstroke}{rgb}{0.000000,0.750000,0.750000}%
\pgfsetstrokecolor{currentstroke}%
\pgfsetdash{}{0pt}%
\pgfpathmoveto{\pgfqpoint{1.305057in}{1.147435in}}%
\pgfpathlineto{\pgfqpoint{1.286935in}{1.111191in}}%
\pgfpathclose%
\pgfusepath{stroke,fill}%
\end{pgfscope}%
\begin{pgfscope}%
\pgfpathrectangle{\pgfqpoint{0.506010in}{1.121191in}}{\pgfqpoint{2.325000in}{1.400000in}} %
\pgfusepath{clip}%
\pgfsetbuttcap%
\pgfsetroundjoin%
\definecolor{currentfill}{rgb}{0.000000,0.750000,0.750000}%
\pgfsetfillcolor{currentfill}%
\pgfsetlinewidth{1.003750pt}%
\definecolor{currentstroke}{rgb}{0.000000,0.750000,0.750000}%
\pgfsetstrokecolor{currentstroke}%
\pgfsetdash{}{0pt}%
\pgfpathmoveto{\pgfqpoint{1.081382in}{1.147547in}}%
\pgfpathlineto{\pgfqpoint{1.063204in}{1.111191in}}%
\pgfpathclose%
\pgfusepath{stroke,fill}%
\end{pgfscope}%
\begin{pgfscope}%
\pgfpathrectangle{\pgfqpoint{0.506010in}{1.121191in}}{\pgfqpoint{2.325000in}{1.400000in}} %
\pgfusepath{clip}%
\pgfsetbuttcap%
\pgfsetroundjoin%
\definecolor{currentfill}{rgb}{0.000000,0.750000,0.750000}%
\pgfsetfillcolor{currentfill}%
\pgfsetlinewidth{1.003750pt}%
\definecolor{currentstroke}{rgb}{0.000000,0.750000,0.750000}%
\pgfsetstrokecolor{currentstroke}%
\pgfsetdash{}{0pt}%
\pgfpathmoveto{\pgfqpoint{0.810364in}{1.147715in}}%
\pgfpathlineto{\pgfqpoint{0.792102in}{1.111191in}}%
\pgfpathclose%
\pgfusepath{stroke,fill}%
\end{pgfscope}%
\begin{pgfscope}%
\pgfpathrectangle{\pgfqpoint{0.506010in}{1.121191in}}{\pgfqpoint{2.325000in}{1.400000in}} %
\pgfusepath{clip}%
\pgfsetbuttcap%
\pgfsetroundjoin%
\definecolor{currentfill}{rgb}{0.000000,0.750000,0.750000}%
\pgfsetfillcolor{currentfill}%
\pgfsetlinewidth{1.003750pt}%
\definecolor{currentstroke}{rgb}{0.000000,0.750000,0.750000}%
\pgfsetstrokecolor{currentstroke}%
\pgfsetdash{}{0pt}%
\pgfpathmoveto{\pgfqpoint{1.366871in}{1.147827in}}%
\pgfpathlineto{\pgfqpoint{1.348552in}{1.111191in}}%
\pgfpathclose%
\pgfusepath{stroke,fill}%
\end{pgfscope}%
\begin{pgfscope}%
\pgfpathrectangle{\pgfqpoint{0.506010in}{1.121191in}}{\pgfqpoint{2.325000in}{1.400000in}} %
\pgfusepath{clip}%
\pgfsetbuttcap%
\pgfsetroundjoin%
\definecolor{currentfill}{rgb}{0.000000,0.750000,0.750000}%
\pgfsetfillcolor{currentfill}%
\pgfsetlinewidth{1.003750pt}%
\definecolor{currentstroke}{rgb}{0.000000,0.750000,0.750000}%
\pgfsetstrokecolor{currentstroke}%
\pgfsetdash{}{0pt}%
\pgfpathmoveto{\pgfqpoint{1.443680in}{1.147827in}}%
\pgfpathlineto{\pgfqpoint{1.425362in}{1.111191in}}%
\pgfpathclose%
\pgfusepath{stroke,fill}%
\end{pgfscope}%
\begin{pgfscope}%
\pgfpathrectangle{\pgfqpoint{0.506010in}{1.121191in}}{\pgfqpoint{2.325000in}{1.400000in}} %
\pgfusepath{clip}%
\pgfsetbuttcap%
\pgfsetroundjoin%
\definecolor{currentfill}{rgb}{0.000000,0.750000,0.750000}%
\pgfsetfillcolor{currentfill}%
\pgfsetlinewidth{1.003750pt}%
\definecolor{currentstroke}{rgb}{0.000000,0.750000,0.750000}%
\pgfsetstrokecolor{currentstroke}%
\pgfsetdash{}{0pt}%
\pgfpathmoveto{\pgfqpoint{1.406887in}{1.148107in}}%
\pgfpathlineto{\pgfqpoint{1.388429in}{1.111191in}}%
\pgfpathclose%
\pgfusepath{stroke,fill}%
\end{pgfscope}%
\begin{pgfscope}%
\pgfpathrectangle{\pgfqpoint{0.506010in}{1.121191in}}{\pgfqpoint{2.325000in}{1.400000in}} %
\pgfusepath{clip}%
\pgfsetbuttcap%
\pgfsetroundjoin%
\definecolor{currentfill}{rgb}{0.000000,0.750000,0.750000}%
\pgfsetfillcolor{currentfill}%
\pgfsetlinewidth{1.003750pt}%
\definecolor{currentstroke}{rgb}{0.000000,0.750000,0.750000}%
\pgfsetstrokecolor{currentstroke}%
\pgfsetdash{}{0pt}%
\pgfpathmoveto{\pgfqpoint{1.108974in}{1.148303in}}%
\pgfpathlineto{\pgfqpoint{1.090418in}{1.111191in}}%
\pgfpathclose%
\pgfusepath{stroke,fill}%
\end{pgfscope}%
\begin{pgfscope}%
\pgfpathrectangle{\pgfqpoint{0.506010in}{1.121191in}}{\pgfqpoint{2.325000in}{1.400000in}} %
\pgfusepath{clip}%
\pgfsetbuttcap%
\pgfsetroundjoin%
\definecolor{currentfill}{rgb}{0.000000,0.750000,0.750000}%
\pgfsetfillcolor{currentfill}%
\pgfsetlinewidth{1.003750pt}%
\definecolor{currentstroke}{rgb}{0.000000,0.750000,0.750000}%
\pgfsetstrokecolor{currentstroke}%
\pgfsetdash{}{0pt}%
\pgfpathmoveto{\pgfqpoint{0.821928in}{1.148443in}}%
\pgfpathlineto{\pgfqpoint{0.803302in}{1.111191in}}%
\pgfpathclose%
\pgfusepath{stroke,fill}%
\end{pgfscope}%
\begin{pgfscope}%
\pgfpathrectangle{\pgfqpoint{0.506010in}{1.121191in}}{\pgfqpoint{2.325000in}{1.400000in}} %
\pgfusepath{clip}%
\pgfsetbuttcap%
\pgfsetroundjoin%
\definecolor{currentfill}{rgb}{0.000000,0.750000,0.750000}%
\pgfsetfillcolor{currentfill}%
\pgfsetlinewidth{1.003750pt}%
\definecolor{currentstroke}{rgb}{0.000000,0.750000,0.750000}%
\pgfsetstrokecolor{currentstroke}%
\pgfsetdash{}{0pt}%
\pgfpathmoveto{\pgfqpoint{1.443680in}{1.148471in}}%
\pgfpathlineto{\pgfqpoint{1.425040in}{1.111191in}}%
\pgfpathclose%
\pgfusepath{stroke,fill}%
\end{pgfscope}%
\begin{pgfscope}%
\pgfpathrectangle{\pgfqpoint{0.506010in}{1.121191in}}{\pgfqpoint{2.325000in}{1.400000in}} %
\pgfusepath{clip}%
\pgfsetbuttcap%
\pgfsetroundjoin%
\definecolor{currentfill}{rgb}{0.000000,0.750000,0.750000}%
\pgfsetfillcolor{currentfill}%
\pgfsetlinewidth{1.003750pt}%
\definecolor{currentstroke}{rgb}{0.000000,0.750000,0.750000}%
\pgfsetstrokecolor{currentstroke}%
\pgfsetdash{}{0pt}%
\pgfpathmoveto{\pgfqpoint{1.379114in}{1.148527in}}%
\pgfpathlineto{\pgfqpoint{1.360445in}{1.111191in}}%
\pgfpathclose%
\pgfusepath{stroke,fill}%
\end{pgfscope}%
\begin{pgfscope}%
\pgfpathrectangle{\pgfqpoint{0.506010in}{1.121191in}}{\pgfqpoint{2.325000in}{1.400000in}} %
\pgfusepath{clip}%
\pgfsetbuttcap%
\pgfsetroundjoin%
\definecolor{currentfill}{rgb}{0.000000,0.750000,0.750000}%
\pgfsetfillcolor{currentfill}%
\pgfsetlinewidth{1.003750pt}%
\definecolor{currentstroke}{rgb}{0.000000,0.750000,0.750000}%
\pgfsetstrokecolor{currentstroke}%
\pgfsetdash{}{0pt}%
\pgfpathmoveto{\pgfqpoint{1.081382in}{1.148611in}}%
\pgfpathlineto{\pgfqpoint{1.062672in}{1.111191in}}%
\pgfpathclose%
\pgfusepath{stroke,fill}%
\end{pgfscope}%
\begin{pgfscope}%
\pgfpathrectangle{\pgfqpoint{0.506010in}{1.121191in}}{\pgfqpoint{2.325000in}{1.400000in}} %
\pgfusepath{clip}%
\pgfsetbuttcap%
\pgfsetroundjoin%
\definecolor{currentfill}{rgb}{0.000000,0.750000,0.750000}%
\pgfsetfillcolor{currentfill}%
\pgfsetlinewidth{1.003750pt}%
\definecolor{currentstroke}{rgb}{0.000000,0.750000,0.750000}%
\pgfsetstrokecolor{currentstroke}%
\pgfsetdash{}{0pt}%
\pgfpathmoveto{\pgfqpoint{1.108974in}{1.148611in}}%
\pgfpathlineto{\pgfqpoint{1.090264in}{1.111191in}}%
\pgfpathclose%
\pgfusepath{stroke,fill}%
\end{pgfscope}%
\begin{pgfscope}%
\pgfpathrectangle{\pgfqpoint{0.506010in}{1.121191in}}{\pgfqpoint{2.325000in}{1.400000in}} %
\pgfusepath{clip}%
\pgfsetbuttcap%
\pgfsetroundjoin%
\definecolor{currentfill}{rgb}{0.000000,0.750000,0.750000}%
\pgfsetfillcolor{currentfill}%
\pgfsetlinewidth{1.003750pt}%
\definecolor{currentstroke}{rgb}{0.000000,0.750000,0.750000}%
\pgfsetstrokecolor{currentstroke}%
\pgfsetdash{}{0pt}%
\pgfpathmoveto{\pgfqpoint{0.681255in}{1.148667in}}%
\pgfpathlineto{\pgfqpoint{0.662517in}{1.111191in}}%
\pgfpathclose%
\pgfusepath{stroke,fill}%
\end{pgfscope}%
\begin{pgfscope}%
\pgfpathrectangle{\pgfqpoint{0.506010in}{1.121191in}}{\pgfqpoint{2.325000in}{1.400000in}} %
\pgfusepath{clip}%
\pgfsetbuttcap%
\pgfsetroundjoin%
\definecolor{currentfill}{rgb}{0.000000,0.750000,0.750000}%
\pgfsetfillcolor{currentfill}%
\pgfsetlinewidth{1.003750pt}%
\definecolor{currentstroke}{rgb}{0.000000,0.750000,0.750000}%
\pgfsetstrokecolor{currentstroke}%
\pgfsetdash{}{0pt}%
\pgfpathmoveto{\pgfqpoint{1.081382in}{1.148723in}}%
\pgfpathlineto{\pgfqpoint{1.062616in}{1.111191in}}%
\pgfpathclose%
\pgfusepath{stroke,fill}%
\end{pgfscope}%
\begin{pgfscope}%
\pgfpathrectangle{\pgfqpoint{0.506010in}{1.121191in}}{\pgfqpoint{2.325000in}{1.400000in}} %
\pgfusepath{clip}%
\pgfsetbuttcap%
\pgfsetroundjoin%
\definecolor{currentfill}{rgb}{0.000000,0.750000,0.750000}%
\pgfsetfillcolor{currentfill}%
\pgfsetlinewidth{1.003750pt}%
\definecolor{currentstroke}{rgb}{0.000000,0.750000,0.750000}%
\pgfsetstrokecolor{currentstroke}%
\pgfsetdash{}{0pt}%
\pgfpathmoveto{\pgfqpoint{1.081382in}{1.148863in}}%
\pgfpathlineto{\pgfqpoint{1.062546in}{1.111191in}}%
\pgfpathclose%
\pgfusepath{stroke,fill}%
\end{pgfscope}%
\begin{pgfscope}%
\pgfpathrectangle{\pgfqpoint{0.506010in}{1.121191in}}{\pgfqpoint{2.325000in}{1.400000in}} %
\pgfusepath{clip}%
\pgfsetbuttcap%
\pgfsetroundjoin%
\definecolor{currentfill}{rgb}{0.000000,0.750000,0.750000}%
\pgfsetfillcolor{currentfill}%
\pgfsetlinewidth{1.003750pt}%
\definecolor{currentstroke}{rgb}{0.000000,0.750000,0.750000}%
\pgfsetstrokecolor{currentstroke}%
\pgfsetdash{}{0pt}%
\pgfpathmoveto{\pgfqpoint{1.007822in}{1.148947in}}%
\pgfpathlineto{\pgfqpoint{0.988944in}{1.111191in}}%
\pgfpathclose%
\pgfusepath{stroke,fill}%
\end{pgfscope}%
\begin{pgfscope}%
\pgfpathrectangle{\pgfqpoint{0.506010in}{1.121191in}}{\pgfqpoint{2.325000in}{1.400000in}} %
\pgfusepath{clip}%
\pgfsetbuttcap%
\pgfsetroundjoin%
\definecolor{currentfill}{rgb}{0.000000,0.750000,0.750000}%
\pgfsetfillcolor{currentfill}%
\pgfsetlinewidth{1.003750pt}%
\definecolor{currentstroke}{rgb}{0.000000,0.750000,0.750000}%
\pgfsetstrokecolor{currentstroke}%
\pgfsetdash{}{0pt}%
\pgfpathmoveto{\pgfqpoint{0.986187in}{1.148975in}}%
\pgfpathlineto{\pgfqpoint{0.967295in}{1.111191in}}%
\pgfpathclose%
\pgfusepath{stroke,fill}%
\end{pgfscope}%
\begin{pgfscope}%
\pgfpathrectangle{\pgfqpoint{0.506010in}{1.121191in}}{\pgfqpoint{2.325000in}{1.400000in}} %
\pgfusepath{clip}%
\pgfsetbuttcap%
\pgfsetroundjoin%
\definecolor{currentfill}{rgb}{0.000000,0.750000,0.750000}%
\pgfsetfillcolor{currentfill}%
\pgfsetlinewidth{1.003750pt}%
\definecolor{currentstroke}{rgb}{0.000000,0.750000,0.750000}%
\pgfsetstrokecolor{currentstroke}%
\pgfsetdash{}{0pt}%
\pgfpathmoveto{\pgfqpoint{1.108974in}{1.149451in}}%
\pgfpathlineto{\pgfqpoint{1.089844in}{1.111191in}}%
\pgfpathclose%
\pgfusepath{stroke,fill}%
\end{pgfscope}%
\begin{pgfscope}%
\pgfpathrectangle{\pgfqpoint{0.506010in}{1.121191in}}{\pgfqpoint{2.325000in}{1.400000in}} %
\pgfusepath{clip}%
\pgfsetbuttcap%
\pgfsetroundjoin%
\definecolor{currentfill}{rgb}{0.000000,0.750000,0.750000}%
\pgfsetfillcolor{currentfill}%
\pgfsetlinewidth{1.003750pt}%
\definecolor{currentstroke}{rgb}{0.000000,0.750000,0.750000}%
\pgfsetstrokecolor{currentstroke}%
\pgfsetdash{}{0pt}%
\pgfpathmoveto{\pgfqpoint{1.081382in}{1.149479in}}%
\pgfpathlineto{\pgfqpoint{1.062238in}{1.111191in}}%
\pgfpathclose%
\pgfusepath{stroke,fill}%
\end{pgfscope}%
\begin{pgfscope}%
\pgfpathrectangle{\pgfqpoint{0.506010in}{1.121191in}}{\pgfqpoint{2.325000in}{1.400000in}} %
\pgfusepath{clip}%
\pgfsetbuttcap%
\pgfsetroundjoin%
\definecolor{currentfill}{rgb}{0.000000,0.750000,0.750000}%
\pgfsetfillcolor{currentfill}%
\pgfsetlinewidth{1.003750pt}%
\definecolor{currentstroke}{rgb}{0.000000,0.750000,0.750000}%
\pgfsetstrokecolor{currentstroke}%
\pgfsetdash{}{0pt}%
\pgfpathmoveto{\pgfqpoint{1.474029in}{1.149563in}}%
\pgfpathlineto{\pgfqpoint{1.454843in}{1.111191in}}%
\pgfpathclose%
\pgfusepath{stroke,fill}%
\end{pgfscope}%
\begin{pgfscope}%
\pgfpathrectangle{\pgfqpoint{0.506010in}{1.121191in}}{\pgfqpoint{2.325000in}{1.400000in}} %
\pgfusepath{clip}%
\pgfsetbuttcap%
\pgfsetroundjoin%
\definecolor{currentfill}{rgb}{0.000000,0.750000,0.750000}%
\pgfsetfillcolor{currentfill}%
\pgfsetlinewidth{1.003750pt}%
\definecolor{currentstroke}{rgb}{0.000000,0.750000,0.750000}%
\pgfsetstrokecolor{currentstroke}%
\pgfsetdash{}{0pt}%
\pgfpathmoveto{\pgfqpoint{1.007822in}{1.149619in}}%
\pgfpathlineto{\pgfqpoint{0.988608in}{1.111191in}}%
\pgfpathclose%
\pgfusepath{stroke,fill}%
\end{pgfscope}%
\begin{pgfscope}%
\pgfpathrectangle{\pgfqpoint{0.506010in}{1.121191in}}{\pgfqpoint{2.325000in}{1.400000in}} %
\pgfusepath{clip}%
\pgfsetbuttcap%
\pgfsetroundjoin%
\definecolor{currentfill}{rgb}{0.000000,0.750000,0.750000}%
\pgfsetfillcolor{currentfill}%
\pgfsetlinewidth{1.003750pt}%
\definecolor{currentstroke}{rgb}{0.000000,0.750000,0.750000}%
\pgfsetstrokecolor{currentstroke}%
\pgfsetdash{}{0pt}%
\pgfpathmoveto{\pgfqpoint{1.081382in}{1.149675in}}%
\pgfpathlineto{\pgfqpoint{1.062140in}{1.111191in}}%
\pgfpathclose%
\pgfusepath{stroke,fill}%
\end{pgfscope}%
\begin{pgfscope}%
\pgfpathrectangle{\pgfqpoint{0.506010in}{1.121191in}}{\pgfqpoint{2.325000in}{1.400000in}} %
\pgfusepath{clip}%
\pgfsetbuttcap%
\pgfsetroundjoin%
\definecolor{currentfill}{rgb}{0.000000,0.750000,0.750000}%
\pgfsetfillcolor{currentfill}%
\pgfsetlinewidth{1.003750pt}%
\definecolor{currentstroke}{rgb}{0.000000,0.750000,0.750000}%
\pgfsetstrokecolor{currentstroke}%
\pgfsetdash{}{0pt}%
\pgfpathmoveto{\pgfqpoint{1.081382in}{1.150123in}}%
\pgfpathlineto{\pgfqpoint{1.061916in}{1.111191in}}%
\pgfpathclose%
\pgfusepath{stroke,fill}%
\end{pgfscope}%
\begin{pgfscope}%
\pgfpathrectangle{\pgfqpoint{0.506010in}{1.121191in}}{\pgfqpoint{2.325000in}{1.400000in}} %
\pgfusepath{clip}%
\pgfsetbuttcap%
\pgfsetroundjoin%
\definecolor{currentfill}{rgb}{0.000000,0.750000,0.750000}%
\pgfsetfillcolor{currentfill}%
\pgfsetlinewidth{1.003750pt}%
\definecolor{currentstroke}{rgb}{0.000000,0.750000,0.750000}%
\pgfsetstrokecolor{currentstroke}%
\pgfsetdash{}{0pt}%
\pgfpathmoveto{\pgfqpoint{1.081382in}{1.150263in}}%
\pgfpathlineto{\pgfqpoint{1.061846in}{1.111191in}}%
\pgfpathclose%
\pgfusepath{stroke,fill}%
\end{pgfscope}%
\begin{pgfscope}%
\pgfpathrectangle{\pgfqpoint{0.506010in}{1.121191in}}{\pgfqpoint{2.325000in}{1.400000in}} %
\pgfusepath{clip}%
\pgfsetbuttcap%
\pgfsetroundjoin%
\definecolor{currentfill}{rgb}{0.000000,0.750000,0.750000}%
\pgfsetfillcolor{currentfill}%
\pgfsetlinewidth{1.003750pt}%
\definecolor{currentstroke}{rgb}{0.000000,0.750000,0.750000}%
\pgfsetstrokecolor{currentstroke}%
\pgfsetdash{}{0pt}%
\pgfpathmoveto{\pgfqpoint{1.007822in}{1.150963in}}%
\pgfpathlineto{\pgfqpoint{0.987936in}{1.111191in}}%
\pgfpathclose%
\pgfusepath{stroke,fill}%
\end{pgfscope}%
\begin{pgfscope}%
\pgfpathrectangle{\pgfqpoint{0.506010in}{1.121191in}}{\pgfqpoint{2.325000in}{1.400000in}} %
\pgfusepath{clip}%
\pgfsetbuttcap%
\pgfsetroundjoin%
\definecolor{currentfill}{rgb}{0.000000,0.750000,0.750000}%
\pgfsetfillcolor{currentfill}%
\pgfsetlinewidth{1.003750pt}%
\definecolor{currentstroke}{rgb}{0.000000,0.750000,0.750000}%
\pgfsetstrokecolor{currentstroke}%
\pgfsetdash{}{0pt}%
\pgfpathmoveto{\pgfqpoint{1.007822in}{1.151691in}}%
\pgfpathlineto{\pgfqpoint{0.987572in}{1.111191in}}%
\pgfpathclose%
\pgfusepath{stroke,fill}%
\end{pgfscope}%
\begin{pgfscope}%
\pgfpathrectangle{\pgfqpoint{0.506010in}{1.121191in}}{\pgfqpoint{2.325000in}{1.400000in}} %
\pgfusepath{clip}%
\pgfsetbuttcap%
\pgfsetroundjoin%
\definecolor{currentfill}{rgb}{0.000000,0.750000,0.750000}%
\pgfsetfillcolor{currentfill}%
\pgfsetlinewidth{1.003750pt}%
\definecolor{currentstroke}{rgb}{0.000000,0.750000,0.750000}%
\pgfsetstrokecolor{currentstroke}%
\pgfsetdash{}{0pt}%
\pgfpathmoveto{\pgfqpoint{1.108974in}{1.152867in}}%
\pgfpathlineto{\pgfqpoint{1.088136in}{1.111191in}}%
\pgfpathclose%
\pgfusepath{stroke,fill}%
\end{pgfscope}%
\begin{pgfscope}%
\pgfpathrectangle{\pgfqpoint{0.506010in}{1.121191in}}{\pgfqpoint{2.325000in}{1.400000in}} %
\pgfusepath{clip}%
\pgfsetbuttcap%
\pgfsetroundjoin%
\definecolor{currentfill}{rgb}{0.000000,0.750000,0.750000}%
\pgfsetfillcolor{currentfill}%
\pgfsetlinewidth{1.003750pt}%
\definecolor{currentstroke}{rgb}{0.000000,0.750000,0.750000}%
\pgfsetstrokecolor{currentstroke}%
\pgfsetdash{}{0pt}%
\pgfpathmoveto{\pgfqpoint{1.474029in}{1.152923in}}%
\pgfpathlineto{\pgfqpoint{1.453163in}{1.111191in}}%
\pgfpathclose%
\pgfusepath{stroke,fill}%
\end{pgfscope}%
\begin{pgfscope}%
\pgfpathrectangle{\pgfqpoint{0.506010in}{1.121191in}}{\pgfqpoint{2.325000in}{1.400000in}} %
\pgfusepath{clip}%
\pgfsetbuttcap%
\pgfsetroundjoin%
\definecolor{currentfill}{rgb}{0.000000,0.750000,0.750000}%
\pgfsetfillcolor{currentfill}%
\pgfsetlinewidth{1.003750pt}%
\definecolor{currentstroke}{rgb}{0.000000,0.750000,0.750000}%
\pgfsetstrokecolor{currentstroke}%
\pgfsetdash{}{0pt}%
\pgfpathmoveto{\pgfqpoint{1.081382in}{1.153483in}}%
\pgfpathlineto{\pgfqpoint{1.060236in}{1.111191in}}%
\pgfpathclose%
\pgfusepath{stroke,fill}%
\end{pgfscope}%
\begin{pgfscope}%
\pgfpathrectangle{\pgfqpoint{0.506010in}{1.121191in}}{\pgfqpoint{2.325000in}{1.400000in}} %
\pgfusepath{clip}%
\pgfsetbuttcap%
\pgfsetroundjoin%
\definecolor{currentfill}{rgb}{0.000000,0.750000,0.750000}%
\pgfsetfillcolor{currentfill}%
\pgfsetlinewidth{1.003750pt}%
\definecolor{currentstroke}{rgb}{0.000000,0.750000,0.750000}%
\pgfsetstrokecolor{currentstroke}%
\pgfsetdash{}{0pt}%
\pgfpathmoveto{\pgfqpoint{1.081382in}{1.153483in}}%
\pgfpathlineto{\pgfqpoint{1.060236in}{1.111191in}}%
\pgfpathclose%
\pgfusepath{stroke,fill}%
\end{pgfscope}%
\begin{pgfscope}%
\pgfpathrectangle{\pgfqpoint{0.506010in}{1.121191in}}{\pgfqpoint{2.325000in}{1.400000in}} %
\pgfusepath{clip}%
\pgfsetbuttcap%
\pgfsetroundjoin%
\definecolor{currentfill}{rgb}{0.000000,0.750000,0.750000}%
\pgfsetfillcolor{currentfill}%
\pgfsetlinewidth{1.003750pt}%
\definecolor{currentstroke}{rgb}{0.000000,0.750000,0.750000}%
\pgfsetstrokecolor{currentstroke}%
\pgfsetdash{}{0pt}%
\pgfpathmoveto{\pgfqpoint{1.007822in}{1.154099in}}%
\pgfpathlineto{\pgfqpoint{0.986368in}{1.111191in}}%
\pgfpathclose%
\pgfusepath{stroke,fill}%
\end{pgfscope}%
\begin{pgfscope}%
\pgfpathrectangle{\pgfqpoint{0.506010in}{1.121191in}}{\pgfqpoint{2.325000in}{1.400000in}} %
\pgfusepath{clip}%
\pgfsetbuttcap%
\pgfsetroundjoin%
\definecolor{currentfill}{rgb}{0.000000,0.750000,0.750000}%
\pgfsetfillcolor{currentfill}%
\pgfsetlinewidth{1.003750pt}%
\definecolor{currentstroke}{rgb}{0.000000,0.750000,0.750000}%
\pgfsetstrokecolor{currentstroke}%
\pgfsetdash{}{0pt}%
\pgfpathmoveto{\pgfqpoint{1.108974in}{1.154183in}}%
\pgfpathlineto{\pgfqpoint{1.087478in}{1.111191in}}%
\pgfpathclose%
\pgfusepath{stroke,fill}%
\end{pgfscope}%
\begin{pgfscope}%
\pgfpathrectangle{\pgfqpoint{0.506010in}{1.121191in}}{\pgfqpoint{2.325000in}{1.400000in}} %
\pgfusepath{clip}%
\pgfsetbuttcap%
\pgfsetroundjoin%
\definecolor{currentfill}{rgb}{0.000000,0.750000,0.750000}%
\pgfsetfillcolor{currentfill}%
\pgfsetlinewidth{1.003750pt}%
\definecolor{currentstroke}{rgb}{0.000000,0.750000,0.750000}%
\pgfsetstrokecolor{currentstroke}%
\pgfsetdash{}{0pt}%
\pgfpathmoveto{\pgfqpoint{1.081382in}{1.154995in}}%
\pgfpathlineto{\pgfqpoint{1.059480in}{1.111191in}}%
\pgfpathclose%
\pgfusepath{stroke,fill}%
\end{pgfscope}%
\begin{pgfscope}%
\pgfpathrectangle{\pgfqpoint{0.506010in}{1.121191in}}{\pgfqpoint{2.325000in}{1.400000in}} %
\pgfusepath{clip}%
\pgfsetbuttcap%
\pgfsetroundjoin%
\definecolor{currentfill}{rgb}{0.000000,0.750000,0.750000}%
\pgfsetfillcolor{currentfill}%
\pgfsetlinewidth{1.003750pt}%
\definecolor{currentstroke}{rgb}{0.000000,0.750000,0.750000}%
\pgfsetstrokecolor{currentstroke}%
\pgfsetdash{}{0pt}%
\pgfpathmoveto{\pgfqpoint{1.007822in}{1.155275in}}%
\pgfpathlineto{\pgfqpoint{0.985862in}{1.111354in}}%
\pgfpathlineto{\pgfqpoint{1.029783in}{1.111354in}}%
\pgfpathclose%
\pgfusepath{stroke,fill}%
\end{pgfscope}%
\begin{pgfscope}%
\pgfpathrectangle{\pgfqpoint{0.506010in}{1.121191in}}{\pgfqpoint{2.325000in}{1.400000in}} %
\pgfusepath{clip}%
\pgfsetbuttcap%
\pgfsetroundjoin%
\definecolor{currentfill}{rgb}{0.000000,0.750000,0.750000}%
\pgfsetfillcolor{currentfill}%
\pgfsetlinewidth{1.003750pt}%
\definecolor{currentstroke}{rgb}{0.000000,0.750000,0.750000}%
\pgfsetstrokecolor{currentstroke}%
\pgfsetdash{}{0pt}%
\pgfpathmoveto{\pgfqpoint{1.081382in}{1.155443in}}%
\pgfpathlineto{\pgfqpoint{1.059422in}{1.111522in}}%
\pgfpathlineto{\pgfqpoint{1.103342in}{1.111522in}}%
\pgfpathclose%
\pgfusepath{stroke,fill}%
\end{pgfscope}%
\begin{pgfscope}%
\pgfpathrectangle{\pgfqpoint{0.506010in}{1.121191in}}{\pgfqpoint{2.325000in}{1.400000in}} %
\pgfusepath{clip}%
\pgfsetbuttcap%
\pgfsetroundjoin%
\definecolor{currentfill}{rgb}{0.000000,0.750000,0.750000}%
\pgfsetfillcolor{currentfill}%
\pgfsetlinewidth{1.003750pt}%
\definecolor{currentstroke}{rgb}{0.000000,0.750000,0.750000}%
\pgfsetstrokecolor{currentstroke}%
\pgfsetdash{}{0pt}%
\pgfpathmoveto{\pgfqpoint{1.081382in}{1.155611in}}%
\pgfpathlineto{\pgfqpoint{1.059422in}{1.111690in}}%
\pgfpathlineto{\pgfqpoint{1.103342in}{1.111690in}}%
\pgfpathclose%
\pgfusepath{stroke,fill}%
\end{pgfscope}%
\begin{pgfscope}%
\pgfpathrectangle{\pgfqpoint{0.506010in}{1.121191in}}{\pgfqpoint{2.325000in}{1.400000in}} %
\pgfusepath{clip}%
\pgfsetbuttcap%
\pgfsetroundjoin%
\definecolor{currentfill}{rgb}{0.000000,0.750000,0.750000}%
\pgfsetfillcolor{currentfill}%
\pgfsetlinewidth{1.003750pt}%
\definecolor{currentstroke}{rgb}{0.000000,0.750000,0.750000}%
\pgfsetstrokecolor{currentstroke}%
\pgfsetdash{}{0pt}%
\pgfpathmoveto{\pgfqpoint{1.081382in}{1.155891in}}%
\pgfpathlineto{\pgfqpoint{1.059422in}{1.111970in}}%
\pgfpathlineto{\pgfqpoint{1.103342in}{1.111970in}}%
\pgfpathclose%
\pgfusepath{stroke,fill}%
\end{pgfscope}%
\begin{pgfscope}%
\pgfpathrectangle{\pgfqpoint{0.506010in}{1.121191in}}{\pgfqpoint{2.325000in}{1.400000in}} %
\pgfusepath{clip}%
\pgfsetbuttcap%
\pgfsetroundjoin%
\definecolor{currentfill}{rgb}{0.000000,0.750000,0.750000}%
\pgfsetfillcolor{currentfill}%
\pgfsetlinewidth{1.003750pt}%
\definecolor{currentstroke}{rgb}{0.000000,0.750000,0.750000}%
\pgfsetstrokecolor{currentstroke}%
\pgfsetdash{}{0pt}%
\pgfpathmoveto{\pgfqpoint{1.081382in}{1.155975in}}%
\pgfpathlineto{\pgfqpoint{1.059422in}{1.112054in}}%
\pgfpathlineto{\pgfqpoint{1.103342in}{1.112054in}}%
\pgfpathclose%
\pgfusepath{stroke,fill}%
\end{pgfscope}%
\begin{pgfscope}%
\pgfpathrectangle{\pgfqpoint{0.506010in}{1.121191in}}{\pgfqpoint{2.325000in}{1.400000in}} %
\pgfusepath{clip}%
\pgfsetbuttcap%
\pgfsetroundjoin%
\definecolor{currentfill}{rgb}{0.000000,0.750000,0.750000}%
\pgfsetfillcolor{currentfill}%
\pgfsetlinewidth{1.003750pt}%
\definecolor{currentstroke}{rgb}{0.000000,0.750000,0.750000}%
\pgfsetstrokecolor{currentstroke}%
\pgfsetdash{}{0pt}%
\pgfpathmoveto{\pgfqpoint{1.007822in}{1.156283in}}%
\pgfpathlineto{\pgfqpoint{0.985862in}{1.112362in}}%
\pgfpathlineto{\pgfqpoint{1.029783in}{1.112362in}}%
\pgfpathclose%
\pgfusepath{stroke,fill}%
\end{pgfscope}%
\begin{pgfscope}%
\pgfpathrectangle{\pgfqpoint{0.506010in}{1.121191in}}{\pgfqpoint{2.325000in}{1.400000in}} %
\pgfusepath{clip}%
\pgfsetbuttcap%
\pgfsetroundjoin%
\definecolor{currentfill}{rgb}{0.000000,0.750000,0.750000}%
\pgfsetfillcolor{currentfill}%
\pgfsetlinewidth{1.003750pt}%
\definecolor{currentstroke}{rgb}{0.000000,0.750000,0.750000}%
\pgfsetstrokecolor{currentstroke}%
\pgfsetdash{}{0pt}%
\pgfpathmoveto{\pgfqpoint{0.788675in}{1.156311in}}%
\pgfpathlineto{\pgfqpoint{0.766715in}{1.112390in}}%
\pgfpathlineto{\pgfqpoint{0.810635in}{1.112390in}}%
\pgfpathclose%
\pgfusepath{stroke,fill}%
\end{pgfscope}%
\begin{pgfscope}%
\pgfpathrectangle{\pgfqpoint{0.506010in}{1.121191in}}{\pgfqpoint{2.325000in}{1.400000in}} %
\pgfusepath{clip}%
\pgfsetbuttcap%
\pgfsetroundjoin%
\definecolor{currentfill}{rgb}{0.000000,0.750000,0.750000}%
\pgfsetfillcolor{currentfill}%
\pgfsetlinewidth{1.003750pt}%
\definecolor{currentstroke}{rgb}{0.000000,0.750000,0.750000}%
\pgfsetstrokecolor{currentstroke}%
\pgfsetdash{}{0pt}%
\pgfpathmoveto{\pgfqpoint{2.480299in}{1.156367in}}%
\pgfpathlineto{\pgfqpoint{2.458339in}{1.112446in}}%
\pgfpathlineto{\pgfqpoint{2.502259in}{1.112446in}}%
\pgfpathclose%
\pgfusepath{stroke,fill}%
\end{pgfscope}%
\begin{pgfscope}%
\pgfpathrectangle{\pgfqpoint{0.506010in}{1.121191in}}{\pgfqpoint{2.325000in}{1.400000in}} %
\pgfusepath{clip}%
\pgfsetbuttcap%
\pgfsetroundjoin%
\definecolor{currentfill}{rgb}{0.000000,0.750000,0.750000}%
\pgfsetfillcolor{currentfill}%
\pgfsetlinewidth{1.003750pt}%
\definecolor{currentstroke}{rgb}{0.000000,0.750000,0.750000}%
\pgfsetstrokecolor{currentstroke}%
\pgfsetdash{}{0pt}%
\pgfpathmoveto{\pgfqpoint{1.139222in}{1.156535in}}%
\pgfpathlineto{\pgfqpoint{1.117261in}{1.112614in}}%
\pgfpathlineto{\pgfqpoint{1.161182in}{1.112614in}}%
\pgfpathclose%
\pgfusepath{stroke,fill}%
\end{pgfscope}%
\begin{pgfscope}%
\pgfpathrectangle{\pgfqpoint{0.506010in}{1.121191in}}{\pgfqpoint{2.325000in}{1.400000in}} %
\pgfusepath{clip}%
\pgfsetbuttcap%
\pgfsetroundjoin%
\definecolor{currentfill}{rgb}{0.000000,0.750000,0.750000}%
\pgfsetfillcolor{currentfill}%
\pgfsetlinewidth{1.003750pt}%
\definecolor{currentstroke}{rgb}{0.000000,0.750000,0.750000}%
\pgfsetstrokecolor{currentstroke}%
\pgfsetdash{}{0pt}%
\pgfpathmoveto{\pgfqpoint{1.007822in}{1.156591in}}%
\pgfpathlineto{\pgfqpoint{0.985862in}{1.112670in}}%
\pgfpathlineto{\pgfqpoint{1.029783in}{1.112670in}}%
\pgfpathclose%
\pgfusepath{stroke,fill}%
\end{pgfscope}%
\begin{pgfscope}%
\pgfpathrectangle{\pgfqpoint{0.506010in}{1.121191in}}{\pgfqpoint{2.325000in}{1.400000in}} %
\pgfusepath{clip}%
\pgfsetbuttcap%
\pgfsetroundjoin%
\definecolor{currentfill}{rgb}{0.000000,0.750000,0.750000}%
\pgfsetfillcolor{currentfill}%
\pgfsetlinewidth{1.003750pt}%
\definecolor{currentstroke}{rgb}{0.000000,0.750000,0.750000}%
\pgfsetstrokecolor{currentstroke}%
\pgfsetdash{}{0pt}%
\pgfpathmoveto{\pgfqpoint{1.007822in}{1.156619in}}%
\pgfpathlineto{\pgfqpoint{0.985862in}{1.112698in}}%
\pgfpathlineto{\pgfqpoint{1.029783in}{1.112698in}}%
\pgfpathclose%
\pgfusepath{stroke,fill}%
\end{pgfscope}%
\begin{pgfscope}%
\pgfpathrectangle{\pgfqpoint{0.506010in}{1.121191in}}{\pgfqpoint{2.325000in}{1.400000in}} %
\pgfusepath{clip}%
\pgfsetbuttcap%
\pgfsetroundjoin%
\definecolor{currentfill}{rgb}{0.000000,0.750000,0.750000}%
\pgfsetfillcolor{currentfill}%
\pgfsetlinewidth{1.003750pt}%
\definecolor{currentstroke}{rgb}{0.000000,0.750000,0.750000}%
\pgfsetstrokecolor{currentstroke}%
\pgfsetdash{}{0pt}%
\pgfpathmoveto{\pgfqpoint{1.007822in}{1.156927in}}%
\pgfpathlineto{\pgfqpoint{0.985862in}{1.113006in}}%
\pgfpathlineto{\pgfqpoint{1.029783in}{1.113006in}}%
\pgfpathclose%
\pgfusepath{stroke,fill}%
\end{pgfscope}%
\begin{pgfscope}%
\pgfpathrectangle{\pgfqpoint{0.506010in}{1.121191in}}{\pgfqpoint{2.325000in}{1.400000in}} %
\pgfusepath{clip}%
\pgfsetbuttcap%
\pgfsetroundjoin%
\definecolor{currentfill}{rgb}{0.000000,0.750000,0.750000}%
\pgfsetfillcolor{currentfill}%
\pgfsetlinewidth{1.003750pt}%
\definecolor{currentstroke}{rgb}{0.000000,0.750000,0.750000}%
\pgfsetstrokecolor{currentstroke}%
\pgfsetdash{}{0pt}%
\pgfpathmoveto{\pgfqpoint{1.108974in}{1.156955in}}%
\pgfpathlineto{\pgfqpoint{1.087014in}{1.113034in}}%
\pgfpathlineto{\pgfqpoint{1.130934in}{1.113034in}}%
\pgfpathclose%
\pgfusepath{stroke,fill}%
\end{pgfscope}%
\begin{pgfscope}%
\pgfpathrectangle{\pgfqpoint{0.506010in}{1.121191in}}{\pgfqpoint{2.325000in}{1.400000in}} %
\pgfusepath{clip}%
\pgfsetbuttcap%
\pgfsetroundjoin%
\definecolor{currentfill}{rgb}{0.000000,0.750000,0.750000}%
\pgfsetfillcolor{currentfill}%
\pgfsetlinewidth{1.003750pt}%
\definecolor{currentstroke}{rgb}{0.000000,0.750000,0.750000}%
\pgfsetstrokecolor{currentstroke}%
\pgfsetdash{}{0pt}%
\pgfpathmoveto{\pgfqpoint{1.108974in}{1.157151in}}%
\pgfpathlineto{\pgfqpoint{1.087014in}{1.113230in}}%
\pgfpathlineto{\pgfqpoint{1.130934in}{1.113230in}}%
\pgfpathclose%
\pgfusepath{stroke,fill}%
\end{pgfscope}%
\begin{pgfscope}%
\pgfpathrectangle{\pgfqpoint{0.506010in}{1.121191in}}{\pgfqpoint{2.325000in}{1.400000in}} %
\pgfusepath{clip}%
\pgfsetbuttcap%
\pgfsetroundjoin%
\definecolor{currentfill}{rgb}{0.000000,0.750000,0.750000}%
\pgfsetfillcolor{currentfill}%
\pgfsetlinewidth{1.003750pt}%
\definecolor{currentstroke}{rgb}{0.000000,0.750000,0.750000}%
\pgfsetstrokecolor{currentstroke}%
\pgfsetdash{}{0pt}%
\pgfpathmoveto{\pgfqpoint{1.081382in}{1.157151in}}%
\pgfpathlineto{\pgfqpoint{1.059422in}{1.113230in}}%
\pgfpathlineto{\pgfqpoint{1.103342in}{1.113230in}}%
\pgfpathclose%
\pgfusepath{stroke,fill}%
\end{pgfscope}%
\begin{pgfscope}%
\pgfpathrectangle{\pgfqpoint{0.506010in}{1.121191in}}{\pgfqpoint{2.325000in}{1.400000in}} %
\pgfusepath{clip}%
\pgfsetbuttcap%
\pgfsetroundjoin%
\definecolor{currentfill}{rgb}{0.000000,0.750000,0.750000}%
\pgfsetfillcolor{currentfill}%
\pgfsetlinewidth{1.003750pt}%
\definecolor{currentstroke}{rgb}{0.000000,0.750000,0.750000}%
\pgfsetstrokecolor{currentstroke}%
\pgfsetdash{}{0pt}%
\pgfpathmoveto{\pgfqpoint{0.820559in}{1.157347in}}%
\pgfpathlineto{\pgfqpoint{0.798599in}{1.113426in}}%
\pgfpathlineto{\pgfqpoint{0.842520in}{1.113426in}}%
\pgfpathclose%
\pgfusepath{stroke,fill}%
\end{pgfscope}%
\begin{pgfscope}%
\pgfpathrectangle{\pgfqpoint{0.506010in}{1.121191in}}{\pgfqpoint{2.325000in}{1.400000in}} %
\pgfusepath{clip}%
\pgfsetbuttcap%
\pgfsetroundjoin%
\definecolor{currentfill}{rgb}{0.000000,0.750000,0.750000}%
\pgfsetfillcolor{currentfill}%
\pgfsetlinewidth{1.003750pt}%
\definecolor{currentstroke}{rgb}{0.000000,0.750000,0.750000}%
\pgfsetstrokecolor{currentstroke}%
\pgfsetdash{}{0pt}%
\pgfpathmoveto{\pgfqpoint{1.167543in}{1.158691in}}%
\pgfpathlineto{\pgfqpoint{1.145583in}{1.114770in}}%
\pgfpathlineto{\pgfqpoint{1.189503in}{1.114770in}}%
\pgfpathclose%
\pgfusepath{stroke,fill}%
\end{pgfscope}%
\begin{pgfscope}%
\pgfpathrectangle{\pgfqpoint{0.506010in}{1.121191in}}{\pgfqpoint{2.325000in}{1.400000in}} %
\pgfusepath{clip}%
\pgfsetbuttcap%
\pgfsetroundjoin%
\definecolor{currentfill}{rgb}{0.000000,0.750000,0.750000}%
\pgfsetfillcolor{currentfill}%
\pgfsetlinewidth{1.003750pt}%
\definecolor{currentstroke}{rgb}{0.000000,0.750000,0.750000}%
\pgfsetstrokecolor{currentstroke}%
\pgfsetdash{}{0pt}%
\pgfpathmoveto{\pgfqpoint{1.139222in}{1.158831in}}%
\pgfpathlineto{\pgfqpoint{1.117261in}{1.114910in}}%
\pgfpathlineto{\pgfqpoint{1.161182in}{1.114910in}}%
\pgfpathclose%
\pgfusepath{stroke,fill}%
\end{pgfscope}%
\begin{pgfscope}%
\pgfpathrectangle{\pgfqpoint{0.506010in}{1.121191in}}{\pgfqpoint{2.325000in}{1.400000in}} %
\pgfusepath{clip}%
\pgfsetbuttcap%
\pgfsetroundjoin%
\definecolor{currentfill}{rgb}{0.000000,0.750000,0.750000}%
\pgfsetfillcolor{currentfill}%
\pgfsetlinewidth{1.003750pt}%
\definecolor{currentstroke}{rgb}{0.000000,0.750000,0.750000}%
\pgfsetstrokecolor{currentstroke}%
\pgfsetdash{}{0pt}%
\pgfpathmoveto{\pgfqpoint{2.191066in}{1.252015in}}%
\pgfpathlineto{\pgfqpoint{2.169106in}{1.208094in}}%
\pgfpathlineto{\pgfqpoint{2.213026in}{1.208094in}}%
\pgfpathclose%
\pgfusepath{stroke,fill}%
\end{pgfscope}%
\begin{pgfscope}%
\pgfpathrectangle{\pgfqpoint{0.506010in}{1.121191in}}{\pgfqpoint{2.325000in}{1.400000in}} %
\pgfusepath{clip}%
\pgfsetbuttcap%
\pgfsetroundjoin%
\definecolor{currentfill}{rgb}{0.000000,0.750000,0.750000}%
\pgfsetfillcolor{currentfill}%
\pgfsetlinewidth{1.003750pt}%
\definecolor{currentstroke}{rgb}{0.000000,0.750000,0.750000}%
\pgfsetstrokecolor{currentstroke}%
\pgfsetdash{}{0pt}%
\pgfpathmoveto{\pgfqpoint{2.265364in}{2.244671in}}%
\pgfpathlineto{\pgfqpoint{2.243404in}{2.200750in}}%
\pgfpathlineto{\pgfqpoint{2.287324in}{2.200750in}}%
\pgfpathclose%
\pgfusepath{stroke,fill}%
\end{pgfscope}%
\begin{pgfscope}%
\pgfpathrectangle{\pgfqpoint{0.506010in}{1.121191in}}{\pgfqpoint{2.325000in}{1.400000in}} %
\pgfusepath{clip}%
\pgfsetbuttcap%
\pgfsetroundjoin%
\definecolor{currentfill}{rgb}{0.000000,0.750000,0.750000}%
\pgfsetfillcolor{currentfill}%
\pgfsetlinewidth{1.003750pt}%
\definecolor{currentstroke}{rgb}{0.000000,0.750000,0.750000}%
\pgfsetstrokecolor{currentstroke}%
\pgfsetdash{}{0pt}%
\pgfpathmoveto{\pgfqpoint{2.185713in}{2.275947in}}%
\pgfpathlineto{\pgfqpoint{2.163753in}{2.232026in}}%
\pgfpathlineto{\pgfqpoint{2.207673in}{2.232026in}}%
\pgfpathclose%
\pgfusepath{stroke,fill}%
\end{pgfscope}%
\begin{pgfscope}%
\pgfpathrectangle{\pgfqpoint{0.506010in}{1.121191in}}{\pgfqpoint{2.325000in}{1.400000in}} %
\pgfusepath{clip}%
\pgfsetbuttcap%
\pgfsetroundjoin%
\definecolor{currentfill}{rgb}{0.000000,0.750000,0.750000}%
\pgfsetfillcolor{currentfill}%
\pgfsetlinewidth{1.003750pt}%
\definecolor{currentstroke}{rgb}{0.000000,0.750000,0.750000}%
\pgfsetstrokecolor{currentstroke}%
\pgfsetdash{}{0pt}%
\pgfpathmoveto{\pgfqpoint{2.217160in}{6.186427in}}%
\pgfpathclose%
\pgfusepath{stroke,fill}%
\end{pgfscope}%
\begin{pgfscope}%
\pgfpathrectangle{\pgfqpoint{0.506010in}{1.121191in}}{\pgfqpoint{2.325000in}{1.400000in}} %
\pgfusepath{clip}%
\pgfsetbuttcap%
\pgfsetroundjoin%
\definecolor{currentfill}{rgb}{0.000000,0.750000,0.750000}%
\pgfsetfillcolor{currentfill}%
\pgfsetlinewidth{1.003750pt}%
\definecolor{currentstroke}{rgb}{0.000000,0.750000,0.750000}%
\pgfsetstrokecolor{currentstroke}%
\pgfsetdash{}{0pt}%
\pgfpathmoveto{\pgfqpoint{2.262181in}{6.186427in}}%
\pgfpathclose%
\pgfusepath{stroke,fill}%
\end{pgfscope}%
\begin{pgfscope}%
\pgfpathrectangle{\pgfqpoint{0.506010in}{1.121191in}}{\pgfqpoint{2.325000in}{1.400000in}} %
\pgfusepath{clip}%
\pgfsetbuttcap%
\pgfsetroundjoin%
\definecolor{currentfill}{rgb}{0.000000,0.750000,0.750000}%
\pgfsetfillcolor{currentfill}%
\pgfsetlinewidth{1.003750pt}%
\definecolor{currentstroke}{rgb}{0.000000,0.750000,0.750000}%
\pgfsetstrokecolor{currentstroke}%
\pgfsetdash{}{0pt}%
\pgfpathmoveto{\pgfqpoint{2.216102in}{6.186483in}}%
\pgfpathclose%
\pgfusepath{stroke,fill}%
\end{pgfscope}%
\begin{pgfscope}%
\pgfpathrectangle{\pgfqpoint{0.506010in}{1.121191in}}{\pgfqpoint{2.325000in}{1.400000in}} %
\pgfusepath{clip}%
\pgfsetbuttcap%
\pgfsetroundjoin%
\definecolor{currentfill}{rgb}{0.000000,0.750000,0.750000}%
\pgfsetfillcolor{currentfill}%
\pgfsetlinewidth{1.003750pt}%
\definecolor{currentstroke}{rgb}{0.000000,0.750000,0.750000}%
\pgfsetstrokecolor{currentstroke}%
\pgfsetdash{}{0pt}%
\pgfpathmoveto{\pgfqpoint{2.234732in}{6.186483in}}%
\pgfpathclose%
\pgfusepath{stroke,fill}%
\end{pgfscope}%
\begin{pgfscope}%
\pgfpathrectangle{\pgfqpoint{0.506010in}{1.121191in}}{\pgfqpoint{2.325000in}{1.400000in}} %
\pgfusepath{clip}%
\pgfsetbuttcap%
\pgfsetroundjoin%
\definecolor{currentfill}{rgb}{0.000000,0.750000,0.750000}%
\pgfsetfillcolor{currentfill}%
\pgfsetlinewidth{1.003750pt}%
\definecolor{currentstroke}{rgb}{0.000000,0.750000,0.750000}%
\pgfsetstrokecolor{currentstroke}%
\pgfsetdash{}{0pt}%
\pgfpathmoveto{\pgfqpoint{2.201136in}{6.186511in}}%
\pgfpathclose%
\pgfusepath{stroke,fill}%
\end{pgfscope}%
\begin{pgfscope}%
\pgfpathrectangle{\pgfqpoint{0.506010in}{1.121191in}}{\pgfqpoint{2.325000in}{1.400000in}} %
\pgfusepath{clip}%
\pgfsetbuttcap%
\pgfsetroundjoin%
\definecolor{currentfill}{rgb}{0.000000,0.750000,0.750000}%
\pgfsetfillcolor{currentfill}%
\pgfsetlinewidth{1.003750pt}%
\definecolor{currentstroke}{rgb}{0.000000,0.750000,0.750000}%
\pgfsetstrokecolor{currentstroke}%
\pgfsetdash{}{0pt}%
\pgfpathmoveto{\pgfqpoint{2.233373in}{6.186511in}}%
\pgfpathclose%
\pgfusepath{stroke,fill}%
\end{pgfscope}%
\begin{pgfscope}%
\pgfpathrectangle{\pgfqpoint{0.506010in}{1.121191in}}{\pgfqpoint{2.325000in}{1.400000in}} %
\pgfusepath{clip}%
\pgfsetbuttcap%
\pgfsetroundjoin%
\definecolor{currentfill}{rgb}{0.000000,0.750000,0.750000}%
\pgfsetfillcolor{currentfill}%
\pgfsetlinewidth{1.003750pt}%
\definecolor{currentstroke}{rgb}{0.000000,0.750000,0.750000}%
\pgfsetstrokecolor{currentstroke}%
\pgfsetdash{}{0pt}%
\pgfpathmoveto{\pgfqpoint{2.222779in}{6.186539in}}%
\pgfpathclose%
\pgfusepath{stroke,fill}%
\end{pgfscope}%
\begin{pgfscope}%
\pgfpathrectangle{\pgfqpoint{0.506010in}{1.121191in}}{\pgfqpoint{2.325000in}{1.400000in}} %
\pgfusepath{clip}%
\pgfsetbuttcap%
\pgfsetroundjoin%
\definecolor{currentfill}{rgb}{0.000000,0.750000,0.750000}%
\pgfsetfillcolor{currentfill}%
\pgfsetlinewidth{1.003750pt}%
\definecolor{currentstroke}{rgb}{0.000000,0.750000,0.750000}%
\pgfsetstrokecolor{currentstroke}%
\pgfsetdash{}{0pt}%
\pgfpathmoveto{\pgfqpoint{2.248922in}{6.186539in}}%
\pgfpathclose%
\pgfusepath{stroke,fill}%
\end{pgfscope}%
\begin{pgfscope}%
\pgfpathrectangle{\pgfqpoint{0.506010in}{1.121191in}}{\pgfqpoint{2.325000in}{1.400000in}} %
\pgfusepath{clip}%
\pgfsetbuttcap%
\pgfsetroundjoin%
\definecolor{currentfill}{rgb}{0.000000,0.750000,0.750000}%
\pgfsetfillcolor{currentfill}%
\pgfsetlinewidth{1.003750pt}%
\definecolor{currentstroke}{rgb}{0.000000,0.750000,0.750000}%
\pgfsetstrokecolor{currentstroke}%
\pgfsetdash{}{0pt}%
\pgfpathmoveto{\pgfqpoint{2.246565in}{6.186651in}}%
\pgfpathclose%
\pgfusepath{stroke,fill}%
\end{pgfscope}%
\begin{pgfscope}%
\pgfpathrectangle{\pgfqpoint{0.506010in}{1.121191in}}{\pgfqpoint{2.325000in}{1.400000in}} %
\pgfusepath{clip}%
\pgfsetbuttcap%
\pgfsetroundjoin%
\definecolor{currentfill}{rgb}{0.000000,0.750000,0.750000}%
\pgfsetfillcolor{currentfill}%
\pgfsetlinewidth{1.003750pt}%
\definecolor{currentstroke}{rgb}{0.000000,0.750000,0.750000}%
\pgfsetstrokecolor{currentstroke}%
\pgfsetdash{}{0pt}%
\pgfpathmoveto{\pgfqpoint{2.233956in}{6.186651in}}%
\pgfpathclose%
\pgfusepath{stroke,fill}%
\end{pgfscope}%
\begin{pgfscope}%
\pgfpathrectangle{\pgfqpoint{0.506010in}{1.121191in}}{\pgfqpoint{2.325000in}{1.400000in}} %
\pgfusepath{clip}%
\pgfsetbuttcap%
\pgfsetroundjoin%
\definecolor{currentfill}{rgb}{0.000000,0.750000,0.750000}%
\pgfsetfillcolor{currentfill}%
\pgfsetlinewidth{1.003750pt}%
\definecolor{currentstroke}{rgb}{0.000000,0.750000,0.750000}%
\pgfsetstrokecolor{currentstroke}%
\pgfsetdash{}{0pt}%
\pgfpathmoveto{\pgfqpoint{2.251609in}{6.186651in}}%
\pgfpathclose%
\pgfusepath{stroke,fill}%
\end{pgfscope}%
\begin{pgfscope}%
\pgfpathrectangle{\pgfqpoint{0.506010in}{1.121191in}}{\pgfqpoint{2.325000in}{1.400000in}} %
\pgfusepath{clip}%
\pgfsetbuttcap%
\pgfsetroundjoin%
\definecolor{currentfill}{rgb}{0.000000,0.750000,0.750000}%
\pgfsetfillcolor{currentfill}%
\pgfsetlinewidth{1.003750pt}%
\definecolor{currentstroke}{rgb}{0.000000,0.750000,0.750000}%
\pgfsetstrokecolor{currentstroke}%
\pgfsetdash{}{0pt}%
\pgfpathmoveto{\pgfqpoint{2.238184in}{6.186651in}}%
\pgfpathclose%
\pgfusepath{stroke,fill}%
\end{pgfscope}%
\begin{pgfscope}%
\pgfpathrectangle{\pgfqpoint{0.506010in}{1.121191in}}{\pgfqpoint{2.325000in}{1.400000in}} %
\pgfusepath{clip}%
\pgfsetbuttcap%
\pgfsetroundjoin%
\definecolor{currentfill}{rgb}{0.000000,0.750000,0.750000}%
\pgfsetfillcolor{currentfill}%
\pgfsetlinewidth{1.003750pt}%
\definecolor{currentstroke}{rgb}{0.000000,0.750000,0.750000}%
\pgfsetstrokecolor{currentstroke}%
\pgfsetdash{}{0pt}%
\pgfpathmoveto{\pgfqpoint{2.248922in}{6.186651in}}%
\pgfpathclose%
\pgfusepath{stroke,fill}%
\end{pgfscope}%
\begin{pgfscope}%
\pgfpathrectangle{\pgfqpoint{0.506010in}{1.121191in}}{\pgfqpoint{2.325000in}{1.400000in}} %
\pgfusepath{clip}%
\pgfsetbuttcap%
\pgfsetroundjoin%
\definecolor{currentfill}{rgb}{0.000000,0.750000,0.750000}%
\pgfsetfillcolor{currentfill}%
\pgfsetlinewidth{1.003750pt}%
\definecolor{currentstroke}{rgb}{0.000000,0.750000,0.750000}%
\pgfsetstrokecolor{currentstroke}%
\pgfsetdash{}{0pt}%
\pgfpathmoveto{\pgfqpoint{2.262855in}{6.186735in}}%
\pgfpathclose%
\pgfusepath{stroke,fill}%
\end{pgfscope}%
\begin{pgfscope}%
\pgfpathrectangle{\pgfqpoint{0.506010in}{1.121191in}}{\pgfqpoint{2.325000in}{1.400000in}} %
\pgfusepath{clip}%
\pgfsetbuttcap%
\pgfsetroundjoin%
\definecolor{currentfill}{rgb}{0.000000,0.750000,0.750000}%
\pgfsetfillcolor{currentfill}%
\pgfsetlinewidth{1.003750pt}%
\definecolor{currentstroke}{rgb}{0.000000,0.750000,0.750000}%
\pgfsetstrokecolor{currentstroke}%
\pgfsetdash{}{0pt}%
\pgfpathmoveto{\pgfqpoint{2.233956in}{6.186735in}}%
\pgfpathclose%
\pgfusepath{stroke,fill}%
\end{pgfscope}%
\begin{pgfscope}%
\pgfpathrectangle{\pgfqpoint{0.506010in}{1.121191in}}{\pgfqpoint{2.325000in}{1.400000in}} %
\pgfusepath{clip}%
\pgfsetbuttcap%
\pgfsetroundjoin%
\definecolor{currentfill}{rgb}{0.000000,0.750000,0.750000}%
\pgfsetfillcolor{currentfill}%
\pgfsetlinewidth{1.003750pt}%
\definecolor{currentstroke}{rgb}{0.000000,0.750000,0.750000}%
\pgfsetstrokecolor{currentstroke}%
\pgfsetdash{}{0pt}%
\pgfpathmoveto{\pgfqpoint{2.217371in}{6.186819in}}%
\pgfpathclose%
\pgfusepath{stroke,fill}%
\end{pgfscope}%
\begin{pgfscope}%
\pgfpathrectangle{\pgfqpoint{0.506010in}{1.121191in}}{\pgfqpoint{2.325000in}{1.400000in}} %
\pgfusepath{clip}%
\pgfsetbuttcap%
\pgfsetroundjoin%
\definecolor{currentfill}{rgb}{0.000000,0.750000,0.750000}%
\pgfsetfillcolor{currentfill}%
\pgfsetlinewidth{1.003750pt}%
\definecolor{currentstroke}{rgb}{0.000000,0.750000,0.750000}%
\pgfsetstrokecolor{currentstroke}%
\pgfsetdash{}{0pt}%
\pgfpathmoveto{\pgfqpoint{2.184228in}{6.186847in}}%
\pgfpathclose%
\pgfusepath{stroke,fill}%
\end{pgfscope}%
\begin{pgfscope}%
\pgfpathrectangle{\pgfqpoint{0.506010in}{1.121191in}}{\pgfqpoint{2.325000in}{1.400000in}} %
\pgfusepath{clip}%
\pgfsetbuttcap%
\pgfsetroundjoin%
\definecolor{currentfill}{rgb}{0.000000,0.750000,0.750000}%
\pgfsetfillcolor{currentfill}%
\pgfsetlinewidth{1.003750pt}%
\definecolor{currentstroke}{rgb}{0.000000,0.750000,0.750000}%
\pgfsetstrokecolor{currentstroke}%
\pgfsetdash{}{0pt}%
\pgfpathmoveto{\pgfqpoint{2.201136in}{6.186875in}}%
\pgfpathclose%
\pgfusepath{stroke,fill}%
\end{pgfscope}%
\begin{pgfscope}%
\pgfpathrectangle{\pgfqpoint{0.506010in}{1.121191in}}{\pgfqpoint{2.325000in}{1.400000in}} %
\pgfusepath{clip}%
\pgfsetbuttcap%
\pgfsetroundjoin%
\pgfsetlinewidth{0.501875pt}%
\definecolor{currentstroke}{rgb}{0.000000,0.000000,0.000000}%
\pgfsetstrokecolor{currentstroke}%
\pgfsetdash{{1.000000pt}{3.000000pt}}{0.000000pt}%
\pgfpathmoveto{\pgfqpoint{0.506010in}{1.121191in}}%
\pgfpathlineto{\pgfqpoint{0.506010in}{2.521191in}}%
\pgfusepath{stroke}%
\end{pgfscope}%
\begin{pgfscope}%
\pgfsetbuttcap%
\pgfsetroundjoin%
\definecolor{currentfill}{rgb}{0.000000,0.000000,0.000000}%
\pgfsetfillcolor{currentfill}%
\pgfsetlinewidth{0.501875pt}%
\definecolor{currentstroke}{rgb}{0.000000,0.000000,0.000000}%
\pgfsetstrokecolor{currentstroke}%
\pgfsetdash{}{0pt}%
\pgfsys@defobject{currentmarker}{\pgfqpoint{0.000000in}{0.000000in}}{\pgfqpoint{0.000000in}{0.055556in}}{%
\pgfpathmoveto{\pgfqpoint{0.000000in}{0.000000in}}%
\pgfpathlineto{\pgfqpoint{0.000000in}{0.055556in}}%
\pgfusepath{stroke,fill}%
}%
\begin{pgfscope}%
\pgfsys@transformshift{0.506010in}{1.121191in}%
\pgfsys@useobject{currentmarker}{}%
\end{pgfscope}%
\end{pgfscope}%
\begin{pgfscope}%
\pgfsetbuttcap%
\pgfsetroundjoin%
\definecolor{currentfill}{rgb}{0.000000,0.000000,0.000000}%
\pgfsetfillcolor{currentfill}%
\pgfsetlinewidth{0.501875pt}%
\definecolor{currentstroke}{rgb}{0.000000,0.000000,0.000000}%
\pgfsetstrokecolor{currentstroke}%
\pgfsetdash{}{0pt}%
\pgfsys@defobject{currentmarker}{\pgfqpoint{0.000000in}{-0.055556in}}{\pgfqpoint{0.000000in}{0.000000in}}{%
\pgfpathmoveto{\pgfqpoint{0.000000in}{0.000000in}}%
\pgfpathlineto{\pgfqpoint{0.000000in}{-0.055556in}}%
\pgfusepath{stroke,fill}%
}%
\begin{pgfscope}%
\pgfsys@transformshift{0.506010in}{2.521191in}%
\pgfsys@useobject{currentmarker}{}%
\end{pgfscope}%
\end{pgfscope}%
\begin{pgfscope}%
\pgftext[x=0.506010in,y=1.065635in,,top]{{\rmfamily\fontsize{8.328000}{9.993600}\selectfont 0.00001}}%
\end{pgfscope}%
\begin{pgfscope}%
\pgfpathrectangle{\pgfqpoint{0.506010in}{1.121191in}}{\pgfqpoint{2.325000in}{1.400000in}} %
\pgfusepath{clip}%
\pgfsetbuttcap%
\pgfsetroundjoin%
\pgfsetlinewidth{0.501875pt}%
\definecolor{currentstroke}{rgb}{0.000000,0.000000,0.000000}%
\pgfsetstrokecolor{currentstroke}%
\pgfsetdash{{1.000000pt}{3.000000pt}}{0.000000pt}%
\pgfpathmoveto{\pgfqpoint{0.971010in}{1.121191in}}%
\pgfpathlineto{\pgfqpoint{0.971010in}{2.521191in}}%
\pgfusepath{stroke}%
\end{pgfscope}%
\begin{pgfscope}%
\pgfsetbuttcap%
\pgfsetroundjoin%
\definecolor{currentfill}{rgb}{0.000000,0.000000,0.000000}%
\pgfsetfillcolor{currentfill}%
\pgfsetlinewidth{0.501875pt}%
\definecolor{currentstroke}{rgb}{0.000000,0.000000,0.000000}%
\pgfsetstrokecolor{currentstroke}%
\pgfsetdash{}{0pt}%
\pgfsys@defobject{currentmarker}{\pgfqpoint{0.000000in}{0.000000in}}{\pgfqpoint{0.000000in}{0.055556in}}{%
\pgfpathmoveto{\pgfqpoint{0.000000in}{0.000000in}}%
\pgfpathlineto{\pgfqpoint{0.000000in}{0.055556in}}%
\pgfusepath{stroke,fill}%
}%
\begin{pgfscope}%
\pgfsys@transformshift{0.971010in}{1.121191in}%
\pgfsys@useobject{currentmarker}{}%
\end{pgfscope}%
\end{pgfscope}%
\begin{pgfscope}%
\pgfsetbuttcap%
\pgfsetroundjoin%
\definecolor{currentfill}{rgb}{0.000000,0.000000,0.000000}%
\pgfsetfillcolor{currentfill}%
\pgfsetlinewidth{0.501875pt}%
\definecolor{currentstroke}{rgb}{0.000000,0.000000,0.000000}%
\pgfsetstrokecolor{currentstroke}%
\pgfsetdash{}{0pt}%
\pgfsys@defobject{currentmarker}{\pgfqpoint{0.000000in}{-0.055556in}}{\pgfqpoint{0.000000in}{0.000000in}}{%
\pgfpathmoveto{\pgfqpoint{0.000000in}{0.000000in}}%
\pgfpathlineto{\pgfqpoint{0.000000in}{-0.055556in}}%
\pgfusepath{stroke,fill}%
}%
\begin{pgfscope}%
\pgfsys@transformshift{0.971010in}{2.521191in}%
\pgfsys@useobject{currentmarker}{}%
\end{pgfscope}%
\end{pgfscope}%
\begin{pgfscope}%
\pgftext[x=0.971010in,y=1.065635in,,top]{{\rmfamily\fontsize{8.328000}{9.993600}\selectfont 0.0001}}%
\end{pgfscope}%
\begin{pgfscope}%
\pgfpathrectangle{\pgfqpoint{0.506010in}{1.121191in}}{\pgfqpoint{2.325000in}{1.400000in}} %
\pgfusepath{clip}%
\pgfsetbuttcap%
\pgfsetroundjoin%
\pgfsetlinewidth{0.501875pt}%
\definecolor{currentstroke}{rgb}{0.000000,0.000000,0.000000}%
\pgfsetstrokecolor{currentstroke}%
\pgfsetdash{{1.000000pt}{3.000000pt}}{0.000000pt}%
\pgfpathmoveto{\pgfqpoint{1.436010in}{1.121191in}}%
\pgfpathlineto{\pgfqpoint{1.436010in}{2.521191in}}%
\pgfusepath{stroke}%
\end{pgfscope}%
\begin{pgfscope}%
\pgfsetbuttcap%
\pgfsetroundjoin%
\definecolor{currentfill}{rgb}{0.000000,0.000000,0.000000}%
\pgfsetfillcolor{currentfill}%
\pgfsetlinewidth{0.501875pt}%
\definecolor{currentstroke}{rgb}{0.000000,0.000000,0.000000}%
\pgfsetstrokecolor{currentstroke}%
\pgfsetdash{}{0pt}%
\pgfsys@defobject{currentmarker}{\pgfqpoint{0.000000in}{0.000000in}}{\pgfqpoint{0.000000in}{0.055556in}}{%
\pgfpathmoveto{\pgfqpoint{0.000000in}{0.000000in}}%
\pgfpathlineto{\pgfqpoint{0.000000in}{0.055556in}}%
\pgfusepath{stroke,fill}%
}%
\begin{pgfscope}%
\pgfsys@transformshift{1.436010in}{1.121191in}%
\pgfsys@useobject{currentmarker}{}%
\end{pgfscope}%
\end{pgfscope}%
\begin{pgfscope}%
\pgfsetbuttcap%
\pgfsetroundjoin%
\definecolor{currentfill}{rgb}{0.000000,0.000000,0.000000}%
\pgfsetfillcolor{currentfill}%
\pgfsetlinewidth{0.501875pt}%
\definecolor{currentstroke}{rgb}{0.000000,0.000000,0.000000}%
\pgfsetstrokecolor{currentstroke}%
\pgfsetdash{}{0pt}%
\pgfsys@defobject{currentmarker}{\pgfqpoint{0.000000in}{-0.055556in}}{\pgfqpoint{0.000000in}{0.000000in}}{%
\pgfpathmoveto{\pgfqpoint{0.000000in}{0.000000in}}%
\pgfpathlineto{\pgfqpoint{0.000000in}{-0.055556in}}%
\pgfusepath{stroke,fill}%
}%
\begin{pgfscope}%
\pgfsys@transformshift{1.436010in}{2.521191in}%
\pgfsys@useobject{currentmarker}{}%
\end{pgfscope}%
\end{pgfscope}%
\begin{pgfscope}%
\pgftext[x=1.436010in,y=1.065635in,,top]{{\rmfamily\fontsize{8.328000}{9.993600}\selectfont 0.001}}%
\end{pgfscope}%
\begin{pgfscope}%
\pgfpathrectangle{\pgfqpoint{0.506010in}{1.121191in}}{\pgfqpoint{2.325000in}{1.400000in}} %
\pgfusepath{clip}%
\pgfsetbuttcap%
\pgfsetroundjoin%
\pgfsetlinewidth{0.501875pt}%
\definecolor{currentstroke}{rgb}{0.000000,0.000000,0.000000}%
\pgfsetstrokecolor{currentstroke}%
\pgfsetdash{{1.000000pt}{3.000000pt}}{0.000000pt}%
\pgfpathmoveto{\pgfqpoint{1.901010in}{1.121191in}}%
\pgfpathlineto{\pgfqpoint{1.901010in}{2.521191in}}%
\pgfusepath{stroke}%
\end{pgfscope}%
\begin{pgfscope}%
\pgfsetbuttcap%
\pgfsetroundjoin%
\definecolor{currentfill}{rgb}{0.000000,0.000000,0.000000}%
\pgfsetfillcolor{currentfill}%
\pgfsetlinewidth{0.501875pt}%
\definecolor{currentstroke}{rgb}{0.000000,0.000000,0.000000}%
\pgfsetstrokecolor{currentstroke}%
\pgfsetdash{}{0pt}%
\pgfsys@defobject{currentmarker}{\pgfqpoint{0.000000in}{0.000000in}}{\pgfqpoint{0.000000in}{0.055556in}}{%
\pgfpathmoveto{\pgfqpoint{0.000000in}{0.000000in}}%
\pgfpathlineto{\pgfqpoint{0.000000in}{0.055556in}}%
\pgfusepath{stroke,fill}%
}%
\begin{pgfscope}%
\pgfsys@transformshift{1.901010in}{1.121191in}%
\pgfsys@useobject{currentmarker}{}%
\end{pgfscope}%
\end{pgfscope}%
\begin{pgfscope}%
\pgfsetbuttcap%
\pgfsetroundjoin%
\definecolor{currentfill}{rgb}{0.000000,0.000000,0.000000}%
\pgfsetfillcolor{currentfill}%
\pgfsetlinewidth{0.501875pt}%
\definecolor{currentstroke}{rgb}{0.000000,0.000000,0.000000}%
\pgfsetstrokecolor{currentstroke}%
\pgfsetdash{}{0pt}%
\pgfsys@defobject{currentmarker}{\pgfqpoint{0.000000in}{-0.055556in}}{\pgfqpoint{0.000000in}{0.000000in}}{%
\pgfpathmoveto{\pgfqpoint{0.000000in}{0.000000in}}%
\pgfpathlineto{\pgfqpoint{0.000000in}{-0.055556in}}%
\pgfusepath{stroke,fill}%
}%
\begin{pgfscope}%
\pgfsys@transformshift{1.901010in}{2.521191in}%
\pgfsys@useobject{currentmarker}{}%
\end{pgfscope}%
\end{pgfscope}%
\begin{pgfscope}%
\pgftext[x=1.901010in,y=1.065635in,,top]{{\rmfamily\fontsize{8.328000}{9.993600}\selectfont 0.01}}%
\end{pgfscope}%
\begin{pgfscope}%
\pgfpathrectangle{\pgfqpoint{0.506010in}{1.121191in}}{\pgfqpoint{2.325000in}{1.400000in}} %
\pgfusepath{clip}%
\pgfsetbuttcap%
\pgfsetroundjoin%
\pgfsetlinewidth{0.501875pt}%
\definecolor{currentstroke}{rgb}{0.000000,0.000000,0.000000}%
\pgfsetstrokecolor{currentstroke}%
\pgfsetdash{{1.000000pt}{3.000000pt}}{0.000000pt}%
\pgfpathmoveto{\pgfqpoint{2.366010in}{1.121191in}}%
\pgfpathlineto{\pgfqpoint{2.366010in}{2.521191in}}%
\pgfusepath{stroke}%
\end{pgfscope}%
\begin{pgfscope}%
\pgfsetbuttcap%
\pgfsetroundjoin%
\definecolor{currentfill}{rgb}{0.000000,0.000000,0.000000}%
\pgfsetfillcolor{currentfill}%
\pgfsetlinewidth{0.501875pt}%
\definecolor{currentstroke}{rgb}{0.000000,0.000000,0.000000}%
\pgfsetstrokecolor{currentstroke}%
\pgfsetdash{}{0pt}%
\pgfsys@defobject{currentmarker}{\pgfqpoint{0.000000in}{0.000000in}}{\pgfqpoint{0.000000in}{0.055556in}}{%
\pgfpathmoveto{\pgfqpoint{0.000000in}{0.000000in}}%
\pgfpathlineto{\pgfqpoint{0.000000in}{0.055556in}}%
\pgfusepath{stroke,fill}%
}%
\begin{pgfscope}%
\pgfsys@transformshift{2.366010in}{1.121191in}%
\pgfsys@useobject{currentmarker}{}%
\end{pgfscope}%
\end{pgfscope}%
\begin{pgfscope}%
\pgfsetbuttcap%
\pgfsetroundjoin%
\definecolor{currentfill}{rgb}{0.000000,0.000000,0.000000}%
\pgfsetfillcolor{currentfill}%
\pgfsetlinewidth{0.501875pt}%
\definecolor{currentstroke}{rgb}{0.000000,0.000000,0.000000}%
\pgfsetstrokecolor{currentstroke}%
\pgfsetdash{}{0pt}%
\pgfsys@defobject{currentmarker}{\pgfqpoint{0.000000in}{-0.055556in}}{\pgfqpoint{0.000000in}{0.000000in}}{%
\pgfpathmoveto{\pgfqpoint{0.000000in}{0.000000in}}%
\pgfpathlineto{\pgfqpoint{0.000000in}{-0.055556in}}%
\pgfusepath{stroke,fill}%
}%
\begin{pgfscope}%
\pgfsys@transformshift{2.366010in}{2.521191in}%
\pgfsys@useobject{currentmarker}{}%
\end{pgfscope}%
\end{pgfscope}%
\begin{pgfscope}%
\pgftext[x=2.366010in,y=1.065635in,,top]{{\rmfamily\fontsize{8.328000}{9.993600}\selectfont 0.1}}%
\end{pgfscope}%
\begin{pgfscope}%
\pgfpathrectangle{\pgfqpoint{0.506010in}{1.121191in}}{\pgfqpoint{2.325000in}{1.400000in}} %
\pgfusepath{clip}%
\pgfsetbuttcap%
\pgfsetroundjoin%
\pgfsetlinewidth{0.501875pt}%
\definecolor{currentstroke}{rgb}{0.000000,0.000000,0.000000}%
\pgfsetstrokecolor{currentstroke}%
\pgfsetdash{{1.000000pt}{3.000000pt}}{0.000000pt}%
\pgfpathmoveto{\pgfqpoint{2.831010in}{1.121191in}}%
\pgfpathlineto{\pgfqpoint{2.831010in}{2.521191in}}%
\pgfusepath{stroke}%
\end{pgfscope}%
\begin{pgfscope}%
\pgfsetbuttcap%
\pgfsetroundjoin%
\definecolor{currentfill}{rgb}{0.000000,0.000000,0.000000}%
\pgfsetfillcolor{currentfill}%
\pgfsetlinewidth{0.501875pt}%
\definecolor{currentstroke}{rgb}{0.000000,0.000000,0.000000}%
\pgfsetstrokecolor{currentstroke}%
\pgfsetdash{}{0pt}%
\pgfsys@defobject{currentmarker}{\pgfqpoint{0.000000in}{0.000000in}}{\pgfqpoint{0.000000in}{0.055556in}}{%
\pgfpathmoveto{\pgfqpoint{0.000000in}{0.000000in}}%
\pgfpathlineto{\pgfqpoint{0.000000in}{0.055556in}}%
\pgfusepath{stroke,fill}%
}%
\begin{pgfscope}%
\pgfsys@transformshift{2.831010in}{1.121191in}%
\pgfsys@useobject{currentmarker}{}%
\end{pgfscope}%
\end{pgfscope}%
\begin{pgfscope}%
\pgfsetbuttcap%
\pgfsetroundjoin%
\definecolor{currentfill}{rgb}{0.000000,0.000000,0.000000}%
\pgfsetfillcolor{currentfill}%
\pgfsetlinewidth{0.501875pt}%
\definecolor{currentstroke}{rgb}{0.000000,0.000000,0.000000}%
\pgfsetstrokecolor{currentstroke}%
\pgfsetdash{}{0pt}%
\pgfsys@defobject{currentmarker}{\pgfqpoint{0.000000in}{-0.055556in}}{\pgfqpoint{0.000000in}{0.000000in}}{%
\pgfpathmoveto{\pgfqpoint{0.000000in}{0.000000in}}%
\pgfpathlineto{\pgfqpoint{0.000000in}{-0.055556in}}%
\pgfusepath{stroke,fill}%
}%
\begin{pgfscope}%
\pgfsys@transformshift{2.831010in}{2.521191in}%
\pgfsys@useobject{currentmarker}{}%
\end{pgfscope}%
\end{pgfscope}%
\begin{pgfscope}%
\pgftext[x=2.831010in,y=1.065635in,,top]{{\rmfamily\fontsize{8.328000}{9.993600}\selectfont 1}}%
\end{pgfscope}%
\begin{pgfscope}%
\pgfsetbuttcap%
\pgfsetroundjoin%
\definecolor{currentfill}{rgb}{0.000000,0.000000,0.000000}%
\pgfsetfillcolor{currentfill}%
\pgfsetlinewidth{0.501875pt}%
\definecolor{currentstroke}{rgb}{0.000000,0.000000,0.000000}%
\pgfsetstrokecolor{currentstroke}%
\pgfsetdash{}{0pt}%
\pgfsys@defobject{currentmarker}{\pgfqpoint{0.000000in}{0.000000in}}{\pgfqpoint{0.000000in}{0.027778in}}{%
\pgfpathmoveto{\pgfqpoint{0.000000in}{0.000000in}}%
\pgfpathlineto{\pgfqpoint{0.000000in}{0.027778in}}%
\pgfusepath{stroke,fill}%
}%
\begin{pgfscope}%
\pgfsys@transformshift{0.645989in}{1.121191in}%
\pgfsys@useobject{currentmarker}{}%
\end{pgfscope}%
\end{pgfscope}%
\begin{pgfscope}%
\pgfsetbuttcap%
\pgfsetroundjoin%
\definecolor{currentfill}{rgb}{0.000000,0.000000,0.000000}%
\pgfsetfillcolor{currentfill}%
\pgfsetlinewidth{0.501875pt}%
\definecolor{currentstroke}{rgb}{0.000000,0.000000,0.000000}%
\pgfsetstrokecolor{currentstroke}%
\pgfsetdash{}{0pt}%
\pgfsys@defobject{currentmarker}{\pgfqpoint{0.000000in}{-0.027778in}}{\pgfqpoint{0.000000in}{0.000000in}}{%
\pgfpathmoveto{\pgfqpoint{0.000000in}{0.000000in}}%
\pgfpathlineto{\pgfqpoint{0.000000in}{-0.027778in}}%
\pgfusepath{stroke,fill}%
}%
\begin{pgfscope}%
\pgfsys@transformshift{0.645989in}{2.521191in}%
\pgfsys@useobject{currentmarker}{}%
\end{pgfscope}%
\end{pgfscope}%
\begin{pgfscope}%
\pgfsetbuttcap%
\pgfsetroundjoin%
\definecolor{currentfill}{rgb}{0.000000,0.000000,0.000000}%
\pgfsetfillcolor{currentfill}%
\pgfsetlinewidth{0.501875pt}%
\definecolor{currentstroke}{rgb}{0.000000,0.000000,0.000000}%
\pgfsetstrokecolor{currentstroke}%
\pgfsetdash{}{0pt}%
\pgfsys@defobject{currentmarker}{\pgfqpoint{0.000000in}{0.000000in}}{\pgfqpoint{0.000000in}{0.027778in}}{%
\pgfpathmoveto{\pgfqpoint{0.000000in}{0.000000in}}%
\pgfpathlineto{\pgfqpoint{0.000000in}{0.027778in}}%
\pgfusepath{stroke,fill}%
}%
\begin{pgfscope}%
\pgfsys@transformshift{0.727871in}{1.121191in}%
\pgfsys@useobject{currentmarker}{}%
\end{pgfscope}%
\end{pgfscope}%
\begin{pgfscope}%
\pgfsetbuttcap%
\pgfsetroundjoin%
\definecolor{currentfill}{rgb}{0.000000,0.000000,0.000000}%
\pgfsetfillcolor{currentfill}%
\pgfsetlinewidth{0.501875pt}%
\definecolor{currentstroke}{rgb}{0.000000,0.000000,0.000000}%
\pgfsetstrokecolor{currentstroke}%
\pgfsetdash{}{0pt}%
\pgfsys@defobject{currentmarker}{\pgfqpoint{0.000000in}{-0.027778in}}{\pgfqpoint{0.000000in}{0.000000in}}{%
\pgfpathmoveto{\pgfqpoint{0.000000in}{0.000000in}}%
\pgfpathlineto{\pgfqpoint{0.000000in}{-0.027778in}}%
\pgfusepath{stroke,fill}%
}%
\begin{pgfscope}%
\pgfsys@transformshift{0.727871in}{2.521191in}%
\pgfsys@useobject{currentmarker}{}%
\end{pgfscope}%
\end{pgfscope}%
\begin{pgfscope}%
\pgfsetbuttcap%
\pgfsetroundjoin%
\definecolor{currentfill}{rgb}{0.000000,0.000000,0.000000}%
\pgfsetfillcolor{currentfill}%
\pgfsetlinewidth{0.501875pt}%
\definecolor{currentstroke}{rgb}{0.000000,0.000000,0.000000}%
\pgfsetstrokecolor{currentstroke}%
\pgfsetdash{}{0pt}%
\pgfsys@defobject{currentmarker}{\pgfqpoint{0.000000in}{0.000000in}}{\pgfqpoint{0.000000in}{0.027778in}}{%
\pgfpathmoveto{\pgfqpoint{0.000000in}{0.000000in}}%
\pgfpathlineto{\pgfqpoint{0.000000in}{0.027778in}}%
\pgfusepath{stroke,fill}%
}%
\begin{pgfscope}%
\pgfsys@transformshift{0.785968in}{1.121191in}%
\pgfsys@useobject{currentmarker}{}%
\end{pgfscope}%
\end{pgfscope}%
\begin{pgfscope}%
\pgfsetbuttcap%
\pgfsetroundjoin%
\definecolor{currentfill}{rgb}{0.000000,0.000000,0.000000}%
\pgfsetfillcolor{currentfill}%
\pgfsetlinewidth{0.501875pt}%
\definecolor{currentstroke}{rgb}{0.000000,0.000000,0.000000}%
\pgfsetstrokecolor{currentstroke}%
\pgfsetdash{}{0pt}%
\pgfsys@defobject{currentmarker}{\pgfqpoint{0.000000in}{-0.027778in}}{\pgfqpoint{0.000000in}{0.000000in}}{%
\pgfpathmoveto{\pgfqpoint{0.000000in}{0.000000in}}%
\pgfpathlineto{\pgfqpoint{0.000000in}{-0.027778in}}%
\pgfusepath{stroke,fill}%
}%
\begin{pgfscope}%
\pgfsys@transformshift{0.785968in}{2.521191in}%
\pgfsys@useobject{currentmarker}{}%
\end{pgfscope}%
\end{pgfscope}%
\begin{pgfscope}%
\pgfsetbuttcap%
\pgfsetroundjoin%
\definecolor{currentfill}{rgb}{0.000000,0.000000,0.000000}%
\pgfsetfillcolor{currentfill}%
\pgfsetlinewidth{0.501875pt}%
\definecolor{currentstroke}{rgb}{0.000000,0.000000,0.000000}%
\pgfsetstrokecolor{currentstroke}%
\pgfsetdash{}{0pt}%
\pgfsys@defobject{currentmarker}{\pgfqpoint{0.000000in}{0.000000in}}{\pgfqpoint{0.000000in}{0.027778in}}{%
\pgfpathmoveto{\pgfqpoint{0.000000in}{0.000000in}}%
\pgfpathlineto{\pgfqpoint{0.000000in}{0.027778in}}%
\pgfusepath{stroke,fill}%
}%
\begin{pgfscope}%
\pgfsys@transformshift{0.831031in}{1.121191in}%
\pgfsys@useobject{currentmarker}{}%
\end{pgfscope}%
\end{pgfscope}%
\begin{pgfscope}%
\pgfsetbuttcap%
\pgfsetroundjoin%
\definecolor{currentfill}{rgb}{0.000000,0.000000,0.000000}%
\pgfsetfillcolor{currentfill}%
\pgfsetlinewidth{0.501875pt}%
\definecolor{currentstroke}{rgb}{0.000000,0.000000,0.000000}%
\pgfsetstrokecolor{currentstroke}%
\pgfsetdash{}{0pt}%
\pgfsys@defobject{currentmarker}{\pgfqpoint{0.000000in}{-0.027778in}}{\pgfqpoint{0.000000in}{0.000000in}}{%
\pgfpathmoveto{\pgfqpoint{0.000000in}{0.000000in}}%
\pgfpathlineto{\pgfqpoint{0.000000in}{-0.027778in}}%
\pgfusepath{stroke,fill}%
}%
\begin{pgfscope}%
\pgfsys@transformshift{0.831031in}{2.521191in}%
\pgfsys@useobject{currentmarker}{}%
\end{pgfscope}%
\end{pgfscope}%
\begin{pgfscope}%
\pgfsetbuttcap%
\pgfsetroundjoin%
\definecolor{currentfill}{rgb}{0.000000,0.000000,0.000000}%
\pgfsetfillcolor{currentfill}%
\pgfsetlinewidth{0.501875pt}%
\definecolor{currentstroke}{rgb}{0.000000,0.000000,0.000000}%
\pgfsetstrokecolor{currentstroke}%
\pgfsetdash{}{0pt}%
\pgfsys@defobject{currentmarker}{\pgfqpoint{0.000000in}{0.000000in}}{\pgfqpoint{0.000000in}{0.027778in}}{%
\pgfpathmoveto{\pgfqpoint{0.000000in}{0.000000in}}%
\pgfpathlineto{\pgfqpoint{0.000000in}{0.027778in}}%
\pgfusepath{stroke,fill}%
}%
\begin{pgfscope}%
\pgfsys@transformshift{0.867850in}{1.121191in}%
\pgfsys@useobject{currentmarker}{}%
\end{pgfscope}%
\end{pgfscope}%
\begin{pgfscope}%
\pgfsetbuttcap%
\pgfsetroundjoin%
\definecolor{currentfill}{rgb}{0.000000,0.000000,0.000000}%
\pgfsetfillcolor{currentfill}%
\pgfsetlinewidth{0.501875pt}%
\definecolor{currentstroke}{rgb}{0.000000,0.000000,0.000000}%
\pgfsetstrokecolor{currentstroke}%
\pgfsetdash{}{0pt}%
\pgfsys@defobject{currentmarker}{\pgfqpoint{0.000000in}{-0.027778in}}{\pgfqpoint{0.000000in}{0.000000in}}{%
\pgfpathmoveto{\pgfqpoint{0.000000in}{0.000000in}}%
\pgfpathlineto{\pgfqpoint{0.000000in}{-0.027778in}}%
\pgfusepath{stroke,fill}%
}%
\begin{pgfscope}%
\pgfsys@transformshift{0.867850in}{2.521191in}%
\pgfsys@useobject{currentmarker}{}%
\end{pgfscope}%
\end{pgfscope}%
\begin{pgfscope}%
\pgfsetbuttcap%
\pgfsetroundjoin%
\definecolor{currentfill}{rgb}{0.000000,0.000000,0.000000}%
\pgfsetfillcolor{currentfill}%
\pgfsetlinewidth{0.501875pt}%
\definecolor{currentstroke}{rgb}{0.000000,0.000000,0.000000}%
\pgfsetstrokecolor{currentstroke}%
\pgfsetdash{}{0pt}%
\pgfsys@defobject{currentmarker}{\pgfqpoint{0.000000in}{0.000000in}}{\pgfqpoint{0.000000in}{0.027778in}}{%
\pgfpathmoveto{\pgfqpoint{0.000000in}{0.000000in}}%
\pgfpathlineto{\pgfqpoint{0.000000in}{0.027778in}}%
\pgfusepath{stroke,fill}%
}%
\begin{pgfscope}%
\pgfsys@transformshift{0.898980in}{1.121191in}%
\pgfsys@useobject{currentmarker}{}%
\end{pgfscope}%
\end{pgfscope}%
\begin{pgfscope}%
\pgfsetbuttcap%
\pgfsetroundjoin%
\definecolor{currentfill}{rgb}{0.000000,0.000000,0.000000}%
\pgfsetfillcolor{currentfill}%
\pgfsetlinewidth{0.501875pt}%
\definecolor{currentstroke}{rgb}{0.000000,0.000000,0.000000}%
\pgfsetstrokecolor{currentstroke}%
\pgfsetdash{}{0pt}%
\pgfsys@defobject{currentmarker}{\pgfqpoint{0.000000in}{-0.027778in}}{\pgfqpoint{0.000000in}{0.000000in}}{%
\pgfpathmoveto{\pgfqpoint{0.000000in}{0.000000in}}%
\pgfpathlineto{\pgfqpoint{0.000000in}{-0.027778in}}%
\pgfusepath{stroke,fill}%
}%
\begin{pgfscope}%
\pgfsys@transformshift{0.898980in}{2.521191in}%
\pgfsys@useobject{currentmarker}{}%
\end{pgfscope}%
\end{pgfscope}%
\begin{pgfscope}%
\pgfsetbuttcap%
\pgfsetroundjoin%
\definecolor{currentfill}{rgb}{0.000000,0.000000,0.000000}%
\pgfsetfillcolor{currentfill}%
\pgfsetlinewidth{0.501875pt}%
\definecolor{currentstroke}{rgb}{0.000000,0.000000,0.000000}%
\pgfsetstrokecolor{currentstroke}%
\pgfsetdash{}{0pt}%
\pgfsys@defobject{currentmarker}{\pgfqpoint{0.000000in}{0.000000in}}{\pgfqpoint{0.000000in}{0.027778in}}{%
\pgfpathmoveto{\pgfqpoint{0.000000in}{0.000000in}}%
\pgfpathlineto{\pgfqpoint{0.000000in}{0.027778in}}%
\pgfusepath{stroke,fill}%
}%
\begin{pgfscope}%
\pgfsys@transformshift{0.925947in}{1.121191in}%
\pgfsys@useobject{currentmarker}{}%
\end{pgfscope}%
\end{pgfscope}%
\begin{pgfscope}%
\pgfsetbuttcap%
\pgfsetroundjoin%
\definecolor{currentfill}{rgb}{0.000000,0.000000,0.000000}%
\pgfsetfillcolor{currentfill}%
\pgfsetlinewidth{0.501875pt}%
\definecolor{currentstroke}{rgb}{0.000000,0.000000,0.000000}%
\pgfsetstrokecolor{currentstroke}%
\pgfsetdash{}{0pt}%
\pgfsys@defobject{currentmarker}{\pgfqpoint{0.000000in}{-0.027778in}}{\pgfqpoint{0.000000in}{0.000000in}}{%
\pgfpathmoveto{\pgfqpoint{0.000000in}{0.000000in}}%
\pgfpathlineto{\pgfqpoint{0.000000in}{-0.027778in}}%
\pgfusepath{stroke,fill}%
}%
\begin{pgfscope}%
\pgfsys@transformshift{0.925947in}{2.521191in}%
\pgfsys@useobject{currentmarker}{}%
\end{pgfscope}%
\end{pgfscope}%
\begin{pgfscope}%
\pgfsetbuttcap%
\pgfsetroundjoin%
\definecolor{currentfill}{rgb}{0.000000,0.000000,0.000000}%
\pgfsetfillcolor{currentfill}%
\pgfsetlinewidth{0.501875pt}%
\definecolor{currentstroke}{rgb}{0.000000,0.000000,0.000000}%
\pgfsetstrokecolor{currentstroke}%
\pgfsetdash{}{0pt}%
\pgfsys@defobject{currentmarker}{\pgfqpoint{0.000000in}{0.000000in}}{\pgfqpoint{0.000000in}{0.027778in}}{%
\pgfpathmoveto{\pgfqpoint{0.000000in}{0.000000in}}%
\pgfpathlineto{\pgfqpoint{0.000000in}{0.027778in}}%
\pgfusepath{stroke,fill}%
}%
\begin{pgfscope}%
\pgfsys@transformshift{0.949733in}{1.121191in}%
\pgfsys@useobject{currentmarker}{}%
\end{pgfscope}%
\end{pgfscope}%
\begin{pgfscope}%
\pgfsetbuttcap%
\pgfsetroundjoin%
\definecolor{currentfill}{rgb}{0.000000,0.000000,0.000000}%
\pgfsetfillcolor{currentfill}%
\pgfsetlinewidth{0.501875pt}%
\definecolor{currentstroke}{rgb}{0.000000,0.000000,0.000000}%
\pgfsetstrokecolor{currentstroke}%
\pgfsetdash{}{0pt}%
\pgfsys@defobject{currentmarker}{\pgfqpoint{0.000000in}{-0.027778in}}{\pgfqpoint{0.000000in}{0.000000in}}{%
\pgfpathmoveto{\pgfqpoint{0.000000in}{0.000000in}}%
\pgfpathlineto{\pgfqpoint{0.000000in}{-0.027778in}}%
\pgfusepath{stroke,fill}%
}%
\begin{pgfscope}%
\pgfsys@transformshift{0.949733in}{2.521191in}%
\pgfsys@useobject{currentmarker}{}%
\end{pgfscope}%
\end{pgfscope}%
\begin{pgfscope}%
\pgfsetbuttcap%
\pgfsetroundjoin%
\definecolor{currentfill}{rgb}{0.000000,0.000000,0.000000}%
\pgfsetfillcolor{currentfill}%
\pgfsetlinewidth{0.501875pt}%
\definecolor{currentstroke}{rgb}{0.000000,0.000000,0.000000}%
\pgfsetstrokecolor{currentstroke}%
\pgfsetdash{}{0pt}%
\pgfsys@defobject{currentmarker}{\pgfqpoint{0.000000in}{0.000000in}}{\pgfqpoint{0.000000in}{0.027778in}}{%
\pgfpathmoveto{\pgfqpoint{0.000000in}{0.000000in}}%
\pgfpathlineto{\pgfqpoint{0.000000in}{0.027778in}}%
\pgfusepath{stroke,fill}%
}%
\begin{pgfscope}%
\pgfsys@transformshift{1.110989in}{1.121191in}%
\pgfsys@useobject{currentmarker}{}%
\end{pgfscope}%
\end{pgfscope}%
\begin{pgfscope}%
\pgfsetbuttcap%
\pgfsetroundjoin%
\definecolor{currentfill}{rgb}{0.000000,0.000000,0.000000}%
\pgfsetfillcolor{currentfill}%
\pgfsetlinewidth{0.501875pt}%
\definecolor{currentstroke}{rgb}{0.000000,0.000000,0.000000}%
\pgfsetstrokecolor{currentstroke}%
\pgfsetdash{}{0pt}%
\pgfsys@defobject{currentmarker}{\pgfqpoint{0.000000in}{-0.027778in}}{\pgfqpoint{0.000000in}{0.000000in}}{%
\pgfpathmoveto{\pgfqpoint{0.000000in}{0.000000in}}%
\pgfpathlineto{\pgfqpoint{0.000000in}{-0.027778in}}%
\pgfusepath{stroke,fill}%
}%
\begin{pgfscope}%
\pgfsys@transformshift{1.110989in}{2.521191in}%
\pgfsys@useobject{currentmarker}{}%
\end{pgfscope}%
\end{pgfscope}%
\begin{pgfscope}%
\pgfsetbuttcap%
\pgfsetroundjoin%
\definecolor{currentfill}{rgb}{0.000000,0.000000,0.000000}%
\pgfsetfillcolor{currentfill}%
\pgfsetlinewidth{0.501875pt}%
\definecolor{currentstroke}{rgb}{0.000000,0.000000,0.000000}%
\pgfsetstrokecolor{currentstroke}%
\pgfsetdash{}{0pt}%
\pgfsys@defobject{currentmarker}{\pgfqpoint{0.000000in}{0.000000in}}{\pgfqpoint{0.000000in}{0.027778in}}{%
\pgfpathmoveto{\pgfqpoint{0.000000in}{0.000000in}}%
\pgfpathlineto{\pgfqpoint{0.000000in}{0.027778in}}%
\pgfusepath{stroke,fill}%
}%
\begin{pgfscope}%
\pgfsys@transformshift{1.192871in}{1.121191in}%
\pgfsys@useobject{currentmarker}{}%
\end{pgfscope}%
\end{pgfscope}%
\begin{pgfscope}%
\pgfsetbuttcap%
\pgfsetroundjoin%
\definecolor{currentfill}{rgb}{0.000000,0.000000,0.000000}%
\pgfsetfillcolor{currentfill}%
\pgfsetlinewidth{0.501875pt}%
\definecolor{currentstroke}{rgb}{0.000000,0.000000,0.000000}%
\pgfsetstrokecolor{currentstroke}%
\pgfsetdash{}{0pt}%
\pgfsys@defobject{currentmarker}{\pgfqpoint{0.000000in}{-0.027778in}}{\pgfqpoint{0.000000in}{0.000000in}}{%
\pgfpathmoveto{\pgfqpoint{0.000000in}{0.000000in}}%
\pgfpathlineto{\pgfqpoint{0.000000in}{-0.027778in}}%
\pgfusepath{stroke,fill}%
}%
\begin{pgfscope}%
\pgfsys@transformshift{1.192871in}{2.521191in}%
\pgfsys@useobject{currentmarker}{}%
\end{pgfscope}%
\end{pgfscope}%
\begin{pgfscope}%
\pgfsetbuttcap%
\pgfsetroundjoin%
\definecolor{currentfill}{rgb}{0.000000,0.000000,0.000000}%
\pgfsetfillcolor{currentfill}%
\pgfsetlinewidth{0.501875pt}%
\definecolor{currentstroke}{rgb}{0.000000,0.000000,0.000000}%
\pgfsetstrokecolor{currentstroke}%
\pgfsetdash{}{0pt}%
\pgfsys@defobject{currentmarker}{\pgfqpoint{0.000000in}{0.000000in}}{\pgfqpoint{0.000000in}{0.027778in}}{%
\pgfpathmoveto{\pgfqpoint{0.000000in}{0.000000in}}%
\pgfpathlineto{\pgfqpoint{0.000000in}{0.027778in}}%
\pgfusepath{stroke,fill}%
}%
\begin{pgfscope}%
\pgfsys@transformshift{1.250968in}{1.121191in}%
\pgfsys@useobject{currentmarker}{}%
\end{pgfscope}%
\end{pgfscope}%
\begin{pgfscope}%
\pgfsetbuttcap%
\pgfsetroundjoin%
\definecolor{currentfill}{rgb}{0.000000,0.000000,0.000000}%
\pgfsetfillcolor{currentfill}%
\pgfsetlinewidth{0.501875pt}%
\definecolor{currentstroke}{rgb}{0.000000,0.000000,0.000000}%
\pgfsetstrokecolor{currentstroke}%
\pgfsetdash{}{0pt}%
\pgfsys@defobject{currentmarker}{\pgfqpoint{0.000000in}{-0.027778in}}{\pgfqpoint{0.000000in}{0.000000in}}{%
\pgfpathmoveto{\pgfqpoint{0.000000in}{0.000000in}}%
\pgfpathlineto{\pgfqpoint{0.000000in}{-0.027778in}}%
\pgfusepath{stroke,fill}%
}%
\begin{pgfscope}%
\pgfsys@transformshift{1.250968in}{2.521191in}%
\pgfsys@useobject{currentmarker}{}%
\end{pgfscope}%
\end{pgfscope}%
\begin{pgfscope}%
\pgfsetbuttcap%
\pgfsetroundjoin%
\definecolor{currentfill}{rgb}{0.000000,0.000000,0.000000}%
\pgfsetfillcolor{currentfill}%
\pgfsetlinewidth{0.501875pt}%
\definecolor{currentstroke}{rgb}{0.000000,0.000000,0.000000}%
\pgfsetstrokecolor{currentstroke}%
\pgfsetdash{}{0pt}%
\pgfsys@defobject{currentmarker}{\pgfqpoint{0.000000in}{0.000000in}}{\pgfqpoint{0.000000in}{0.027778in}}{%
\pgfpathmoveto{\pgfqpoint{0.000000in}{0.000000in}}%
\pgfpathlineto{\pgfqpoint{0.000000in}{0.027778in}}%
\pgfusepath{stroke,fill}%
}%
\begin{pgfscope}%
\pgfsys@transformshift{1.296031in}{1.121191in}%
\pgfsys@useobject{currentmarker}{}%
\end{pgfscope}%
\end{pgfscope}%
\begin{pgfscope}%
\pgfsetbuttcap%
\pgfsetroundjoin%
\definecolor{currentfill}{rgb}{0.000000,0.000000,0.000000}%
\pgfsetfillcolor{currentfill}%
\pgfsetlinewidth{0.501875pt}%
\definecolor{currentstroke}{rgb}{0.000000,0.000000,0.000000}%
\pgfsetstrokecolor{currentstroke}%
\pgfsetdash{}{0pt}%
\pgfsys@defobject{currentmarker}{\pgfqpoint{0.000000in}{-0.027778in}}{\pgfqpoint{0.000000in}{0.000000in}}{%
\pgfpathmoveto{\pgfqpoint{0.000000in}{0.000000in}}%
\pgfpathlineto{\pgfqpoint{0.000000in}{-0.027778in}}%
\pgfusepath{stroke,fill}%
}%
\begin{pgfscope}%
\pgfsys@transformshift{1.296031in}{2.521191in}%
\pgfsys@useobject{currentmarker}{}%
\end{pgfscope}%
\end{pgfscope}%
\begin{pgfscope}%
\pgfsetbuttcap%
\pgfsetroundjoin%
\definecolor{currentfill}{rgb}{0.000000,0.000000,0.000000}%
\pgfsetfillcolor{currentfill}%
\pgfsetlinewidth{0.501875pt}%
\definecolor{currentstroke}{rgb}{0.000000,0.000000,0.000000}%
\pgfsetstrokecolor{currentstroke}%
\pgfsetdash{}{0pt}%
\pgfsys@defobject{currentmarker}{\pgfqpoint{0.000000in}{0.000000in}}{\pgfqpoint{0.000000in}{0.027778in}}{%
\pgfpathmoveto{\pgfqpoint{0.000000in}{0.000000in}}%
\pgfpathlineto{\pgfqpoint{0.000000in}{0.027778in}}%
\pgfusepath{stroke,fill}%
}%
\begin{pgfscope}%
\pgfsys@transformshift{1.332850in}{1.121191in}%
\pgfsys@useobject{currentmarker}{}%
\end{pgfscope}%
\end{pgfscope}%
\begin{pgfscope}%
\pgfsetbuttcap%
\pgfsetroundjoin%
\definecolor{currentfill}{rgb}{0.000000,0.000000,0.000000}%
\pgfsetfillcolor{currentfill}%
\pgfsetlinewidth{0.501875pt}%
\definecolor{currentstroke}{rgb}{0.000000,0.000000,0.000000}%
\pgfsetstrokecolor{currentstroke}%
\pgfsetdash{}{0pt}%
\pgfsys@defobject{currentmarker}{\pgfqpoint{0.000000in}{-0.027778in}}{\pgfqpoint{0.000000in}{0.000000in}}{%
\pgfpathmoveto{\pgfqpoint{0.000000in}{0.000000in}}%
\pgfpathlineto{\pgfqpoint{0.000000in}{-0.027778in}}%
\pgfusepath{stroke,fill}%
}%
\begin{pgfscope}%
\pgfsys@transformshift{1.332850in}{2.521191in}%
\pgfsys@useobject{currentmarker}{}%
\end{pgfscope}%
\end{pgfscope}%
\begin{pgfscope}%
\pgfsetbuttcap%
\pgfsetroundjoin%
\definecolor{currentfill}{rgb}{0.000000,0.000000,0.000000}%
\pgfsetfillcolor{currentfill}%
\pgfsetlinewidth{0.501875pt}%
\definecolor{currentstroke}{rgb}{0.000000,0.000000,0.000000}%
\pgfsetstrokecolor{currentstroke}%
\pgfsetdash{}{0pt}%
\pgfsys@defobject{currentmarker}{\pgfqpoint{0.000000in}{0.000000in}}{\pgfqpoint{0.000000in}{0.027778in}}{%
\pgfpathmoveto{\pgfqpoint{0.000000in}{0.000000in}}%
\pgfpathlineto{\pgfqpoint{0.000000in}{0.027778in}}%
\pgfusepath{stroke,fill}%
}%
\begin{pgfscope}%
\pgfsys@transformshift{1.363980in}{1.121191in}%
\pgfsys@useobject{currentmarker}{}%
\end{pgfscope}%
\end{pgfscope}%
\begin{pgfscope}%
\pgfsetbuttcap%
\pgfsetroundjoin%
\definecolor{currentfill}{rgb}{0.000000,0.000000,0.000000}%
\pgfsetfillcolor{currentfill}%
\pgfsetlinewidth{0.501875pt}%
\definecolor{currentstroke}{rgb}{0.000000,0.000000,0.000000}%
\pgfsetstrokecolor{currentstroke}%
\pgfsetdash{}{0pt}%
\pgfsys@defobject{currentmarker}{\pgfqpoint{0.000000in}{-0.027778in}}{\pgfqpoint{0.000000in}{0.000000in}}{%
\pgfpathmoveto{\pgfqpoint{0.000000in}{0.000000in}}%
\pgfpathlineto{\pgfqpoint{0.000000in}{-0.027778in}}%
\pgfusepath{stroke,fill}%
}%
\begin{pgfscope}%
\pgfsys@transformshift{1.363980in}{2.521191in}%
\pgfsys@useobject{currentmarker}{}%
\end{pgfscope}%
\end{pgfscope}%
\begin{pgfscope}%
\pgfsetbuttcap%
\pgfsetroundjoin%
\definecolor{currentfill}{rgb}{0.000000,0.000000,0.000000}%
\pgfsetfillcolor{currentfill}%
\pgfsetlinewidth{0.501875pt}%
\definecolor{currentstroke}{rgb}{0.000000,0.000000,0.000000}%
\pgfsetstrokecolor{currentstroke}%
\pgfsetdash{}{0pt}%
\pgfsys@defobject{currentmarker}{\pgfqpoint{0.000000in}{0.000000in}}{\pgfqpoint{0.000000in}{0.027778in}}{%
\pgfpathmoveto{\pgfqpoint{0.000000in}{0.000000in}}%
\pgfpathlineto{\pgfqpoint{0.000000in}{0.027778in}}%
\pgfusepath{stroke,fill}%
}%
\begin{pgfscope}%
\pgfsys@transformshift{1.390947in}{1.121191in}%
\pgfsys@useobject{currentmarker}{}%
\end{pgfscope}%
\end{pgfscope}%
\begin{pgfscope}%
\pgfsetbuttcap%
\pgfsetroundjoin%
\definecolor{currentfill}{rgb}{0.000000,0.000000,0.000000}%
\pgfsetfillcolor{currentfill}%
\pgfsetlinewidth{0.501875pt}%
\definecolor{currentstroke}{rgb}{0.000000,0.000000,0.000000}%
\pgfsetstrokecolor{currentstroke}%
\pgfsetdash{}{0pt}%
\pgfsys@defobject{currentmarker}{\pgfqpoint{0.000000in}{-0.027778in}}{\pgfqpoint{0.000000in}{0.000000in}}{%
\pgfpathmoveto{\pgfqpoint{0.000000in}{0.000000in}}%
\pgfpathlineto{\pgfqpoint{0.000000in}{-0.027778in}}%
\pgfusepath{stroke,fill}%
}%
\begin{pgfscope}%
\pgfsys@transformshift{1.390947in}{2.521191in}%
\pgfsys@useobject{currentmarker}{}%
\end{pgfscope}%
\end{pgfscope}%
\begin{pgfscope}%
\pgfsetbuttcap%
\pgfsetroundjoin%
\definecolor{currentfill}{rgb}{0.000000,0.000000,0.000000}%
\pgfsetfillcolor{currentfill}%
\pgfsetlinewidth{0.501875pt}%
\definecolor{currentstroke}{rgb}{0.000000,0.000000,0.000000}%
\pgfsetstrokecolor{currentstroke}%
\pgfsetdash{}{0pt}%
\pgfsys@defobject{currentmarker}{\pgfqpoint{0.000000in}{0.000000in}}{\pgfqpoint{0.000000in}{0.027778in}}{%
\pgfpathmoveto{\pgfqpoint{0.000000in}{0.000000in}}%
\pgfpathlineto{\pgfqpoint{0.000000in}{0.027778in}}%
\pgfusepath{stroke,fill}%
}%
\begin{pgfscope}%
\pgfsys@transformshift{1.414733in}{1.121191in}%
\pgfsys@useobject{currentmarker}{}%
\end{pgfscope}%
\end{pgfscope}%
\begin{pgfscope}%
\pgfsetbuttcap%
\pgfsetroundjoin%
\definecolor{currentfill}{rgb}{0.000000,0.000000,0.000000}%
\pgfsetfillcolor{currentfill}%
\pgfsetlinewidth{0.501875pt}%
\definecolor{currentstroke}{rgb}{0.000000,0.000000,0.000000}%
\pgfsetstrokecolor{currentstroke}%
\pgfsetdash{}{0pt}%
\pgfsys@defobject{currentmarker}{\pgfqpoint{0.000000in}{-0.027778in}}{\pgfqpoint{0.000000in}{0.000000in}}{%
\pgfpathmoveto{\pgfqpoint{0.000000in}{0.000000in}}%
\pgfpathlineto{\pgfqpoint{0.000000in}{-0.027778in}}%
\pgfusepath{stroke,fill}%
}%
\begin{pgfscope}%
\pgfsys@transformshift{1.414733in}{2.521191in}%
\pgfsys@useobject{currentmarker}{}%
\end{pgfscope}%
\end{pgfscope}%
\begin{pgfscope}%
\pgfsetbuttcap%
\pgfsetroundjoin%
\definecolor{currentfill}{rgb}{0.000000,0.000000,0.000000}%
\pgfsetfillcolor{currentfill}%
\pgfsetlinewidth{0.501875pt}%
\definecolor{currentstroke}{rgb}{0.000000,0.000000,0.000000}%
\pgfsetstrokecolor{currentstroke}%
\pgfsetdash{}{0pt}%
\pgfsys@defobject{currentmarker}{\pgfqpoint{0.000000in}{0.000000in}}{\pgfqpoint{0.000000in}{0.027778in}}{%
\pgfpathmoveto{\pgfqpoint{0.000000in}{0.000000in}}%
\pgfpathlineto{\pgfqpoint{0.000000in}{0.027778in}}%
\pgfusepath{stroke,fill}%
}%
\begin{pgfscope}%
\pgfsys@transformshift{1.575989in}{1.121191in}%
\pgfsys@useobject{currentmarker}{}%
\end{pgfscope}%
\end{pgfscope}%
\begin{pgfscope}%
\pgfsetbuttcap%
\pgfsetroundjoin%
\definecolor{currentfill}{rgb}{0.000000,0.000000,0.000000}%
\pgfsetfillcolor{currentfill}%
\pgfsetlinewidth{0.501875pt}%
\definecolor{currentstroke}{rgb}{0.000000,0.000000,0.000000}%
\pgfsetstrokecolor{currentstroke}%
\pgfsetdash{}{0pt}%
\pgfsys@defobject{currentmarker}{\pgfqpoint{0.000000in}{-0.027778in}}{\pgfqpoint{0.000000in}{0.000000in}}{%
\pgfpathmoveto{\pgfqpoint{0.000000in}{0.000000in}}%
\pgfpathlineto{\pgfqpoint{0.000000in}{-0.027778in}}%
\pgfusepath{stroke,fill}%
}%
\begin{pgfscope}%
\pgfsys@transformshift{1.575989in}{2.521191in}%
\pgfsys@useobject{currentmarker}{}%
\end{pgfscope}%
\end{pgfscope}%
\begin{pgfscope}%
\pgfsetbuttcap%
\pgfsetroundjoin%
\definecolor{currentfill}{rgb}{0.000000,0.000000,0.000000}%
\pgfsetfillcolor{currentfill}%
\pgfsetlinewidth{0.501875pt}%
\definecolor{currentstroke}{rgb}{0.000000,0.000000,0.000000}%
\pgfsetstrokecolor{currentstroke}%
\pgfsetdash{}{0pt}%
\pgfsys@defobject{currentmarker}{\pgfqpoint{0.000000in}{0.000000in}}{\pgfqpoint{0.000000in}{0.027778in}}{%
\pgfpathmoveto{\pgfqpoint{0.000000in}{0.000000in}}%
\pgfpathlineto{\pgfqpoint{0.000000in}{0.027778in}}%
\pgfusepath{stroke,fill}%
}%
\begin{pgfscope}%
\pgfsys@transformshift{1.657871in}{1.121191in}%
\pgfsys@useobject{currentmarker}{}%
\end{pgfscope}%
\end{pgfscope}%
\begin{pgfscope}%
\pgfsetbuttcap%
\pgfsetroundjoin%
\definecolor{currentfill}{rgb}{0.000000,0.000000,0.000000}%
\pgfsetfillcolor{currentfill}%
\pgfsetlinewidth{0.501875pt}%
\definecolor{currentstroke}{rgb}{0.000000,0.000000,0.000000}%
\pgfsetstrokecolor{currentstroke}%
\pgfsetdash{}{0pt}%
\pgfsys@defobject{currentmarker}{\pgfqpoint{0.000000in}{-0.027778in}}{\pgfqpoint{0.000000in}{0.000000in}}{%
\pgfpathmoveto{\pgfqpoint{0.000000in}{0.000000in}}%
\pgfpathlineto{\pgfqpoint{0.000000in}{-0.027778in}}%
\pgfusepath{stroke,fill}%
}%
\begin{pgfscope}%
\pgfsys@transformshift{1.657871in}{2.521191in}%
\pgfsys@useobject{currentmarker}{}%
\end{pgfscope}%
\end{pgfscope}%
\begin{pgfscope}%
\pgfsetbuttcap%
\pgfsetroundjoin%
\definecolor{currentfill}{rgb}{0.000000,0.000000,0.000000}%
\pgfsetfillcolor{currentfill}%
\pgfsetlinewidth{0.501875pt}%
\definecolor{currentstroke}{rgb}{0.000000,0.000000,0.000000}%
\pgfsetstrokecolor{currentstroke}%
\pgfsetdash{}{0pt}%
\pgfsys@defobject{currentmarker}{\pgfqpoint{0.000000in}{0.000000in}}{\pgfqpoint{0.000000in}{0.027778in}}{%
\pgfpathmoveto{\pgfqpoint{0.000000in}{0.000000in}}%
\pgfpathlineto{\pgfqpoint{0.000000in}{0.027778in}}%
\pgfusepath{stroke,fill}%
}%
\begin{pgfscope}%
\pgfsys@transformshift{1.715968in}{1.121191in}%
\pgfsys@useobject{currentmarker}{}%
\end{pgfscope}%
\end{pgfscope}%
\begin{pgfscope}%
\pgfsetbuttcap%
\pgfsetroundjoin%
\definecolor{currentfill}{rgb}{0.000000,0.000000,0.000000}%
\pgfsetfillcolor{currentfill}%
\pgfsetlinewidth{0.501875pt}%
\definecolor{currentstroke}{rgb}{0.000000,0.000000,0.000000}%
\pgfsetstrokecolor{currentstroke}%
\pgfsetdash{}{0pt}%
\pgfsys@defobject{currentmarker}{\pgfqpoint{0.000000in}{-0.027778in}}{\pgfqpoint{0.000000in}{0.000000in}}{%
\pgfpathmoveto{\pgfqpoint{0.000000in}{0.000000in}}%
\pgfpathlineto{\pgfqpoint{0.000000in}{-0.027778in}}%
\pgfusepath{stroke,fill}%
}%
\begin{pgfscope}%
\pgfsys@transformshift{1.715968in}{2.521191in}%
\pgfsys@useobject{currentmarker}{}%
\end{pgfscope}%
\end{pgfscope}%
\begin{pgfscope}%
\pgfsetbuttcap%
\pgfsetroundjoin%
\definecolor{currentfill}{rgb}{0.000000,0.000000,0.000000}%
\pgfsetfillcolor{currentfill}%
\pgfsetlinewidth{0.501875pt}%
\definecolor{currentstroke}{rgb}{0.000000,0.000000,0.000000}%
\pgfsetstrokecolor{currentstroke}%
\pgfsetdash{}{0pt}%
\pgfsys@defobject{currentmarker}{\pgfqpoint{0.000000in}{0.000000in}}{\pgfqpoint{0.000000in}{0.027778in}}{%
\pgfpathmoveto{\pgfqpoint{0.000000in}{0.000000in}}%
\pgfpathlineto{\pgfqpoint{0.000000in}{0.027778in}}%
\pgfusepath{stroke,fill}%
}%
\begin{pgfscope}%
\pgfsys@transformshift{1.761031in}{1.121191in}%
\pgfsys@useobject{currentmarker}{}%
\end{pgfscope}%
\end{pgfscope}%
\begin{pgfscope}%
\pgfsetbuttcap%
\pgfsetroundjoin%
\definecolor{currentfill}{rgb}{0.000000,0.000000,0.000000}%
\pgfsetfillcolor{currentfill}%
\pgfsetlinewidth{0.501875pt}%
\definecolor{currentstroke}{rgb}{0.000000,0.000000,0.000000}%
\pgfsetstrokecolor{currentstroke}%
\pgfsetdash{}{0pt}%
\pgfsys@defobject{currentmarker}{\pgfqpoint{0.000000in}{-0.027778in}}{\pgfqpoint{0.000000in}{0.000000in}}{%
\pgfpathmoveto{\pgfqpoint{0.000000in}{0.000000in}}%
\pgfpathlineto{\pgfqpoint{0.000000in}{-0.027778in}}%
\pgfusepath{stroke,fill}%
}%
\begin{pgfscope}%
\pgfsys@transformshift{1.761031in}{2.521191in}%
\pgfsys@useobject{currentmarker}{}%
\end{pgfscope}%
\end{pgfscope}%
\begin{pgfscope}%
\pgfsetbuttcap%
\pgfsetroundjoin%
\definecolor{currentfill}{rgb}{0.000000,0.000000,0.000000}%
\pgfsetfillcolor{currentfill}%
\pgfsetlinewidth{0.501875pt}%
\definecolor{currentstroke}{rgb}{0.000000,0.000000,0.000000}%
\pgfsetstrokecolor{currentstroke}%
\pgfsetdash{}{0pt}%
\pgfsys@defobject{currentmarker}{\pgfqpoint{0.000000in}{0.000000in}}{\pgfqpoint{0.000000in}{0.027778in}}{%
\pgfpathmoveto{\pgfqpoint{0.000000in}{0.000000in}}%
\pgfpathlineto{\pgfqpoint{0.000000in}{0.027778in}}%
\pgfusepath{stroke,fill}%
}%
\begin{pgfscope}%
\pgfsys@transformshift{1.797850in}{1.121191in}%
\pgfsys@useobject{currentmarker}{}%
\end{pgfscope}%
\end{pgfscope}%
\begin{pgfscope}%
\pgfsetbuttcap%
\pgfsetroundjoin%
\definecolor{currentfill}{rgb}{0.000000,0.000000,0.000000}%
\pgfsetfillcolor{currentfill}%
\pgfsetlinewidth{0.501875pt}%
\definecolor{currentstroke}{rgb}{0.000000,0.000000,0.000000}%
\pgfsetstrokecolor{currentstroke}%
\pgfsetdash{}{0pt}%
\pgfsys@defobject{currentmarker}{\pgfqpoint{0.000000in}{-0.027778in}}{\pgfqpoint{0.000000in}{0.000000in}}{%
\pgfpathmoveto{\pgfqpoint{0.000000in}{0.000000in}}%
\pgfpathlineto{\pgfqpoint{0.000000in}{-0.027778in}}%
\pgfusepath{stroke,fill}%
}%
\begin{pgfscope}%
\pgfsys@transformshift{1.797850in}{2.521191in}%
\pgfsys@useobject{currentmarker}{}%
\end{pgfscope}%
\end{pgfscope}%
\begin{pgfscope}%
\pgfsetbuttcap%
\pgfsetroundjoin%
\definecolor{currentfill}{rgb}{0.000000,0.000000,0.000000}%
\pgfsetfillcolor{currentfill}%
\pgfsetlinewidth{0.501875pt}%
\definecolor{currentstroke}{rgb}{0.000000,0.000000,0.000000}%
\pgfsetstrokecolor{currentstroke}%
\pgfsetdash{}{0pt}%
\pgfsys@defobject{currentmarker}{\pgfqpoint{0.000000in}{0.000000in}}{\pgfqpoint{0.000000in}{0.027778in}}{%
\pgfpathmoveto{\pgfqpoint{0.000000in}{0.000000in}}%
\pgfpathlineto{\pgfqpoint{0.000000in}{0.027778in}}%
\pgfusepath{stroke,fill}%
}%
\begin{pgfscope}%
\pgfsys@transformshift{1.828980in}{1.121191in}%
\pgfsys@useobject{currentmarker}{}%
\end{pgfscope}%
\end{pgfscope}%
\begin{pgfscope}%
\pgfsetbuttcap%
\pgfsetroundjoin%
\definecolor{currentfill}{rgb}{0.000000,0.000000,0.000000}%
\pgfsetfillcolor{currentfill}%
\pgfsetlinewidth{0.501875pt}%
\definecolor{currentstroke}{rgb}{0.000000,0.000000,0.000000}%
\pgfsetstrokecolor{currentstroke}%
\pgfsetdash{}{0pt}%
\pgfsys@defobject{currentmarker}{\pgfqpoint{0.000000in}{-0.027778in}}{\pgfqpoint{0.000000in}{0.000000in}}{%
\pgfpathmoveto{\pgfqpoint{0.000000in}{0.000000in}}%
\pgfpathlineto{\pgfqpoint{0.000000in}{-0.027778in}}%
\pgfusepath{stroke,fill}%
}%
\begin{pgfscope}%
\pgfsys@transformshift{1.828980in}{2.521191in}%
\pgfsys@useobject{currentmarker}{}%
\end{pgfscope}%
\end{pgfscope}%
\begin{pgfscope}%
\pgfsetbuttcap%
\pgfsetroundjoin%
\definecolor{currentfill}{rgb}{0.000000,0.000000,0.000000}%
\pgfsetfillcolor{currentfill}%
\pgfsetlinewidth{0.501875pt}%
\definecolor{currentstroke}{rgb}{0.000000,0.000000,0.000000}%
\pgfsetstrokecolor{currentstroke}%
\pgfsetdash{}{0pt}%
\pgfsys@defobject{currentmarker}{\pgfqpoint{0.000000in}{0.000000in}}{\pgfqpoint{0.000000in}{0.027778in}}{%
\pgfpathmoveto{\pgfqpoint{0.000000in}{0.000000in}}%
\pgfpathlineto{\pgfqpoint{0.000000in}{0.027778in}}%
\pgfusepath{stroke,fill}%
}%
\begin{pgfscope}%
\pgfsys@transformshift{1.855947in}{1.121191in}%
\pgfsys@useobject{currentmarker}{}%
\end{pgfscope}%
\end{pgfscope}%
\begin{pgfscope}%
\pgfsetbuttcap%
\pgfsetroundjoin%
\definecolor{currentfill}{rgb}{0.000000,0.000000,0.000000}%
\pgfsetfillcolor{currentfill}%
\pgfsetlinewidth{0.501875pt}%
\definecolor{currentstroke}{rgb}{0.000000,0.000000,0.000000}%
\pgfsetstrokecolor{currentstroke}%
\pgfsetdash{}{0pt}%
\pgfsys@defobject{currentmarker}{\pgfqpoint{0.000000in}{-0.027778in}}{\pgfqpoint{0.000000in}{0.000000in}}{%
\pgfpathmoveto{\pgfqpoint{0.000000in}{0.000000in}}%
\pgfpathlineto{\pgfqpoint{0.000000in}{-0.027778in}}%
\pgfusepath{stroke,fill}%
}%
\begin{pgfscope}%
\pgfsys@transformshift{1.855947in}{2.521191in}%
\pgfsys@useobject{currentmarker}{}%
\end{pgfscope}%
\end{pgfscope}%
\begin{pgfscope}%
\pgfsetbuttcap%
\pgfsetroundjoin%
\definecolor{currentfill}{rgb}{0.000000,0.000000,0.000000}%
\pgfsetfillcolor{currentfill}%
\pgfsetlinewidth{0.501875pt}%
\definecolor{currentstroke}{rgb}{0.000000,0.000000,0.000000}%
\pgfsetstrokecolor{currentstroke}%
\pgfsetdash{}{0pt}%
\pgfsys@defobject{currentmarker}{\pgfqpoint{0.000000in}{0.000000in}}{\pgfqpoint{0.000000in}{0.027778in}}{%
\pgfpathmoveto{\pgfqpoint{0.000000in}{0.000000in}}%
\pgfpathlineto{\pgfqpoint{0.000000in}{0.027778in}}%
\pgfusepath{stroke,fill}%
}%
\begin{pgfscope}%
\pgfsys@transformshift{1.879733in}{1.121191in}%
\pgfsys@useobject{currentmarker}{}%
\end{pgfscope}%
\end{pgfscope}%
\begin{pgfscope}%
\pgfsetbuttcap%
\pgfsetroundjoin%
\definecolor{currentfill}{rgb}{0.000000,0.000000,0.000000}%
\pgfsetfillcolor{currentfill}%
\pgfsetlinewidth{0.501875pt}%
\definecolor{currentstroke}{rgb}{0.000000,0.000000,0.000000}%
\pgfsetstrokecolor{currentstroke}%
\pgfsetdash{}{0pt}%
\pgfsys@defobject{currentmarker}{\pgfqpoint{0.000000in}{-0.027778in}}{\pgfqpoint{0.000000in}{0.000000in}}{%
\pgfpathmoveto{\pgfqpoint{0.000000in}{0.000000in}}%
\pgfpathlineto{\pgfqpoint{0.000000in}{-0.027778in}}%
\pgfusepath{stroke,fill}%
}%
\begin{pgfscope}%
\pgfsys@transformshift{1.879733in}{2.521191in}%
\pgfsys@useobject{currentmarker}{}%
\end{pgfscope}%
\end{pgfscope}%
\begin{pgfscope}%
\pgfsetbuttcap%
\pgfsetroundjoin%
\definecolor{currentfill}{rgb}{0.000000,0.000000,0.000000}%
\pgfsetfillcolor{currentfill}%
\pgfsetlinewidth{0.501875pt}%
\definecolor{currentstroke}{rgb}{0.000000,0.000000,0.000000}%
\pgfsetstrokecolor{currentstroke}%
\pgfsetdash{}{0pt}%
\pgfsys@defobject{currentmarker}{\pgfqpoint{0.000000in}{0.000000in}}{\pgfqpoint{0.000000in}{0.027778in}}{%
\pgfpathmoveto{\pgfqpoint{0.000000in}{0.000000in}}%
\pgfpathlineto{\pgfqpoint{0.000000in}{0.027778in}}%
\pgfusepath{stroke,fill}%
}%
\begin{pgfscope}%
\pgfsys@transformshift{2.040989in}{1.121191in}%
\pgfsys@useobject{currentmarker}{}%
\end{pgfscope}%
\end{pgfscope}%
\begin{pgfscope}%
\pgfsetbuttcap%
\pgfsetroundjoin%
\definecolor{currentfill}{rgb}{0.000000,0.000000,0.000000}%
\pgfsetfillcolor{currentfill}%
\pgfsetlinewidth{0.501875pt}%
\definecolor{currentstroke}{rgb}{0.000000,0.000000,0.000000}%
\pgfsetstrokecolor{currentstroke}%
\pgfsetdash{}{0pt}%
\pgfsys@defobject{currentmarker}{\pgfqpoint{0.000000in}{-0.027778in}}{\pgfqpoint{0.000000in}{0.000000in}}{%
\pgfpathmoveto{\pgfqpoint{0.000000in}{0.000000in}}%
\pgfpathlineto{\pgfqpoint{0.000000in}{-0.027778in}}%
\pgfusepath{stroke,fill}%
}%
\begin{pgfscope}%
\pgfsys@transformshift{2.040989in}{2.521191in}%
\pgfsys@useobject{currentmarker}{}%
\end{pgfscope}%
\end{pgfscope}%
\begin{pgfscope}%
\pgfsetbuttcap%
\pgfsetroundjoin%
\definecolor{currentfill}{rgb}{0.000000,0.000000,0.000000}%
\pgfsetfillcolor{currentfill}%
\pgfsetlinewidth{0.501875pt}%
\definecolor{currentstroke}{rgb}{0.000000,0.000000,0.000000}%
\pgfsetstrokecolor{currentstroke}%
\pgfsetdash{}{0pt}%
\pgfsys@defobject{currentmarker}{\pgfqpoint{0.000000in}{0.000000in}}{\pgfqpoint{0.000000in}{0.027778in}}{%
\pgfpathmoveto{\pgfqpoint{0.000000in}{0.000000in}}%
\pgfpathlineto{\pgfqpoint{0.000000in}{0.027778in}}%
\pgfusepath{stroke,fill}%
}%
\begin{pgfscope}%
\pgfsys@transformshift{2.122871in}{1.121191in}%
\pgfsys@useobject{currentmarker}{}%
\end{pgfscope}%
\end{pgfscope}%
\begin{pgfscope}%
\pgfsetbuttcap%
\pgfsetroundjoin%
\definecolor{currentfill}{rgb}{0.000000,0.000000,0.000000}%
\pgfsetfillcolor{currentfill}%
\pgfsetlinewidth{0.501875pt}%
\definecolor{currentstroke}{rgb}{0.000000,0.000000,0.000000}%
\pgfsetstrokecolor{currentstroke}%
\pgfsetdash{}{0pt}%
\pgfsys@defobject{currentmarker}{\pgfqpoint{0.000000in}{-0.027778in}}{\pgfqpoint{0.000000in}{0.000000in}}{%
\pgfpathmoveto{\pgfqpoint{0.000000in}{0.000000in}}%
\pgfpathlineto{\pgfqpoint{0.000000in}{-0.027778in}}%
\pgfusepath{stroke,fill}%
}%
\begin{pgfscope}%
\pgfsys@transformshift{2.122871in}{2.521191in}%
\pgfsys@useobject{currentmarker}{}%
\end{pgfscope}%
\end{pgfscope}%
\begin{pgfscope}%
\pgfsetbuttcap%
\pgfsetroundjoin%
\definecolor{currentfill}{rgb}{0.000000,0.000000,0.000000}%
\pgfsetfillcolor{currentfill}%
\pgfsetlinewidth{0.501875pt}%
\definecolor{currentstroke}{rgb}{0.000000,0.000000,0.000000}%
\pgfsetstrokecolor{currentstroke}%
\pgfsetdash{}{0pt}%
\pgfsys@defobject{currentmarker}{\pgfqpoint{0.000000in}{0.000000in}}{\pgfqpoint{0.000000in}{0.027778in}}{%
\pgfpathmoveto{\pgfqpoint{0.000000in}{0.000000in}}%
\pgfpathlineto{\pgfqpoint{0.000000in}{0.027778in}}%
\pgfusepath{stroke,fill}%
}%
\begin{pgfscope}%
\pgfsys@transformshift{2.180968in}{1.121191in}%
\pgfsys@useobject{currentmarker}{}%
\end{pgfscope}%
\end{pgfscope}%
\begin{pgfscope}%
\pgfsetbuttcap%
\pgfsetroundjoin%
\definecolor{currentfill}{rgb}{0.000000,0.000000,0.000000}%
\pgfsetfillcolor{currentfill}%
\pgfsetlinewidth{0.501875pt}%
\definecolor{currentstroke}{rgb}{0.000000,0.000000,0.000000}%
\pgfsetstrokecolor{currentstroke}%
\pgfsetdash{}{0pt}%
\pgfsys@defobject{currentmarker}{\pgfqpoint{0.000000in}{-0.027778in}}{\pgfqpoint{0.000000in}{0.000000in}}{%
\pgfpathmoveto{\pgfqpoint{0.000000in}{0.000000in}}%
\pgfpathlineto{\pgfqpoint{0.000000in}{-0.027778in}}%
\pgfusepath{stroke,fill}%
}%
\begin{pgfscope}%
\pgfsys@transformshift{2.180968in}{2.521191in}%
\pgfsys@useobject{currentmarker}{}%
\end{pgfscope}%
\end{pgfscope}%
\begin{pgfscope}%
\pgfsetbuttcap%
\pgfsetroundjoin%
\definecolor{currentfill}{rgb}{0.000000,0.000000,0.000000}%
\pgfsetfillcolor{currentfill}%
\pgfsetlinewidth{0.501875pt}%
\definecolor{currentstroke}{rgb}{0.000000,0.000000,0.000000}%
\pgfsetstrokecolor{currentstroke}%
\pgfsetdash{}{0pt}%
\pgfsys@defobject{currentmarker}{\pgfqpoint{0.000000in}{0.000000in}}{\pgfqpoint{0.000000in}{0.027778in}}{%
\pgfpathmoveto{\pgfqpoint{0.000000in}{0.000000in}}%
\pgfpathlineto{\pgfqpoint{0.000000in}{0.027778in}}%
\pgfusepath{stroke,fill}%
}%
\begin{pgfscope}%
\pgfsys@transformshift{2.226031in}{1.121191in}%
\pgfsys@useobject{currentmarker}{}%
\end{pgfscope}%
\end{pgfscope}%
\begin{pgfscope}%
\pgfsetbuttcap%
\pgfsetroundjoin%
\definecolor{currentfill}{rgb}{0.000000,0.000000,0.000000}%
\pgfsetfillcolor{currentfill}%
\pgfsetlinewidth{0.501875pt}%
\definecolor{currentstroke}{rgb}{0.000000,0.000000,0.000000}%
\pgfsetstrokecolor{currentstroke}%
\pgfsetdash{}{0pt}%
\pgfsys@defobject{currentmarker}{\pgfqpoint{0.000000in}{-0.027778in}}{\pgfqpoint{0.000000in}{0.000000in}}{%
\pgfpathmoveto{\pgfqpoint{0.000000in}{0.000000in}}%
\pgfpathlineto{\pgfqpoint{0.000000in}{-0.027778in}}%
\pgfusepath{stroke,fill}%
}%
\begin{pgfscope}%
\pgfsys@transformshift{2.226031in}{2.521191in}%
\pgfsys@useobject{currentmarker}{}%
\end{pgfscope}%
\end{pgfscope}%
\begin{pgfscope}%
\pgfsetbuttcap%
\pgfsetroundjoin%
\definecolor{currentfill}{rgb}{0.000000,0.000000,0.000000}%
\pgfsetfillcolor{currentfill}%
\pgfsetlinewidth{0.501875pt}%
\definecolor{currentstroke}{rgb}{0.000000,0.000000,0.000000}%
\pgfsetstrokecolor{currentstroke}%
\pgfsetdash{}{0pt}%
\pgfsys@defobject{currentmarker}{\pgfqpoint{0.000000in}{0.000000in}}{\pgfqpoint{0.000000in}{0.027778in}}{%
\pgfpathmoveto{\pgfqpoint{0.000000in}{0.000000in}}%
\pgfpathlineto{\pgfqpoint{0.000000in}{0.027778in}}%
\pgfusepath{stroke,fill}%
}%
\begin{pgfscope}%
\pgfsys@transformshift{2.262850in}{1.121191in}%
\pgfsys@useobject{currentmarker}{}%
\end{pgfscope}%
\end{pgfscope}%
\begin{pgfscope}%
\pgfsetbuttcap%
\pgfsetroundjoin%
\definecolor{currentfill}{rgb}{0.000000,0.000000,0.000000}%
\pgfsetfillcolor{currentfill}%
\pgfsetlinewidth{0.501875pt}%
\definecolor{currentstroke}{rgb}{0.000000,0.000000,0.000000}%
\pgfsetstrokecolor{currentstroke}%
\pgfsetdash{}{0pt}%
\pgfsys@defobject{currentmarker}{\pgfqpoint{0.000000in}{-0.027778in}}{\pgfqpoint{0.000000in}{0.000000in}}{%
\pgfpathmoveto{\pgfqpoint{0.000000in}{0.000000in}}%
\pgfpathlineto{\pgfqpoint{0.000000in}{-0.027778in}}%
\pgfusepath{stroke,fill}%
}%
\begin{pgfscope}%
\pgfsys@transformshift{2.262850in}{2.521191in}%
\pgfsys@useobject{currentmarker}{}%
\end{pgfscope}%
\end{pgfscope}%
\begin{pgfscope}%
\pgfsetbuttcap%
\pgfsetroundjoin%
\definecolor{currentfill}{rgb}{0.000000,0.000000,0.000000}%
\pgfsetfillcolor{currentfill}%
\pgfsetlinewidth{0.501875pt}%
\definecolor{currentstroke}{rgb}{0.000000,0.000000,0.000000}%
\pgfsetstrokecolor{currentstroke}%
\pgfsetdash{}{0pt}%
\pgfsys@defobject{currentmarker}{\pgfqpoint{0.000000in}{0.000000in}}{\pgfqpoint{0.000000in}{0.027778in}}{%
\pgfpathmoveto{\pgfqpoint{0.000000in}{0.000000in}}%
\pgfpathlineto{\pgfqpoint{0.000000in}{0.027778in}}%
\pgfusepath{stroke,fill}%
}%
\begin{pgfscope}%
\pgfsys@transformshift{2.293980in}{1.121191in}%
\pgfsys@useobject{currentmarker}{}%
\end{pgfscope}%
\end{pgfscope}%
\begin{pgfscope}%
\pgfsetbuttcap%
\pgfsetroundjoin%
\definecolor{currentfill}{rgb}{0.000000,0.000000,0.000000}%
\pgfsetfillcolor{currentfill}%
\pgfsetlinewidth{0.501875pt}%
\definecolor{currentstroke}{rgb}{0.000000,0.000000,0.000000}%
\pgfsetstrokecolor{currentstroke}%
\pgfsetdash{}{0pt}%
\pgfsys@defobject{currentmarker}{\pgfqpoint{0.000000in}{-0.027778in}}{\pgfqpoint{0.000000in}{0.000000in}}{%
\pgfpathmoveto{\pgfqpoint{0.000000in}{0.000000in}}%
\pgfpathlineto{\pgfqpoint{0.000000in}{-0.027778in}}%
\pgfusepath{stroke,fill}%
}%
\begin{pgfscope}%
\pgfsys@transformshift{2.293980in}{2.521191in}%
\pgfsys@useobject{currentmarker}{}%
\end{pgfscope}%
\end{pgfscope}%
\begin{pgfscope}%
\pgfsetbuttcap%
\pgfsetroundjoin%
\definecolor{currentfill}{rgb}{0.000000,0.000000,0.000000}%
\pgfsetfillcolor{currentfill}%
\pgfsetlinewidth{0.501875pt}%
\definecolor{currentstroke}{rgb}{0.000000,0.000000,0.000000}%
\pgfsetstrokecolor{currentstroke}%
\pgfsetdash{}{0pt}%
\pgfsys@defobject{currentmarker}{\pgfqpoint{0.000000in}{0.000000in}}{\pgfqpoint{0.000000in}{0.027778in}}{%
\pgfpathmoveto{\pgfqpoint{0.000000in}{0.000000in}}%
\pgfpathlineto{\pgfqpoint{0.000000in}{0.027778in}}%
\pgfusepath{stroke,fill}%
}%
\begin{pgfscope}%
\pgfsys@transformshift{2.320947in}{1.121191in}%
\pgfsys@useobject{currentmarker}{}%
\end{pgfscope}%
\end{pgfscope}%
\begin{pgfscope}%
\pgfsetbuttcap%
\pgfsetroundjoin%
\definecolor{currentfill}{rgb}{0.000000,0.000000,0.000000}%
\pgfsetfillcolor{currentfill}%
\pgfsetlinewidth{0.501875pt}%
\definecolor{currentstroke}{rgb}{0.000000,0.000000,0.000000}%
\pgfsetstrokecolor{currentstroke}%
\pgfsetdash{}{0pt}%
\pgfsys@defobject{currentmarker}{\pgfqpoint{0.000000in}{-0.027778in}}{\pgfqpoint{0.000000in}{0.000000in}}{%
\pgfpathmoveto{\pgfqpoint{0.000000in}{0.000000in}}%
\pgfpathlineto{\pgfqpoint{0.000000in}{-0.027778in}}%
\pgfusepath{stroke,fill}%
}%
\begin{pgfscope}%
\pgfsys@transformshift{2.320947in}{2.521191in}%
\pgfsys@useobject{currentmarker}{}%
\end{pgfscope}%
\end{pgfscope}%
\begin{pgfscope}%
\pgfsetbuttcap%
\pgfsetroundjoin%
\definecolor{currentfill}{rgb}{0.000000,0.000000,0.000000}%
\pgfsetfillcolor{currentfill}%
\pgfsetlinewidth{0.501875pt}%
\definecolor{currentstroke}{rgb}{0.000000,0.000000,0.000000}%
\pgfsetstrokecolor{currentstroke}%
\pgfsetdash{}{0pt}%
\pgfsys@defobject{currentmarker}{\pgfqpoint{0.000000in}{0.000000in}}{\pgfqpoint{0.000000in}{0.027778in}}{%
\pgfpathmoveto{\pgfqpoint{0.000000in}{0.000000in}}%
\pgfpathlineto{\pgfqpoint{0.000000in}{0.027778in}}%
\pgfusepath{stroke,fill}%
}%
\begin{pgfscope}%
\pgfsys@transformshift{2.344733in}{1.121191in}%
\pgfsys@useobject{currentmarker}{}%
\end{pgfscope}%
\end{pgfscope}%
\begin{pgfscope}%
\pgfsetbuttcap%
\pgfsetroundjoin%
\definecolor{currentfill}{rgb}{0.000000,0.000000,0.000000}%
\pgfsetfillcolor{currentfill}%
\pgfsetlinewidth{0.501875pt}%
\definecolor{currentstroke}{rgb}{0.000000,0.000000,0.000000}%
\pgfsetstrokecolor{currentstroke}%
\pgfsetdash{}{0pt}%
\pgfsys@defobject{currentmarker}{\pgfqpoint{0.000000in}{-0.027778in}}{\pgfqpoint{0.000000in}{0.000000in}}{%
\pgfpathmoveto{\pgfqpoint{0.000000in}{0.000000in}}%
\pgfpathlineto{\pgfqpoint{0.000000in}{-0.027778in}}%
\pgfusepath{stroke,fill}%
}%
\begin{pgfscope}%
\pgfsys@transformshift{2.344733in}{2.521191in}%
\pgfsys@useobject{currentmarker}{}%
\end{pgfscope}%
\end{pgfscope}%
\begin{pgfscope}%
\pgfsetbuttcap%
\pgfsetroundjoin%
\definecolor{currentfill}{rgb}{0.000000,0.000000,0.000000}%
\pgfsetfillcolor{currentfill}%
\pgfsetlinewidth{0.501875pt}%
\definecolor{currentstroke}{rgb}{0.000000,0.000000,0.000000}%
\pgfsetstrokecolor{currentstroke}%
\pgfsetdash{}{0pt}%
\pgfsys@defobject{currentmarker}{\pgfqpoint{0.000000in}{0.000000in}}{\pgfqpoint{0.000000in}{0.027778in}}{%
\pgfpathmoveto{\pgfqpoint{0.000000in}{0.000000in}}%
\pgfpathlineto{\pgfqpoint{0.000000in}{0.027778in}}%
\pgfusepath{stroke,fill}%
}%
\begin{pgfscope}%
\pgfsys@transformshift{2.505989in}{1.121191in}%
\pgfsys@useobject{currentmarker}{}%
\end{pgfscope}%
\end{pgfscope}%
\begin{pgfscope}%
\pgfsetbuttcap%
\pgfsetroundjoin%
\definecolor{currentfill}{rgb}{0.000000,0.000000,0.000000}%
\pgfsetfillcolor{currentfill}%
\pgfsetlinewidth{0.501875pt}%
\definecolor{currentstroke}{rgb}{0.000000,0.000000,0.000000}%
\pgfsetstrokecolor{currentstroke}%
\pgfsetdash{}{0pt}%
\pgfsys@defobject{currentmarker}{\pgfqpoint{0.000000in}{-0.027778in}}{\pgfqpoint{0.000000in}{0.000000in}}{%
\pgfpathmoveto{\pgfqpoint{0.000000in}{0.000000in}}%
\pgfpathlineto{\pgfqpoint{0.000000in}{-0.027778in}}%
\pgfusepath{stroke,fill}%
}%
\begin{pgfscope}%
\pgfsys@transformshift{2.505989in}{2.521191in}%
\pgfsys@useobject{currentmarker}{}%
\end{pgfscope}%
\end{pgfscope}%
\begin{pgfscope}%
\pgfsetbuttcap%
\pgfsetroundjoin%
\definecolor{currentfill}{rgb}{0.000000,0.000000,0.000000}%
\pgfsetfillcolor{currentfill}%
\pgfsetlinewidth{0.501875pt}%
\definecolor{currentstroke}{rgb}{0.000000,0.000000,0.000000}%
\pgfsetstrokecolor{currentstroke}%
\pgfsetdash{}{0pt}%
\pgfsys@defobject{currentmarker}{\pgfqpoint{0.000000in}{0.000000in}}{\pgfqpoint{0.000000in}{0.027778in}}{%
\pgfpathmoveto{\pgfqpoint{0.000000in}{0.000000in}}%
\pgfpathlineto{\pgfqpoint{0.000000in}{0.027778in}}%
\pgfusepath{stroke,fill}%
}%
\begin{pgfscope}%
\pgfsys@transformshift{2.587871in}{1.121191in}%
\pgfsys@useobject{currentmarker}{}%
\end{pgfscope}%
\end{pgfscope}%
\begin{pgfscope}%
\pgfsetbuttcap%
\pgfsetroundjoin%
\definecolor{currentfill}{rgb}{0.000000,0.000000,0.000000}%
\pgfsetfillcolor{currentfill}%
\pgfsetlinewidth{0.501875pt}%
\definecolor{currentstroke}{rgb}{0.000000,0.000000,0.000000}%
\pgfsetstrokecolor{currentstroke}%
\pgfsetdash{}{0pt}%
\pgfsys@defobject{currentmarker}{\pgfqpoint{0.000000in}{-0.027778in}}{\pgfqpoint{0.000000in}{0.000000in}}{%
\pgfpathmoveto{\pgfqpoint{0.000000in}{0.000000in}}%
\pgfpathlineto{\pgfqpoint{0.000000in}{-0.027778in}}%
\pgfusepath{stroke,fill}%
}%
\begin{pgfscope}%
\pgfsys@transformshift{2.587871in}{2.521191in}%
\pgfsys@useobject{currentmarker}{}%
\end{pgfscope}%
\end{pgfscope}%
\begin{pgfscope}%
\pgfsetbuttcap%
\pgfsetroundjoin%
\definecolor{currentfill}{rgb}{0.000000,0.000000,0.000000}%
\pgfsetfillcolor{currentfill}%
\pgfsetlinewidth{0.501875pt}%
\definecolor{currentstroke}{rgb}{0.000000,0.000000,0.000000}%
\pgfsetstrokecolor{currentstroke}%
\pgfsetdash{}{0pt}%
\pgfsys@defobject{currentmarker}{\pgfqpoint{0.000000in}{0.000000in}}{\pgfqpoint{0.000000in}{0.027778in}}{%
\pgfpathmoveto{\pgfqpoint{0.000000in}{0.000000in}}%
\pgfpathlineto{\pgfqpoint{0.000000in}{0.027778in}}%
\pgfusepath{stroke,fill}%
}%
\begin{pgfscope}%
\pgfsys@transformshift{2.645968in}{1.121191in}%
\pgfsys@useobject{currentmarker}{}%
\end{pgfscope}%
\end{pgfscope}%
\begin{pgfscope}%
\pgfsetbuttcap%
\pgfsetroundjoin%
\definecolor{currentfill}{rgb}{0.000000,0.000000,0.000000}%
\pgfsetfillcolor{currentfill}%
\pgfsetlinewidth{0.501875pt}%
\definecolor{currentstroke}{rgb}{0.000000,0.000000,0.000000}%
\pgfsetstrokecolor{currentstroke}%
\pgfsetdash{}{0pt}%
\pgfsys@defobject{currentmarker}{\pgfqpoint{0.000000in}{-0.027778in}}{\pgfqpoint{0.000000in}{0.000000in}}{%
\pgfpathmoveto{\pgfqpoint{0.000000in}{0.000000in}}%
\pgfpathlineto{\pgfqpoint{0.000000in}{-0.027778in}}%
\pgfusepath{stroke,fill}%
}%
\begin{pgfscope}%
\pgfsys@transformshift{2.645968in}{2.521191in}%
\pgfsys@useobject{currentmarker}{}%
\end{pgfscope}%
\end{pgfscope}%
\begin{pgfscope}%
\pgfsetbuttcap%
\pgfsetroundjoin%
\definecolor{currentfill}{rgb}{0.000000,0.000000,0.000000}%
\pgfsetfillcolor{currentfill}%
\pgfsetlinewidth{0.501875pt}%
\definecolor{currentstroke}{rgb}{0.000000,0.000000,0.000000}%
\pgfsetstrokecolor{currentstroke}%
\pgfsetdash{}{0pt}%
\pgfsys@defobject{currentmarker}{\pgfqpoint{0.000000in}{0.000000in}}{\pgfqpoint{0.000000in}{0.027778in}}{%
\pgfpathmoveto{\pgfqpoint{0.000000in}{0.000000in}}%
\pgfpathlineto{\pgfqpoint{0.000000in}{0.027778in}}%
\pgfusepath{stroke,fill}%
}%
\begin{pgfscope}%
\pgfsys@transformshift{2.691031in}{1.121191in}%
\pgfsys@useobject{currentmarker}{}%
\end{pgfscope}%
\end{pgfscope}%
\begin{pgfscope}%
\pgfsetbuttcap%
\pgfsetroundjoin%
\definecolor{currentfill}{rgb}{0.000000,0.000000,0.000000}%
\pgfsetfillcolor{currentfill}%
\pgfsetlinewidth{0.501875pt}%
\definecolor{currentstroke}{rgb}{0.000000,0.000000,0.000000}%
\pgfsetstrokecolor{currentstroke}%
\pgfsetdash{}{0pt}%
\pgfsys@defobject{currentmarker}{\pgfqpoint{0.000000in}{-0.027778in}}{\pgfqpoint{0.000000in}{0.000000in}}{%
\pgfpathmoveto{\pgfqpoint{0.000000in}{0.000000in}}%
\pgfpathlineto{\pgfqpoint{0.000000in}{-0.027778in}}%
\pgfusepath{stroke,fill}%
}%
\begin{pgfscope}%
\pgfsys@transformshift{2.691031in}{2.521191in}%
\pgfsys@useobject{currentmarker}{}%
\end{pgfscope}%
\end{pgfscope}%
\begin{pgfscope}%
\pgfsetbuttcap%
\pgfsetroundjoin%
\definecolor{currentfill}{rgb}{0.000000,0.000000,0.000000}%
\pgfsetfillcolor{currentfill}%
\pgfsetlinewidth{0.501875pt}%
\definecolor{currentstroke}{rgb}{0.000000,0.000000,0.000000}%
\pgfsetstrokecolor{currentstroke}%
\pgfsetdash{}{0pt}%
\pgfsys@defobject{currentmarker}{\pgfqpoint{0.000000in}{0.000000in}}{\pgfqpoint{0.000000in}{0.027778in}}{%
\pgfpathmoveto{\pgfqpoint{0.000000in}{0.000000in}}%
\pgfpathlineto{\pgfqpoint{0.000000in}{0.027778in}}%
\pgfusepath{stroke,fill}%
}%
\begin{pgfscope}%
\pgfsys@transformshift{2.727850in}{1.121191in}%
\pgfsys@useobject{currentmarker}{}%
\end{pgfscope}%
\end{pgfscope}%
\begin{pgfscope}%
\pgfsetbuttcap%
\pgfsetroundjoin%
\definecolor{currentfill}{rgb}{0.000000,0.000000,0.000000}%
\pgfsetfillcolor{currentfill}%
\pgfsetlinewidth{0.501875pt}%
\definecolor{currentstroke}{rgb}{0.000000,0.000000,0.000000}%
\pgfsetstrokecolor{currentstroke}%
\pgfsetdash{}{0pt}%
\pgfsys@defobject{currentmarker}{\pgfqpoint{0.000000in}{-0.027778in}}{\pgfqpoint{0.000000in}{0.000000in}}{%
\pgfpathmoveto{\pgfqpoint{0.000000in}{0.000000in}}%
\pgfpathlineto{\pgfqpoint{0.000000in}{-0.027778in}}%
\pgfusepath{stroke,fill}%
}%
\begin{pgfscope}%
\pgfsys@transformshift{2.727850in}{2.521191in}%
\pgfsys@useobject{currentmarker}{}%
\end{pgfscope}%
\end{pgfscope}%
\begin{pgfscope}%
\pgfsetbuttcap%
\pgfsetroundjoin%
\definecolor{currentfill}{rgb}{0.000000,0.000000,0.000000}%
\pgfsetfillcolor{currentfill}%
\pgfsetlinewidth{0.501875pt}%
\definecolor{currentstroke}{rgb}{0.000000,0.000000,0.000000}%
\pgfsetstrokecolor{currentstroke}%
\pgfsetdash{}{0pt}%
\pgfsys@defobject{currentmarker}{\pgfqpoint{0.000000in}{0.000000in}}{\pgfqpoint{0.000000in}{0.027778in}}{%
\pgfpathmoveto{\pgfqpoint{0.000000in}{0.000000in}}%
\pgfpathlineto{\pgfqpoint{0.000000in}{0.027778in}}%
\pgfusepath{stroke,fill}%
}%
\begin{pgfscope}%
\pgfsys@transformshift{2.758980in}{1.121191in}%
\pgfsys@useobject{currentmarker}{}%
\end{pgfscope}%
\end{pgfscope}%
\begin{pgfscope}%
\pgfsetbuttcap%
\pgfsetroundjoin%
\definecolor{currentfill}{rgb}{0.000000,0.000000,0.000000}%
\pgfsetfillcolor{currentfill}%
\pgfsetlinewidth{0.501875pt}%
\definecolor{currentstroke}{rgb}{0.000000,0.000000,0.000000}%
\pgfsetstrokecolor{currentstroke}%
\pgfsetdash{}{0pt}%
\pgfsys@defobject{currentmarker}{\pgfqpoint{0.000000in}{-0.027778in}}{\pgfqpoint{0.000000in}{0.000000in}}{%
\pgfpathmoveto{\pgfqpoint{0.000000in}{0.000000in}}%
\pgfpathlineto{\pgfqpoint{0.000000in}{-0.027778in}}%
\pgfusepath{stroke,fill}%
}%
\begin{pgfscope}%
\pgfsys@transformshift{2.758980in}{2.521191in}%
\pgfsys@useobject{currentmarker}{}%
\end{pgfscope}%
\end{pgfscope}%
\begin{pgfscope}%
\pgfsetbuttcap%
\pgfsetroundjoin%
\definecolor{currentfill}{rgb}{0.000000,0.000000,0.000000}%
\pgfsetfillcolor{currentfill}%
\pgfsetlinewidth{0.501875pt}%
\definecolor{currentstroke}{rgb}{0.000000,0.000000,0.000000}%
\pgfsetstrokecolor{currentstroke}%
\pgfsetdash{}{0pt}%
\pgfsys@defobject{currentmarker}{\pgfqpoint{0.000000in}{0.000000in}}{\pgfqpoint{0.000000in}{0.027778in}}{%
\pgfpathmoveto{\pgfqpoint{0.000000in}{0.000000in}}%
\pgfpathlineto{\pgfqpoint{0.000000in}{0.027778in}}%
\pgfusepath{stroke,fill}%
}%
\begin{pgfscope}%
\pgfsys@transformshift{2.785947in}{1.121191in}%
\pgfsys@useobject{currentmarker}{}%
\end{pgfscope}%
\end{pgfscope}%
\begin{pgfscope}%
\pgfsetbuttcap%
\pgfsetroundjoin%
\definecolor{currentfill}{rgb}{0.000000,0.000000,0.000000}%
\pgfsetfillcolor{currentfill}%
\pgfsetlinewidth{0.501875pt}%
\definecolor{currentstroke}{rgb}{0.000000,0.000000,0.000000}%
\pgfsetstrokecolor{currentstroke}%
\pgfsetdash{}{0pt}%
\pgfsys@defobject{currentmarker}{\pgfqpoint{0.000000in}{-0.027778in}}{\pgfqpoint{0.000000in}{0.000000in}}{%
\pgfpathmoveto{\pgfqpoint{0.000000in}{0.000000in}}%
\pgfpathlineto{\pgfqpoint{0.000000in}{-0.027778in}}%
\pgfusepath{stroke,fill}%
}%
\begin{pgfscope}%
\pgfsys@transformshift{2.785947in}{2.521191in}%
\pgfsys@useobject{currentmarker}{}%
\end{pgfscope}%
\end{pgfscope}%
\begin{pgfscope}%
\pgfsetbuttcap%
\pgfsetroundjoin%
\definecolor{currentfill}{rgb}{0.000000,0.000000,0.000000}%
\pgfsetfillcolor{currentfill}%
\pgfsetlinewidth{0.501875pt}%
\definecolor{currentstroke}{rgb}{0.000000,0.000000,0.000000}%
\pgfsetstrokecolor{currentstroke}%
\pgfsetdash{}{0pt}%
\pgfsys@defobject{currentmarker}{\pgfqpoint{0.000000in}{0.000000in}}{\pgfqpoint{0.000000in}{0.027778in}}{%
\pgfpathmoveto{\pgfqpoint{0.000000in}{0.000000in}}%
\pgfpathlineto{\pgfqpoint{0.000000in}{0.027778in}}%
\pgfusepath{stroke,fill}%
}%
\begin{pgfscope}%
\pgfsys@transformshift{2.809733in}{1.121191in}%
\pgfsys@useobject{currentmarker}{}%
\end{pgfscope}%
\end{pgfscope}%
\begin{pgfscope}%
\pgfsetbuttcap%
\pgfsetroundjoin%
\definecolor{currentfill}{rgb}{0.000000,0.000000,0.000000}%
\pgfsetfillcolor{currentfill}%
\pgfsetlinewidth{0.501875pt}%
\definecolor{currentstroke}{rgb}{0.000000,0.000000,0.000000}%
\pgfsetstrokecolor{currentstroke}%
\pgfsetdash{}{0pt}%
\pgfsys@defobject{currentmarker}{\pgfqpoint{0.000000in}{-0.027778in}}{\pgfqpoint{0.000000in}{0.000000in}}{%
\pgfpathmoveto{\pgfqpoint{0.000000in}{0.000000in}}%
\pgfpathlineto{\pgfqpoint{0.000000in}{-0.027778in}}%
\pgfusepath{stroke,fill}%
}%
\begin{pgfscope}%
\pgfsys@transformshift{2.809733in}{2.521191in}%
\pgfsys@useobject{currentmarker}{}%
\end{pgfscope}%
\end{pgfscope}%
\begin{pgfscope}%
\pgftext[x=1.668510in,y=0.892267in,,top]{{\rmfamily\fontsize{8.328000}{9.993600}\selectfont Density \(\displaystyle \frac{2m}{n(n-1)}\)}}%
\end{pgfscope}%
\begin{pgfscope}%
\pgfpathrectangle{\pgfqpoint{0.506010in}{1.121191in}}{\pgfqpoint{2.325000in}{1.400000in}} %
\pgfusepath{clip}%
\pgfsetbuttcap%
\pgfsetroundjoin%
\pgfsetlinewidth{0.501875pt}%
\definecolor{currentstroke}{rgb}{0.000000,0.000000,0.000000}%
\pgfsetstrokecolor{currentstroke}%
\pgfsetdash{{1.000000pt}{3.000000pt}}{0.000000pt}%
\pgfpathmoveto{\pgfqpoint{0.506010in}{1.121191in}}%
\pgfpathlineto{\pgfqpoint{2.831010in}{1.121191in}}%
\pgfusepath{stroke}%
\end{pgfscope}%
\begin{pgfscope}%
\pgfsetbuttcap%
\pgfsetroundjoin%
\definecolor{currentfill}{rgb}{0.000000,0.000000,0.000000}%
\pgfsetfillcolor{currentfill}%
\pgfsetlinewidth{0.501875pt}%
\definecolor{currentstroke}{rgb}{0.000000,0.000000,0.000000}%
\pgfsetstrokecolor{currentstroke}%
\pgfsetdash{}{0pt}%
\pgfsys@defobject{currentmarker}{\pgfqpoint{0.000000in}{0.000000in}}{\pgfqpoint{0.055556in}{0.000000in}}{%
\pgfpathmoveto{\pgfqpoint{0.000000in}{0.000000in}}%
\pgfpathlineto{\pgfqpoint{0.055556in}{0.000000in}}%
\pgfusepath{stroke,fill}%
}%
\begin{pgfscope}%
\pgfsys@transformshift{0.506010in}{1.121191in}%
\pgfsys@useobject{currentmarker}{}%
\end{pgfscope}%
\end{pgfscope}%
\begin{pgfscope}%
\pgfsetbuttcap%
\pgfsetroundjoin%
\definecolor{currentfill}{rgb}{0.000000,0.000000,0.000000}%
\pgfsetfillcolor{currentfill}%
\pgfsetlinewidth{0.501875pt}%
\definecolor{currentstroke}{rgb}{0.000000,0.000000,0.000000}%
\pgfsetstrokecolor{currentstroke}%
\pgfsetdash{}{0pt}%
\pgfsys@defobject{currentmarker}{\pgfqpoint{-0.055556in}{0.000000in}}{\pgfqpoint{0.000000in}{0.000000in}}{%
\pgfpathmoveto{\pgfqpoint{0.000000in}{0.000000in}}%
\pgfpathlineto{\pgfqpoint{-0.055556in}{0.000000in}}%
\pgfusepath{stroke,fill}%
}%
\begin{pgfscope}%
\pgfsys@transformshift{2.831010in}{1.121191in}%
\pgfsys@useobject{currentmarker}{}%
\end{pgfscope}%
\end{pgfscope}%
\begin{pgfscope}%
\pgftext[x=0.450454in,y=1.121191in,right,]{{\rmfamily\fontsize{8.328000}{9.993600}\selectfont \(\displaystyle 0\)}}%
\end{pgfscope}%
\begin{pgfscope}%
\pgfpathrectangle{\pgfqpoint{0.506010in}{1.121191in}}{\pgfqpoint{2.325000in}{1.400000in}} %
\pgfusepath{clip}%
\pgfsetbuttcap%
\pgfsetroundjoin%
\pgfsetlinewidth{0.501875pt}%
\definecolor{currentstroke}{rgb}{0.000000,0.000000,0.000000}%
\pgfsetstrokecolor{currentstroke}%
\pgfsetdash{{1.000000pt}{3.000000pt}}{0.000000pt}%
\pgfpathmoveto{\pgfqpoint{0.506010in}{1.401191in}}%
\pgfpathlineto{\pgfqpoint{2.831010in}{1.401191in}}%
\pgfusepath{stroke}%
\end{pgfscope}%
\begin{pgfscope}%
\pgfsetbuttcap%
\pgfsetroundjoin%
\definecolor{currentfill}{rgb}{0.000000,0.000000,0.000000}%
\pgfsetfillcolor{currentfill}%
\pgfsetlinewidth{0.501875pt}%
\definecolor{currentstroke}{rgb}{0.000000,0.000000,0.000000}%
\pgfsetstrokecolor{currentstroke}%
\pgfsetdash{}{0pt}%
\pgfsys@defobject{currentmarker}{\pgfqpoint{0.000000in}{0.000000in}}{\pgfqpoint{0.055556in}{0.000000in}}{%
\pgfpathmoveto{\pgfqpoint{0.000000in}{0.000000in}}%
\pgfpathlineto{\pgfqpoint{0.055556in}{0.000000in}}%
\pgfusepath{stroke,fill}%
}%
\begin{pgfscope}%
\pgfsys@transformshift{0.506010in}{1.401191in}%
\pgfsys@useobject{currentmarker}{}%
\end{pgfscope}%
\end{pgfscope}%
\begin{pgfscope}%
\pgfsetbuttcap%
\pgfsetroundjoin%
\definecolor{currentfill}{rgb}{0.000000,0.000000,0.000000}%
\pgfsetfillcolor{currentfill}%
\pgfsetlinewidth{0.501875pt}%
\definecolor{currentstroke}{rgb}{0.000000,0.000000,0.000000}%
\pgfsetstrokecolor{currentstroke}%
\pgfsetdash{}{0pt}%
\pgfsys@defobject{currentmarker}{\pgfqpoint{-0.055556in}{0.000000in}}{\pgfqpoint{0.000000in}{0.000000in}}{%
\pgfpathmoveto{\pgfqpoint{0.000000in}{0.000000in}}%
\pgfpathlineto{\pgfqpoint{-0.055556in}{0.000000in}}%
\pgfusepath{stroke,fill}%
}%
\begin{pgfscope}%
\pgfsys@transformshift{2.831010in}{1.401191in}%
\pgfsys@useobject{currentmarker}{}%
\end{pgfscope}%
\end{pgfscope}%
\begin{pgfscope}%
\pgftext[x=0.450454in,y=1.401191in,right,]{{\rmfamily\fontsize{8.328000}{9.993600}\selectfont \(\displaystyle 100\)}}%
\end{pgfscope}%
\begin{pgfscope}%
\pgfpathrectangle{\pgfqpoint{0.506010in}{1.121191in}}{\pgfqpoint{2.325000in}{1.400000in}} %
\pgfusepath{clip}%
\pgfsetbuttcap%
\pgfsetroundjoin%
\pgfsetlinewidth{0.501875pt}%
\definecolor{currentstroke}{rgb}{0.000000,0.000000,0.000000}%
\pgfsetstrokecolor{currentstroke}%
\pgfsetdash{{1.000000pt}{3.000000pt}}{0.000000pt}%
\pgfpathmoveto{\pgfqpoint{0.506010in}{1.681191in}}%
\pgfpathlineto{\pgfqpoint{2.831010in}{1.681191in}}%
\pgfusepath{stroke}%
\end{pgfscope}%
\begin{pgfscope}%
\pgfsetbuttcap%
\pgfsetroundjoin%
\definecolor{currentfill}{rgb}{0.000000,0.000000,0.000000}%
\pgfsetfillcolor{currentfill}%
\pgfsetlinewidth{0.501875pt}%
\definecolor{currentstroke}{rgb}{0.000000,0.000000,0.000000}%
\pgfsetstrokecolor{currentstroke}%
\pgfsetdash{}{0pt}%
\pgfsys@defobject{currentmarker}{\pgfqpoint{0.000000in}{0.000000in}}{\pgfqpoint{0.055556in}{0.000000in}}{%
\pgfpathmoveto{\pgfqpoint{0.000000in}{0.000000in}}%
\pgfpathlineto{\pgfqpoint{0.055556in}{0.000000in}}%
\pgfusepath{stroke,fill}%
}%
\begin{pgfscope}%
\pgfsys@transformshift{0.506010in}{1.681191in}%
\pgfsys@useobject{currentmarker}{}%
\end{pgfscope}%
\end{pgfscope}%
\begin{pgfscope}%
\pgfsetbuttcap%
\pgfsetroundjoin%
\definecolor{currentfill}{rgb}{0.000000,0.000000,0.000000}%
\pgfsetfillcolor{currentfill}%
\pgfsetlinewidth{0.501875pt}%
\definecolor{currentstroke}{rgb}{0.000000,0.000000,0.000000}%
\pgfsetstrokecolor{currentstroke}%
\pgfsetdash{}{0pt}%
\pgfsys@defobject{currentmarker}{\pgfqpoint{-0.055556in}{0.000000in}}{\pgfqpoint{0.000000in}{0.000000in}}{%
\pgfpathmoveto{\pgfqpoint{0.000000in}{0.000000in}}%
\pgfpathlineto{\pgfqpoint{-0.055556in}{0.000000in}}%
\pgfusepath{stroke,fill}%
}%
\begin{pgfscope}%
\pgfsys@transformshift{2.831010in}{1.681191in}%
\pgfsys@useobject{currentmarker}{}%
\end{pgfscope}%
\end{pgfscope}%
\begin{pgfscope}%
\pgftext[x=0.450454in,y=1.681191in,right,]{{\rmfamily\fontsize{8.328000}{9.993600}\selectfont \(\displaystyle 200\)}}%
\end{pgfscope}%
\begin{pgfscope}%
\pgfpathrectangle{\pgfqpoint{0.506010in}{1.121191in}}{\pgfqpoint{2.325000in}{1.400000in}} %
\pgfusepath{clip}%
\pgfsetbuttcap%
\pgfsetroundjoin%
\pgfsetlinewidth{0.501875pt}%
\definecolor{currentstroke}{rgb}{0.000000,0.000000,0.000000}%
\pgfsetstrokecolor{currentstroke}%
\pgfsetdash{{1.000000pt}{3.000000pt}}{0.000000pt}%
\pgfpathmoveto{\pgfqpoint{0.506010in}{1.961191in}}%
\pgfpathlineto{\pgfqpoint{2.831010in}{1.961191in}}%
\pgfusepath{stroke}%
\end{pgfscope}%
\begin{pgfscope}%
\pgfsetbuttcap%
\pgfsetroundjoin%
\definecolor{currentfill}{rgb}{0.000000,0.000000,0.000000}%
\pgfsetfillcolor{currentfill}%
\pgfsetlinewidth{0.501875pt}%
\definecolor{currentstroke}{rgb}{0.000000,0.000000,0.000000}%
\pgfsetstrokecolor{currentstroke}%
\pgfsetdash{}{0pt}%
\pgfsys@defobject{currentmarker}{\pgfqpoint{0.000000in}{0.000000in}}{\pgfqpoint{0.055556in}{0.000000in}}{%
\pgfpathmoveto{\pgfqpoint{0.000000in}{0.000000in}}%
\pgfpathlineto{\pgfqpoint{0.055556in}{0.000000in}}%
\pgfusepath{stroke,fill}%
}%
\begin{pgfscope}%
\pgfsys@transformshift{0.506010in}{1.961191in}%
\pgfsys@useobject{currentmarker}{}%
\end{pgfscope}%
\end{pgfscope}%
\begin{pgfscope}%
\pgfsetbuttcap%
\pgfsetroundjoin%
\definecolor{currentfill}{rgb}{0.000000,0.000000,0.000000}%
\pgfsetfillcolor{currentfill}%
\pgfsetlinewidth{0.501875pt}%
\definecolor{currentstroke}{rgb}{0.000000,0.000000,0.000000}%
\pgfsetstrokecolor{currentstroke}%
\pgfsetdash{}{0pt}%
\pgfsys@defobject{currentmarker}{\pgfqpoint{-0.055556in}{0.000000in}}{\pgfqpoint{0.000000in}{0.000000in}}{%
\pgfpathmoveto{\pgfqpoint{0.000000in}{0.000000in}}%
\pgfpathlineto{\pgfqpoint{-0.055556in}{0.000000in}}%
\pgfusepath{stroke,fill}%
}%
\begin{pgfscope}%
\pgfsys@transformshift{2.831010in}{1.961191in}%
\pgfsys@useobject{currentmarker}{}%
\end{pgfscope}%
\end{pgfscope}%
\begin{pgfscope}%
\pgftext[x=0.450454in,y=1.961191in,right,]{{\rmfamily\fontsize{8.328000}{9.993600}\selectfont \(\displaystyle 300\)}}%
\end{pgfscope}%
\begin{pgfscope}%
\pgfpathrectangle{\pgfqpoint{0.506010in}{1.121191in}}{\pgfqpoint{2.325000in}{1.400000in}} %
\pgfusepath{clip}%
\pgfsetbuttcap%
\pgfsetroundjoin%
\pgfsetlinewidth{0.501875pt}%
\definecolor{currentstroke}{rgb}{0.000000,0.000000,0.000000}%
\pgfsetstrokecolor{currentstroke}%
\pgfsetdash{{1.000000pt}{3.000000pt}}{0.000000pt}%
\pgfpathmoveto{\pgfqpoint{0.506010in}{2.241191in}}%
\pgfpathlineto{\pgfqpoint{2.831010in}{2.241191in}}%
\pgfusepath{stroke}%
\end{pgfscope}%
\begin{pgfscope}%
\pgfsetbuttcap%
\pgfsetroundjoin%
\definecolor{currentfill}{rgb}{0.000000,0.000000,0.000000}%
\pgfsetfillcolor{currentfill}%
\pgfsetlinewidth{0.501875pt}%
\definecolor{currentstroke}{rgb}{0.000000,0.000000,0.000000}%
\pgfsetstrokecolor{currentstroke}%
\pgfsetdash{}{0pt}%
\pgfsys@defobject{currentmarker}{\pgfqpoint{0.000000in}{0.000000in}}{\pgfqpoint{0.055556in}{0.000000in}}{%
\pgfpathmoveto{\pgfqpoint{0.000000in}{0.000000in}}%
\pgfpathlineto{\pgfqpoint{0.055556in}{0.000000in}}%
\pgfusepath{stroke,fill}%
}%
\begin{pgfscope}%
\pgfsys@transformshift{0.506010in}{2.241191in}%
\pgfsys@useobject{currentmarker}{}%
\end{pgfscope}%
\end{pgfscope}%
\begin{pgfscope}%
\pgfsetbuttcap%
\pgfsetroundjoin%
\definecolor{currentfill}{rgb}{0.000000,0.000000,0.000000}%
\pgfsetfillcolor{currentfill}%
\pgfsetlinewidth{0.501875pt}%
\definecolor{currentstroke}{rgb}{0.000000,0.000000,0.000000}%
\pgfsetstrokecolor{currentstroke}%
\pgfsetdash{}{0pt}%
\pgfsys@defobject{currentmarker}{\pgfqpoint{-0.055556in}{0.000000in}}{\pgfqpoint{0.000000in}{0.000000in}}{%
\pgfpathmoveto{\pgfqpoint{0.000000in}{0.000000in}}%
\pgfpathlineto{\pgfqpoint{-0.055556in}{0.000000in}}%
\pgfusepath{stroke,fill}%
}%
\begin{pgfscope}%
\pgfsys@transformshift{2.831010in}{2.241191in}%
\pgfsys@useobject{currentmarker}{}%
\end{pgfscope}%
\end{pgfscope}%
\begin{pgfscope}%
\pgftext[x=0.450454in,y=2.241191in,right,]{{\rmfamily\fontsize{8.328000}{9.993600}\selectfont \(\displaystyle 400\)}}%
\end{pgfscope}%
\begin{pgfscope}%
\pgfpathrectangle{\pgfqpoint{0.506010in}{1.121191in}}{\pgfqpoint{2.325000in}{1.400000in}} %
\pgfusepath{clip}%
\pgfsetbuttcap%
\pgfsetroundjoin%
\pgfsetlinewidth{0.501875pt}%
\definecolor{currentstroke}{rgb}{0.000000,0.000000,0.000000}%
\pgfsetstrokecolor{currentstroke}%
\pgfsetdash{{1.000000pt}{3.000000pt}}{0.000000pt}%
\pgfpathmoveto{\pgfqpoint{0.506010in}{2.521191in}}%
\pgfpathlineto{\pgfqpoint{2.831010in}{2.521191in}}%
\pgfusepath{stroke}%
\end{pgfscope}%
\begin{pgfscope}%
\pgfsetbuttcap%
\pgfsetroundjoin%
\definecolor{currentfill}{rgb}{0.000000,0.000000,0.000000}%
\pgfsetfillcolor{currentfill}%
\pgfsetlinewidth{0.501875pt}%
\definecolor{currentstroke}{rgb}{0.000000,0.000000,0.000000}%
\pgfsetstrokecolor{currentstroke}%
\pgfsetdash{}{0pt}%
\pgfsys@defobject{currentmarker}{\pgfqpoint{0.000000in}{0.000000in}}{\pgfqpoint{0.055556in}{0.000000in}}{%
\pgfpathmoveto{\pgfqpoint{0.000000in}{0.000000in}}%
\pgfpathlineto{\pgfqpoint{0.055556in}{0.000000in}}%
\pgfusepath{stroke,fill}%
}%
\begin{pgfscope}%
\pgfsys@transformshift{0.506010in}{2.521191in}%
\pgfsys@useobject{currentmarker}{}%
\end{pgfscope}%
\end{pgfscope}%
\begin{pgfscope}%
\pgfsetbuttcap%
\pgfsetroundjoin%
\definecolor{currentfill}{rgb}{0.000000,0.000000,0.000000}%
\pgfsetfillcolor{currentfill}%
\pgfsetlinewidth{0.501875pt}%
\definecolor{currentstroke}{rgb}{0.000000,0.000000,0.000000}%
\pgfsetstrokecolor{currentstroke}%
\pgfsetdash{}{0pt}%
\pgfsys@defobject{currentmarker}{\pgfqpoint{-0.055556in}{0.000000in}}{\pgfqpoint{0.000000in}{0.000000in}}{%
\pgfpathmoveto{\pgfqpoint{0.000000in}{0.000000in}}%
\pgfpathlineto{\pgfqpoint{-0.055556in}{0.000000in}}%
\pgfusepath{stroke,fill}%
}%
\begin{pgfscope}%
\pgfsys@transformshift{2.831010in}{2.521191in}%
\pgfsys@useobject{currentmarker}{}%
\end{pgfscope}%
\end{pgfscope}%
\begin{pgfscope}%
\pgftext[x=0.450454in,y=2.521191in,right,]{{\rmfamily\fontsize{8.328000}{9.993600}\selectfont \(\displaystyle 500\)}}%
\end{pgfscope}%
\begin{pgfscope}%
\pgftext[x=0.203924in,y=1.821191in,,bottom,rotate=90.000000]{{\rmfamily\fontsize{8.328000}{9.993600}\selectfont Time \(\displaystyle t\) (s)}}%
\end{pgfscope}%
\begin{pgfscope}%
\pgfsetbuttcap%
\pgfsetroundjoin%
\pgfsetlinewidth{1.003750pt}%
\definecolor{currentstroke}{rgb}{0.000000,0.000000,0.000000}%
\pgfsetstrokecolor{currentstroke}%
\pgfsetdash{}{0pt}%
\pgfpathmoveto{\pgfqpoint{0.506010in}{2.521191in}}%
\pgfpathlineto{\pgfqpoint{2.831010in}{2.521191in}}%
\pgfusepath{stroke}%
\end{pgfscope}%
\begin{pgfscope}%
\pgfsetbuttcap%
\pgfsetroundjoin%
\pgfsetlinewidth{1.003750pt}%
\definecolor{currentstroke}{rgb}{0.000000,0.000000,0.000000}%
\pgfsetstrokecolor{currentstroke}%
\pgfsetdash{}{0pt}%
\pgfpathmoveto{\pgfqpoint{2.831010in}{1.121191in}}%
\pgfpathlineto{\pgfqpoint{2.831010in}{2.521191in}}%
\pgfusepath{stroke}%
\end{pgfscope}%
\begin{pgfscope}%
\pgfsetbuttcap%
\pgfsetroundjoin%
\pgfsetlinewidth{1.003750pt}%
\definecolor{currentstroke}{rgb}{0.000000,0.000000,0.000000}%
\pgfsetstrokecolor{currentstroke}%
\pgfsetdash{}{0pt}%
\pgfpathmoveto{\pgfqpoint{0.506010in}{1.121191in}}%
\pgfpathlineto{\pgfqpoint{2.831010in}{1.121191in}}%
\pgfusepath{stroke}%
\end{pgfscope}%
\begin{pgfscope}%
\pgfsetbuttcap%
\pgfsetroundjoin%
\pgfsetlinewidth{1.003750pt}%
\definecolor{currentstroke}{rgb}{0.000000,0.000000,0.000000}%
\pgfsetstrokecolor{currentstroke}%
\pgfsetdash{}{0pt}%
\pgfpathmoveto{\pgfqpoint{0.506010in}{1.121191in}}%
\pgfpathlineto{\pgfqpoint{0.506010in}{2.521191in}}%
\pgfusepath{stroke}%
\end{pgfscope}%
\begin{pgfscope}%
\pgftext[x=1.668510in,y=2.590635in,,base]{{\rmfamily\fontsize{8.328000}{9.993600}\selectfont Solving time over density}}%
\end{pgfscope}%
\begin{pgfscope}%
\pgfsetbuttcap%
\pgfsetroundjoin%
\definecolor{currentfill}{rgb}{1.000000,1.000000,1.000000}%
\pgfsetfillcolor{currentfill}%
\pgfsetlinewidth{1.003750pt}%
\definecolor{currentstroke}{rgb}{0.000000,0.000000,0.000000}%
\pgfsetstrokecolor{currentstroke}%
\pgfsetdash{}{0pt}%
\pgfpathmoveto{\pgfqpoint{0.387681in}{0.100000in}}%
\pgfpathlineto{\pgfqpoint{2.949339in}{0.100000in}}%
\pgfpathlineto{\pgfqpoint{2.949339in}{0.462381in}}%
\pgfpathlineto{\pgfqpoint{0.387681in}{0.462381in}}%
\pgfpathlineto{\pgfqpoint{0.387681in}{0.100000in}}%
\pgfpathclose%
\pgfusepath{stroke,fill}%
\end{pgfscope}%
\begin{pgfscope}%
\pgfsetbuttcap%
\pgfsetroundjoin%
\definecolor{currentfill}{rgb}{0.000000,0.500000,0.000000}%
\pgfsetfillcolor{currentfill}%
\pgfsetlinewidth{1.003750pt}%
\definecolor{currentstroke}{rgb}{0.000000,0.500000,0.000000}%
\pgfsetstrokecolor{currentstroke}%
\pgfsetdash{}{0pt}%
\pgfpathmoveto{\pgfqpoint{0.468647in}{0.343550in}}%
\pgfpathcurveto{\pgfqpoint{0.474471in}{0.343550in}}{\pgfqpoint{0.480057in}{0.345864in}}{\pgfqpoint{0.484176in}{0.349982in}}%
\pgfpathcurveto{\pgfqpoint{0.488294in}{0.354100in}}{\pgfqpoint{0.490608in}{0.359687in}}{\pgfqpoint{0.490608in}{0.365511in}}%
\pgfpathcurveto{\pgfqpoint{0.490608in}{0.371334in}}{\pgfqpoint{0.488294in}{0.376921in}}{\pgfqpoint{0.484176in}{0.381039in}}%
\pgfpathcurveto{\pgfqpoint{0.480057in}{0.385157in}}{\pgfqpoint{0.474471in}{0.387471in}}{\pgfqpoint{0.468647in}{0.387471in}}%
\pgfpathcurveto{\pgfqpoint{0.462823in}{0.387471in}}{\pgfqpoint{0.457237in}{0.385157in}}{\pgfqpoint{0.453119in}{0.381039in}}%
\pgfpathcurveto{\pgfqpoint{0.449001in}{0.376921in}}{\pgfqpoint{0.446687in}{0.371334in}}{\pgfqpoint{0.446687in}{0.365511in}}%
\pgfpathcurveto{\pgfqpoint{0.446687in}{0.359687in}}{\pgfqpoint{0.449001in}{0.354100in}}{\pgfqpoint{0.453119in}{0.349982in}}%
\pgfpathcurveto{\pgfqpoint{0.457237in}{0.345864in}}{\pgfqpoint{0.462823in}{0.343550in}}{\pgfqpoint{0.468647in}{0.343550in}}%
\pgfpathclose%
\pgfusepath{stroke,fill}%
\end{pgfscope}%
\begin{pgfscope}%
\pgfsetbuttcap%
\pgfsetroundjoin%
\definecolor{currentfill}{rgb}{0.000000,0.500000,0.000000}%
\pgfsetfillcolor{currentfill}%
\pgfsetlinewidth{1.003750pt}%
\definecolor{currentstroke}{rgb}{0.000000,0.500000,0.000000}%
\pgfsetstrokecolor{currentstroke}%
\pgfsetdash{}{0pt}%
\pgfpathmoveto{\pgfqpoint{0.549614in}{0.353671in}}%
\pgfpathcurveto{\pgfqpoint{0.555438in}{0.353671in}}{\pgfqpoint{0.561024in}{0.355985in}}{\pgfqpoint{0.565142in}{0.360103in}}%
\pgfpathcurveto{\pgfqpoint{0.569260in}{0.364221in}}{\pgfqpoint{0.571574in}{0.369807in}}{\pgfqpoint{0.571574in}{0.375631in}}%
\pgfpathcurveto{\pgfqpoint{0.571574in}{0.381455in}}{\pgfqpoint{0.569260in}{0.387041in}}{\pgfqpoint{0.565142in}{0.391160in}}%
\pgfpathcurveto{\pgfqpoint{0.561024in}{0.395278in}}{\pgfqpoint{0.555438in}{0.397592in}}{\pgfqpoint{0.549614in}{0.397592in}}%
\pgfpathcurveto{\pgfqpoint{0.543790in}{0.397592in}}{\pgfqpoint{0.538204in}{0.395278in}}{\pgfqpoint{0.534086in}{0.391160in}}%
\pgfpathcurveto{\pgfqpoint{0.529968in}{0.387041in}}{\pgfqpoint{0.527654in}{0.381455in}}{\pgfqpoint{0.527654in}{0.375631in}}%
\pgfpathcurveto{\pgfqpoint{0.527654in}{0.369807in}}{\pgfqpoint{0.529968in}{0.364221in}}{\pgfqpoint{0.534086in}{0.360103in}}%
\pgfpathcurveto{\pgfqpoint{0.538204in}{0.355985in}}{\pgfqpoint{0.543790in}{0.353671in}}{\pgfqpoint{0.549614in}{0.353671in}}%
\pgfpathclose%
\pgfusepath{stroke,fill}%
\end{pgfscope}%
\begin{pgfscope}%
\pgfsetbuttcap%
\pgfsetroundjoin%
\definecolor{currentfill}{rgb}{0.000000,0.500000,0.000000}%
\pgfsetfillcolor{currentfill}%
\pgfsetlinewidth{1.003750pt}%
\definecolor{currentstroke}{rgb}{0.000000,0.500000,0.000000}%
\pgfsetstrokecolor{currentstroke}%
\pgfsetdash{}{0pt}%
\pgfpathmoveto{\pgfqpoint{0.630581in}{0.338490in}}%
\pgfpathcurveto{\pgfqpoint{0.636405in}{0.338490in}}{\pgfqpoint{0.641991in}{0.340804in}}{\pgfqpoint{0.646109in}{0.344922in}}%
\pgfpathcurveto{\pgfqpoint{0.650227in}{0.349040in}}{\pgfqpoint{0.652541in}{0.354626in}}{\pgfqpoint{0.652541in}{0.360450in}}%
\pgfpathcurveto{\pgfqpoint{0.652541in}{0.366274in}}{\pgfqpoint{0.650227in}{0.371860in}}{\pgfqpoint{0.646109in}{0.375978in}}%
\pgfpathcurveto{\pgfqpoint{0.641991in}{0.380097in}}{\pgfqpoint{0.636405in}{0.382410in}}{\pgfqpoint{0.630581in}{0.382410in}}%
\pgfpathcurveto{\pgfqpoint{0.624757in}{0.382410in}}{\pgfqpoint{0.619171in}{0.380097in}}{\pgfqpoint{0.615052in}{0.375978in}}%
\pgfpathcurveto{\pgfqpoint{0.610934in}{0.371860in}}{\pgfqpoint{0.608620in}{0.366274in}}{\pgfqpoint{0.608620in}{0.360450in}}%
\pgfpathcurveto{\pgfqpoint{0.608620in}{0.354626in}}{\pgfqpoint{0.610934in}{0.349040in}}{\pgfqpoint{0.615052in}{0.344922in}}%
\pgfpathcurveto{\pgfqpoint{0.619171in}{0.340804in}}{\pgfqpoint{0.624757in}{0.338490in}}{\pgfqpoint{0.630581in}{0.338490in}}%
\pgfpathclose%
\pgfusepath{stroke,fill}%
\end{pgfscope}%
\begin{pgfscope}%
\pgftext[x=0.757814in,y=0.335148in,left,base]{{\rmfamily\fontsize{8.328000}{9.993600}\selectfont MoMC}}%
\end{pgfscope}%
\begin{pgfscope}%
\pgfsetbuttcap%
\pgfsetroundjoin%
\definecolor{currentfill}{rgb}{1.000000,0.000000,0.000000}%
\pgfsetfillcolor{currentfill}%
\pgfsetlinewidth{1.003750pt}%
\definecolor{currentstroke}{rgb}{1.000000,0.000000,0.000000}%
\pgfsetstrokecolor{currentstroke}%
\pgfsetdash{}{0pt}%
\pgfpathmoveto{\pgfqpoint{0.468647in}{0.179710in}}%
\pgfpathlineto{\pgfqpoint{0.490608in}{0.223630in}}%
\pgfpathlineto{\pgfqpoint{0.446687in}{0.223630in}}%
\pgfpathclose%
\pgfusepath{stroke,fill}%
\end{pgfscope}%
\begin{pgfscope}%
\pgfsetbuttcap%
\pgfsetroundjoin%
\definecolor{currentfill}{rgb}{1.000000,0.000000,0.000000}%
\pgfsetfillcolor{currentfill}%
\pgfsetlinewidth{1.003750pt}%
\definecolor{currentstroke}{rgb}{1.000000,0.000000,0.000000}%
\pgfsetstrokecolor{currentstroke}%
\pgfsetdash{}{0pt}%
\pgfpathmoveto{\pgfqpoint{0.549614in}{0.189830in}}%
\pgfpathlineto{\pgfqpoint{0.571574in}{0.233751in}}%
\pgfpathlineto{\pgfqpoint{0.527654in}{0.233751in}}%
\pgfpathclose%
\pgfusepath{stroke,fill}%
\end{pgfscope}%
\begin{pgfscope}%
\pgfsetbuttcap%
\pgfsetroundjoin%
\definecolor{currentfill}{rgb}{1.000000,0.000000,0.000000}%
\pgfsetfillcolor{currentfill}%
\pgfsetlinewidth{1.003750pt}%
\definecolor{currentstroke}{rgb}{1.000000,0.000000,0.000000}%
\pgfsetstrokecolor{currentstroke}%
\pgfsetdash{}{0pt}%
\pgfpathmoveto{\pgfqpoint{0.630581in}{0.174649in}}%
\pgfpathlineto{\pgfqpoint{0.652541in}{0.218570in}}%
\pgfpathlineto{\pgfqpoint{0.608620in}{0.218570in}}%
\pgfpathclose%
\pgfusepath{stroke,fill}%
\end{pgfscope}%
\begin{pgfscope}%
\pgftext[x=0.757814in,y=0.171307in,left,base]{{\rmfamily\fontsize{8.328000}{9.993600}\selectfont RMoMC}}%
\end{pgfscope}%
\begin{pgfscope}%
\pgfsetbuttcap%
\pgfsetroundjoin%
\definecolor{currentfill}{rgb}{0.000000,0.000000,1.000000}%
\pgfsetfillcolor{currentfill}%
\pgfsetlinewidth{1.003750pt}%
\definecolor{currentstroke}{rgb}{0.000000,0.000000,1.000000}%
\pgfsetstrokecolor{currentstroke}%
\pgfsetdash{}{0pt}%
\pgfpathmoveto{\pgfqpoint{1.419671in}{0.343550in}}%
\pgfpathlineto{\pgfqpoint{1.463591in}{0.387471in}}%
\pgfpathmoveto{\pgfqpoint{1.419671in}{0.387471in}}%
\pgfpathlineto{\pgfqpoint{1.463591in}{0.343550in}}%
\pgfusepath{stroke,fill}%
\end{pgfscope}%
\begin{pgfscope}%
\pgfsetbuttcap%
\pgfsetroundjoin%
\definecolor{currentfill}{rgb}{0.000000,0.000000,1.000000}%
\pgfsetfillcolor{currentfill}%
\pgfsetlinewidth{1.003750pt}%
\definecolor{currentstroke}{rgb}{0.000000,0.000000,1.000000}%
\pgfsetstrokecolor{currentstroke}%
\pgfsetdash{}{0pt}%
\pgfpathmoveto{\pgfqpoint{1.500637in}{0.353671in}}%
\pgfpathlineto{\pgfqpoint{1.544558in}{0.397592in}}%
\pgfpathmoveto{\pgfqpoint{1.500637in}{0.397592in}}%
\pgfpathlineto{\pgfqpoint{1.544558in}{0.353671in}}%
\pgfusepath{stroke,fill}%
\end{pgfscope}%
\begin{pgfscope}%
\pgfsetbuttcap%
\pgfsetroundjoin%
\definecolor{currentfill}{rgb}{0.000000,0.000000,1.000000}%
\pgfsetfillcolor{currentfill}%
\pgfsetlinewidth{1.003750pt}%
\definecolor{currentstroke}{rgb}{0.000000,0.000000,1.000000}%
\pgfsetstrokecolor{currentstroke}%
\pgfsetdash{}{0pt}%
\pgfpathmoveto{\pgfqpoint{1.581604in}{0.338490in}}%
\pgfpathlineto{\pgfqpoint{1.625524in}{0.382410in}}%
\pgfpathmoveto{\pgfqpoint{1.581604in}{0.382410in}}%
\pgfpathlineto{\pgfqpoint{1.625524in}{0.338490in}}%
\pgfusepath{stroke,fill}%
\end{pgfscope}%
\begin{pgfscope}%
\pgftext[x=1.730797in,y=0.335148in,left,base]{{\rmfamily\fontsize{8.328000}{9.993600}\selectfont LSBnR}}%
\end{pgfscope}%
\begin{pgfscope}%
\pgfsetbuttcap%
\pgfsetroundjoin%
\definecolor{currentfill}{rgb}{0.750000,0.750000,0.000000}%
\pgfsetfillcolor{currentfill}%
\pgfsetlinewidth{1.003750pt}%
\definecolor{currentstroke}{rgb}{0.750000,0.750000,0.000000}%
\pgfsetstrokecolor{currentstroke}%
\pgfsetdash{}{0pt}%
\pgfpathmoveto{\pgfqpoint{1.441631in}{0.170613in}}%
\pgfpathlineto{\pgfqpoint{1.460265in}{0.201670in}}%
\pgfpathlineto{\pgfqpoint{1.441631in}{0.232726in}}%
\pgfpathlineto{\pgfqpoint{1.422997in}{0.201670in}}%
\pgfpathclose%
\pgfusepath{stroke,fill}%
\end{pgfscope}%
\begin{pgfscope}%
\pgfsetbuttcap%
\pgfsetroundjoin%
\definecolor{currentfill}{rgb}{0.750000,0.750000,0.000000}%
\pgfsetfillcolor{currentfill}%
\pgfsetlinewidth{1.003750pt}%
\definecolor{currentstroke}{rgb}{0.750000,0.750000,0.000000}%
\pgfsetstrokecolor{currentstroke}%
\pgfsetdash{}{0pt}%
\pgfpathmoveto{\pgfqpoint{1.522597in}{0.180734in}}%
\pgfpathlineto{\pgfqpoint{1.541231in}{0.211791in}}%
\pgfpathlineto{\pgfqpoint{1.522597in}{0.242847in}}%
\pgfpathlineto{\pgfqpoint{1.503964in}{0.211791in}}%
\pgfpathclose%
\pgfusepath{stroke,fill}%
\end{pgfscope}%
\begin{pgfscope}%
\pgfsetbuttcap%
\pgfsetroundjoin%
\definecolor{currentfill}{rgb}{0.750000,0.750000,0.000000}%
\pgfsetfillcolor{currentfill}%
\pgfsetlinewidth{1.003750pt}%
\definecolor{currentstroke}{rgb}{0.750000,0.750000,0.000000}%
\pgfsetstrokecolor{currentstroke}%
\pgfsetdash{}{0pt}%
\pgfpathmoveto{\pgfqpoint{1.603564in}{0.165553in}}%
\pgfpathlineto{\pgfqpoint{1.622198in}{0.196609in}}%
\pgfpathlineto{\pgfqpoint{1.603564in}{0.227666in}}%
\pgfpathlineto{\pgfqpoint{1.584930in}{0.196609in}}%
\pgfpathclose%
\pgfusepath{stroke,fill}%
\end{pgfscope}%
\begin{pgfscope}%
\pgftext[x=1.730797in,y=0.171307in,left,base]{{\rmfamily\fontsize{8.328000}{9.993600}\selectfont BnR}}%
\end{pgfscope}%
\begin{pgfscope}%
\pgfsetbuttcap%
\pgfsetroundjoin%
\definecolor{currentfill}{rgb}{0.000000,0.750000,0.750000}%
\pgfsetfillcolor{currentfill}%
\pgfsetlinewidth{1.003750pt}%
\definecolor{currentstroke}{rgb}{0.000000,0.750000,0.750000}%
\pgfsetstrokecolor{currentstroke}%
\pgfsetdash{}{0pt}%
\pgfpathmoveto{\pgfqpoint{2.343943in}{0.387471in}}%
\pgfpathlineto{\pgfqpoint{2.321983in}{0.343550in}}%
\pgfpathlineto{\pgfqpoint{2.365904in}{0.343550in}}%
\pgfpathclose%
\pgfusepath{stroke,fill}%
\end{pgfscope}%
\begin{pgfscope}%
\pgfsetbuttcap%
\pgfsetroundjoin%
\definecolor{currentfill}{rgb}{0.000000,0.750000,0.750000}%
\pgfsetfillcolor{currentfill}%
\pgfsetlinewidth{1.003750pt}%
\definecolor{currentstroke}{rgb}{0.000000,0.750000,0.750000}%
\pgfsetstrokecolor{currentstroke}%
\pgfsetdash{}{0pt}%
\pgfpathmoveto{\pgfqpoint{2.424910in}{0.397592in}}%
\pgfpathlineto{\pgfqpoint{2.402950in}{0.353671in}}%
\pgfpathlineto{\pgfqpoint{2.446870in}{0.353671in}}%
\pgfpathclose%
\pgfusepath{stroke,fill}%
\end{pgfscope}%
\begin{pgfscope}%
\pgfsetbuttcap%
\pgfsetroundjoin%
\definecolor{currentfill}{rgb}{0.000000,0.750000,0.750000}%
\pgfsetfillcolor{currentfill}%
\pgfsetlinewidth{1.003750pt}%
\definecolor{currentstroke}{rgb}{0.000000,0.750000,0.750000}%
\pgfsetstrokecolor{currentstroke}%
\pgfsetdash{}{0pt}%
\pgfpathmoveto{\pgfqpoint{2.505877in}{0.382410in}}%
\pgfpathlineto{\pgfqpoint{2.483916in}{0.338490in}}%
\pgfpathlineto{\pgfqpoint{2.527837in}{0.338490in}}%
\pgfpathclose%
\pgfusepath{stroke,fill}%
\end{pgfscope}%
\begin{pgfscope}%
\pgftext[x=2.633110in,y=0.335148in,left,base]{{\rmfamily\fontsize{8.328000}{9.993600}\selectfont FullA}}%
\end{pgfscope}%
\end{pgfpicture}%
\makeatother%
\endgroup%

  \caption{Time required for each solver over graph density.} 
  \label{fig:solution_time}
\end{figure}
\fi{}

\section{Conclusion}
We presented the winning solver of the PACE 2019 Implementation Challenge Vertex Cover Track. Our algorithm uses a portfolio of techniques, including an aggressive kernelization strategy with all known reduction rules, local search, branch-and-reduce, and a state-of-the-art branch-and-bound solver. Of particular interest is that several of our techniques were not from the literature on the vertex over problem: they were originally published to solve the (complementary) maximum independent set and maximum clique problems. Lastly, our experiments show the impact of the different solver techniques on the number of instances solved during the challenge. In particular, the results emphasize that data reductions play an important tool to boost the performance of branch-and-bound, and local search is highly effective to boost the performance of branch-and-reduce.

\subsection*{Acknowledgments.}
We wish to thank the organizers of the PACE 2019 Implementation Challenge for providing us with the opportunity and means to test our algorithmic ideas. We also are indebted to Takuya Akiba and  Yoichi Iwata for sharing their original branch-and-reduce source code\footnote{\url{https://github.com/wata-orz/vertex_cover}}, and to Chu-Min Li, Hua Jiang, and Felip Many\`a for not only sharing---but even open sourcing---their code for MoMC at our request\footnote{\url{https://home.mis.u-picardie.fr/~cli/EnglishPage.html}}. Their solver was of critical importance to our algorithm's success.

\comment[id=DS]{Put full running times in appendix.}

%\vfill 
%\pagebreak
%\
%\vfill
%\pagebreak
\bibliographystyle{plainurl}
\bibliography{references}
\vfill
\end{document}
