\documentclass[twoside,leqno,twocolumn]{article}

% Comment out the line below if using A4 paper size
\usepackage[letterpaper]{geometry}

\usepackage{ltexpprt}
\usepackage{t1enc}
\usepackage[utf8]{inputenc}
\usepackage{numprint}
\npdecimalsign{.} % we want . not , in numbers
\usepackage{hyperref}
\usepackage{xspace}
\usepackage{doi}
\usepackage{enumerate}
\usepackage{booktabs}

\newcommand{\overbar}[1]{\mkern 1.5mu\overline{\mkern-1.5mu#1\mkern-1.5mu}\mkern 1.5mu}
\def\MdR{\ensuremath{\mathbb{R}}}
\def\MdN{\ensuremath{\mathbb{N}}}
%\DeclareMathOperator{\sgn}{sgn}
\newcommand{\Id}[1]{\texttt{\detokenize{#1}}}
\newcommand{\Is}       {:=}
\newcommand{\setGilt}[2]{\left\{ #1\sodass #2\right\}}
\newcommand{\sodass}{\,:\,}
\newcommand{\set}[1]{\left\{ #1\right\}}
\newcommand{\gilt}{:}
\newcommand{\ie}{i.\,e.,\xspace}
\newcommand{\eg}{e.\,g.,\xspace}
\newcommand{\etal}{et~al.\xspace}
\newcommand{\Wlog}{w.\,l.\,o.\,g.\ }
\newcommand{\wrt}{w.\,r.\,t.\xspace}

\newcommand{\mytitle}{WeGotYouCovered}

\setlength\parfillskip{0pt plus .4\textwidth}
\setlength\emergencystretch{.1\textwidth}
\clubpenalty10000
\widowpenalty10000
\displaywidowpenalty=10000


%\subjclass{G.2.2 Graph Theory -- Graph Algorithms, G.4 Mathematical Software -- Algorithm Design and Analysis} 
%\keywords{kernelization, branch-and-reduce, local search}
%\EventEditors{}
%\EventNoEds{0}
%\EventLongTitle{}
%\EventShortTitle{PACE 2019}
%\EventAcronym{PACE}
%\EventYear{2019}
%\EventDate{}
%\EventLocation{}
%\EventLogo{}
%\SeriesVolume{}
%\ArticleNo{}

\begin{document}
\title{\mytitle}
\author{Demian Hespe\thanks{Karlsruhe Institute of Technology, Karlsruhe, Germany} \and Sebastian Lamm\thanks{Karlsruhe Institute of Technology, Karlsruhe, Germany} \and Christian Schulz\thanks{University of Vienna, Faculty of Computer Science, Austria} \and Darren Strash\thanks{Hamilton College, New York, USA}}

%\affil[1]{Karlsruhe Institute of Technology, Karlsruhe, Germany \\
  %\texttt{hespe@kit.edu}}
%\affil[2]{Karlsruhe Institute of Technology, Karlsruhe, Germany\\
  %\texttt{lamm@kit.edu}}
%\affil[3]{University of Vienna, Faculty of Computer Science, Vienna, Austria\\ \texttt{christian.schulz@univie.ac.at}}
%\affil[4]{Hamilton College, New York, USA,  \texttt{dstrash@hamilton.edu}}

\date{}


%\Copyright{}
\maketitle
\begin{abstract}
We present the winning solver of the PACE 2019 Implementation Challenge Vertex Cover Track. The vertex cover problem is one of a handful of problems for which \emph{kernelization}---the repeated reducing of the input size via \emph{data reduction rules}---is known to be highly effective in practice. Our algorithm uses a portfolio of techniques, including an aggressive kernelization strategy with all known reduction rules, local search, branch-and-reduce, and a state-of-the-art branch-and-bound solver. Of particular interest is that several of our techniques were \emph{not} from the literature on the vertex over problem: they were originally published to solve the (complementary) maximum independent set and maximum clique problems. Lastly, we perform extensive experiments to show the impact of the different solver techniques on the number of instances solved during the challenge.
\end{abstract}

%apply an initial aggressive kernelization strategy, using all known reduction rules for the problem. From there we use local search to produce a high-quality solution on the (hopefully smaller) kernel, which we use as a starting solution for a branch-and-bound solver. Our branch-and-bound solver also applies reduction rules via a branch-and-reduce scheme -- applying rules when possible, and branching otherwise -- though this may be toggled to omit reductions if they are not effective.
%%%%%%%%%%%%%%%%%%%%%%%%%%%%%
\section{Introduction}

A \emph{vertex cover} of a graph $G=(V,E)$ is a set of vertices $S\subseteq V$ of $G$ such that every edge of $G$ has at least one of member of $S$ as an endpoint (i.e., $\forall (u,v) \in E\,\, [u\in S \textrm{ or } v \in S]$).
A minimum vertex cover is a vertex cover of minimum cardinality. 
Complementary to vertex covers are independent sets and cliques. An independent set is a set of vertices $I\subseteq V$, all pairs of which are not adjacent, and an clique is a set of vertices $K\subseteq V$ all pairs of which are adjacent. A maximum independent set (maximum clique) is an independent set (clique) of maximum cardinality. The goal of the maximum independent set problem (maximum clique problem) is to compute a maximum independent set (maximum clique).

Many techniques have been proposed for solving these problems, and papers in the literature usually focus on one of these problems in particular. However, all of these problems are equivalent: a
minimum vertex cover $C$ in $G$ is the complement of a maximum independent set $V\setminus C$ in $G$, which is a maximum clique $V\setminus C$ in $\overline{G}$. Thus, an algorithm that solves one of these problems can be used to~solve~the~others.
For our approach, we use a portfolio of solvers, using techniques from the literature on all three problems. These include data reduction rules and branch-and-reduce for the minimum vertex cover problem~\cite{akiba-tcs-2016}, iterated local search for the maximum independent set problem~\cite{andrade-2012}, and a state-of-the-art branch-and-bound maximum clique solver~\cite{DBLP:journals/cor/LiJM17}.

We first briefly describe releated work. Then we outline each of the techniques that we use, and finally describe how we combine all of the techniques in our final solver that scored most of the points during the PACE 2019 Implementation Challenge. Lastly, we perform an experimental evaluation to show the impact of the components used on the final number of instances solved during the challenge.
\section{Related Work}
\label{sec:related_work}
We now present important related work.
This includes exact branch-and-bound algorithms as well as reduction based approaches. Much research has been devoted to improve exact branch-and-bound algorithms for the independent set and its complementary problems.
These improvements include different pruning methods and sophisticated branching schemes~\cite{ostergaard2002fast,balas1986finding,babel1994fast,warren2006combinatorial}.
Warren and Hicks~\cite{warren2006combinatorial} proposed three combinatorial branch-and-bound algorithms that are able to quickly solve DIMACS and weighted random graphs.
These algorithms use weighted clique covers to generate upper bounds that reduce the search space via pruning.
Furthermore, they all use a branching scheme proposed by Balas and Yu~\cite{balas1986finding}.
In particular, their first algorithm is an extension and improvement of a method by Babel~\cite{babel1994fast}.
Their second one uses a modified version of the algorithm by Balas and Yu that uses clique covers that borrow structural features from the ones by Babel~\cite{babel1994fast}.
Finally, their third approach is a hybrid of both previous algorithms.
Overall, their algorithms are able to quickly solve instances with hundreds of vertices.

An important technique to reduce the base of the exponent for exact branch-and-bound algorithms are so-called \emph{reduction rules}.
Reduction rules are able to reduce the input graph to an irreducible \emph{kernel} by removing well-defined subgraphs.
This is done by selecting certain vertices that are provably part of some maximum independent set, thus maintaining optimality.
We can then extend a solution on the kernel to a solution on the input graph by undoing the previously applied reductions.
There exist several well-known reduction rules for the unweighted vertex cover problem (and in turn for the unweighted MIS problem)~\cite{akiba-tcs-2016}.

As noted by Larson~\cite{larson-2007}, it is possible that in the unweighted case the initial critical set found by Butenko and Trukhanov might be empty.
To prevent this case, Larson~\cite{larson-2007} proposed an algorithm that finds a \emph{maximum} (unweighted) critical independent set.
His algorithm accumulates vertices that are in some critical set and removes their neighborhood.
Additionally, he provides a method to quickly check if a given vertex is part of some critical set.
Later Iwata~\cite{iwata-2014} has shown how to remove a large collection of vertices from a maximum matching all at once; however, it is not known if these reductions are equivalent.

For the maximum weight clique problem, Cai and Lin~\cite{cai2016fast} give an exact branch-and-bound algorithm that interleaves between clique construction and reductions.
In particular, their algorithm picks different starting vertices to form a clique and then maintains a candidate set to iteratively extend this clique.
In each iteration, the vertex to be added is selected using a benefit estimation function and a dynamic best from multiple selection heuristic~\cite{cai2015balance}.
Once the candidate set is empty, the new solution is compared to the best solution found so far.
If an improvement is found, their algorithm then tries to apply reductions and reduce the graph size.
To be more specific, they use two reduction rules that are able to remove a vertex $v$ by computing upper bounds related to the weight of different neighborhoods of $v$. 
We briefly note that their algorithm and reductions are targeted at sparse graphs, and therefore their reductions would likely work well for the maximum weighted independent set problem on \emph{dense} graphs.


\section{Solver Techniques}
We now describe the main techniques that we use within our solver.
\paragraph*{Kernelization.}
The most efficient algorithms for computing a minimum vertex cover in both theory and practice use \emph{data reduction rules} to obtain a much smaller problem instance. If this smaller instance has size bounded by a function of some parameter, it's called a \emph{kernel}. 

We use an extensive (though not exhaustive) collection of data reduction rules whose efficacy was studied by Akiba and Iwata~\cite{akiba-tcs-2016}. To compute a kernel, Akiba and Iwata~\cite{akiba-tcs-2016} apply their
reductions~$r_1, \dots, r_j$ by iterating over all reductions and trying to
apply the current reduction $r_i$ to all vertices. If $r_i$ reduces at
least one vertex, they restart with reduction~$r_1$. When reduction~$r_j$ 
is executed, but does not reduce any vertex, all reductions have been applied
exhaustively, and a kernel is found. Following their study we order the reductions
as follows: degree-one vertex (i.e., pendant) removal, unconfined vertex removal~\cite{Xiao201392}, a well-known linear-programming 
relaxation~\cite{iwata-2014,nemhauser-1975} related to crown removal~\cite{abu-khzam-2007}, vertex folding~\cite{chen-1999}, and twin, funnel, and desk reductions~\cite{Xiao201392}.

%The kernel can then be solved quickly using exact or heuristic algorithms---or by repeatedly kernelizing recursively in the branch-and-reduce paradigm. 

\paragraph*{Branch-and-Reduce.}
Branch-and-reduce is a paradigm that intermixes data reduction rules and branching. We use the algorithm of Akiba and Iwata, which exhaustively applies their full suite of reduction rules before branching, and includes a number of advanced branching rules. When branching, a vertex is chosen at random for inclusion into the vertex cover.

\paragraph*{Branch-and-Bound.} Experiments by Strash~\cite{strash2016power} show that the full power of branch-and-reduce is only needed \emph{very rarely} in real-world instances; kernelization followed by standard branch-and-bound solver is sufficient for many real-world instances. Furthermore, branch-and-reduce does not work well on many synthetic benchmark instances, where data reduction rules are ineffective~\cite{akiba-tcs-2016}, and instead add significant overhead to branch-and-bound. We use a state-of-the-art branch-and-bound maximum clique solver (MoMC) by Li et al.~\cite{DBLP:journals/cor/LiJM17}, which uses incremental MaxSAT reasoning to prune search, and a combination of static and dynamic vertex ordering to select the vertex for branching. We run the clique solver on the complement graph, giving a maximum independent set from which we derive a minimum vertex cover. In preliminary experiments, we found that a kernel can sometimes be harder for the solver than the original input; therefore, we run the algorithm on both the kernel and on the original graph.

\paragraph*{Iterated Local Search.}
Batsyn et al.~\cite{batsyn-mcs-ils-2014} showed that if branch-and-bound search is primed with a high-quality solution from local search, then instances can be solved up to thousands of times faster. 
We use iterated local search algorithm by Andrade et al.~\cite{andrade-2012} to prime the branch-and-reduce solver with a high-quality initial solution. Iterated local search was originally implemented for the maximum independent set problem, and is based on the notion of $(j,k)$-swaps. A $(j,k)$-swap removes $j$ nodes from the current solution and inserts $k$ nodes. The authors present a fast linear-time implementation that, given a maximal independent set, can find a $(1,2)$-swap or prove that none exists. Their algorithm applies $(1,2)$-swaps until reaching a local maximum, then perturbs the solution and repeats. We implemented the algorithm to find a high-quality solution on \emph{the kernel}. Calling local search on the kernel has been shown to produce a high-quality solution much faster than without kernelization~\cite{chang2017computing,dahlum2016accelerating}.

\section{Putting it all Together}
Our algorithm first runs a preprocessing phase, followed by 4 phases of solvers.

%Our solver is a combination of different kernelization techniques \cite{DBLP:conf/alenex/Hespe0S18}, local search~\cite{DBLP:conf/wea/AndradeRW08}, as well as branch-and-reduce~\cite{akiba-tcs-2016,DBLP:journals/cor/LiJM17}.

%
%Our algorithm uses a portfolio of solvers, i.e., a branch-and-bound solver for vertex cover~\cite{akiba-tcs-2016} as well as a branch-and-bound solver for the maximum clique problem \cite{DBLP:journals/cor/LiJM17}.

\begin{description}
\item[Phase 1. (Preprocessing)] Our algorithm starts by computing a kernel of the graph using the reductions by Akiba and Iwata~\cite{akiba-tcs-2016}. 
From there we use iterated local search to produce a high-quality solution $S_{\textrm{init}}$ on the (hopefully smaller) kernel. 
\item[Phase 2. (Branch-and-Reduce, short)]
We prime a branch-and-reduce solver with the initial solution $S_{\textrm{init}}$ and run it with a short time limit.
\item[Phase 3. (Branch-and-Bound, short)]
If Phase 2 is unsuccessful, we run the MoMC~\cite{DBLP:journals/cor/LiJM17} clique solver on the complement of the kernel, also using a short time limit. Sometimes kernelization can make the problem harder for MoMC. Therefore, if the first call was unsuccessful we also run MoMC on the complement of the original (unkernelized) input with the same short time limit.

\item[Phase 4. (Branch-and-Reduce, long)]
If we have still not found a solution, we run branch-and-reduce on the kernel using initial solution $S_{\textrm{init}}$ and a longer time limit. We opt for this second phase because, while most graphs amenable to reductions are solved very quickly with branch-and-reduce (less than a second),
experiments by Akiba and Iwata~\cite{akiba-tcs-2016} showed that other slower instances either finish in at most a few minutes, or take significantly longer---more than the time limit allotted for the challenge. This second phase of branch-and-reduce is meant to catch any instances that still benefit from reductions.

\item[Phase 5. (Branch-and-Bound, remaining time)]
If all previous phases were unsuccessful, we run MoMC on the original (unkernelized) input graph until the end of the time given to the program by the challenge. This is meant to capture only the most hard-to-compute instances.
\end{description}

The ordering and time limits were carefully chosen so that the overall algorithm outputs solutions of the ``easy'' instances \emph{quickly}, while still being able to solve hard instances.
\vfill\pagebreak
\section{Experimental Results}
We now look at the impact of the algorithmic components on the number of instances solved.
Here, we use the public instances -- obtained from \url{https://pacechallenge.org/files/pace2019-vc-exact-public-v2.tar.bz2} -- of the PACE 2019 Track A implementation challenge. This set contains 100 instances overall. Afterwards, we present the results comaring against the second and third best competing algorithms during the challenge. 

\subsection{Methodology and Setup.}
All of our experiments were run on a machine with  four Sixteen-Core Intel Xeon Haswell-EX E7-8867 processors running at $2.5$ GHz, $1$ TB of main memory, and $32768$ KB L2-Cache.
The machine runs Debian GNU/Linux 9 and Linux kernel version 4.9.0-9.
All algorithms were implemented in C++11 and compiled with gcc~version 6.3.0 with optimization flag \texttt{-O3}.
Each algorithm was run sequentially with a time limit of 30 minutes. Our evaluations focus on the total number of instances solved.
\subsection{Evaluation.}
We now explain our main configuration that we use in our experimental setup.
In the following \texttt{MoMC} runs the clique solver~\cite{DBLP:journals/cor/LiJM17} on the complement of the input graph, \texttt{RMoMC} applies reductions to the input graph exhaustively and then runs MoMC on the complement of the kernel graph, \texttt{BnR} applies reductions exhaustively, then runs local search to obtain a high-quality solution on the kernel which is used as a initial bound in the branch-and-reduce algorithm that is run on the kernel, \texttt{BnR-LS} applies reductions and then runs the branch-and-reduce algorithm on the kernel (no local search is used to improve an initial bound), \texttt{FullA} is the full algorithm as described above.

Tables~\ref{tab:detailedresults1} and \ref{tab:detailedresults2} give an overview over the instances that each of the solver solved, about the kernel sizes as well as the optimal vertex cover size, if our full algorithm could solve the instance.
Overall, \texttt{MoMC} can solve 30 out of the 100 instances. 
Using reductions first, enables \texttt{RMoMC} to solve 68 instances. However, there are also instances that \texttt{MoMC} could solve, but \texttt{RMoMC} could not solve. 
In these case, the number of nodes has been reduced, but the number of edges actually increase. This is due to the \emph{Alternative} reduction, which in some cases can create more edges than initially present. This is why our full algorithm also runs MoMC on the input graph (in order to be able to solve those instances as well).
\texttt{BnR} can solve 55 out of the 100 instances. Here, priming the branch-and-reduce algorithm with an initial solution computed by local search has an important impact. Running the branch-and-reduce algorithm on the kernel without using local search can only solve 42 instances. In particular, using local search to find an initial bound helps to solve large instances in which the initial kernelization step does not reduce the graph fully. Our full algorithm \texttt{FullA} can solve 82 out of the 100 instances, and in particular, as expeced, dominates each of the other configurations.

On the private instances, our full algorithm solved 87, the second place (peaty~\cite{}) solved 77, the third place (bogdan~\cite{}) solved 76 instances (of 100 instances). The peaty solver focused on using ..., whereas bogdan focussed on ...
%\section{Material}
%\begin{itemize}
        %\item MoMC solves 30 / 100 instances 
        %\item Kernel + MoMC solves 68 / 100 instances 
        %\item Kernel + BnR solves 55 / 100 instances 
        %\item Kernel + BnR - Local Search solves 42 / 100 instances 
        %\item Full solver solves 82 / 100 instances 
%\end{itemize}

%\begin{description}
%\item[GitHub:] \url{https://github.com/sebalamm/pace-2019/releases/tag/pace-2019}
%\item[DOI:] \url{https://doi.org/10.5281/zenodo.2816116}
%\end{description}
%\section{TODOs}
%\begin{itemize}
        %\item TODO add more exact stuff, heuristics?
%\item TODO insert final three solvers of pace challenge
%\item create a table with instances solved
%\end{itemize}
\begin{table*}
\centering
\caption{Detailed per instance results.}
\label{tab:detailedresults1}
\begin{tabular}{l@{\hskip 25pt} rrrr|ccccc|rc}
\toprule
inst\# & $n$ &$m$& $n'$& $m'$ & MoMC & RMoMC & BnR & BnR-LS & FullA & $|VC|$ \\
                \midrule

001 &\numprint{6160}&\numprint{40207}&\numprint{0}&\numprint{0}&-&X&X&X&X&  \numprint{2586}&\\ 
003 &\numprint{60541}&\numprint{74220}&\numprint{0}&\numprint{0}&-&X&X&X&X&  \numprint{12190}&\\ 
005 &\numprint{200}&\numprint{819}&\numprint{192}&\numprint{800}&X&X&X&X&X&  \numprint{129}&\\ 
007 &\numprint{8794}&\numprint{10130}&\numprint{0}&\numprint{0}&-&X&X&X&X&  \numprint{4397}&\\ 
009 &\numprint{38452}&\numprint{174645}&\numprint{0}&\numprint{0}&-&X&X&X&X&  \numprint{21348}&\\ 
011 &\numprint{9877}&\numprint{25973}&\numprint{0}&\numprint{0}&-&X&X&X&X&  \numprint{4981}&\\ 
013 &\numprint{45307}&\numprint{55440}&\numprint{0}&\numprint{0}&-&X&X&X&X&  \numprint{8610}&\\ 
015 &\numprint{53610}&\numprint{65952}&\numprint{0}&\numprint{0}&-&X&X&X&X&  \numprint{10670}&\\ 
017 &\numprint{23541}&\numprint{51747}&\numprint{0}&\numprint{0}&-&X&X&X&X&  \numprint{12082}&\\ 
019 &\numprint{200}&\numprint{884}&\numprint{194}&\numprint{862}&X&X&X&X&X&  \numprint{130}&\\ 
021 &\numprint{24765}&\numprint{30242}&\numprint{0}&\numprint{0}&-&X&X&X&X&  \numprint{5110}&\\ 
023 &\numprint{27717}&\numprint{133665}&\numprint{0}&\numprint{0}&-&X&X&X&X&  \numprint{16013}&\\ 
025 &\numprint{23194}&\numprint{28221}&\numprint{0}&\numprint{0}&-&X&X&X&X&  \numprint{4899}&\\ 
027 &\numprint{65866}&\numprint{81245}&\numprint{0}&\numprint{0}&-&X&X&X&X&  \numprint{13431}&\\ 
029 &\numprint{13431}&\numprint{21999}&\numprint{0}&\numprint{0}&-&X&X&X&X&  \numprint{6622}&\\ 
031 &\numprint{200}&\numprint{813}&\numprint{198}&\numprint{818}&X&X&X&X&X&  \numprint{136}&\\ 
033 &\numprint{4410}&\numprint{6885}&\numprint{138}&\numprint{471}&-&X&X&X&X&  \numprint{2725}&\\ 
035 &\numprint{200}&\numprint{884}&\numprint{189}&\numprint{859}&X&X&X&X&X&  \numprint{133}&\\ 
037 &\numprint{198}&\numprint{824}&\numprint{194}&\numprint{810}&X&X&X&X&X&  \numprint{131}&\\ 
039 &\numprint{6795}&\numprint{10620}&\numprint{219}&\numprint{753}&-&X&X&X&X&  \numprint{4200}&\\ 
041 &\numprint{200}&\numprint{1040}&\numprint{200}&\numprint{1023}&X&X&X&X&X&  \numprint{139}&\\ 
043 &\numprint{200}&\numprint{841}&\numprint{198}&\numprint{844}&X&X&X&X&X&  \numprint{139}&\\ 
045 &\numprint{200}&\numprint{1044}&\numprint{200}&\numprint{1020}&X&X&X&X&X&  \numprint{137}&\\ 
047 &\numprint{200}&\numprint{1120}&\numprint{198}&\numprint{1080}&X&X&X&X&X&  \numprint{140}&\\ 
049 &\numprint{200}&\numprint{957}&\numprint{198}&\numprint{930}&X&X&X&X&X&  \numprint{136}&\\ 
051 &\numprint{200}&\numprint{1135}&\numprint{200}&\numprint{1098}&X&X&X&X&X&  \numprint{140}&\\ 
053 &\numprint{200}&\numprint{1062}&\numprint{200}&\numprint{1026}&X&X&X&X&X&  \numprint{139}&\\ 
055 &\numprint{200}&\numprint{958}&\numprint{194}&\numprint{938}&X&X&X&X&X&  \numprint{134}&\\ 
057 &\numprint{200}&\numprint{1200}&\numprint{197}&\numprint{1139}&X&X&X&X&X&  \numprint{142}&\\ 
059 &\numprint{200}&\numprint{988}&\numprint{193}&\numprint{954}&X&X&X&X&X&  \numprint{137}&\\ 
061 &\numprint{200}&\numprint{952}&\numprint{198}&\numprint{914}&X&X&X&X&X&  \numprint{135}&\\ 
063 &\numprint{200}&\numprint{1040}&\numprint{200}&\numprint{1011}&X&X&X&X&X&  \numprint{138}&\\ 
065 &\numprint{200}&\numprint{1037}&\numprint{200}&\numprint{1011}&X&X&X&X&X&  \numprint{138}&\\ 
067 &\numprint{200}&\numprint{1201}&\numprint{200}&\numprint{1174}&X&X&X&X&X&  \numprint{143}&\\ 
069 &\numprint{200}&\numprint{1120}&\numprint{196}&\numprint{1077}&X&X&X&X&X&  \numprint{140}&\\ 
071 &\numprint{200}&\numprint{984}&\numprint{200}&\numprint{952}&X&X&X&X&X&  \numprint{136}&\\ 
073 &\numprint{200}&\numprint{1107}&\numprint{200}&\numprint{1078}&X&X&X&X&X&  \numprint{139}&\\ 
075 &\numprint{26300}&\numprint{41500}&\numprint{500}&\numprint{3000}&-&-&X&-&X&  \numprint{16300}&\\ 
077 &\numprint{200}&\numprint{988}&\numprint{193}&\numprint{954}&X&X&X&X&X&  \numprint{137}&\\ 
079 &\numprint{26300}&\numprint{41500}&\numprint{500}&\numprint{3000}&-&-&X&-&X&  \numprint{16300}&\\ 
081 &\numprint{199}&\numprint{1124}&\numprint{197}&\numprint{1087}&X&X&X&X&X&  \numprint{141}&\\ 
083 &\numprint{200}&\numprint{1215}&\numprint{198}&\numprint{1182}&X&X&X&X&X&  \numprint{144}&\\ 
085 &\numprint{11470}&\numprint{17408}&\numprint{3539}&\numprint{25955}&-&-&-&-&-&  &\\ 
087 &\numprint{13590}&\numprint{21240}&\numprint{441}&\numprint{1512}&-&X&-&-&X&  \numprint{8400}&\\ 
089 &\numprint{57316}&\numprint{77978}&\numprint{16834}&\numprint{54847}&-&-&-&-&-&  &\\ 
091 &\numprint{200}&\numprint{1196}&\numprint{200}&\numprint{1163}&X&X&X&X&X&  \numprint{145}&\\ 
093 &\numprint{200}&\numprint{1207}&\numprint{200}&\numprint{1162}&X&X&X&X&X&  \numprint{143}&\\ 
095 &\numprint{15783}&\numprint{24663}&\numprint{510}&\numprint{1746}&-&X&-&-&X&  \numprint{9755}&\\ 
097 &\numprint{18096}&\numprint{28281}&\numprint{579}&\numprint{1995}&-&X&-&-&X&  \numprint{11185}&\\ 
099 &\numprint{26300}&\numprint{41500}&\numprint{500}&\numprint{3000}&-&-&X&-&X&  \numprint{16300}&\\ 
\bottomrule
\end{tabular}
\end{table*}

\begin{table*}
\centering

\caption{Detailed per instance results.}
\label{tab:detailedresults2}
\begin{tabular}{l@{\hskip 25pt} rrrr|ccccc|rc}
\toprule
inst\# & $n$ &$m$& $n'$& $m'$ & MoMC & RMoMC & BnR & BnR-LS & FullA & $|VC|$ \\
                \midrule

101 &\numprint{26300}&\numprint{41500}&\numprint{500}&\numprint{3000}&-&-&X&-&X&  \numprint{16300}&\\ 
103 &\numprint{15783}&\numprint{24663}&\numprint{513}&\numprint{1752}&-&X&-&-&X&  \numprint{9755}&\\ 
105 &\numprint{26300}&\numprint{41500}&\numprint{500}&\numprint{3000}&-&-&X&-&X&  \numprint{16300}&\\ 
107 &\numprint{13590}&\numprint{21240}&\numprint{435}&\numprint{1500}&-&X&-&-&X&  \numprint{8400}&\\ 
109 &\numprint{66992}&\numprint{90970}&\numprint{20336}&\numprint{66350}&-&-&-&-&-&  &\\ 
111 &\numprint{450}&\numprint{17831}&\numprint{450}&\numprint{17831}&X&X&-&-&X&  \numprint{420}&\\ 
113 &\numprint{26300}&\numprint{41500}&\numprint{500}&\numprint{3000}&-&-&X&-&X&  \numprint{16300}&\\ 
115 &\numprint{18096}&\numprint{28281}&\numprint{573}&\numprint{1986}&-&X&-&-&X&  \numprint{11185}&\\ 
117 &\numprint{18096}&\numprint{28281}&\numprint{582}&\numprint{2007}&-&X&-&-&X&  \numprint{11185}&\\ 
119 &\numprint{18096}&\numprint{28281}&\numprint{588}&\numprint{2016}&-&X&-&-&X&  \numprint{11185}&\\ 
121 &\numprint{18096}&\numprint{28281}&\numprint{579}&\numprint{1998}&-&X&-&-&X&  \numprint{11185}&\\ 
123 &\numprint{26300}&\numprint{41500}&\numprint{500}&\numprint{3000}&-&-&X&-&X&  \numprint{16300}&\\ 
125 &\numprint{26300}&\numprint{41500}&\numprint{500}&\numprint{3000}&-&-&X&-&X&  \numprint{16300}&\\ 
127 &\numprint{18096}&\numprint{28281}&\numprint{582}&\numprint{2001}&-&X&-&-&X&  \numprint{11185}&\\ 
129 &\numprint{15783}&\numprint{24663}&\numprint{507}&\numprint{1752}&-&X&-&-&X&  \numprint{9755}&\\ 
131 &\numprint{2980}&\numprint{5360}&\numprint{2179}&\numprint{6951}&X&-&-&-&-&  &\\ 
133 &\numprint{15783}&\numprint{24663}&\numprint{507}&\numprint{1746}&-&X&-&-&X&  \numprint{9755}&\\ 
135 &\numprint{26300}&\numprint{41500}&\numprint{500}&\numprint{3000}&-&-&X&-&X&  \numprint{16300}&\\ 
137 &\numprint{26300}&\numprint{41500}&\numprint{500}&\numprint{3000}&-&-&X&-&X&  \numprint{16300}&\\ 
139 &\numprint{18096}&\numprint{28281}&\numprint{579}&\numprint{1995}&-&X&-&-&X&  \numprint{11185}&\\ 
141 &\numprint{18096}&\numprint{28281}&\numprint{576}&\numprint{1995}&-&X&-&-&X&  \numprint{11185}&\\ 
143 &\numprint{18096}&\numprint{28281}&\numprint{582}&\numprint{2001}&-&X&-&-&X&  \numprint{11185}&\\ 
145 &\numprint{18096}&\numprint{28281}&\numprint{576}&\numprint{1989}&-&X&-&-&X&  \numprint{11185}&\\ 
147 &\numprint{18096}&\numprint{28281}&\numprint{567}&\numprint{1974}&-&X&-&-&X&  \numprint{11185}&\\ 
149 &\numprint{26300}&\numprint{41500}&\numprint{500}&\numprint{3000}&-&-&X&-&X&  \numprint{16300}&\\ 
151 &\numprint{15783}&\numprint{24663}&\numprint{501}&\numprint{1728}&-&X&-&-&X&  \numprint{9755}&\\ 
153 &\numprint{29076}&\numprint{45570}&\numprint{2124}&\numprint{16266}&-&-&-&-&-&  &\\ 
155 &\numprint{26300}&\numprint{41500}&\numprint{500}&\numprint{3000}&-&-&X&-&X&  \numprint{16300}&\\ 
157 &\numprint{2980}&\numprint{5360}&\numprint{2169}&\numprint{6898}&X&-&-&-&-&  &\\ 
159 &\numprint{18096}&\numprint{28281}&\numprint{582}&\numprint{2004}&-&X&-&-&X&  \numprint{11185}&\\ 
161 &\numprint{138141}&\numprint{227241}&\numprint{41926}&\numprint{202869}&-&-&-&-&-&  &\\ 
163 &\numprint{18096}&\numprint{28281}&\numprint{582}&\numprint{2004}&-&X&-&-&X&  \numprint{11185}&\\ 
165 &\numprint{18096}&\numprint{28281}&\numprint{576}&\numprint{1995}&-&X&-&-&X&  \numprint{11185}&\\ 
167 &\numprint{15783}&\numprint{24663}&\numprint{510}&\numprint{1746}&-&X&-&-&X&  \numprint{9755}&\\ 
169 &\numprint{4768}&\numprint{8576}&\numprint{3458}&\numprint{11014}&-&-&-&-&-&  &\\ 
171 &\numprint{18096}&\numprint{28281}&\numprint{576}&\numprint{1989}&-&X&-&-&X&  \numprint{11185}&\\ 
173 &\numprint{56860}&\numprint{77264}&\numprint{17090}&\numprint{55568}&-&-&-&-&-&  &\\ 
175 &\numprint{3523}&\numprint{6446}&\numprint{2723}&\numprint{8570}&-&-&-&-&-&  &\\ 
177 &\numprint{5066}&\numprint{9112}&\numprint{3704}&\numprint{11797}&-&-&-&-&-&  &\\ 
179 &\numprint{15783}&\numprint{24663}&\numprint{504}&\numprint{1740}&-&X&-&-&X&  \numprint{9755}&\\ 
181 &\numprint{18096}&\numprint{28281}&\numprint{573}&\numprint{1989}&-&X&X&-&X&  \numprint{11185}&\\ 
183 &\numprint{72420}&\numprint{118362}&\numprint{30340}&\numprint{133872}&-&-&-&-&-&  &\\ 
185 &\numprint{3523}&\numprint{6446}&\numprint{2723}&\numprint{8568}&-&-&-&-&-&  &\\ 
187 &\numprint{4227}&\numprint{7734}&\numprint{3264}&\numprint{10286}&-&-&-&-&-&  &\\ 
189 &\numprint{7400}&\numprint{13600}&\numprint{5802}&\numprint{18212}&-&-&-&-&-&  &\\ 
191 &\numprint{4579}&\numprint{8378}&\numprint{3539}&\numprint{11137}&-&-&-&-&-&  &\\ 
193 &\numprint{7030}&\numprint{12920}&\numprint{5510}&\numprint{17294}&-&-&-&-&-&  &\\ 
195 &\numprint{1150}&\numprint{81068}&\numprint{1150}&\numprint{81068}&-&-&-&-&-&  &\\ 
197 &\numprint{1534}&\numprint{127011}&\numprint{1534}&\numprint{127011}&-&-&-&-&-&  &\\ 
199 &\numprint{1534}&\numprint{126163}&\numprint{1534}&\numprint{126163}&-&-&-&-&-&  &\\ 

\bottomrule
\end{tabular}
\end{table*}
\vfill 
\pagebreak
\
\vfill
\pagebreak
\bibliographystyle{plainurl}
\bibliography{references}
\vfill
\end{document}
